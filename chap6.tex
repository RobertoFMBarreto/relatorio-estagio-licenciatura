\chapter{Conclusão e futuras implementações}

O desenvolvimento deste projeto resultou numa solução completa preparada para uma primeira versão de teste no mercado. Este desenvolvimento conteve grandes dificuldades que permitiram adquirir novas capacidades e assimilar conhecimentos obtidos.

Entre as diversas dificuldades encontradas é importante destacar a forma como se comunica com o cliente e a importância de uma conversa onde este explica o processo de negócio da empresa, pois, a maioria dos problemas encontrados referentes a especificação do \textit{software} poderiam ter sido resolvidas se esta conversa tivesse sido realizada no início do projeto. Já no desenvolvimento do \textit{software}, as principais dificuldades encontradas foram a implementação dos \textit{links} para abrir um ecrã de uma aplicação, visto que, não funcionam como os \textit{links} na \textit{web}, notificações, uma vez que, tinham de enviar dados específicos para o dispositivo e a implementação de \textit{iOS}. Esta tecnologia nunca tinha sigo explorada devido a limitações da produtora, que obrigam a que seja utilizado um dispositivo \textit{Apple} para o desenvolvimento \textit{iOS}.

Apesar das diversas dificuldades, estas agregaram para novas capacidades, assim como, a exploração de novas tecnologias. A nível da segurança foi explorada a segurança das variáveis de ambiente do \textit{backend}, uma vez que, não teria sido possível ainda explorar estas tecnologias. A nível de \textit{frontend} foi possível explorar desenvolvimento \textit{cross-platform}, tendo sido esta uma experiência completamente diferente, visto que, é necessário a todo o momento pensar em ambas as plataformas de desenvolvimento. Associado ao desenvolvimento \textit{cross-platform} foi aprendida uma nova linguagem de programação, o \textit{Dart} e uma nova ferramenta de desenvolvimento, o \textit{Flutter}, acompanhado de uma nova abordagem ao desenvolvimento de \textit{software} através da utilização de \textit{widgets}. Para além destas novas tecnologias, o \textit{frontend} permitiu explorar funcionalidades, tais como, obter imagens e vídeos da galeria do dispositivo, reprodução de vídeos e abertura de um ecrã da aplicação através das notificações e \textit{links}.

Futuramente seria de grande importância disponibilizar uma versão de testes da aplicação, para assim obter o \emph{feedback} dos utilizadores sobre formas de melhorar a experiência de utilização da aplicação e a resolução de eventuais erros. Para além destas melhorias, seria relevante o desenvolvimento de um reprodutor de vídeo próprio, em vez da utilização do reprodutor de \emph{iOS}. A implementação de funcionalidades como descarregar imagens de outros utilizadores e editar imagens antes de carregar para a plataforma, também seria notável, dado que, forneceria uma melhor comunicação entre utilizadores. 

Estas novas competências adquiridas e todo o aprendizado com as dificuldades encontradas, assim como os erros cometidos agregaram para a experiência em toda a área de desenvolvimento de \emph{software}, sendo este um dos pontos mais críticos no mercado de trabalho.


