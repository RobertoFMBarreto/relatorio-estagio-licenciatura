% \usepackage{color}
% \usepackage{tabularray}
\definecolor{Concrete}{rgb}{0.952,0.952,0.952}
\begin{table}[htb]
\centering
\begin{tblr}{
  width = \linewidth,
  colspec = {Q[252]Q[310]Q[379]},
  row{6} = {Concrete},
  cell{1}{1} = {Concrete},
  cell{1}{2} = {c=2}{0.689\linewidth},
  cell{2}{1} = {Concrete},
  cell{2}{2} = {c=2}{0.689\linewidth},
  cell{3}{1} = {Concrete},
  cell{3}{2} = {c=2}{0.689\linewidth},
  cell{4}{1} = {Concrete},
  cell{4}{2} = {c=2}{0.689\linewidth},
  cell{5}{1} = {Concrete},
  cell{5}{2} = {c=2}{0.689\linewidth},
  cell{6}{2} = {c},
  cell{6}{3} = {c},
  cell{7}{1} = {r=2}{Concrete},
  cell{9}{1} = {r=2}{Concrete},
  vlines,
  hline{1-7,9,11} = {-}{},
  hline{8,10} = {2-3}{},
}
Caso de Uso           & Ver perfil                 &                                 \\
Descrição             & Ver perfil do técnico      &                                 \\
Ator                  & Técnico                    &                                 \\
Pré-condição          & -                          &                                 \\
Pós-condição          & -                          &                                 \\
                      & Ator                       & Sistema                         \\
Fluxo Principal       & 1-Alterar \textit{email}            &                                 \\
                      &                            & 2-Alteração de \textit{email}            \\
Fluxo Alternativo(A1) & 1-Alterar imagem de perfil &                                 \\
                      &                            & 2-Alteração de imagem de perfil 
\end{tblr}
\end{table}