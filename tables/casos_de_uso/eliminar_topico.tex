% \usepackage{color}
% \usepackage{tabularray}
\definecolor{Concrete}{rgb}{0.952,0.952,0.952}
\begin{table}[htb]
\centering
\label{tab:10}
\caption{Tabela de especificação do caso de uso de eliminar tópico}
\begin{tblr}{
  width = \linewidth,
  colspec = {Q[290]Q[371]Q[273]},
  row{6} = {Concrete},
  cell{1}{1} = {Concrete},
  cell{1}{2} = {c=2}{0.644\linewidth},
  cell{2}{1} = {Concrete},
  cell{2}{2} = {c=2}{0.644\linewidth},
  cell{3}{1} = {Concrete},
  cell{3}{2} = {c=2}{0.644\linewidth},
  cell{4}{1} = {Concrete},
  cell{4}{2} = {c=2}{0.644\linewidth},
  cell{5}{1} = {Concrete},
  cell{5}{2} = {c=2}{0.644\linewidth},
  cell{6}{2} = {c},
  cell{6}{3} = {c},
  cell{7}{1} = {r=2}{Concrete},
  cell{9}{1} = {Concrete},
  cell{9}{2} = {c},
  cell{9}{3} = {c},
  vlines,
  hline{1-7,9-10} = {-}{},
  hline{8} = {2-3}{},
}
Caso de Uso           & Eliminar tópico             &                    \\
Descrição             & Eliminar um tópico do fórum &                    \\
Ator                  & Técnico                     &                    \\
Pré-condição          & Clicar no tópico desejado   &                    \\
Pós-condição          & Remoção do tópico           &                    \\
                      & Ator                        & Sistema            \\
Fluxo Principal       & 1-Clicar em remover tópico  &                    \\
                      &                             & 3-Remover o tópico \\
Fluxo Alternativo(A1) & -                           & -                  
\end{tblr}
\end{table}