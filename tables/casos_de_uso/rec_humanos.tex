% \usepackage{color}
% \usepackage{tabularray}
\definecolor{Concrete}{rgb}{0.952,0.952,0.952}
\begin{table}[htb]
\centering
\begin{tblr}{
  width = \linewidth,
  colspec = {Q[331]Q[454]Q[142]},
  row{6} = {Concrete},
  cell{1}{1} = {Concrete},
  cell{1}{2} = {c=2}{0.627\linewidth},
  cell{2}{1} = {Concrete},
  cell{2}{2} = {c=2}{0.627\linewidth},
  cell{3}{1} = {Concrete},
  cell{3}{2} = {c=2}{0.627\linewidth},
  cell{4}{1} = {Concrete},
  cell{4}{2} = {c=2}{0.627\linewidth},
  cell{5}{1} = {Concrete},
  cell{5}{2} = {c=2}{0.627\linewidth},
  cell{6}{2} = {c},
  cell{6}{3} = {c},
  cell{7}{1} = {r=3}{Concrete},
  cell{10}{1} = {Concrete},
  cell{10}{2} = {c},
  cell{10}{3} = {c},
  vlines,
  hline{1-7,10-11} = {-}{},
  hline{8-9} = {2-3}{},
}
Caso de Uso           & Registar técnico                     &                    \\
Descrição             & Registar conta de técnico da empresa &                    \\
Ator                  & Empresa                              &                    \\
Pré-condição          & -                                    &                    \\
Pós-condição          & -                                    &                    \\
                      & Ator                                 & Sistema            \\
Fluxo Principal       & 1-Indicar o nºcontribuinte           &                    \\
                      & 2-Indicar email                      &                    \\
                      &                                      & 3-Registar técnico \\
Fluxo Alternativo(A1) & -                                    & -                  
\end{tblr}
\end{table}