% \usepackage{color}
% \usepackage{tabularray}
\definecolor{Concrete}{rgb}{0.952,0.952,0.952}
\begin{table}[htb]
\centering
\label{tab:11}
\caption{Tabela de especificação de caso de uso de alteração de visibilidade de um tópico}
\begin{tblr}{
 width = \linewidth,
 colspec = {Q[242]Q[344]Q[354]},
 row{6} = {Concrete},
 cell{1}{1} = {Concrete},
 cell{1}{2} = {c=2}{0.698\linewidth},
 cell{2}{1} = {Concrete},
 cell{2}{2} = {c=2}{0.698\linewidth},
 cell{3}{1} = {Concrete},
 cell{3}{2} = {c=2}{0.698\linewidth},
 cell{4}{1} = {Concrete},
 cell{4}{2} = {c=2}{0.698\linewidth},
 cell{5}{1} = {Concrete},
 cell{5}{2} = {c=2}{0.698\linewidth},
 cell{6}{2} = {c},
 cell{6}{3} = {c},
 cell{7}{1} = {r=2}{Concrete},
 cell{9}{1} = {Concrete},
 cell{9}{2} = {c},
 cell{9}{3} = {c},
 vlines,
 hline{1-7,9-10} = {-}{},
 hline{8} = {2-3}{},
}
Caso de Uso      & Alterar visibilidade do tópico               &                  \\
Descrição       & Alterar a visibilidade de um tópico entre público e privado &                  \\
Ator         & Técnico                           &                  \\
Pré-condição     & Clicar no tópico desejado                  &                  \\
Pós-condição     & Alterar visibilidade do tópico               &                  \\
           & Ator                            & Sistema              \\
Fluxo Principal    & 1-Clicar em alterar visibilidade              &                  \\
           &                               & 2-Inverter visibilidade do tópico \\
Fluxo Alternativo(A1) & -                              & -                 
\end{tblr}
\end{table}