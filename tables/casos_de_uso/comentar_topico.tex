% \usepackage{color}
% \usepackage{tabularray}
\definecolor{Concrete}{rgb}{0.952,0.952,0.952}
\begin{table}[htb]
\centering
\label{tab:14}
\caption{Tabela de especificação de caso de uso de comentar um tópico}
\begin{tblr}{
 width = \linewidth,
 colspec = {Q[225]Q[348]Q[367]},
 row{6} = {Concrete},
 cell{1}{1} = {Concrete},
 cell{1}{2} = {c=2}{0.715\linewidth},
 cell{2}{1} = {Concrete},
 cell{2}{2} = {c=2}{0.715\linewidth},
 cell{3}{1} = {Concrete},
 cell{3}{2} = {c=2}{0.715\linewidth},
 cell{4}{1} = {Concrete},
 cell{4}{2} = {c=2}{0.715\linewidth},
 cell{5}{1} = {Concrete},
 cell{5}{2} = {c=2}{0.715\linewidth},
 cell{6}{2} = {c},
 cell{6}{3} = {c},
 cell{7}{1} = {r=4}{Concrete},
 cell{11}{1} = {r=3}{Concrete},
 cell{14}{1} = {Concrete},
 vlines,
 hline{1-7,11,14-15} = {-}{},
 hline{8-10,12-13} = {2-3}{},
}
Caso de Uso      & Comentar o tópico         &                   \\
Descrição       & Comentar um tópico        &                   \\
Ator         & Técnico              &                   \\
Pré-condição     & Clicar no tópico desejado     &                   \\
Pós-condição     & Inserir a resposta no tópico   &                   \\
           & Ator               & Sistema               \\
Fluxo Principal    & 1-Indicar a descrição da resposta &                   \\
           & 2-Anexar Imagem          &                   \\
           & 3-Confirmar a resposta      &                   \\
           &                  & 4-Inserir novo comentário no tópico \\
Fluxo Alternativo(A1) & 1-Indicar a descrição da resposta &                   \\
           & 2-Confirmar a resposta      &                   \\
           &                  & 3-Inserir novo comentário no tópico \\
Fluxo Alternativo(A2) & 1-Cancelar a criação do tópico  &                   
\end{tblr}
\end{table}