% \usepackage{color}
% \usepackage{tabularray}
\definecolor{Concrete}{rgb}{0.952,0.952,0.952}
\begin{table}[htb]
\centering
\label{tab:18}
\caption{Tabela de especificação de caso de uso de responder a comentário}
\begin{tblr}{
  width = \linewidth,
  colspec = {Q[229]Q[381]Q[333]},
  row{6} = {Concrete},
  cell{1}{1} = {Concrete},
  cell{1}{2} = {c=2}{0.714\linewidth},
  cell{2}{1} = {Concrete},
  cell{2}{2} = {c=2}{0.714\linewidth},
  cell{3}{1} = {Concrete},
  cell{3}{2} = {c=2}{0.714\linewidth},
  cell{4}{1} = {Concrete},
  cell{4}{2} = {c=2}{0.714\linewidth},
  cell{5}{1} = {Concrete},
  cell{5}{2} = {c=2}{0.714\linewidth},
  cell{6}{2} = {c},
  cell{6}{3} = {c},
  cell{7}{1} = {r=2}{Concrete},
  cell{9}{1} = {Concrete},
  cell{9}{2} = {c},
  cell{9}{3} = {c},
  vlines,
  hline{1-7,9-10} = {-}{},
  hline{8} = {2-3}{},
}
Caso de Uso           & Responder a comentário                 &                                 \\
Descrição             & Responder a um comentário de um tópico &                                 \\
Ator                  & Técnico                                &                                 \\
Pré-condição          & Clicar no tópico desejado              &                                 \\
Pós-condição          & Novo comentário                        &                                 \\
                      & Ator                                   & Sistema                         \\
Fluxo Principal       & 1-Clicar em responder a comentário     &                                 \\
                      &                                        & 2-Inserir resposta a comentário \\
Fluxo Alternativo(A1) & -                                      & -                               
\end{tblr}
\end{table}