% \usepackage{color}
% \usepackage{tabularray}
\definecolor{Concrete}{rgb}{0.952,0.952,0.952}
\begin{table}[htb]
\centering
\begin{tblr}{
 width = \linewidth,
 colspec = {Q[287]Q[285]Q[363]},
 row{6} = {Concrete},
 cell{1}{1} = {Concrete},
 cell{1}{2} = {c=2}{0.647\linewidth},
 cell{2}{1} = {Concrete},
 cell{2}{2} = {c=2}{0.647\linewidth},
 cell{3}{1} = {Concrete},
 cell{3}{2} = {c=2}{0.647\linewidth},
 cell{4}{1} = {Concrete},
 cell{4}{2} = {c=2}{0.647\linewidth},
 cell{5}{1} = {Concrete},
 cell{5}{2} = {c=2}{0.647\linewidth},
 cell{6}{2} = {c},
 cell{6}{3} = {c},
 cell{7}{1} = {r=2}{Concrete},
 cell{9}{1} = {r=4}{Concrete},
 vlines,
 hline{1-7,9,13} = {-}{},
 hline{8,10-12} = {2-3}{},
}
Caso de Uso      & Ver Notificações      &              \\
Descrição       & Ver notificações do técnico &              \\
Ator         & Técnico           &              \\
Pré-condição     & -              &              \\
Pós-condição     & -              &              \\
           & Ator            & Sistema          \\
Fluxo Principal    & 1-Ver notificações     &              \\
           &               & 2-Listagem de notificações \\
Fluxo Alternativo(A1) & 1-Ver notificações     &              \\
           &               & 2-Listagem de notificações \\
           & 3-Apagar notificação    &              \\
           &               & 4-Eliminar notificação   
\end{tblr}
\end{table}