% \usepackage{color}
% \usepackage{tabularray}
\definecolor{Concrete}{rgb}{0.952,0.952,0.952}
\begin{table}[htb]
\centering
\begin{tblr}{
  width = \linewidth,
  colspec = {Q[223]Q[356]Q[362]},
  row{6} = {Concrete},
  cell{1}{1} = {Concrete},
  cell{1}{2} = {c=2}{0.706\linewidth},
  cell{2}{1} = {Concrete},
  cell{2}{2} = {c=2}{0.706\linewidth},
  cell{3}{1} = {Concrete},
  cell{3}{2} = {c=2}{0.706\linewidth},
  cell{4}{1} = {Concrete},
  cell{4}{2} = {c=2}{0.706\linewidth,c},
  cell{5}{1} = {Concrete},
  cell{5}{2} = {c=2}{0.706\linewidth},
  cell{6}{2} = {c},
  cell{6}{3} = {c},
  cell{7}{1} = {r=3}{Concrete},
  cell{10}{1} = {Concrete},
  cell{10}{2} = {c},
  cell{10}{3} = {c},
  vlines,
  hline{1-7,10-11} = {-}{},
  hline{8-9} = {2-3}{},
}
Caso de Uso           & Pedir reenvio de código de ativação          &                                 \\
Descrição             & Pedir reenvio de email de código de ativação &                                 \\
Ator                  & Técnico                                      &                                 \\
Pré-condição          & -                                            &                                 \\
Pós-condição          & Email de verificação de código               &                                 \\
                      & Ator                                         & Sistema                         \\
Fluxo Principal       & 1-Pedir novo código de ativação              &                                 \\
                      &                                              & 2-Gerar novo código de ativação \\
                      &                                              & 3-Enviar novo email             \\
Fluxo Alternativo(A1) & -                                            & -                               
\end{tblr}
\end{table}