% \usepackage{color}
% \usepackage{tabularray}
\definecolor{Concrete}{rgb}{0.952,0.952,0.952}
\begin{table}[htb]
\centering
\begin{tblr}{
 width = \linewidth,
 colspec = {Q[185]Q[337]Q[419]},
 row{6} = {Concrete},
 cell{1}{1} = {Concrete},
 cell{1}{2} = {c=2}{0.756\linewidth},
 cell{2}{1} = {Concrete},
 cell{2}{2} = {c=2}{0.756\linewidth},
 cell{3}{1} = {Concrete},
 cell{3}{2} = {c=2}{0.756\linewidth},
 cell{4}{1} = {Concrete},
 cell{4}{2} = {c=2}{0.756\linewidth},
 cell{5}{1} = {Concrete},
 cell{5}{2} = {c=2}{0.756\linewidth,c},
 cell{6}{2} = {c},
 cell{6}{3} = {c},
 cell{7}{1} = {r=2}{Concrete},
 cell{9}{1} = {r=2}{Concrete},
 vlines,
 hline{1-7,9,11} = {-}{},
 hline{8,10} = {2-3}{},
}
Caso de Uso      & Ver todos os comentários                 &                           \\
Descrição       & Ver todos os comentários, sejam de apenas técnicos ou não &                           \\
Ator         & Utilizador                          &                           \\
Pré-condição     & Clicar no tópico desejado                 &                           \\
Pós-condição     & -                             &                           \\
           & Ator                           & Sistema                       \\
Fluxo Principal    & 1-Clicar em todos os comentários             &                           \\
           &                              & 2-Lista de todos os comentários           \\
Fluxo Alternativo(A1) & 1-Clicar em comentários de profissionais         &                           \\
           &                              & 2-Filtragem da lista de comentários de profissionais 
\end{tblr}
\end{table}