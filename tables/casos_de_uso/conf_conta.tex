% \usepackage{color}
% \usepackage{tabularray}
\definecolor{Concrete}{rgb}{0.952,0.952,0.952}
\begin{table}[htb]
\centering
\begin{tblr}{
 width = \linewidth,
 colspec = {Q[258]Q[429]Q[254]},
 row{6} = {Concrete},
 cell{1}{1} = {Concrete},
 cell{1}{2} = {c=2}{0.683\linewidth},
 cell{2}{1} = {Concrete},
 cell{2}{2} = {c=2}{0.683\linewidth},
 cell{3}{1} = {Concrete},
 cell{3}{2} = {c=2}{0.683\linewidth},
 cell{4}{1} = {Concrete},
 cell{4}{2} = {c=2}{0.683\linewidth,c},
 cell{5}{1} = {Concrete},
 cell{5}{2} = {c=2}{0.683\linewidth,c},
 cell{6}{2} = {c},
 cell{6}{3} = {c},
 cell{7}{1} = {r=6}{Concrete},
 cell{13}{1} = {Concrete},
 vlines,
 hline{1-7,13-14} = {-}{},
 hline{8-12} = {2-3}{},
}
Caso de Uso      & Confirmar conta          &            \\
Descrição       & Confirmar conta de técnico    &            \\
Ator         & Técnico              &            \\
Pré-condição     & -                 &            \\
Pós-condição     & -                 &            \\
           & Ator               & Sistema        \\
Fluxo Principal    & 1-Inserir nome de utilizador   &            \\
           & 2-Indicar Password        &            \\
           & 3-Indicar Confirmação de \textit{password} &            \\
           &                  & 4-Verificar se registo \\
           &                  & 5-Conta registada   \\
           &                  & 6-Validar conta    \\
Fluxo Alternativo(A1) & 1-Cancelar Ativação de conta   &            
\end{tblr}
\end{table}