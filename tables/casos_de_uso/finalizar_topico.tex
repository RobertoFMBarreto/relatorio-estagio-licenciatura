% \usepackage{color}
% \usepackage{tabularray}
\definecolor{Concrete}{rgb}{0.952,0.952,0.952}
\begin{table}[htb]
\centering
\label{tab:8}
\caption{Tabela de especificação de caso de uso de finalizar tópico}
\begin{tblr}{
  width = \linewidth,
  colspec = {Q[254]Q[319]Q[365]},
  row{6} = {Concrete},
  cell{1}{1} = {Concrete},
  cell{1}{2} = {c=2}{0.683\linewidth},
  cell{2}{1} = {Concrete},
  cell{2}{2} = {c=2}{0.683\linewidth},
  cell{3}{1} = {Concrete},
  cell{3}{2} = {c=2}{0.683\linewidth},
  cell{4}{1} = {Concrete},
  cell{4}{2} = {c=2}{0.683\linewidth},
  cell{5}{1} = {Concrete},
  cell{5}{2} = {c=2}{0.683\linewidth},
  cell{6}{2} = {c},
  cell{6}{3} = {c},
  cell{7}{1} = {r=2}{Concrete},
  cell{9}{1} = {Concrete},
  cell{9}{2} = {c},
  cell{9}{3} = {c},
  vlines,
  hline{1-7,9-10} = {-}{},
  hline{8} = {2-3}{},
}
Caso de Uso           & Finalizar tópico                                     &                                  \\
Descrição             & Finalizar um tópico para indicar que está respondido &                                  \\
Ator                  & Técnico                                              &                                  \\
Pré-condição          & Clicar no tópico desejado                            &                                  \\
Pós-condição          & Alterar tópico para finalizado                       &                                  \\
                      & Ator                                                 & Sistema                          \\
Fluxo Principal       & 1-Clicar em finalizar tópico                         &                                  \\
                      &                                                      & 2-Alterar tópico para finalizado \\
Fluxo Alternativo(A1) & -                                                    & -                                
\end{tblr}
\end{table}