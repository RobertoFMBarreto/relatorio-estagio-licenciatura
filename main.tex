\documentclass[a4paper,12pt,twoside]{book}

% nivel de numeração de titulos
\setcounter{secnumdepth}{3}
% nivel de leitura de titulos do indice
\setcounter{tocdepth}{3}

\usepackage{lesi/lesi}
\usepackage{color}
\usepackage{tabularray}
\usepackage{longtable}


\title{App Install \& Go}
\author{Roberto Filipe Manso Barreto}
\nrAluno{21123}
\regimeDiurno         % ou \regimePosLaboral
\LESI
\date{2022/2023}

% Caso tenham mais que um orientador, colocar
% \orientador{Nome professor \AND Nome professor}
\orientador{Luís Gonzaga Martins Ferreira}

% comentar estas três linhas para projectos
\empresa{Motorline Eletrocelos S.A}
\enderecoEmpresa{Travessa do Sobreiro, 29 Rio Côvo (Sta. Eugénia) 4755-474 Barcelos}
\supervisor{Eng. Helder Remelhe}

% comentar se não for para usar glossários
%\makeglossaries

\begin{document}

\frontmatter

\maketitle


\begin{resumo}
Este documento trata o processo de análise, especificação e implementação da solução Install\&Go. Esta vem resolver o problema de comunicação entre a empresa Motorline e os seus técnicos, uma vez que no acontecimento de um problema estes deverão telefonar para a empresa, o que gera sobrecarga.

A resolução deste problema vem com o desenvolvimento de uma plataforma de fórum onde as empresas podem registar os seus técnicos, sendo que estes conseguem então expor as suas questões no fórum onde outros técnicos de outras empresas e/ou Motorline poderão auxiliar. Esta permite também que no caso de um técnico possuir um problema já resolvido, este poderá pesquisar pela solução dada ao mesmo na plataforma.

O desenvolvimento desta solução proveu a aquisição de novas capacidades técnológicas como o desenvolvimento cross-platform e as suas frameworks, sendo que destas foi explorado flutter. Este permitiu também a assimilar capacidades como análise, especificação de projetos e comunicação com clientes. Por fim foi possível desenvolver completamente a solução de acordo com as necessidades e expectativas do cliente.

% Resumo do trabalho realizado. Deve ser sucinto, e cobrir todo o relatório: uma introdução ao problema que se pretendeu resolver, um pequeno resumo da abordagem realizada, e algumas conclusões do trabalho atingido.

% Poderão ser criados vários parágrafos, até para que cada um corresponda às três fases de introdução, desenvolvimento e conclusão.

% Não é relevante colocar no resumo o local de estágio ou a referência ao curso. Essa informação já consta da capa.
\end{resumo}

\begin{abstract}
This document relates the analysis, specification and development of the Install\&Go software. This software solves the comunication problem between Motorline and their professionals, since that in the event of a problem they must call the company, which generates an overload.

The solution to this problem comes with the development of a forum where companies can register their professionals, and these cand then expose their questions so that other professionals from other companies or Motorline it self can help. This also allow that if a professional has a problem that was already solved, he can search for the solution on the platform.

The development of this solution provided the acquisition of new technical skills such as cross-plataform development and its frameworks, of which flutter was explored. It also allowed the assimilation of skill such as project analysis, specification and comunication with clients. At the end, it was possible to fully develop the solution according to the client's needs and expectations.

% This is the translation of the previous text. It should say the exact same thing. Please do not use directly Google Translator.
\end{abstract}

% Comment the following part if you are not acknowledging anybody
\begin{agradecimentos}
% [A secção de agradecimentos é a parte pessoal do documento, e o único sítio onde o aluno pode escrever de forma menos formal, usando o tipo de linguagem que lhe parecer adequado para as pessoas a quem agradece.]

Em primeiro lugar, gostaria de agradecer à minha família, com destaque à minha mãe, ao meu padrinho, aos meus avós e à minha namorada, que com todo o carinho orientaram-me nesta caminhada que foi a licenciatura.

Também, gostaria de salientar um caloroso agradecimento à empresa Motorline, por me terem sempre feito sentir um membro da equipa, com destaque ao supervisor Helder Remelhe que sempre se apresentou disponível para eventuais dúvidas que surgissem no desenvolvimento do projeto.

Por fim, mas não menos importante, gostaria de enfatizar toda a orientação, disponibilidade, conversa e ensinamentos proporcionados pelo doutor professor Luís Ferreira no de e também aos meus colegas de curso Henrique Cartucho e João Castro por todo o apoio e todos os momentos vividos ao longo da licenciatura.
\end{agradecimentos}


\tableofcontents

% comentar se nao tiver figuraIs
%\listoffigures

% comentar se nao tiver tabelas
%\listoftables

% comentar se nao se quiser lista de listagens
%\lstlistoflistings

% Commentar proximas duas linhas se nao for para usar acronimos
% \newacronym{api}{API}{Application Programming Interface}
 \newacronym{mvc}{MVC}{Model View Controller}
 \newacronym{json}{JSON}{JavaScript Object Notation}
 \newacronym{us}{US}{User Stories}
 \newacronym{uc}{UC}{Use Cases}
 
%\printglossary[type=\acronymtype,title={Siglas \& Acrónimos}]

% Commentar proximas duas linhas se nao for para usar glossarios
%\input{glossario}
%\printglossary


%\printglossary

 
\mainmatter


% novos capitulos em paginas impares
% 
\chapter{Introdução}
Este projeto intitulado \textit{Install \& Go}, refere-se a uma aplicação para \textit{smartphone} direcionada a todos os clientes e 
técnicos Motorline, com o propósito de agilizar todo o processo de resolução de problemas e de acesso 
aos produtos. Para alcançar estes objetivos, a aplicação conta com um catálogo para todos os utilizadores 
e também com um fórum para os clientes e técnicos Motorline.

No início de estágio foi indicado pelo supervisor da empresa a subdivisão do projeto em duas grandes componentes. Também este projeto foi desenvolvido com uma colega que ficou encarregue de desenvolver o \textit{frontend} da componente do catálogo. Já o desenvolvimento completo do \textit{backend}, aquisição do catálogo de produtos do \textit{website} da empresa e o \textit{frontend} para o fórum, gestão de utilizadores e autenticação ficaram a meu cargo.
%//TODO: Fala com prof sobre a utilização do meu aqui, usar neste relatório será abordado

O fórum da aplicação é uma plataforma que permite aos clientes e técnicos criar publicações que permitem realizar questões e/ou expor problemas para a comunidade. Estas publicações podem ser comentadas, onde também é possível realizar uma comunicação.

O suporte \textit{backend} da aplicação foi realizado através de uma \textit{api backend}, sendo inicialmente apenas necessário para fórum. Contudo, foi posteriormente acrescentado suporte para o catálogo de produtos.

\newpage

\section{Objetivos}
A plataforma do fórum da aplicação deverá permitir que a comunidade partilhe as suas publicações. Para isso, o utilizador necessitará conseguir criar estas com a indicação do problema e uma descrição do mesmo, com a possibilidade de anexar imagens, assim como referenciar o produto, para facilitar a identificação e resolução do problema.

A comunidade deverá também ter a possibilidade de comentar as publicações e comunicar nesta secção. O responsável pela publicação necessitará ter a possibilidade de a colocar em privado caso tenha como objetivo que apenas técnicos Motorline respondam à mesma. Este também deverá ter a seu dispor a possibilidade de indicar quando a publicação estiver finalizada e qual a melhor resposta que obteve. Os utilizadores deverão também conseguir gostar de publicações e comentários para destacar os mesmos perante a restante comunidade.

A comunidade deverá conseguir ver as publicações em destaque, as mais recentes, as suas publicações, as que não estão finalizadas e os técnicos Motorline necessitarão de conseguir ver as publicações privadas existentes. A comunidade deverá ter a possibilidade de pesquisar por publicações com assuntos específicos, pesquisa por nome e por produto. Para filtragem de pesquisa a comunidade deverá também conseguir selecionar o tipo de publicação.

\section{Contexto}
O projeto foi desenvolvido na empresa Motorline Eletrocelos S.A daqui em diante designada Motorline, esta empresa é especializada na produção e comercialização de automatismos para portas e portões, sistemas de controlo de acessos, sistemas de segurança, entre outros produtos relacionados com o setor da automação.

Este projeto vem por este meio resolver o problema de comunicação com o cliente, que atualmente para solucionar as suas questões tem de contactar a assistência técnica que por vezes pode estar sobrecarregada, pelo que deverão preencher um formulário para expor a sua questão.

Com esta solução os clientes e técnicos Motorline poderão procurar ou expor os seus problemas com a comunidade onde têm a possibilidade de ser respondidos, o que torna o processo de resolução de problemas mais ágil.

\newpage
\section{Estrutura do documento}

Este documento contém a descrição de todo o processo de engenharia de \textit{software}, desenvolvimento e pensamento sobre o problema em mãos, sendo estes pontos divididos sobre diversos tópicos:
\begin{enumerate}
    \item Estado da Arte, onde é abordado as ferramentas utilizadas, tecnologias exploradas e planificação de trabalho.
    \item Análise e especificação, onde é abordado o estado da arte, descrito o modelo de negócio e apresentada toda a engenharia de \textit{software}
    \item Trabalho desenvolvido, onde é descrito todo o processo de desenvolvimento do projeto.
    \item Análise de resultados, onde é discutido os resultados obtidos.
    \item Conclusão e trabalho futuro, onde é abordada a conclusão sobre o projeto e também futuras implementações que poderiam ser realizadas.
\end{enumerate}
% 
\chapter{Estado da arte}
%Para a organização do trabalho foi utilizada a técnica de desenvolvimento ágil SCRUM
De forma a organizar todo o trabalho a desenvolver visto que este é dividido com mais uma colega, 
foi então utilizada a técnica de desenvolvimento ágil, conseguindo assim organizar todas as tarefas 
entre os elementos de desenvolvimento do projeto.

\section{Ferramentas de trabalho utilizadas}

A organização de todas as tarefas, foi realizada na ferramenta \textit{Github Projetcs}, esta 
permite ligar um projeto a um repositório de \textit{Github}, conseguindo também personalizar completamente todo 
o projeto e parâmetros das tarefas o que permite uma organização minuciosa destas.

O Microsoft Excel foi também utilizado para a engenharia de \textit{software} onde foram descritos os requisitos 
do projeto, \textit{user stories} e também a especificação de casos de uso. Esta ferramenta foi também utilizada 
para a organização de reuniões com o cliente e redação de tópicos a abordar e abordados nesta.

Para o desenvolvimento do \textit{design do software} foi utilizado a ferramenta \textit{figma}, que permite o \textit{design} de 
todas as componentes tendo em conta as reais dimensões de um dispositivo. Esta ferramenta 
permite também criar uma apresentação interativa que consegue demonstrar o comportamento da aplicação 
como completamente desenvolvida, dando também suporte à implementação.

O \textit{draw.io} foi também utilizado para o desenho das arquiteturas do projeto tendo se revelado de grande auxílio 
visto que este permite uma grande liberdade no desenho. Esta ferramenta permite também 
conexão com \textit{github} conseguindo assim facilmente guardar estes projetos e ter acesso a partir de qualquer 
dispositivo.

A engenharia de \textit{software} foi realizada através da utilização \textit{Visual Studio paradigm}, esta é uma ferramenta muito 
completa contendo modelos e regras para a engenharia de \textit{software}. Esta ferramenta tornou-se um grande 
auxílio no desenvolvimento da base de dados, pois é possível desenhar o modelo e exportar para um ficheiro 
de criação de base de dados.

\newpage

\section{Tecnologias utilizadas}

\subsection{Serviços Backend}

De forma a realizar a integração entre a aplicação \emph{frontend} e os dados, foi necessário desenvolver uma API para dar suporte a todos os serviços necessários para a aplicação.
API sigla para \emph{Application Programming Interface} disponibiliza um conjunto de funções e dados que facilita as interações entre aplicações e permite que troquem informação ~\cite{rest_cookbook}.
Esta ferramenta apesar de ser desenvolvida para trabalhar em conjunto com outros programas, ela são em sua grande maioria desenvolvidas para serem entendidas e utilizadas por outros programadores no
desenvolvimento dos seus programas ~\cite{api_design}.

\subsubsection{Serviços REST Full}
% //TODO:
Explicar o que é


\subsubsection{Typescript}
% //TODO:
Explicar o que é

\subsubsection{Logs e Logging}
\textit{Logs} "(...)\emph{are a very useful source of information for computer system resource management (printers, disk systems, battery backup systems, operating systems, etc.), user and application management (login and logout, application access, etc.), and security}(...)"\citep{Logging}. Um Log é "(...)\emph{what a computer system, device, software, etc. generates in response to some sort of stimuli.What exactly the stimuli are greatly depends on the source of the log message}(...)"\citep{Logging}, ou seja, perante um estímulo desejado, um \textit{log} poderá ser gerado. Os dados dos \textit{logs} são "(...)\emph{the intrinsic meaning that a log message has}(...)"\citep{Logging}, o que significa que estes contêm apenas dados relevantes ao objetivo do \textit{log}. \textit{Logging} é o nome que se dá ao processo de geração de \textit{logs}.

\newpage

Neste contexto \textit{logging} poderá ser utilizado para realizar a monitorização de pedidos e erros. Estas informações poderão até auxiliar na toma de decisões sobre o \textit{software} e em quais funcionalidades colocar mais atenção.

\subsubsection{Morgan}

\textit{Morgan} é uma biblioteca que permite extrair dados de um pedido, assim como a criação de \textit{logs}. Este atua como um \textit{middleware} do servidor, que recebe qual o tipo de \textit{log} a ser escrito, estes estão definidos na biblioteca. Os principais dados obtidos por este são, a data e hora do pedido, o tipo, o serviço, os dados recebidos, a resposta devolvida e também a descrição do sistema utilizado. Com estes dados é possível saber que plataforma é mais utilizada no \textit{software}, quais as horas de maior utilização e quais os serviços mais executados. Estes dados permitem direcionar mais recursos para uma indicada plataforma e/ou serviço, assim como escolher os melhores horários de manutenção dos servidores.

\newpage


\subsubsection{Envio de emails}
Para o envio de emails para os utilizadores, foi utilizada a ferramenta nodemailer, EXPLICAR O QUE é nodemailer..... Esta ferramenta foi escolhida devido a ser uma das mais utilizadas para esta tipo de necessidade, o que permite que exista mais informação sobre a mesma facilitando a resolução e identificação de erros. 

Para desenvolver o conteúdo dos emails foi utilizada a ferramenta Tabular Email, esta ferramenta permite realizar o design do conteúdo de um email, sendo possível de seguida exportar o mesmo para html, a dificuldade desta ferramenta é que não permite a utilização de acentuação e visto que o html é gerado por uma máquina este torna-se complicado de navegar e traduzir.

Após obter o servidor a utilizar e o conteúdo a enviar, é então utilizado o objeto do servidor de email e no envio é definido o destinatário, o assunto e o conteúdo do email.


\subsubsection{Agendamento de tarefas}

Um requisito para este projeto é o envio de emails com relatório diário de notificação todos os dias ao final do dia. Para realizar este primeiramente foi pesquisado que ferramentas existem para realizar este tipo de ações, pelo que foram encontradas o cronetab e o node-cron. A grande diferença este estas duas ferramentas é, o cronetab funciona a nivel de servidor sendo que sempre que se encontra na hora programada, este executa um comando indicado, este comando poderá por exemplo executar um código para enviar emails. Já o node-cron trata-se de uma biblioteca de NodeJs que trabalha com base no crontab, este permite o fácil agendamento de tarefas de forma programática assim como também a indicação do código a ser executado sem necessidade de criar comandos de execução de código.

Visto que a hora de execução do código de envio de emails poderá variar e necessitar de reprogramação foi então optado pela utilização do node-cron devido á sua facilidade de utilização, agilizando assim o processo de reprogramação de horas de envio de relatório.

\subsubsection{Encriptação de passwords}
De forma a garantir a segurança das passwords dos utilizadores é necessário encriptar estas, a encriptação poderá ser feita manualmente ou com o auxílio de ferramentas, a grande diferença é que manualmente poderá não se obter uma cifra tão segura como com o auxílio de uma ferramenta, sendo assim foi decidido utilizar uma ferramenta para encriptar passwords, a ferramenta escolhida foi bcrypt, esta foi escolhida devido a ser vastamente utilizada e até ao momento sem problemas em relação à sua cifra sendo que não existem registos de ataques bem sucedidos a esta ferramenta. Esta ferramenta oferece um conjunto de métodos sendo tendo sido utilizados os métodos de cifra e de comparação. O método de cifra permite através de um valor, chamado salt, indicar a complexidade a aplicar sobre a cifra, sendo de seguida devolvida a password cifrada. O método comparação permite comparar uma password cifrada com uma password sem cifra, devolvendo verdadeiro ou falso conforme as passwords sejam iguais ou não.

\newpage

\subsubsection{Cifragem de configurações do servidor}
De forma a garantir um nivel de segurança maior foram realizadas pesquisas sobre as principais falhas de segurança no NodeJs, pelo que foi descoberto que as principais formas de ataque a esta ferramenta é o desenvolvimento de bibliotecas de malware e o ataque às bibliotecas com o objetivo de obter dados de acesso a servidores que se encontram nos ficheiros de configuração.

Por norma todas as configurações de servidores são colocadas num ficheiro env, este ficheiro no momento de iniciar o servidor é utilizado para carregar todas as variáveis para o ambiente do mesmo, sendo assim qualquer um com acesso ao ficheiro ou às variáveis de ambiente poderá ver todas as configurações do servidor.

A solução mais indicada para este problema é a cifragem do ficheiro env e das variáveis de ambiente. A biblioteca mais utilizada para este objetivo é a secur-env, esta permite realizar a cifragem de um ficheiro indicando uma password. A password deverá ser indicada no processo de inicialização do servidor de forma a ser possível ao mesmo decifrar o ficheiro, sendo que a gestão das variáveis cifradas passa então a estar encarregue desta biblioteca.

Mesmo com esta solução existem possibilidades de ataque, pois é possível ver o histórico do terminal do servidor, pelo que e possível obter a password escrita no mesmo, para resolver este problema é indicada a biblioteca readline, pois esta possui o modo de password que apaga o histórico do terminal sempre que utilizado, esta contém a vertente assincrona, readline e a vertente síncrona, readline-sync. Para o projeto foi utilizada a versão síncrona da biblioteca visto que o objetivo é o servidor apenas inciar após a indicação da password, sem nenhum serviço a correr em simultâneo.

\newpage

\section{Planificação do trabalho}

De forma a obter uma visão geral do projeto e uma previsão de finalização foi realizada uma planificação 
expectável de tarefas.

\begin{figure}[htb]
    \centering
    
    \includegraphics[width=0.75\textwidth]{images/etapa1_sprint_planning.png}
    \caption{Planeamento de sprints}
    \label{fig:1}
\end{figure}




% 
\chapter{Análise e especificação}

Na análise e especificação está contida a descrição do modelo de negócio, onde serão identificados os principais problemas do modelo atual e será indicada a necessidade de implementação do \textit{software}. Nesta análise estarão identificados os objetivos de negócio e o principal impacto deste \textit{software} neste modelo de negócio. Nesta análise estará também contida a engenharia de \textit{software} com toda a especificação do projeto em mãos.


% \section{Modelo de negócio}

Este projeto surgiu com a necessidade da Motorline melhorar a comunicação e a experiência das 
empresas clientes e técnicos. 
Atualmente sempre que um técnico possui uma questão este deverá contactar o suporte técnico o que pode 
levar a sobrecarga deste, ou então este deverá se juntar ao grupo do Facebook de técnicos e colocar 
neste esta questão. Sempre que é necessário um manual de produto ou sempre que o utilizador pretender ver 
o catalogo este deverá se deslocar ao site da Motorline e procurar o produto no catálogo não possuindo um
acesso rápido a estes.
Visto que este processo acaba por não ser prático para o utilizador, foi então 
decidido suprir este problema com o projeto Install \& Go. Tendo estas questões em mente surgiu então 
a ideia de implementar um fórum de resolução de problemas.


% \section{Objetivos de negócio}

Este software visa minimizar o problema acima descrito. Visto que acontece que muitas das 
questões dos técnicos são comuns, então surgiu a ideia de implementar um fórum onde o técnico poderá 
pesquisar por questões semelhantes à sua, encontrando solução sem necessidade de contactar o suporte 
técnico. O técnico poderá também expor a sua questão anexando imagens, facilitando o processo de 
resolução da sua questão. Para existir partilha de conhecimento o técnico poderá também responder 
a tópicos, pois caso encontre alguma questão que já tenha resolvido poderá ajudar outro técnico. O técnico 
também poderá alterar a visibilidade do seu tópico caso deseje que apenas técnicos visualizem a sua questão.

A empresa poderá realizar as mesmas ações que o técnico, mas esta poderá também criar contas para os seus 
técnicos e realizar a gestão destas sendo importante permitir apagar e impedir acesso à conta em caso de 
necessidade.


% \section{Domínio de aplicação do sistema}

Com o diagrama abaixo representado (Figura~\ref{fig:2}) é possível visualizar todos os atores do software 
e as suas ações, assim como também os sistemas envolvidos na aplicação e as suas funções.
Destes é possível identificar que este software contém três atores principais, o Utilizador que é um 
utilizador sem sessão iniciada, o Técnico que é um utilizador com sessão iniciada, já a Empresa é uma empresa
cliente da Motorline. Também é possível visualizar os diferentes sistemas integrados no 
projeto, como Servidor Motorline onde serão obtidas informações de catálogo de produtos, 
Servidor Install \& Go onde estão todas as funções de suporte ao software, o Servidor de Imagens onde 
serão guardadas todas as imagens do fórum e por fim o Servidor de Email que enviará email com o código de 
validação de conta para os clientes assim que se registarem no software .

\begin{figure}[htb]
    \centering
    
    \includegraphics[width=\textwidth]{images/diagramas/diagrama_contexto.png}
    \caption{Diagrama de contexto da aplicação}
    \label{fig:2}
\end{figure}


% \newpage

% \section{Operações a realizar no sistema}
A primeira tarefa a realizar no desenvolvimento desta etapa do projeto foi o levantamento dos \acrfull{rf} (Tabela~\ref{tab:1}) e \acrfull{rnf}, tendo sido posteriormente validados com cliente. Este levantamento de requisitos funcionais, foi realizado num conjunto de reuniões realizadas com os clientes do projeto.

\definecolor{Mercury}{rgb}{0.905,0.901,0.901}
\begin{longtblr}
[
caption={Tabela de requisitos funcionais},
label={tab:1},
]
{
 width = \linewidth,
 colspec = {Q[75]Q[400]Q[160]Q[120]},
 row{1} = {Mercury},
 row{2} = {Mercury,c},
 row{10} = {Mercury,c},
 row{20} = {Mercury,c},
 row{26} = {Mercury,c},
 row{31} = {Mercury,c},
 row{35} = {Mercury,c},
 row{40} = {Mercury,c},
 row{45} = {Mercury,c},
 row{50} = {Mercury,c},
 cell{1}{1} = {c},
 cell{1}{3} = {c},
 cell{1}{4} = {c},
 cell{3}{1} = {c},
 cell{3}{3} = {c},
 cell{3}{4} = {c},
 cell{4}{1} = {c},
 cell{4}{3} = {c},
 cell{4}{4} = {c},
 cell{5}{1} = {c},
 cell{5}{3} = {c},
 cell{5}{4} = {c},
 cell{6}{1} = {c},
 cell{6}{3} = {c},
 cell{6}{4} = {c},
 cell{7}{1} = {c},
 cell{7}{3} = {c},
 cell{7}{4} = {c},
 cell{8}{1} = {c},
 cell{8}{3} = {c},
 cell{8}{4} = {c},
 cell{9}{1} = {c},
 cell{9}{3} = {c},
 cell{9}{4} = {c},
 cell{11}{1} = {c},
 cell{11}{3} = {c},
 cell{11}{4} = {c},
 cell{12}{1} = {c},
 cell{12}{3} = {c},
 cell{12}{4} = {c},
 cell{13}{1} = {c},
 cell{13}{3} = {c},
 cell{13}{4} = {c},
 cell{14}{1} = {c},
 cell{14}{3} = {c},
 cell{14}{4} = {c},
 cell{15}{1} = {c},
 cell{15}{3} = {c},
 cell{15}{4} = {c},
 cell{16}{1} = {c},
 cell{16}{3} = {c},
 cell{16}{4} = {c},
 cell{17}{1} = {c},
 cell{17}{3} = {c},
 cell{17}{4} = {c},
 cell{18}{1} = {c},
 cell{18}{3} = {c},
 cell{18}{4} = {c},
 cell{19}{1} = {c},
 cell{19}{3} = {c},
 cell{19}{4} = {c},
 cell{21}{1} = {c},
 cell{21}{3} = {c},
 cell{21}{4} = {c},
 cell{22}{1} = {c},
 cell{22}{3} = {c},
 cell{22}{4} = {c},
 cell{23}{1} = {c},
 cell{23}{3} = {c},
 cell{23}{4} = {c},
 cell{24}{1} = {c},
 cell{24}{3} = {c},
 cell{24}{4} = {c},
 cell{25}{1} = {c},
 cell{25}{3} = {c},
 cell{25}{4} = {c},
 cell{27}{1} = {c},
 cell{27}{3} = {c},
 cell{27}{4} = {c},
 cell{28}{1} = {c},
 cell{28}{3} = {c},
 cell{28}{4} = {c},
 cell{29}{1} = {c},
 cell{29}{3} = {c},
 cell{29}{4} = {c},
 cell{30}{1} = {c},
 cell{30}{3} = {c},
 cell{30}{4} = {c},
 cell{32}{1} = {c},
 cell{32}{3} = {c},
 cell{32}{4} = {c},
 cell{33}{1} = {c},
 cell{33}{3} = {c},
 cell{33}{4} = {c},
 cell{34}{1} = {c},
 cell{34}{3} = {c},
 cell{34}{4} = {c},
 cell{36}{1} = {c},
 cell{36}{3} = {c},
 cell{36}{4} = {c},
 cell{37}{1} = {c},
 cell{37}{3} = {c},
 cell{37}{4} = {c},
 cell{38}{1} = {c},
 cell{38}{3} = {c},
 cell{38}{4} = {c},
 cell{39}{1} = {c},
 cell{39}{3} = {c},
 cell{39}{4} = {c},
 cell{41}{1} = {c},
 cell{41}{3} = {c},
 cell{41}{4} = {c},
 cell{42}{1} = {c},
 cell{42}{3} = {c},
 cell{42}{4} = {c},
 cell{43}{1} = {c},
 cell{43}{3} = {c},
 cell{43}{4} = {c},
 cell{44}{1} = {c},
 cell{44}{3} = {c},
 cell{44}{4} = {c},
 cell{46}{1} = {c},
 cell{46}{3} = {c},
 cell{46}{4} = {c},
 cell{47}{1} = {c},
 cell{47}{3} = {c},
 cell{47}{4} = {c},
 cell{48}{1} = {c},
 cell{48}{3} = {c},
 cell{48}{4} = {c},
 cell{49}{1} = {c},
 cell{49}{3} = {c},
 cell{49}{4} = {c},
 cell{51}{1} = {c},
 cell{51}{3} = {c},
 cell{51}{4} = {c},
 cell{52}{1} = {c},
 cell{52}{3} = {c},
 cell{52}{4} = {c},
 cell{53}{1} = {c},
 cell{53}{3} = {c},
 cell{53}{4} = {c},
 cell{54}{1} = {c},
 cell{54}{3} = {c},
 cell{54}{4} = {c},
 hlines,
 vlines,
}
\#  & Descrição                                                                              & Fonte     & Data   \\
   & Autenticação                                                                            &        &      \\
RF01 & O Utilizador deverá conseguir visualizar o catálogo                                                         & Helder Remelhe & 2/13/2023 \\
RF02 & A Empresa deverá conseguir realizar o registo na aplicação                                                     & Helder Remelhe & 2/13/2023 \\
RF03 & O Técnico deverá conseguir realizar o login na aplicação                                                      & Helder Remelhe & 2/13/2023 \\
RF04 & O login deverá ser realizado utilizando número de contribuinte e \textit{password}                                              & Helder Remelhe & 2/13/2023 \\
RF05 & Assim que o registo é realizado, a motorline deverá validar a conta sendo posteriormente enviado um \textit{email} para a empresa ativar e utilizar a conta         & Rafael Viana  & 2/13/2023 \\
RF06 & O Técnico deverá conseguir pedir reenvio de código de verificação de conta                                             & Rafael Viana  & 2/27/2023 \\
RF07 & Os Técnicos deverão ser identificados como técnicos certificados ou técnicos oficiais                                        & Helder Remelhe & 2/27/2023 \\
   & Fórum                                                                                &        &      \\
RF08 & O Técnico deverá conseguir aceder ao fórum e realizar operações                                                   & Helder Remelhe & 2/13/2023 \\
RF09 & O Técnico deverá conseguir visualizar os tópicos mais recentes                                                   & Brainstorming & 2/14/2023 \\
RF10 & O Técnico deverá conseguir visualizar os tópicos em destaque                                                    & Brainstorming & 2/14/2023 \\
RF11 & O Técnico deverá conseguir visualizar os tópicos por responder                                                   & Brainstorming & 2/14/2023 \\
RF12 & O Técnico deverá conseguir acessar aos tópicos privados do fórum                                                  & Helder Remelhe & 2/13/2023 \\
RF13 & O Técnico deverá conseguir pesquisar por tópicos referentes a um assunto desejado                                          & Brainstorming & 2/14/2023 \\
RF14 & O Técnico deverá conseguir pesquisar por tópicos referentes a um produto desejado                                          & Brainstorming & 2/14/2023 \\
RF15 & O Técnico deverá conseguir pesquisar por tópicos referentes a um produto por código QR                                       & Brainstorming & 2/14/2023 \\
RF16 & O Técnico deverá conseguir visualizar os seus tópicos                                                        & Brainstorming & 2/14/2023 \\
   & Criar Tópico                                                                            &        &      \\
RF17 & O Técnico deverá conseguir criar tópicos para expor a sua questão                                                  & Helder Remelhe & 2/13/2023 \\
RF18 & O Técnico deverá conseguir colocar o seu tópico público ou privado para assim apenas outras empresas conseguirem ver                        & Helder Remelhe & 2/13/2023 \\
RF19 & O Técnico deverá conseguir indicar tipo de tópico para agilizar a identificação do mesmo                                      & Helder Remelhe & 2/14/2023 \\
RF20 & O Técnico deverá conseguir indicar o produto referente ao tópico para agilizar a identificação da sua questão                           & Brainstorming & 2/14/2023 \\
RF21 & O Técnico deverá conseguir anexar imagens ou vídeos ao tópico em questão para agilizar a comunicação e identificação do seu problema                & Helder Remelhe & 2/13/2023 \\
   & Gestão de tópico                                                                          &        &      \\
RF22 & O Técnico deverá conseguir indicar qual a melhor resposta ao seu tópico & Helder Remelhe & 2/14/2023 \\
RF23 & O Técnico deverá conseguir indicar que o tópico se encontra finalizado quando o seu problema se encontrar resolvido                         & Helder Remelhe & 2/13/2023 \\
RF24 & O Técnico deverá conseguir remover o seu tópico                                                           & Brainstorming & 2/14/2023 \\
RF25 & O Técnico deverá conseguir alterar a visibilidade do seu tópico                                                   & Helder Remelhe & 2/13/2023 \\
   & Tópicos                                                                               &        &      \\*
RF26 & O Técnico deverá conseguir ver todas as respostas ao tópico                                                     & Helder Remelhe & 2/13/2023 \\
RF27 & O Técnico deverá conseguir gostar do tópico para dar destaque ao mesmo                                               & Brainstorming & 2/14/2023 \\
RF28 & O Técnico deverá conseguir remover uma resposta que colocou em um tópico                                              & Brainstorming & 2/14/2023 \\
   & Respostas a Tópicos                                                                         &        &      \\
RF29 & O Técnico deverá conseguir comentar tópicos                                                             & Helder Remelhe & 2/13/2023 \\
RF30 & O Técnico deverá conseguir responder a comentários de tópicos                                                    & Helder Remelhe & 2/13/2023 \\
RF31 & O Técnico deverá conseguir anexar imagens e videos à sua resposta                                                  & Helder Remelhe & 2/13/2023 \\
RF32 & O Técnico deverá conseguir gostar de alguma resposta para dar destaque a esta resposta                                       & Helder Remelhe & 2/13/2023 \\
   & Perfil                                                                               &        &      \\
RF33 & O Técnico deverá conseguir alterar o seu \textit{email}                                                           & Helder Remelhe & 2/27/2023 \\
RF34 & O Técnico deverá conseguir alterar imagem de perfil                                                         & Brainstorming & 2/27/2023 \\
RF35 & O Técnico deverá conseguir alterar nome                                                               & Brainstorming & 2/27/2023 \\
RF36 & O Técnico deverá ser identificado como Técnico oficial ou certificado                                                & Helder Remelhe & 2/28/2023 \\
   & Notificações                                                                            &        &      \\
RF37 & O Técnico deverá conseguir receber notificações por \textit{email} e/ou push                                                 & Rafael Viana  & 2/27/2023 \\
RF38 & O Técnico deverá conseguir alterar o tipo de notificação ente relatório diário e notificação direta para ambos os tipos                       & Rafael Viana  & 2/27/2023 \\
RF39 & O Técnico deverá conseguir ver as suas notificações                                                         & Helder Remelhe & 2/27/2023 \\
RF40 & O Técnico deverá conseguir apagar as suas notificações                                                       & Helder Remelhe & 2/27/2023 \\
   & Gestão de recursos humanos                                                                     &        &      \\*
RF41 & A Empresa deverá conseguir criar contas para os seus técnicos                                                    & Brainstorming & 2/13/2023 \\*
RF42 & Assim que a conta de técnico for criada, este deverá receber um \textit{email} para registar as restantes informações e ativar a conta                    & Helder Remelhe & 2/27/2023 \\
RF43 & A Empresa deverá conseguir impedir acesso a uma conta de técnico                                                  & Helder Remelhe & 2/27/2023 \\
RF44 & A Empresa deverá conseguir remover uma conta de técnico da empresa                                                 & Helder Remelhe & 2/27/2023 
\end{longtblr}

% \section{Descrição dos intervenientes}
O projeto envolve um conjunto de intervenientes, sendo estes, o utilizador, a empresa e o técnico. 
Estes desempenham um papel fundamental e podem realizar diferentes ações.

O utilizador sem sessão iniciada terá um fácil e rápido acesso ao catálogo de produtos, o que irá 
facilitar quando este deseja realizar a consulta do mesmo, terá também acesso apenas de visão de questões 
públicas do fórum, não conseguindo realizar nenhuma operação.

O técnico conseguirá realizar as mesmas ações que o utilizador, mas este ator conseguirá também ter 
acesso total ao fórum e a questões privadas. O fórum permite expor questões, com auxílio de imagens, 
permite a ligação da 
questão a uma categoria de questões e um produto em específico para facilitar a resolução da sua questão. 
As questões poderão ser públicas para assim qualquer um poder ver, ou então estas podem ser 
privadas para apenas técnicos conseguirem ver. Assim que o técnico estiver satisfeito com a sua 
questão este poderá indicar a melhor resposta que obteve para a destacar e então finalizar o tópico, 
para assim este ser indicado como finalizado.

O técnico pode também realizar pesquisas por questões em caso de ter algum problema. Com isto, este 
evita um telefonema ou o preenchimento de um formulário para contactar um técnico. Ao realizar pesquisas 
por questões este pode responder a outras questões, pode também pesquisar por questões em categoria 
e produto. As respostas podem conter imagens anexadas e podem também responder a outras 
respostas para manter uma comunicação continua. O técnico poderá também destacar tópicos e respostas de 
tópicos gostando destas.

A empresa pode realizar a gestão de contas de técnicos, conseguindo criar, impedir acesso e 
eliminar em caso de necessidade.


% \section{Partes Interessadas}

Este projeto foi proposto pelo supervisor de estágio, sendo então este com a empresa Motorline representada 
por, Rafael Viana e André Viana, as partes interessadas.

% \newpage

% \newpage
\section{Condições Específicas}

Os requisitos não funcionais são caraterísticas do \textit{software} que não interferem diretamente com o funcionamento normal, estes podem referir-se a caraterísticas como segurança e cultura. Na Tabela~\ref{tab:2} é possível visualizar os requisitos não funcionais levantados.

\definecolor{Alto}{rgb}{0.85,0.85,0.85}
\begin{table}[htb]
\centering
\caption{Tabela de requisitos não funcionais}
\label{tab:2}
\begin{tblr}{
  width = \linewidth,
  colspec = {Q[77]Q[215]Q[638]},
  row{1} = {Alto},
  hlines,
  vlines,
  vline{3} = {-}{black},
}
\#    & Tipo            & Descrição                                                                        \\
RNF01 & Cultural        & O \textit{software} deverá suportar vários idiomas (prioritariamente, português e inglês) \\
RNF02 & Configuração    & A falha dos servidores, implica a inutilidade total do fórum da aplicação                 \\
RNF03 & Conexão         & Necessário o uso de dados móveis ou WIFI                                         \\
RNF04 & Segurança       & O \textit{software} tem de ser seguro para proteger os dados do consumidor          \\
RNF05 & Desenvolvimento & A aplicação deverá ser desenvolvida com a utilização da tecnologia \textit{Flutter}              
\end{tblr}
\end{table}

% \newpage

% \section{Esquematização do conteúdo das páginas}

Para ser possibilitar a perceção dos dados necessários para alimentar o \textit{software}, o que apresentar em cada página e também como navegar entre os ecrãs da aplicação foi então desenhado um esquema 
(Figura~\ref{fig:3}).  Encontra-se no documento de anexos, no anexo 12, uma versão mais detalhada da Figura~\ref*{fig:3}.

\begin{figure}[htb]
  \centering
  
  \includegraphics[width=\textwidth]{images/Arquiteturas/diagrama_superficial_de_aplicacao.png}
  \caption{Esquema de organização de páginas do \textit{software}}
  \label{fig:3}
\end{figure}

\newpage

\subsection{Autenticação e página Inicial}

Através deste esquema é possível perceber que do ecrã principal, o utilizador tem acesso ao 
catálogo de produtos e ao fórum, neste pode também realizar o \textit{login} e o \textit{logout} que redirecionam para os respetivos ecrãs.

No ecrã de \textit{login} é necessário o utilizador indicar o número de contribuinte e a \textit{password}, neste ele pode também pedir recuperação de \textit{password} e/ou redirecionar para o registo onde necessitará de número de contribuinte, \textit{password} e \textit{email} para o realizar.

Em caso de o utilizador não possuir a conta ativa, este será encaminhado para o ecrã de validação conta em que deverá indicar o código de validação, cancelar a operação e pedir o reenvio do código de validação.

Em caso de se tratar de um técnico que necessita de confirmar a sua conta, este conseguirá ver as informações registadas, introduzir o seu nome, alterar o seu \textit{email} e \textit{password}.

\begin{figure}[htb]
  \centering
  \includegraphics[height=0.9\textwidth]{images/Arquiteturas/superficial_de_app/home_auth.png}
  \caption{Esquema de organização das páginas de autenticação e página inicial}
  \label{fig:4}
\end{figure}

\newpage

\subsection{Fórum}

Através do ecrã inicial o utilizador pode direcionar-se para o ecrã do fórum. Neste ecrã, poderá pesquisar por tópicos, ou então aceder a tópicos em alta, mais recentes ou sem resposta.
O técnico consegue também criar e ver os seus tópicos.

\begin{figure}[htb]
  \centering
  
  \includegraphics[height=0.4\textwidth]{images/Arquiteturas/superficial_de_app/forum.png}
  \caption{Esquema de organização da página de fórum}
  \label{fig:5}
\end{figure}

\subsection{Criar nova tópico}

Quando o técnico decide criar uma tópico, este tem de indicar o título e a descrição, de seguida poderá indicar se é privado ou não, o tipo de tópico, o produto referente e adicionar anexos.

\begin{figure}[htb]
  \centering
  
  \includegraphics[height=0.3\textwidth]{images/Arquiteturas/superficial_de_app/criar_topico.png}
  \caption{Esquema de organização da página de criação de tópicos}
  \label{fig:6}
\end{figure}

\newpage

\subsection{Detalhes de tópicos}

O técnico pode também ver os detalhes do tópico, responder e gostar de uma resposta.
Caso esta seja do mesmo, este pode indicar a melhor resposta, apagar o tópico, tornar público ou privado e indicar como completo.

\begin{figure}[htb]
  \centering
  
  \includegraphics[height=0.35\textwidth]{images/Arquiteturas/superficial_de_app/detalhes_topico.png}
  \caption{Esquema de organização da página de detalhes de tópico}
  \label{fig:7}
\end{figure}

\subsection{Pesquisa de tópicos}

A página de pesquisa permite ao técnico procurar por tópicos específicos tanto por nome como por produto. Para além da pesquisa o utilizador pode também realizar a filtragem dos 
tópicos por tipo e categoria.
\begin{figure}[htb]
  \centering
  
  \includegraphics[height=0.3\textwidth]{images/Arquiteturas/superficial_de_app/pesquisa_forum.png}
  \caption{Esquema de organização da página de pesquisa de tópicos}
  \label{fig:8}
\end{figure}

\newpage

\subsection{Notificações}

A página de notificações permite ao técnico visualizar as suas notificações, assim como também apagar.
\begin{figure}[htb]
  \centering
  
  \includegraphics[height=0.2\textwidth]{images/Arquiteturas/superficial_de_app/notificacoes.png}
  \caption{Esquema de organização da página de notificações}
  \label{fig:9}
\end{figure}

\subsection{Perfil}

A página de perfil de técnico permite visualizar as suas informações, assim como alterar 
o seu \textit{email} e configurar as notificações. Caso se trate de uma empresa a visualizar o seu perfil, esta poderá ter acesso à gestão de recursos humanos, para gerir os seus técnicos.
\begin{figure}[htb]
  \centering
  
  \includegraphics[height=0.35\textwidth]{images/Arquiteturas/superficial_de_app/perfil.png}
  \caption{Esquema de organização da página de perfil}
  \label{fig:10}
\end{figure}

\newpage

\subsection{Gestão de recursos humanos}

A página de gestão de recursos humanos permite à empresa gerir todos os seus técnicos registados e criar contas. Assim que a empresa seleciona um técnico, esta vê o seu perfil, 
com estatísticas.
\begin{figure}[htb]
  \centering
  
  \includegraphics[height=0.7\textwidth]{images/Arquiteturas/superficial_de_app/gestao_recursos_humanos.png}
  \caption{Esquema de organização da página de gestão de recursos humanos}
  \label{fig:11}
\end{figure}

% \section{\textit{User Stories}}
Antes de desenvolver os casos de uso foram criadas \textit{\acrfull{us}} para ser possível descrever os objetivos ao realizar uma ação.

% \usepackage{color}
% \usepackage{tabularray}
\definecolor{Concrete}{rgb}{0.952,0.952,0.952}
\begin{longtblr}
[
caption={Tabela de histórias de utilizador},
label={tab:3},
]{
     width = \linewidth,
     colspec = {Q[75]Q[110]Q[700]},
     row{1} = {Concrete},
  row{2} = {Concrete},
  row{8} = {Concrete},
  row{19} = {Concrete},
  row{25} = {Concrete},
  row{30} = {Concrete},
  row{34} = {Concrete},
  row{39} = {Concrete},
  row{43} = {Concrete},
  row{47} = {Concrete},
  column{2} = {c},
  cell{1}{1} = {c},
  cell{2}{2} = {c=2}{0.891\linewidth},
  cell{3}{1} = {c},
  cell{4}{1} = {c},
  cell{5}{1} = {c},
  cell{8}{2} = {c=2}{0.891\linewidth},
  cell{9}{1} = {c},
  cell{10}{1} = {c},
  cell{11}{1} = {c},
  cell{12}{1} = {c},
  cell{13}{1} = {c},
  cell{14}{1} = {c},
  cell{15}{1} = {c},
  cell{16}{1} = {c},
  cell{17}{1} = {c},
  cell{18}{1} = {c},
  cell{19}{2} = {c=2}{0.891\linewidth},
  cell{20}{1} = {c},
  cell{21}{1} = {c},
  cell{22}{1} = {c},
  cell{23}{1} = {c},
  cell{24}{1} = {c},
  cell{25}{2} = {c=2}{0.891\linewidth},
  cell{26}{1} = {c},
  cell{27}{1} = {c},
  cell{28}{1} = {c},
  cell{29}{1} = {c},
  cell{30}{2} = {c=2}{0.891\linewidth},
  cell{31}{1} = {c},
  cell{32}{1} = {c},
  cell{33}{1} = {c},
  cell{34}{2} = {c=2}{0.891\linewidth},
  cell{35}{1} = {c},
  cell{36}{1} = {c},
  cell{37}{1} = {c},
  cell{38}{1} = {c},
  cell{39}{2} = {c=2}{0.891\linewidth},
  cell{40}{1} = {c},
  cell{41}{1} = {c},
  cell{42}{1} = {c},
  cell{43}{2} = {c=2}{0.891\linewidth},
  cell{44}{1} = {c},
  cell{45}{1} = {c},
  cell{46}{1} = {c},
  cell{47}{2} = {c=2}{0.891\linewidth},
  cell{48}{1} = {c},
  cell{49}{1} = {c},
  cell{50}{1} = {c},
  hlines,
  vlines,
}
\#   & Ator                       & Descrição                                                                                                                                                                              \\
     & Autenticação               &                                                                                                                                                                                        \\
US01 & Utilizador                 & Eu como Utilizador, quero conseguir utilizar a aplicação sem realizar o login                                                                                                          \\
US02 & Empresa                    & Eu como Empresa, quero conseguir realizar o registo na aplicação                                                                                                                       \\
US03 & Técnico                    & Eu como Técnico, quero conseguir realizar o login na aplicação utilizando o número de contribuinte e password                                                                          \\
US04 & Técnico                    & Eu como Técnico, quero conseguir pedir reenvio de código de ativação de conta caso eu não receba o código                                                                              \\
US05 & Técnico                    & Eu como Técnico, quero conseguir ser identificado como tal na aplicação                                                                                                                \\
     & Fórum                      &                                                                                                                                                                                        \\*
US06 & Utilizador                 & Eu como Utilizador, quero conseguir acessar ao fórum                                                                                                                                      \\*
US07 & Utilizador                 & Eu como Utilizador, quero conseguir visualizar os tópicos mais recentes, de forma a conseguir ver os mais falados no dia atual                                                            \\
US08 & Utilizador                 & Eu como Utilizador, quero conseguir visualizar os tópicos em destaque, de forma a ver quais são mais falados                                                                      \\
US09 & Técnico                    & Eu como Técnico, quero conseguir ver os meus tópicos de forma a conseguir aceder a estes facilmente                                                                                    \\
US10 & Utilizador                 & Eu como Utilizador, quero conseguir visualizar os tópicos por responder, de forma a conseguir ajudar alguém com maior facilidade                                                          \\
US11 & Técnico                    & Eu como Técnico, quero conseguir visualizar os tópicos privados                                                                                                                        \\
US12 & Utilizador                 & Eu como Utilizador, quero conseguir realizar filtragem de tópicos por tipo                                                                                \\
US13 & Utilizador                 & Eu como Utilizador, quero conseguir pesquisar por um tópico relativo a um assunto de forma a obter a solução                                                                              \\
US14 & Utilizador                 & Eu como Utilizador, quero conseguir pesquisar por um tópico relativo a um produto de forma a encontrar questões comuns a este                                                             \\
US15 & Utilizador                 & Eu como Utilizador, quero conseguir pesquisar por código QR de um produto de forma a encontrar tópicos referentes ao mesmo mais facilmente                                                 \\
     & Criar Tópico               &                                                                                                                                                                                        \\
US16 & Técnico                    & Eu como Técnico, quero conseguir criar tópicos de forma a conseguir expor questões                                                                                                     \\
US17 & Técnico                    & Eu como Técnico, quero conseguir indicar se o meu tópico é publico ou privado, de forma a conseguir respostas de qualquer cliente, ou apenas de técnicos                               \\
US18 & Técnico                    & Eu como Técnico, quero conseguir indicar o tipo de tópico em que o este se enquadra de forma a facilitar a sua identificação                                                          \\
US19 & Técnico                    & Eu como Técnico, quero conseguir indicar o produto referente ao tópico para facilitar a identificação do mesmo                                                                         \\
US20 & Técnico                    & Eu como Técnico, quero conseguir anexar imagens ao tópico de forma a facilitar a comunicação e identificação do problema                                                               \\
     & Gestão de Tópico           &                                                                                                                                                                                        \\*
US21 & Técnico                    & Eu como Técnico quero conseguir indicar qual a melhor resposta ao meu tópico de forma a facilitar o encontro da solução do problema a outros clientes ou técnicos com o mesmo problema \\*
US22 & Técnico                    & Eu como Técnico quero conseguir indicar que o tópico se encontra finalizado quando o problema está resolvido                                                                           \\
US23 & Técnico                    & Eu como Técnico quero conseguir remover o meu tópico                                                                                                                                   \\
US24 & Técnico                    & Eu como Técnico quero conseguir alterar a visibilidade do meu tópico                                                                                                                   \\
     & Tópicos                    &                                                                                                                                                                                        \\
US25 & Utilizador                 & Eu como Utilizador, quero conseguir ver todas as respostas a um tópico                                                                                                                    \\
US26 & Técnico                    & Eu como Técnico, quero conseguir gostar de um tópico caso o ache relevante                                                                                                             \\
US27 & Técnico                    & Eu como Técnico, quero conseguir apagar uma resposta minha                                                                                                                             \\
     & Respostas a Tópicos        &                                                                                                                                                                                        \\
US28 & Técnico                    & Eu como Técnico, quero conseguir comentar um tópico de forma a dar a minha resposta                                                                                                    \\
US29 & Técnico                    & Eu como Técnico, quero conseguir responder a um comentário de forma a comunicar                                                                                                        \\
US30 & Técnico                    & Eu como Técnico, quero conseguir anexar imagens ao meu comentário                                                                                                                      \\
US31 & Técnico                    & Eu como Técnico, quero conseguir gostar de um comentário caso ache este relevante                                                                                                      \\
     & Perfil                     &                                                                                                                                                                                        \\
US32 & Técnico                    & Eu como Técnico quero conseguir alterar o meu email                                                                                                                                    \\
US33 & Técnico                    & Eu como Técnico quero conseguir alterar a minha imagem de perfil                                                                                                                       \\
US34 & Técnico                    & Eu como Técnico quero conseguir alterar o meu nome                                                                                                                                     \\
     & Notificações               &                                                                                                                                                                                        \\
US35 & Técnico                    & Eu como Técnico quero conseguir receber notificações por email e/ou push de forma a manter-me atualizado das minhas questões                                                           \\
US36 & Técnico                    & Eu como Técnico quero conseguir alterar o tipo de notificação que recebo entre relatório diário e tempo real                                                                           \\
US37 & Técnico                    & Eu como Técnico quero conseguir apagar as minhas notificações de forma a evitar aglomeração                                                                                            \\
     & Gestão de recursos humanos &                                                                                                                                                                                        \\*
US38 & Empresa                    & Eu como empresa quero conseguir criar conta para os meus técnicos utilizarem o fórum                                                                                                   \\
US39 & Empresa                    & Eu como empresa quero conseguir impedir acesso a uma conta de técnico                                                                                                                  \\
US40 & Empresa                    & Eu como empresa quero conseguir remover uma conta de técnico em caso de este já não pertencer à empresa                                                                                
\end{longtblr}

% \newpage

% \section{Casos de uso}
De forma a transformar as histórias de utilizador em ações e especificar todas as ações dos atores do 
software e todas as reações do sistema com o qual o ator interage foram desenvolvidos casos de uso.

% \usepackage{color}
% \usepackage{tabularray}
\definecolor{Concrete}{rgb}{0.952,0.952,0.952}
\definecolor{Gallery}{rgb}{0.937,0.937,0.937}
\begin{longtblr}
[
caption={Tabela de casos de uso},
label={tab:4},
]
{
  width = \linewidth,
  colspec = {Q[130]Q[81]Q[120]Q[242]Q[375]},
  row{1} = {Concrete},
  row{2} = {Concrete,c},
  row{4} = {Concrete,c},
  row{8} = {Concrete,c},
  row{10} = {Concrete,c},
  row{18} = {Concrete,c},
  row{23} = {Concrete,c},
  row{26} = {Concrete,c},
  row{28} = {Concrete,c},
  row{31} = {Concrete,c},
  column{2} = {c},
  column{3} = {c},
  cell{1}{1} = {c},
  cell{2}{1} = {c=5}{0.936\linewidth},
  cell{4}{1} = {c=5}{0.936\linewidth},
  cell{8}{1} = {c=5}{0.936\linewidth},
  cell{10}{1} = {c=5}{0.936\linewidth},
  cell{18}{1} = {c=5}{0.936\linewidth},
  cell{23}{1} = {c=5}{0.936\linewidth},
  cell{26}{1} = {c=5}{0.936\linewidth},
  cell{28}{1} = {c=5}{0.936\linewidth},
  cell{31}{1} = {c=5}{0.936\linewidth},
  hlines,
  vlines,
}
\#                         & User Story         & Ator       & Nome                                & Descrição                                                   \\
Criar tópico               &                    &            &                                     &                                                             \\
UC 1.0                     & US15               & Técnico    & Criar novo tópico                   & Criação de um novo tópico no fórum                          \\
Pesquisa de tópicos        &                    &            &                                     &                                                             \\
UC 1.1                     & US11               & Utilizador & Pesquisar tópicos específicos       & Pesquisar por tópicos no fórum                              \\
UC 1.1.1                   & US12 e US13        & Utilizador & Pesquisa escrita                    & Pesquisar tópicos por assunto                               \\
UC 1.1.2                   & US14               & Utilizador & Pesquisa por código QR              & Pesquisar tópicos referentes a um produto                   \\
Listagens de tópicos       &                    &            &                                     &                                                             \\
UC 1.2                     & US05               & Utilizador & Ver tópicos                         & Ver listagens de tópicos do fórum                           \\
Detalhes de tópico         &                    &            &                                     &                                                             \\
UC 1.3                     & US08               & Utilizador & Selecionar tópico                   & Ver detalhes de um tópico selecionado                       \\
UC 1.3.1                   & US21               & Técnico    & Finalizar tópico                    & Finalizar um tópico para indicar que está respondido        \\
UC 1.3.2                   & US20               & Técnico    & Selecionar melhor resposta          & Selecionar a melhor resposta do tópico                      \\
UC 1.3.3                   & US22               & Técnico    & Eliminar tópico                     & Eliminar um tópico do fórum                                 \\
UC 1.3.4                   & US23               & Técnico    & Alterar visibilidade do tópico      & Alterar a visibilidade de um tópico entre publico e privado \\
UC 1.3.5                   & US28               & Técnico    & Comentar o tópico                   & Comentar um tópico                                          \\
UC 1.3.6                   & US25               & Técnico    & Gostar de tópico                    & Gostar de um tópico                                         \\
Comentários                &                    &            &                                     &                                                             \\
UC 1.3.7                   & US24               & Utilizador & Ver comentários                     & Ver comentários do tópico                                   \\
UC 1.3.7.1                 & US27               & Técnico    & Apagar comentário                   & Apagar comentário de um tópico                              \\
UC 1.3.7.2                 & US29               & Técnico    & Responder a comentário              & Responder a um comentário de um tópico                      \\
UC 1.3.7.3                 & US31               & Técnico    & Gostar de comentário                & Gostar de um comentário                                     \\
Ativação de conta          &                    &            &                                     &                                                             \\
UC 1.4                     & -                  & Técnico    & Ativação de conta                   & Ativar conta de cliente                                     \\
UC 1.4.1                   & US04               & Técnico    & Pedir reenvio de código de ativação & Pedir reenvio de email de código de ativação                \\
Perfil                     &                    &            &                                     &                                                             \\
UC 1.5                     & US31 - US32 - US33 & Técnico    & Ver Perfil                          & Ver perfil de utilizador                                    \\
Notificações               &                    &            &                                     &                                                             \\
UC1.6                      & US34 e US36        & Técnico    & Ver notificações                    & Ver todas as notificações                                   \\
UC1.7                      & US34 e US35        & Técnico    & Configuração de notificações        & Configurar o modo e tipo de notificações a receber          \\
Gestão de recursos humanos &                    &            &                                     &                                                             \\
UC1.8                      & US37               & Empresa    & Registar Técnico                    & Registar conta de técnico da empresa                        \\
UC1.9                      & US38               & Empresa    & Impedir acesso a técnico            & Registar conta de técnico da empresa                        \\
UC1.10                     & US39               & Empresa    & Remover conta de técnico            & Registar conta de técnico da empresa                        
\end{longtblr}
\newpage

\subsection{Especificação de casos de uso}

De forma a demonstrar todas as interações entre os atores e o sistema, assim como também todas as ações 
destes e fluxos possíveis, foram realizadas especificações de casos de uso.

\subsubsection{Especificação de caso de uso de listagem de tópicos}

Através da listagem de tópicos é possível visualizar todas as listagens de tópicos que o utilizador poderá 
visualizar, sendo que o técnico consegue para além destas listagens, ver os seus tópicos e ver os tópicos 
privados.

%\definecolor{Concrete}{rgb}{0.952,0.952,0.952}
\begin{longtblr}
[
caption={Tabela de especificação de caso de uso de listagem de tópicos do utilizador},
label={tab:8},
]
{
  width = \linewidth,
  colspec = {Q[225]Q[331]Q[383]},
  row{6} = {Concrete},
  cell{1}{1} = {Concrete},
  cell{1}{2} = {c=2}{0.714\linewidth},
  cell{2}{1} = {Concrete},
  cell{2}{2} = {c=2}{0.714\linewidth},
  cell{3}{1} = {Concrete},
  cell{3}{2} = {c=2}{0.714\linewidth},
  cell{4}{1} = {Concrete},
  cell{4}{2} = {c=2}{0.714\linewidth,c},
  cell{5}{1} = {Concrete},
  cell{5}{2} = {c=2}{0.714\linewidth,c},
  cell{6}{2} = {c},
  cell{6}{3} = {c},
  cell{7}{1} = {r=2}{Concrete},
  cell{9}{1} = {r=2}{Concrete},
  cell{11}{1} = {r=2}{Concrete},
  vlines,
  hline{1-7,9,11,13} = {-}{},
  hline{8,10,12} = {2-3}{},
}
Caso de Uso           & Ver listagem de tópicos                                               &                                  \\
Descrição             & Ver a listagem de tópicos existentes no fórum por diversas categorias &                                  \\
Ator                  & Utilizador                                                            &                                  \\
Pré-condição          & -                                                                     &                                  \\
Pós-condição          & -                                                                     &                                  \\
                      & Ator                                                                  & Sistema                          \\
Fluxo Principal       & 1-Ver tópicos populares                                               &                                  \\
                      &                                                                       & 2-Lista de tópicos populares     \\
Fluxo Alternativo(A1) & 1-Ver tópicos mais recentes                                           &                                  \\
                      &                                                                       & 2-Lista de tópicos mais recentes \\
Fluxo Alternativo(A2) & 1-Ver tópicos por responder                                           &                                  \\
                      &                                                                       & 2-Lista de tópicos por responder 
\end{longtblr}

\input{tables/casos_de_uso/listagem_topicos_tecnico}

\newpage

\subsubsection{Especificação de caso de uso de criar novo tópico}

Aquando a criação de um tópico um técnico poderá realizar diversas ações sendo que obrigatoriamente 
terá sempre de indicar o título, descrição do tópico e tipo de tópico, para além desta informação o técnico poderá também
anexar imagens, referenciar um produto e indicar a visibilidade.

\definecolor{Concrete}{rgb}{0.952,0.952,0.952}
\begin{longtblr}
[
caption={Tabela de especificação de caso de uso login},
label={tab:6},
]{
  width = \linewidth,
  colspec = {Q[212]Q[360]Q[369]},
  row{6} = {Concrete},
  cell{1}{1} = {Concrete},
  cell{1}{2} = {c=2}{0.725\linewidth},
  cell{2}{1} = {Concrete},
  cell{2}{2} = {c=2}{0.725\linewidth},
  cell{3}{1} = {Concrete},
  cell{3}{2} = {c=2}{0.725\linewidth},
  cell{4}{1} = {Concrete},
  cell{4}{2} = {c=2}{0.725\linewidth},
  cell{5}{1} = {Concrete},
  cell{5}{2} = {c=2}{0.725\linewidth,c},
  cell{6}{2} = {c},
  cell{6}{3} = {c},
  cell{7}{1} = {r=10}{Concrete,c},
  cell{17}{1} = {Concrete},
  cell{18}{1} = {r=6}{Concrete},
  vlines,
  hline{1-7,17-18,24} = {-}{},
  hline{8-16,19-23} = {2-3}{},
}
Caso de Uso           & Criar novo tópico                      &                                        \\
Descrição             & Criação de um novo tópico no fórum     &                                        \\
Ator                  & Técnico                                &                                        \\
Pré-condição          & Clicar em adicionar novo tópico        &                                        \\
Pós-condição          & -                                      &                                        \\
                      & Ator                                   & Sistema                                \\
Fluxo Principal       & 1-Indicar o título do tópico           &                                        \\
                      & 2-Indicar a descrição do tópico        &                                        \\
                      & 3-Indicar se o tópico é privado        &                                        \\
                      & 4-Indicar o tipo do tópico             &                                        \\
                      & 5-Indicar o produto referido no tópico &                                        \\
                      & 6-Adicionar imagens de anexo           &                                        \\
                      & 7-Confirmar a criação do tópico        &                                        \\
                      &                                        & 8-Verificar se titulo está inserido    \\
                      &                                        & 9-Verificar se descrição está inserida \\
                      &                                        & 10-Inserir novo tópico no fórum        \\
Fluxo Alternativo(A1) & 1-Cancelar a criação do tópico         &                                        \\
Fluxo Alternativo(A2) & 1-Indicar o título do tópico           &                                        \\*
                      & 2-Indicar se o tópico é privado        &                                        \\*
                      & 3-Confirmar a criação do tópico        &                                        \\*
                      &                                        & 4-Verificar se titulo está inserido    \\*
                      &                                        & 5-Verificar se descrição está inserida \\*
                      &                                        & 6-Erro descrição em falta         
\end{longtblr}

\subsubsection{Especificação de caso de uso de pesquisar tópicos por escrito}

Assim que um utilizador deseje pesquisar por um assunto específico de tópico este poderá realizar uma 
pesquisa escrita onde conseguirá realizar filtragem por tipo de tópico também.

% \usepackage{color}
% \usepackage{tabularray}
\definecolor{Concrete}{rgb}{0.952,0.952,0.952}
\begin{longtblr}
[
caption={Tabela de especificação de caso de uso de pesquisa por escrito},
label={tab:7},
]{
  width = \linewidth,
  colspec = {Q[190]Q[217]Q[533]},
  row{6} = {Concrete},
  cell{1}{1} = {Concrete},
  cell{1}{2} = {c=2}{0.702\linewidth},
  cell{2}{1} = {Concrete},
  cell{2}{2} = {c=2}{0.702\linewidth},
  cell{3}{1} = {Concrete},
  cell{3}{2} = {c=2}{0.702\linewidth},
  cell{4}{1} = {Concrete},
  cell{4}{2} = {c=2}{0.702\linewidth},
  cell{5}{1} = {Concrete},
  cell{5}{2} = {c=2}{0.702\linewidth,c},
  cell{6}{2} = {c},
  cell{6}{3} = {c},
  cell{7}{1} = {r=4}{Concrete},
  cell{11}{1} = {r=2}{Concrete},
  vlines,
  hline{1-7,11,13} = {-}{},
  hline{8-10,12} = {2-3}{},
}
Caso de Uso           & Pesquisa por escrita           &                                            \\
Descrição             & Pesquisar por tópicos no fórum &                                            \\
Ator                  & Utilizador                     &                                            \\
Pré-condição          & Selecionar pesquisa de fórum   &                                            \\
Pós-condição          & -                              &                                            \\
                      & Ator                           & Sistema                                    \\
Fluxo Principal       & 1-Pesquisar assunto            &                                            \\
                      &                                & 2-Lista de tópicos do assunto              \\
                      & 3-Filtrar por tipo             &                                            \\
                      &                                & 3-Filtragem de tópicos do assunto por tipo \\
Fluxo Alternativo(A1) & 1-Pesquisar assunto            &                                            \\
                      &                                & 2-Lista de tópicos do assunto              
\end{longtblr}


\subsubsection{Especificação de caso de uso de ver finalizar tópico}

Quando um técnico já se encontra satisfeito com a solução do problema este poderá indicar que o tópico se 
encontra finalizado sinalizando então para outros técnicos que o tópico tem solução.

% \usepackage{color}
% \usepackage{tabularray}
\definecolor{Concrete}{rgb}{0.952,0.952,0.952}
\begin{table}[htb]
\centering
\label{tab:8}
\caption{Tabela de especificação de caso de uso de finalizar tópico}
\begin{tblr}{
 width = \linewidth,
 colspec = {Q[254]Q[319]Q[365]},
 row{6} = {Concrete},
 cell{1}{1} = {Concrete},
 cell{1}{2} = {c=2}{0.683\linewidth},
 cell{2}{1} = {Concrete},
 cell{2}{2} = {c=2}{0.683\linewidth},
 cell{3}{1} = {Concrete},
 cell{3}{2} = {c=2}{0.683\linewidth},
 cell{4}{1} = {Concrete},
 cell{4}{2} = {c=2}{0.683\linewidth},
 cell{5}{1} = {Concrete},
 cell{5}{2} = {c=2}{0.683\linewidth},
 cell{6}{2} = {c},
 cell{6}{3} = {c},
 cell{7}{1} = {r=2}{Concrete},
 cell{9}{1} = {Concrete},
 cell{9}{2} = {c},
 cell{9}{3} = {c},
 vlines,
 hline{1-7,9-10} = {-}{},
 hline{8} = {2-3}{},
}
Caso de Uso      & Finalizar tópico                   &                 \\
Descrição       & Finalizar um tópico para indicar que está respondido &                 \\
Ator         & Técnico                       &                 \\
Pré-condição     & Clicar no tópico desejado              &                 \\
Pós-condição     & Alterar tópico para finalizado            &                 \\
           & Ator                         & Sistema             \\
Fluxo Principal    & 1-Clicar em finalizar tópico             &                 \\
           &                           & 2-Alterar tópico para finalizado \\
Fluxo Alternativo(A1) & -                          & -                
\end{tblr}
\end{table}

\newpage

\subsubsection{Especificação de caso de uso de selecionar melhor resposta}

Sempre que o técnico encontrar uma resposta no seu tópico que se destaca na solução da sua questão, 
este poderá colocar esta resposta como melhor resposta do tópico, caso já exista uma melhor resposta no 
tópico, esta é removida de melhor resposta e a nova resposta é colocada como melhor resposta.

% \usepackage{color}
% \usepackage{tabularray}
\definecolor{Concrete}{rgb}{0.952,0.952,0.952}
\begin{table}[htb]
\centering
\label{tab:9}
\caption{Tabela de especificação de caso de uso de selecionar melhor resposta}
\begin{tblr}{
 width = \linewidth,
 colspec = {Q[181]Q[235]Q[525]},
 row{6} = {Concrete},
 cell{1}{1} = {Concrete},
 cell{1}{2} = {c=2}{0.76\linewidth},
 cell{2}{1} = {Concrete},
 cell{2}{2} = {c=2}{0.76\linewidth},
 cell{3}{1} = {Concrete},
 cell{3}{2} = {c=2}{0.76\linewidth},
 cell{4}{1} = {Concrete},
 cell{4}{2} = {c=2}{0.76\linewidth},
 cell{5}{1} = {Concrete},
 cell{5}{2} = {c=2}{0.76\linewidth},
 cell{6}{2} = {c},
 cell{6}{3} = {c},
 cell{7}{1} = {r=3}{Concrete},
 cell{10}{1} = {r=4}{Concrete},
 cell{14}{1} = {r=4}{Concrete},
 vlines,
 hline{1-7,10,14,18} = {-}{},
 hline{8-9,11-13,15-17} = {2-3}{},
}
Caso de Uso      & Selecionar melhor resposta       &                                 \\
Descrição       & Selecionar a melhor resposta do tópico &                                 \\
Ator         & Técnico                 &                                 \\
Pré-condição     & Clicar no tópico desejado        &                                 \\
Pós-condição     & Alterar a resposta para melhor resposta &                                 \\
           & Ator                  & Sistema                             \\
Fluxo Principal    & 1-Clicar em melhor resposta       &                                 \\
           &                     & 2-Verificar se já existe uma melhor resposta - Não        \\
           &                     & 3- Colocar a resposta como melhor resposta do tópico       \\
Fluxo Alternativo(A1) & 1-Clicar em melhor resposta       &                                 \\
           &                     & 2-Verificar se já existe uma melhor resposta - Sim        \\
           &                     & 3- Verificar se a resposta existente é a mesma selecionada - Não \\
           &                     & 4-Alterar melhor resposta                    \\
Fluxo Alternativo(A2) & 1-Clicar em melhor resposta       &                                 \\
           &                     & 2-Verificar se já existe uma melhor resposta - Sim        \\
           &                     & 3- Verificar se a resposta existente é a mesma selecionada - Sim \\
           &                     & 4-Remover melhor resposta                    
\end{tblr}
\end{table}

\newpage

\subsubsection{Especificação de caso de uso de eliminar tópico}

O técnico sempre que desejar poderá eliminar o tópico, conseguindo assim remover este tópico do fórum, 
não sendo este novamente mostrado.

% \usepackage{color}
% \usepackage{tabularray}
\definecolor{Concrete}{rgb}{0.952,0.952,0.952}
\begin{table}[htb]
\centering
\label{tab:10}
\caption{Tabela de especificação do caso de uso de eliminar tópico}
\begin{tblr}{
 width = \linewidth,
 colspec = {Q[290]Q[371]Q[273]},
 row{6} = {Concrete},
 cell{1}{1} = {Concrete},
 cell{1}{2} = {c=2}{0.644\linewidth},
 cell{2}{1} = {Concrete},
 cell{2}{2} = {c=2}{0.644\linewidth},
 cell{3}{1} = {Concrete},
 cell{3}{2} = {c=2}{0.644\linewidth},
 cell{4}{1} = {Concrete},
 cell{4}{2} = {c=2}{0.644\linewidth},
 cell{5}{1} = {Concrete},
 cell{5}{2} = {c=2}{0.644\linewidth},
 cell{6}{2} = {c},
 cell{6}{3} = {c},
 cell{7}{1} = {r=2}{Concrete},
 cell{9}{1} = {Concrete},
 cell{9}{2} = {c},
 cell{9}{3} = {c},
 vlines,
 hline{1-7,9-10} = {-}{},
 hline{8} = {2-3}{},
}
Caso de Uso      & Eliminar tópico       &          \\
Descrição       & Eliminar um tópico do fórum &          \\
Ator         & Técnico           &          \\
Pré-condição     & Clicar no tópico desejado  &          \\
Pós-condição     & Remoção do tópico      &          \\
           & Ator            & Sistema      \\
Fluxo Principal    & 1-Clicar em remover tópico &          \\
           &               & 3-Remover o tópico \\
Fluxo Alternativo(A1) & -              & -         
\end{tblr}
\end{table}

\subsubsection{Especificação de caso de uso de alterar visibilidade de um tópico}

Quando um técnico pública um tópico este pode desejar alterar a sua visibilidade para apenas técnicos ou 
todos os utilizadores conseguirem ver.

% \usepackage{color}
% \usepackage{tabularray}
\definecolor{Concrete}{rgb}{0.952,0.952,0.952}
\begin{table}[htb]
\centering
\label{tab:11}
\caption{Tabela de especificação de caso de uso de alteração de visibilidade de um tópico}
\begin{tblr}{
 width = \linewidth,
 colspec = {Q[242]Q[344]Q[354]},
 row{6} = {Concrete},
 cell{1}{1} = {Concrete},
 cell{1}{2} = {c=2}{0.698\linewidth},
 cell{2}{1} = {Concrete},
 cell{2}{2} = {c=2}{0.698\linewidth},
 cell{3}{1} = {Concrete},
 cell{3}{2} = {c=2}{0.698\linewidth},
 cell{4}{1} = {Concrete},
 cell{4}{2} = {c=2}{0.698\linewidth},
 cell{5}{1} = {Concrete},
 cell{5}{2} = {c=2}{0.698\linewidth},
 cell{6}{2} = {c},
 cell{6}{3} = {c},
 cell{7}{1} = {r=2}{Concrete},
 cell{9}{1} = {Concrete},
 cell{9}{2} = {c},
 cell{9}{3} = {c},
 vlines,
 hline{1-7,9-10} = {-}{},
 hline{8} = {2-3}{},
}
Caso de Uso      & Alterar visibilidade do tópico               &                  \\
Descrição       & Alterar a visibilidade de um tópico entre público e privado &                  \\
Ator         & Técnico                           &                  \\
Pré-condição     & Clicar no tópico desejado                  &                  \\
Pós-condição     & Alterar visibilidade do tópico               &                  \\
           & Ator                            & Sistema              \\
Fluxo Principal    & 1-Clicar em alterar visibilidade              &                  \\
           &                               & 2-Inverter visibilidade do tópico \\
Fluxo Alternativo(A1) & -                              & -                 
\end{tblr}
\end{table}

\newpage

\subsubsection{Especificação de caso de uso gostar de um tópico}

O técnico sempre que encontra um tópico que identifica como útil, este poderá gostar o tópico dando assim 
destaque a este.

% \usepackage{color}
% \usepackage{tabularray}
\definecolor{Concrete}{rgb}{0.952,0.952,0.952}
\begin{table}[htb]
\centering
\label{tab:12}
\caption{Tabela de especificação de caso de uso de gostar de um tópico}
\begin{tblr}{
 width = \linewidth,
 colspec = {Q[260]Q[221]Q[458]},
 row{6} = {Concrete},
 cell{1}{1} = {Concrete},
 cell{1}{2} = {c=2}{0.679\linewidth},
 cell{2}{1} = {Concrete},
 cell{2}{2} = {c=2}{0.679\linewidth},
 cell{3}{1} = {Concrete},
 cell{3}{2} = {c=2}{0.679\linewidth},
 cell{4}{1} = {Concrete},
 cell{4}{2} = {c=2}{0.679\linewidth},
 cell{5}{1} = {Concrete},
 cell{5}{2} = {c=2}{0.679\linewidth},
 cell{6}{2} = {c},
 cell{6}{3} = {c},
 cell{7}{1} = {r=3}{Concrete},
 cell{10}{1} = {r=3}{Concrete},
 vlines,
 hline{1-7,10,13} = {-}{},
 hline{8-9,11-12} = {2-3}{},
}
Caso de Uso      & Gostar do tópico     &                    \\
Descrição       & Gostar de um tópico    &                    \\
Ator         & Técnico          &                    \\
Pré-condição     & Clicar no tópico desejado &                    \\
Pós-condição     & Alterar gostos do tópico &                    \\
           & Ator           & Sistema                \\
Fluxo Principal    & 1-Clicar em gosto     &                    \\
           &              & 2-Verificar se o gosto já existe - Não \\
           &              & 3-Acrescentar gosto ao tópico     \\
Fluxo Alternativo(A1) & 1-Clicar em gosto     &                    \\
           &              & 2-Verificar se o gosto já existe - Sim \\
           &              & 3-Remover gosto do tópico       
\end{tblr}
\end{table}


\subsubsection{Especificação de caso de uso gostar de um comentário}

Sempre que um técnico identificar um comentário como útil este poderá gostar dos comentários, dando 
assim destaque a este.

\input{tables/casos_de_uso/gostar_comentario}

\newpage

\subsubsection{Especificação de caso de uso de comentar tópico}

Sempre que um técnico encontra um tópico que sobre uma questão que poderá ajudar este consegue responder 
a um tópico, criando assim um comentário, onde este poderá anexar também imagens.

% \usepackage{color}
% \usepackage{tabularray}
\definecolor{Concrete}{rgb}{0.952,0.952,0.952}
\begin{table}[htb]
\centering
\label{tab:14}
\caption{Tabela de especificação de caso de uso de comentar um tópico}
\begin{tblr}{
 width = \linewidth,
 colspec = {Q[225]Q[348]Q[367]},
 row{6} = {Concrete},
 cell{1}{1} = {Concrete},
 cell{1}{2} = {c=2}{0.715\linewidth},
 cell{2}{1} = {Concrete},
 cell{2}{2} = {c=2}{0.715\linewidth},
 cell{3}{1} = {Concrete},
 cell{3}{2} = {c=2}{0.715\linewidth},
 cell{4}{1} = {Concrete},
 cell{4}{2} = {c=2}{0.715\linewidth},
 cell{5}{1} = {Concrete},
 cell{5}{2} = {c=2}{0.715\linewidth},
 cell{6}{2} = {c},
 cell{6}{3} = {c},
 cell{7}{1} = {r=4}{Concrete},
 cell{11}{1} = {r=3}{Concrete},
 cell{14}{1} = {Concrete},
 vlines,
 hline{1-7,11,14-15} = {-}{},
 hline{8-10,12-13} = {2-3}{},
}
Caso de Uso      & Comentar o tópico         &                   \\
Descrição       & Comentar um tópico        &                   \\
Ator         & Técnico              &                   \\
Pré-condição     & Clicar no tópico desejado     &                   \\
Pós-condição     & Inserir a resposta no tópico   &                   \\
           & Ator               & Sistema               \\
Fluxo Principal    & 1-Indicar a descrição da resposta &                   \\
           & 2-Anexar Imagem          &                   \\
           & 3-Confirmar a resposta      &                   \\
           &                  & 4-Inserir novo comentário no tópico \\
Fluxo Alternativo(A1) & 1-Indicar a descrição da resposta &                   \\
           & 2-Confirmar a resposta      &                   \\
           &                  & 3-Inserir novo comentário no tópico \\
Fluxo Alternativo(A2) & 1-Cancelar a criação do tópico  &                   
\end{tblr}
\end{table}

\newpage

\subsubsection{Especificação de caso de uso ativar conta}

O técnico de forma a ativar a sua conta deverá introduzir o código de ativação de conta correto, 
caso contrário não será possível ativar a sua conta.

% \usepackage{color}
% \usepackage{tabularray}
\definecolor{Concrete}{rgb}{0.952,0.952,0.952}
\begin{longtblr}
  [
  caption={Tabela de especificação de caso de uso ativação de conta},
  label={tab:15},
  ]{
  width = \linewidth,
  colspec = {Q[219]Q[300]Q[419]},
  row{6} = {Concrete},
  cell{1}{1} = {Concrete},
  cell{1}{2} = {c=2}{0.719\linewidth},
  cell{2}{1} = {Concrete},
  cell{2}{2} = {c=2}{0.719\linewidth},
  cell{3}{1} = {Concrete},
  cell{3}{2} = {c=2}{0.719\linewidth},
  cell{4}{1} = {Concrete},
  cell{4}{2} = {c=2}{0.719\linewidth,c},
  cell{5}{1} = {Concrete},
  cell{5}{2} = {c=2}{0.719\linewidth,c},
  cell{6}{2} = {c},
  cell{6}{3} = {c},
  cell{7}{1} = {r=4}{Concrete},
  cell{11}{1} = {Concrete},
  cell{12}{1} = {r=4}{Concrete},
  vlines,
  hline{1-7,11-12,16} = {-}{},
  hline{8-10,13-15} = {2-3}{},
}
Caso de Uso           & Ativar conta                  &                                           \\
Descrição             & Ativar conta de aplicação     &                                           \\
Ator                  & Técnico                       &                                           \\
Pré-condição          & -                             &                                           \\
Pós-condição          & -                             &                                           \\
                      & Ator                          & Sistema                                   \\
Fluxo Principal       & 1-Inserir código de validação &                                           \\
                      & 2-Validar conta               &                                           \\
                      &                               & 3-Verificar se o código está correto- Sim \\
                      &                               & 4-Validar conta                           \\
Fluxo Alternativo(A1) & 1-Cancelar Ativação de conta  &                                           \\
Fluxo Alternativo(A2) & 1-Inserir código de validação &                                           \\
                      & 2-Validar conta               &                                           \\
                      &                               & 3-Verificar se o código está correto- Não \\
                      &                               & 4-Código de valdiação incorreto           
\end{longtblr}

\newpage

\subsubsection{Especificação de caso de uso configurar notificações}

As notificações da aplicação poderão ser personalizadas, sendo possível escolher entre email, push e ambos,
para além destas configurações, é também possivel persolinzar o tipo de notificação para cada método, seja 
relatório diário de todas as notificações ou notificações em tempo real.

% \usepackage{color}
% \usepackage{tabularray}
\definecolor{Concrete}{rgb}{0.952,0.952,0.952}
\begin{table}[htb]
\centering
\label{tab:16}
\caption{Tabela de especificação de caso de uso de configuração de notificações}
\begin{tblr}{
  width = \linewidth,
  colspec = {Q[258]Q[575]Q[108]},
  row{6} = {Concrete},
  cell{1}{1} = {Concrete},
  cell{1}{2} = {c=2}{0.682\linewidth},
  cell{2}{1} = {Concrete},
  cell{2}{2} = {c=2}{0.682\linewidth},
  cell{3}{1} = {Concrete},
  cell{3}{2} = {c=2}{0.682\linewidth},
  cell{4}{1} = {Concrete},
  cell{4}{2} = {c=2}{0.682\linewidth},
  cell{5}{1} = {Concrete},
  cell{5}{2} = {c=2}{0.682\linewidth},
  cell{6}{2} = {c},
  cell{6}{3} = {c},
  cell{7}{1} = {r=2}{Concrete},
  cell{9}{1} = {Concrete},
  vlines,
  hline{1-7,9-10} = {-}{},
  hline{8} = {2-3}{},
}
Caso de Uso           & Configuração de notificações                     &         \\
Descrição             & Configuração de notificações do técnico          &         \\
Ator                  & Técnico                                          &         \\
Pré-condição          & -                                                &         \\
Pós-condição          & -                                                &         \\
                      & Ator                                             & Sistema \\
Fluxo Principal       & 1-Indicar preferência de receção de notificações &         \\
                      & 2-Indicar tipo de receção de notificações        &         \\
Fluxo Alternativo(A1) & 1-Ver notificações                               &         
\end{tblr}
\end{table}

\subsubsection{Especificação de caso de uso registar técnico}

Sempre que uma empresa deseja realizar o registo de técnicos em seu nome, esta poderá indicar o nºcontribuinte
do mesmo e email, com isto este receberá um email para confirmar o registo de conta.

% \usepackage{color}
% \usepackage{tabularray}
\definecolor{Concrete}{rgb}{0.952,0.952,0.952}
\begin{table}[htb]
\centering
\label{tab:17}
\caption{Tabela de especificação de caso de uso de registar técnico}
\begin{tblr}{
 width = \linewidth,
 colspec = {Q[331]Q[454]Q[142]},
 row{6} = {Concrete},
 cell{1}{1} = {Concrete},
 cell{1}{2} = {c=2}{0.627\linewidth},
 cell{2}{1} = {Concrete},
 cell{2}{2} = {c=2}{0.627\linewidth},
 cell{3}{1} = {Concrete},
 cell{3}{2} = {c=2}{0.627\linewidth},
 cell{4}{1} = {Concrete},
 cell{4}{2} = {c=2}{0.627\linewidth},
 cell{5}{1} = {Concrete},
 cell{5}{2} = {c=2}{0.627\linewidth},
 cell{6}{2} = {c},
 cell{6}{3} = {c},
 cell{7}{1} = {r=3}{Concrete},
 cell{10}{1} = {Concrete},
 cell{10}{2} = {c},
 cell{10}{3} = {c},
 vlines,
 hline{1-7,10-11} = {-}{},
 hline{8-9} = {2-3}{},
}
Caso de Uso      & Registar técnico           &          \\
Descrição       & Registar conta de técnico da empresa &          \\
Ator         & Empresa               &          \\
Pré-condição     & -                  &          \\
Pós-condição     & -                  &          \\
           & Ator                 & Sistema      \\
Fluxo Principal    & 1-Indicar o nºcontribuinte      &          \\
           & 2-Indicar \textit{email}           &          \\
           &                   & 3-Registar técnico \\
Fluxo Alternativo(A1) & -                  & -         
\end{tblr}
\end{table}


% \subsubsection{Especificação de caso de uso responder a comentário}

% O técnico poderá manter uma conversa com outros técnicos através da resposta a outros comentários, 
% a qual poderá também incluir imagens.

% % \usepackage{color}
% \usepackage{tabularray}
\definecolor{Concrete}{rgb}{0.952,0.952,0.952}
\begin{table}[htb]
\centering
\label{tab:18}
\caption{Tabela de especificação de caso de uso de responder a comentário}
\begin{tblr}{
  width = \linewidth,
  colspec = {Q[229]Q[381]Q[333]},
  row{6} = {Concrete},
  cell{1}{1} = {Concrete},
  cell{1}{2} = {c=2}{0.714\linewidth},
  cell{2}{1} = {Concrete},
  cell{2}{2} = {c=2}{0.714\linewidth},
  cell{3}{1} = {Concrete},
  cell{3}{2} = {c=2}{0.714\linewidth},
  cell{4}{1} = {Concrete},
  cell{4}{2} = {c=2}{0.714\linewidth},
  cell{5}{1} = {Concrete},
  cell{5}{2} = {c=2}{0.714\linewidth},
  cell{6}{2} = {c},
  cell{6}{3} = {c},
  cell{7}{1} = {r=2}{Concrete},
  cell{9}{1} = {Concrete},
  cell{9}{2} = {c},
  cell{9}{3} = {c},
  vlines,
  hline{1-7,9-10} = {-}{},
  hline{8} = {2-3}{},
}
Caso de Uso           & Responder a comentário                 &                                 \\
Descrição             & Responder a um comentário de um tópico &                                 \\
Ator                  & Técnico                                &                                 \\
Pré-condição          & Clicar no tópico desejado              &                                 \\
Pós-condição          & Novo comentário                        &                                 \\
                      & Ator                                   & Sistema                         \\
Fluxo Principal       & 1-Clicar em responder a comentário     &                                 \\
                      &                                        & 2-Inserir resposta a comentário \\
Fluxo Alternativo(A1) & -                                      & -                               
\end{tblr}
\end{table}

%---------------------------------------------------------------------------------

% \subsubsection{Especificação de caso de uso registo}

% A empresa deverá realizar o registo utilizando o número de contribuinte, email e password. Após este 
% registo, a Motorline validará o registo e de seguida um email é enviado para confirmar o registo na app.

% % \usepackage{color}
% \usepackage{tabularray}
\definecolor{Concrete}{rgb}{0.952,0.952,0.952}
\begin{table}[htb]
\centering
\label{tab:21}
\caption{Tabela de especificação de caso de uso de registo}
\begin{tblr}{
 width = \linewidth,
 colspec = {Q[267]Q[348]Q[323]},
 row{6} = {Concrete},
 cell{1}{1} = {Concrete},
 cell{1}{2} = {c=2}{0.671\linewidth},
 cell{2}{1} = {Concrete},
 cell{2}{2} = {c=2}{0.671\linewidth},
 cell{3}{1} = {Concrete},
 cell{3}{2} = {c=2}{0.671\linewidth},
 cell{4}{1} = {Concrete},
 cell{4}{2} = {c=2}{0.671\linewidth},
 cell{5}{1} = {Concrete},
 cell{5}{2} = {c=2}{0.671\linewidth},
 cell{6}{2} = {c},
 cell{6}{3} = {c},
 cell{7}{1} = {r=7}{Concrete},
 cell{14}{1} = {Concrete},
 vlines,
 hline{1-7,14-15} = {-}{},
 hline{8-13} = {2-3}{},
}
Caso de Uso      & Registo de cliente ou técnico       &            \\
Descrição       & Registo de cliente ou técnico na aplicação &            \\
Ator         & Cliente                  &            \\
Pré-condição     & -                     &            \\
Pós-condição     & Email de verificação de código       &            \\
           & Ator                    & Sistema        \\
Fluxo Principal    & 1-Indicar Nº Contribuinte         &            \\
           & 2-Indicar nome de empresa         &            \\
           & 3-Indicar Email              &            \\
           & 4-Password                 &            \\
           & 5-Confirmação Password           &            \\
           &                      & 6-Verificar Registo  \\
           &                      & 7-Mensagem de Sucesso \\
Fluxo Alternativo(A1) & 1-Cancelar Registo             &            
\end{tblr}
\end{table}

%---------------------------------------------------------------------------------

% \subsubsection{Especificação de caso de uso pedir reenvio de código de verificação}

% O técnico poderá aquando a validação da sua conta pedir o reenvio de um novo código de validação em caso 
% de algum imprevisto.

% % \usepackage{color}
% \usepackage{tabularray}
\definecolor{Concrete}{rgb}{0.952,0.952,0.952}
\begin{table}[htb]
\centering
\begin{tblr}{
 width = \linewidth,
 colspec = {Q[223]Q[356]Q[362]},
 row{6} = {Concrete},
 cell{1}{1} = {Concrete},
 cell{1}{2} = {c=2}{0.706\linewidth},
 cell{2}{1} = {Concrete},
 cell{2}{2} = {c=2}{0.706\linewidth},
 cell{3}{1} = {Concrete},
 cell{3}{2} = {c=2}{0.706\linewidth},
 cell{4}{1} = {Concrete},
 cell{4}{2} = {c=2}{0.706\linewidth,c},
 cell{5}{1} = {Concrete},
 cell{5}{2} = {c=2}{0.706\linewidth},
 cell{6}{2} = {c},
 cell{6}{3} = {c},
 cell{7}{1} = {r=3}{Concrete},
 cell{10}{1} = {Concrete},
 cell{10}{2} = {c},
 cell{10}{3} = {c},
 vlines,
 hline{1-7,10-11} = {-}{},
 hline{8-9} = {2-3}{},
}
Caso de Uso      & Pedir reenvio de código de ativação     &                 \\
Descrição       & Pedir reenvio de \textit{email} de código de ativação &                 \\
Ator         & Técnico                   &                 \\
Pré-condição     & -                      &                 \\
Pós-condição     & Email de verificação de código        &                 \\
           & Ator                     & Sistema             \\
Fluxo Principal    & 1-Pedir novo código de ativação       &                 \\
           &                       & 2-Gerar novo código de ativação \\
           &                       & 3-Enviar novo \textit{email}       \\
Fluxo Alternativo(A1) & -                      & -                
\end{tblr}
\end{table}

%---------------------------------------------------------------------------------

% \subsubsection{Especificação de caso de uso confirmar conta}

% Sempre que uma conta de técnico é criada, este deverá proceder à confirmação da conta, nesta confirmação
% o técnico tem de inidicar o seu nome de utilizador, poderá alterar o email de registo e terá de indicar a
% password e confirmação de password, procedendo depois à ativação da conta.

% % \usepackage{color}
% \usepackage{tabularray}
\definecolor{Concrete}{rgb}{0.952,0.952,0.952}
\begin{table}[htb]
\centering
\begin{tblr}{
  width = \linewidth,
  colspec = {Q[258]Q[429]Q[254]},
  row{6} = {Concrete},
  cell{1}{1} = {Concrete},
  cell{1}{2} = {c=2}{0.683\linewidth},
  cell{2}{1} = {Concrete},
  cell{2}{2} = {c=2}{0.683\linewidth},
  cell{3}{1} = {Concrete},
  cell{3}{2} = {c=2}{0.683\linewidth},
  cell{4}{1} = {Concrete},
  cell{4}{2} = {c=2}{0.683\linewidth,c},
  cell{5}{1} = {Concrete},
  cell{5}{2} = {c=2}{0.683\linewidth,c},
  cell{6}{2} = {c},
  cell{6}{3} = {c},
  cell{7}{1} = {r=6}{Concrete},
  cell{13}{1} = {Concrete},
  vlines,
  hline{1-7,13-14} = {-}{},
  hline{8-12} = {2-3}{},
}
Caso de Uso           & Confirmar conta                   &                        \\
Descrição             & Confirmar conta de técnico        &                        \\
Ator                  & Técnico                           &                        \\
Pré-condição          & -                                 &                        \\
Pós-condição          & -                                 &                        \\
                      & Ator                              & Sistema                \\
Fluxo Principal       & 1-Inserir nome de utilizador      &                        \\
                      & 2-Indicar Password                &                        \\
                      & 3-Indicar Confirmação de password &                        \\
                      &                                   & 4-Verificar se registo \\
                      &                                   & 5-Conta registada      \\
                      &                                   & 6-Validar conta        \\
Fluxo Alternativo(A1) & 1-Cancelar Ativação de conta      &                        
\end{tblr}
\end{table}

%---------------------------------------------------------------------------------

% \subsubsection{Especificação de caso de uso ver perfil}

% Sempre que um técnico desejar alterar alguma informação sua, este poderá se dirigir ao seu perfil onde 
% consegue alterar o seu email e imagem de perfil.

% % \usepackage{color}
% \usepackage{tabularray}
\definecolor{Concrete}{rgb}{0.952,0.952,0.952}
\begin{table}[htb]
\centering
\begin{tblr}{
  width = \linewidth,
  colspec = {Q[252]Q[310]Q[379]},
  row{6} = {Concrete},
  cell{1}{1} = {Concrete},
  cell{1}{2} = {c=2}{0.689\linewidth},
  cell{2}{1} = {Concrete},
  cell{2}{2} = {c=2}{0.689\linewidth},
  cell{3}{1} = {Concrete},
  cell{3}{2} = {c=2}{0.689\linewidth},
  cell{4}{1} = {Concrete},
  cell{4}{2} = {c=2}{0.689\linewidth},
  cell{5}{1} = {Concrete},
  cell{5}{2} = {c=2}{0.689\linewidth},
  cell{6}{2} = {c},
  cell{6}{3} = {c},
  cell{7}{1} = {r=2}{Concrete},
  cell{9}{1} = {r=2}{Concrete},
  vlines,
  hline{1-7,9,11} = {-}{},
  hline{8,10} = {2-3}{},
}
Caso de Uso           & Ver perfil                 &                                 \\
Descrição             & Ver perfil do técnico      &                                 \\
Ator                  & Técnico                    &                                 \\
Pré-condição          & -                          &                                 \\
Pós-condição          & -                          &                                 \\
                      & Ator                       & Sistema                         \\
Fluxo Principal       & 1-Alterar \textit{email}            &                                 \\
                      &                            & 2-Alteração de \textit{email}            \\
Fluxo Alternativo(A1) & 1-Alterar imagem de perfil &                                 \\
                      &                            & 2-Alteração de imagem de perfil 
\end{tblr}
\end{table}

%---------------------------------------------------------------------------------

% \subsubsection{Especificação de caso de uso ver notificações}

% Sempre que um técnico desejar ver todas as suas notificações este poderá ver esta listagem, 
% conseguindo também apagar notificações que já não deseja ver.

% % \usepackage{color}
% \usepackage{tabularray}
\definecolor{Concrete}{rgb}{0.952,0.952,0.952}
\begin{table}[htb]
\centering
\begin{tblr}{
  width = \linewidth,
  colspec = {Q[287]Q[285]Q[363]},
  row{6} = {Concrete},
  cell{1}{1} = {Concrete},
  cell{1}{2} = {c=2}{0.647\linewidth},
  cell{2}{1} = {Concrete},
  cell{2}{2} = {c=2}{0.647\linewidth},
  cell{3}{1} = {Concrete},
  cell{3}{2} = {c=2}{0.647\linewidth},
  cell{4}{1} = {Concrete},
  cell{4}{2} = {c=2}{0.647\linewidth},
  cell{5}{1} = {Concrete},
  cell{5}{2} = {c=2}{0.647\linewidth},
  cell{6}{2} = {c},
  cell{6}{3} = {c},
  cell{7}{1} = {r=2}{Concrete},
  cell{9}{1} = {r=4}{Concrete},
  vlines,
  hline{1-7,9,13} = {-}{},
  hline{8,10-12} = {2-3}{},
}
Caso de Uso           & Ver Notificações            &                            \\
Descrição             & Ver notificações do técnico &                            \\
Ator                  & Técnico                     &                            \\
Pré-condição          & -                           &                            \\
Pós-condição          & -                           &                            \\
                      & Ator                        & Sistema                    \\
Fluxo Principal       & 1-Ver notificações          &                            \\
                      &                             & 2-Listagem de notificações \\
Fluxo Alternativo(A1) & 1-Ver notificações          &                            \\
                      &                             & 2-Listagem de notificações \\
                      & 3-Apagar notificação        &                            \\
                      &                             & 4-Eliminar notificação     
\end{tblr}
\end{table}

%---------------------------------------------------------------------------------

% \subsubsection{Especificação de caso de uso apagar comentário}

% Sempre que o técnico cria um comentário este tem a possibilidade de o remover a qualquer momento que 
% desejar.

% % \usepackage{color}
% \usepackage{tabularray}
\definecolor{Concrete}{rgb}{0.952,0.952,0.952}
\begin{table}[htb]
\centering
\label{tab:17}
\caption{Tabela de especificação de caso de uso de apagar comentário}
\begin{tblr}{
 width = \linewidth,
 colspec = {Q[271]Q[394]Q[273]},
 row{6} = {Concrete},
 cell{1}{1} = {Concrete},
 cell{1}{2} = {c=2}{0.667\linewidth},
 cell{2}{1} = {Concrete},
 cell{2}{2} = {c=2}{0.667\linewidth},
 cell{3}{1} = {Concrete},
 cell{3}{2} = {c=2}{0.667\linewidth},
 cell{4}{1} = {Concrete},
 cell{4}{2} = {c=2}{0.667\linewidth},
 cell{5}{1} = {Concrete},
 cell{5}{2} = {c=2}{0.667\linewidth},
 cell{6}{2} = {c},
 cell{6}{3} = {c},
 cell{7}{1} = {r=2}{Concrete},
 cell{9}{1} = {Concrete},
 cell{9}{2} = {c},
 cell{9}{3} = {c},
 vlines,
 hline{1-7,9-10} = {-}{},
 hline{8} = {2-3}{},
}
Caso de Uso      & Apagar comentário       &           \\
Descrição       & Apagar comentário de um tópico &           \\
Ator         & Técnico            &           \\
Pré-condição     & Clicar no tópico desejado   &           \\
Pós-condição     & Comentário apagado       &           \\
           & Ator              & Sistema       \\
Fluxo Principal    & 1-Clicar em apagar comentário &           \\
           &                & 2-Apagar comentário \\
Fluxo Alternativo(A1) & -               & -          
\end{tblr}
\end{table}

%---------------------------------------------------------------------------------

% \subsubsection{Especificação de caso de uso login}

% O técnico deverá realizar o login utilizando o número de contribuinte e password.

% % \usepackage{color}
% \usepackage{tabularray}
\definecolor{Concrete}{rgb}{0.952,0.952,0.952}
\begin{longtblr}
 [
 caption={Tabela de especificação de caso de uso criação de novo tópico},
 label={tab:5},
 ]{
  width = \linewidth,
  colspec = {Q[262]Q[373]Q[306]},
  row{6} = {Concrete},
  cell{1}{1} = {Concrete},
  cell{1}{2} = {c=2}{0.679\linewidth},
  cell{2}{1} = {Concrete},
  cell{2}{2} = {c=2}{0.679\linewidth},
  cell{3}{1} = {Concrete},
  cell{3}{2} = {c=2}{0.679\linewidth},
  cell{4}{1} = {Concrete},
  cell{4}{2} = {c=2}{0.679\linewidth},
  cell{5}{1} = {Concrete},
  cell{5}{2} = {c=2}{0.679\linewidth},
  cell{6}{2} = {c},
  cell{6}{3} = {c},
  cell{7}{1} = {r=4}{Concrete},
  cell{11}{1} = {Concrete},
  cell{12}{1} = {Concrete},
  cell{13}{1} = {r=4}{Concrete},
  vlines,
  hline{1-7,11-13,17} = {-}{},
  hline{8-10,14-16} = {2-3}{},
 }
 Caso de Uso      & Login              &             \\
 Descrição       & Iniciar sessão na aplicação   &             \\
 Ator         & Técnico             &             \\
 Pré-condição     & Código de verificação confirmado &             \\
 Pós-condição     & Home Screen           &             \\
            & Ator               & Sistema         \\
 Fluxo Principal    & 1-Indicar Nº Contribuinte    &             \\
            & 2-Indicar Password        &             \\
            &                 & 3-Verificar Login    \\
            &                 & 4-Devolver Sessão    \\
 Fluxo Alternativo(A1) & 1-Clicar em não iniciar sessão  &             \\
 Fluxo Alternativo(A2) & 1-Cancelar Login         &             \\
 Fluxo Alternativo(A3) & 1-Indicar Nº Contribuinte    &             \\
            & 2-Indicar Password        &             \\
            &                 & 3-Verificar Login    \\
            &                 & 4-Erro conta não ativada 
\end{longtblr}

%---------------------------------------------------------------------------------

%\subsubsection{Especificação de caso de uso de pesquisar tópicos por código QR}

%Assim que um utilizador deseje pesquisar por tópicos relativos a um produto, este poderá utilizar 
%o código QR do mesmo, conseguindo também realizar uma pesquisa por escrito.    

%% \usepackage{color}
% \usepackage{tabularray}
\definecolor{Concrete}{rgb}{0.952,0.952,0.952}
\begin{longtblr}
[
caption={Tabela de especificação de caso de uso de pesquisa por código QR},
label={tab:7},
]{
  width = \linewidth,
  colspec = {Q[175]Q[219]Q[546]},
  row{6} = {Concrete},
  cell{1}{1} = {Concrete},
  cell{1}{2} = {c=2}{0.74\linewidth},
  cell{2}{1} = {Concrete},
  cell{2}{2} = {c=2}{0.74\linewidth},
  cell{3}{1} = {Concrete},
  cell{3}{2} = {c=2}{0.74\linewidth},
  cell{4}{1} = {Concrete},
  cell{4}{2} = {c=2}{0.74\linewidth},
  cell{5}{1} = {Concrete},
  cell{5}{2} = {c=2}{0.74\linewidth,c},
  cell{6}{2} = {c},
  cell{6}{3} = {c},
  cell{7}{1} = {r=6}{Concrete},
  cell{13}{1} = {r=4}{Concrete},
  vlines,
  hline{1-7,13,17} = {-}{},
  hline{8-12,14-16} = {2-3}{},
}
Caso de Uso           & Pesquisa por código QR                       &                                                        \\
Descrição             & Pesquisar por tópicos no fórum por código QR &                                                        \\
Ator                  & Utilizador                                   &                                                        \\
Pré-condição          & Selecionar pesquisa de fórum                 &                                                        \\
Pós-condição          & -                                            &                                                        \\
                      & Ator                                         & Sistema                                                \\
Fluxo Principal       & 1-Pesquisar por código QR                    &                                                        \\
                      &                                              & 2-Lista de tópicos do produto                          \\
                      & 2-Pesquisar assunto                          &                                                        \\
                      &                                              & 3-Filtragem de tópicos do produto por assunto          \\
                      & 4-Filtrar por tipo                           &                                                        \\
                      &                                              & 5-Filtragem de tópicos do produto por assunto e tópico \\
Fluxo Alternativo(A1) & 1-Pesquisar por código QR                    &                                                        \\
                      &                                              & 2-Lista de tópicos do produto                          \\
                      & 2-Pesquisar assunto                          &                                                        \\
                      &                                              & 3-Filtragem de tópicos do produto por assunto          
\end{longtblr}

%\newpage

% \newpage

% 
\subsection{Diagramas de casos de uso}
Para ser possível visualizar graficamente todas as ações que os 
atores conseguem realizar e para melhorar a comunicação com as 
partes interessadas do projeto, foram desenvolvidos diagramas de 
casos de uso.

\subsubsection{Casos de uso Fórum}
Na Figura~\ref{fig:12}, é possível 
visualizar o diagrama de casos de uso para o fórum.
Neste, o técnico poderá ver as listagens de tópicos em destaque e tópicos mais recentes. Caso seja um técnico oficial terá acesso aos tópicos privados e os seus tópicos.
O técnico poderá também pesquisar por tópicos, dos quais ele terá a possibilidade de selecionar um para visualizar. 
O técnico conseguirá, além disso, criar um novo tópico, onde se dirigirá para a criação de tópicos. Aqui, o técnico será obrigado a inserir um título, descrição e tipo do tópico para o criar, 
mas poderá inserir imagens e indicar produto referente.
Para finalizar a criação o técnico conseguirá confirmar ou cancelar o processo. 

\begin{figure}[htb]
  \centering
  \includegraphics[width=0.6\textwidth]{images/diagramas/casos_de_uso/use_case_forum.png}
  \caption{Diagrama de casos de uso de fórum}
  \label{fig:12}
\end{figure}

\subsubsection{Casos de uso de pesquisar tópicos}

O técnico poderá realizar a pesquisa por tópicos específicos, 
esta será ser realizada por escrito onde indica o assunto a pesquisar e poderá ser filtrada.

O técnico terá também a possibilidade de pesquisar por código QR de produto, uma vez que, o servidor da Motorline esteja desenvolvido para tal.

\begin{figure}[htb]
  \centering
  \includegraphics[width=0.6\textwidth]{images/diagramas/casos_de_uso/use_case_forum_search.png}
  \caption{Diagrama de casos de uso de pesquisa de tópicos}
  \label{fig:13}
\end{figure}

\subsubsection{Casos de uso ver detalhes de tópico}

Assim que um técnico seleciona um tópico, é movido para os 
detalhes, onde consegue visualizar os detalhes, responder e, caso seja o seu tópico, consegue finalizar, selecionar a melhor resposta, remover a melhor resposta, eliminar e alterar a visibilidade do tópico.

\begin{figure}[htb]
  \centering
  
  \includegraphics[width=0.5\textwidth]{images/diagramas/casos_de_uso/use_case_topic_details.png}
  \caption{Diagrama de casos de uso de detalhes de tópico}
  \label{fig:14}
\end{figure}

\subsubsection{Casos de uso ver comentários}

O técnico quando decide visualizar os comentários consegue responder e gostar de uma resposta ou comentário, caso este seja seu ainda o consegue apagar.

\begin{figure}[htb]
  \centering
  
  \includegraphics[width=0.5\textwidth]{images/diagramas/casos_de_uso/use_case_topic_comments.png}
  \caption{Diagrama de casos de uso de ver comentários}
  \label{fig:15}
\end{figure}

\subsubsection{Casos de uso ativação de conta}

Assim que uma conta é confirmada, um \textit{email} de ativação é enviado para técnico e esta deverá ser ativada.
Para isto, o código deverá ser indicado pelo técnico para se proceder à ativação da conta. Este, poderá em caso de necessidade, pedir o reenvio do código de ativação, o qual será gerado novamente e reenviado.

\begin{figure}[htb]
  \centering
  
  \includegraphics[width=0.5\textwidth]{images/diagramas/casos_de_uso/use_case_account_validation.png}
  \caption{Diagrama de casos de uso de ativação de conta}
  \label{fig:16}
\end{figure}

\subsubsection{Casos de uso perfil}

Sempre que o técnico desejar alterar alguma informação, este poderá alterar o seu nome, \textit{email} e imagem de perfil.

\begin{figure}[htb]
  \centering
  
  \includegraphics[width=0.5\textwidth]{images/diagramas/casos_de_uso/use_case_perfil.png}
  \caption{Diagrama de casos de uso de perfil}
  \label{fig:17}
\end{figure}

\newpage

\subsubsection{Casos de uso notificações}

Sempre que o técnico desejar ver as suas notificações, poderá seleciona-las, também dispõe da possibilidade de alterar a configuração das notificações, para apenas as receber por \textit{email} ou push, ou então, ambas. Terá também a possibilidade de personalizar cada método, para receber um relatório diário de notificações ou então, notificações em tempo real.

\begin{figure}[htb]
  \centering
  \includegraphics[width=0.5\textwidth]{images/diagramas/casos_de_uso/use_case_notificacoes.png}
  \caption{Diagrama de casos de notificações}
  \label{fig:18}
\end{figure}

\subsubsection{Casos de uso gestão de recursos humanos}

Uma empresa poderá registar contas para os seus técnicos no seu nome, com a indicação do \textit{email} e nº contribuinte. Esta poderá também impedir acesso a estas contas ou remover completamente a conta da aplicação.

\begin{figure}[htb]
  \centering
  \includegraphics[width=0.5\textwidth]{images/diagramas/casos_de_uso/use_case_rec_humanos.png}
  \caption{Diagrama de casos de uso de recursos humanos}
  \label{fig:19}
\end{figure}

% \newpage

% \section{Diagrama Entidade Relação}

O software Install\&Go é suportado por uma base de dados relacional, 
sendo esta esquematizada tendo por base as necessidades do projeto.

\begin{figure}[htb]
    \centering
    
    \includegraphics[width=\textwidth]{images/diagramas/diagrama_bd.png}
    \caption{Diagrama Entidade Relação base de dados Install\&Go}
    \label{fig:14}
\end{figure}

\newpage

\subsection{Dicionário de termos}

De forma a ser possível entender o propósito de cada tabela e atributo 
foi então criado um dicionário de termos para a base de dados(Tabela X).

\definecolor{Concrete}{rgb}{0.952,0.952,0.952}
\begin{longtblr}
[
caption={Dicionário de termos da base de dados},
label={tab:22},
]{
  width = \linewidth,
  colspec = {Q[170]Q[292]Q[240]Q[215]},
  row{1} = {Concrete},
  column{1} = {c},
  cell{2}{1} = {r=19}{},
  cell{2}{2} = {r=19}{},
  cell{22}{1} = {r=2}{},
  cell{22}{2} = {r=2}{},
  cell{24}{1} = {r=2}{},
  cell{24}{2} = {r=2}{},
  cell{26}{1} = {r=2}{},
  cell{26}{2} = {r=2}{},
  cell{28}{1} = {r=7}{},
  cell{28}{2} = {r=7}{},
  cell{35}{1} = {r=10}{},
  cell{35}{2} = {r=10}{},
  cell{45}{1} = {r=8}{},
  cell{45}{2} = {r=8}{},
  cell{53}{1} = {r=2}{},
  cell{53}{2} = {r=2}{},
  cell{55}{1} = {r=3}{},
  cell{55}{2} = {r=3}{},
  cell{58}{1} = {r=4}{},
  cell{58}{2} = {r=4}{},
  cell{62}{1} = {r=11}{},
  cell{62}{2} = {r=11}{},
  vlines,
  hline{1-2,21-22,24,26,28,35,45,53,55,58,62,73} = {-}{},
  hline{3-20,23,25,27,29-34,36-44,46-52,54,56-57,59-61,63-72} = {3-4}{},
}
Tabela         & Descrição                                                                            & Atributos            & Descrição                                           \\
users          & Tabela encarregue de guardar todos os dados referentes aos utilizadores da aplicação & uid                  & Id do utilizador                                    \\
               &                                                                                      & company\_id          & Id da empresa referente ao técnico                  \\
               &                                                                                      & n\_contribuinte      & Número de contribuinte do utilizador                \\
               &                                                                                      & name                 & Nome do utilizador                                  \\
               &                                                                                      & email                & Email do utilizador                                 \\
               &                                                                                      & password             & Password do utilizador                              \\
               &                                                                                      & profile\_pic         & Imagem de perfil do utilizador                      \\
               &                                                                                      & verif\_code          & Código de verificação do utilizador                 \\
               &                                                                                      & is\_confirmed        & Verificação de se o código está confirmado          \\
               &                                                                                      & refresh\_token       & Token de refresh                                    \\
               &                                                                                      & is\_professional     & Verificação se é profissional                       \\
               &                                                                                      & has\_access          & Verificação se tem acesso à conta                   \\
               &                                                                                      & has\_email\_noti     & Verificação se ativou notificações de email         \\
               &                                                                                      & has\_push\_noti      & Verificação se ativou notificações push             \\
               &                                                                                      & email\_noti\_type    & Tipo de notificação de email                        \\
               &                                                                                      & push\_noti\_type     & Tipo de notificação push                            \\
               &                                                                                      & is\_deleted          & Verificação se a conta se encontra apagada          \\
               &                                                                                      & is\_certified        & Verificação se é um técnico certificado             \\
               &                                                                                      & is\_official         & Verificação se é um técnico oficial                 \\
black\_list    & Tabela que guarda os tokens a bloquear                                               & token                & Token a bloquear                                    \\
liked\_topics  & Tabela encarregue de guardar todos os tópicos gostados pelo utilizador               & uid                  & Id do utilizador                                    \\
               &                                                                                      & topic\_id            & Id do tópico                                        \\
liked\_answers & Tabela encarregue de guardar todas as respostas que receberam gosto do utilizador    & uid                  & Id do utilizador                                    \\
               &                                                                                      & answer\_id           & Id da resposta                                      \\
topic\_types   & Tabela encarregue de guardar os tipos de tópico existentes                           & type\_id             & Id do tipo de tópico                                \\
               &                                                                                      & name                 & Nome do tipo de tópico                              \\
notifications  & Tabela encarregue de guardar todas as notificações do técnico                        & noti\_id             & Id da notificação                                   \\
               &                                                                                      & receiver\_id         & Recetor da notificação                              \\
               &                                                                                      & sender\_id           & Emissor da notificação                              \\
               &                                                                                      & topic\_id            & Id do tópico em caso de estar referente a um tópico \\
               &                                                                                      & is\_deleted          & Verificação se a notificação está apagada           \\
               &                                                                                      & message              & Mensagem da notificação                             \\
               &                                                                                      & date                 & Data de emissão da notificação                      \\
forum\_topics  & Tabela encarregue de guardar todos os tópicos existentes na aplicação                & topic\_id            & Id do tópico                                        \\*
               &                                                                                      & uid                  & Id do dono do tópico                                \\*
               &                                                                                      & product\_id          & Produto referente ao tópico                         \\*
               &                                                                                      & tiype\_id            & Id do tipo referente ao tópico                      \\*
               &                                                                                      & creation\_date       & Data de criação do tópico                           \\*
               &                                                                                      & description          & Descrição do tópico                                 \\*
               &                                                                                      & title                & Título do tópico                                    \\*
               &                                                                                      & is\_complete         & Verificação se o tópico está finalizado             \\*
               &                                                                                      & is\_private          & Verificação se o tópico é privado                   \\*
               &                                                                                      & is\_deleted          & Verificação se o tópico está apagado                \\*
topic\_answers & Tabela encarregue de guardar todas as respostas a um tópico                          & answer\_id           & Id da resposta                                      \\
               &                                                                                      & topic\_id            & Id do tópico                                        \\
               &                                                                                      & uid                  & Id do dono da resposta                              \\
               &                                                                                      & creation\_date       & Data de criação da resposta                         \\
               &                                                                                      & description          & Descrição da resposta                               \\
               &                                                                                      & is\_best             & Verificação se é a melhor resposta                  \\
               &                                                                                      & parent\_id           & Id da resposta pai                                  \\
               &                                                                                      & is\_deleted          & Verificação se o topico se encontra apagado         \\
categories     & Tabela encarregue de guardar todas as categorias de produtos existentes              & category\_id         & Id da categoria                                     \\*
               &                                                                                      & name                 & Nome da categoria                                   \\*
subcategories  & Tabela encarregue de guardar as subcategorias de produtos existentes                 & subcategory\_id      & Id da subcategoria                                  \\*
               &                                                                                      & category\_id         & Id da categoria                                     \\*
               &                                                                                      & name                 & Nome da subcategoria                                \\*
cb\_manuals    & Tabela encarregue de guardar os manuais de utilização das placas de controlo         & cb\_manual\_id       & Id do manual                                        \\
               &                                                                                      & product\_id          & Id do produto                                       \\
               &                                                                                      & name                 & Nome da placa de controlo                           \\
               &                                                                                      & manual               & Url do manual                                       \\
products       & Tabela encarregue de guardar as informações dos produtos do catálogo da empresa      & product\_id          & Id do produto                                       \\
               &                                                                                      & name                 & Nome do produto                                     \\
               &                                                                                      & description          & Descrição do produto                                \\
               &                                                                                      & user\_manual         & Url do manual de utilização do produto              \\
               &                                                                                      & general\_information & Url da informação geral do produto                  \\
               &                                                                                      & technical\_draw      & Url do desenho técnico do produto                   \\
               &                                                                                      & install\_video       & Url do video de instalação do produto                \\
               &                                                                                      & program\_video       & Url do video de programação do dispositivo          \\
               &                                                                                      & category\_id         & Id da categoria de produto                          \\
               &                                                                                      & subcategory\_id      & Id da subcategoria de produto                       \\
               &                                                                                      & is\_featured         & Verificação se o produto é um destaque              
\end{longtblr}

% \newpage

% \section{Diagrama de Classes}

De forma a ser possível prever e organizar o software foi desenvolvido um diagrama de classes que permite visualizar cada classe que se espera conter no software, assim como também os seus atributos e métodos.

\begin{figure}[htb]
    \centering
    
    \includegraphics[width=\textwidth]{images/diagramas/diagrama_classes.png}
    \caption{Diagrama de classes Install\&Go}
    \label{fig:15}
\end{figure}


% \newpage

% \section{Mockups}
A propósito de desenhar um design para seguir e apresentar às partes interessadas antes de iniciar a 
fase de desenvolvimento, então foram realizadas mockups do design da aplicação. Este design foi 
iterativamente revisto pelas partes interessadas e afinado até chegar ao seu estado final.

\subsection{Página Inicial}

A página inicial da aplicação, dá ao utilizador a possibilidade de navegar pelo produtos do catálogo,
conseguindo também filtrar pelas categorias e subcategorias do catálogo, 
assim como também realizar uma pesquisa rápida pelos produtos, por fim este poderá navegar para o fórum.
Caso um técnico esteja com sessão iniciada este poderá também visualizar o icon de notificações e a sua 
imagem de perfil.

\begin{figure}[htb]
    \centering
    \includegraphics[width=0.4\textwidth]{images/mockups/home_screen.png}
    \caption{Página inicial do fórum}
    \label{fig:16}
\end{figure}

\newpage

\subsection{Autenticação - Login e Registo}

Na autenticação primeiramente é possivel realizar o login, onde o técnico poderá iniciar sessão no software 
e registo onde a empresa poderá realizar o registo no software.

\begin{figure}[htb]%
    \centering
    \subfloat[\centering Página de login]{{\includegraphics[width=0.4\textwidth]{images/mockups/login.png} }}%
    \qquad
    \subfloat[\centering Página de registo]{{\includegraphics[width=0.4\textwidth]{images/mockups/register.png} }}%
    \caption{Autenticação - Login e Registo}%
    \label{fig:17}%
\end{figure}

\newpage

\subsection{Autenticação - Ativação e Confirmação de conta}

Na autenticação tem existe a página de confirmação de conta, onde um técnico que tem a sua conta recentemente adicionada poderá
confirmar o registo da conta indicando as informações finais de conta, sendo por fim direcionado para a página
de ativação de conta onde terá de colocar o código de ativação enviado para o seu email, esta página será também aberta 
caso um técnico realize login com uma conta que não foi ativada ou então sempre que um registo é finalizado.

\begin{figure}[htb]%
    \centering
    \subfloat[\centering Página de confirmação de conta]{{\includegraphics[width=0.4\textwidth]{images/mockups/account_confirmation.png} }}%
    \qquad
    \subfloat[\centering Página de ativação de conta]{{\includegraphics[width=0.4\textwidth]{images/mockups/account_verification.png} }}%
    \caption{Autenticação - Ativação e Confirmação de conta}%
    \label{fig:17}%
\end{figure}

\newpage

\subsection{Página inicial fórum}

O utilizador assim que se dirige ao fórum entrará na página inicial do mesmo, esta página permite navegar 
entre as diferentes listagens de tópicos acessíveis ao utilizador, aceder à página de pesquisas, filtrar a 
listagem atual por tipo de tópico e caso um técnico entre nesta página ele poderá também criar um novo 
tópico.

\begin{figure}[htb]
    \centering
    \includegraphics[width=0.5\textwidth]{images/mockups/forum_home.png}
    \caption{Página inicial do fórum}
    \label{fig:18}
\end{figure}

\newpage

\subsection{Página de detalhes de um tópico}

Assim que o utilizador seleciona um tópico ele será encaminhado para a página de detalhes de um tópico, onde 
é indicado o nome do dono do tópico, a sua imagem de perfil, a hora de criação do mesmo, a quantidade de 
gostos, o título, descrição, imagens e comentários do tópico. Nesta página o utilizador poderá visualizar 
todas as respostas. 
Por fim o técnico consegue, além disso, gostar do tópico, gostar de respostas, responder ao tópico e a outras respostas.

Se o tópico for do técnico que está a visualizar o mesmo, este poderá também concluir, eliminar o tópico e alterar 
a sua visibilidade.

\begin{figure}[htb]%
    \centering
    \subfloat[\centering Página de detalhes de um tópico]{{\includegraphics[width=0.4\textwidth]{images/mockups/topic_not_user.png} }}%
    \qquad
    \subfloat[\centering Página de detalhes de um tópico do técnico]{{\includegraphics[width=0.4\textwidth]{images/mockups/user_topic.png} }}%
    \caption{Página de detalhes de tópico do software}%
    \label{fig:20}%
\end{figure}

\newpage

\subsection{Página de criação de um tópico}

Quando um técnico inicia a criação de um tópico, este será direcionado para a página de criação de tópico, 
nesta página o técnico poderá inserir o título e descrição do tópico, assim como indicar a visibilidade do 
tópico, o tipo de tópico, o produto referente e por fim o técnico poderá anexar imagens. 
A qualquer momento o técnico poderá cancelar ou confirmar a criação do tópico.

\begin{figure}[htb]
    \centering
    \includegraphics[width=0.5\textwidth]{images/mockups/forum_create_topic.png}
    \caption{Página de criação de tópico}
    \label{fig:21}
\end{figure}

\subsection{Página de notificações}

Um técnico sempre que desejar poderá visualizar as suas notificações, para isso deverá se dirigir à página
de notificações, neste é possível ver todas as notificações identificando quem enviou, qual a descrição da 
notificação e data de receção desta. O técnico poderá também apagar a notificação se assim desejar.

\begin{figure}[htb]
    \centering
    \includegraphics[width=0.5\textwidth]{images/mockups/notifications.png}
    \caption{Página de notificações}
    \label{fig:22}
\end{figure}

\newpage

\subsection{Página de perfil de utilizador}

O técnico sempre que desejar alterar as suas informações, deverá se deslocar ao seu perfil no qual é possível
alterar o se nome, email e imagem de perfil. Neste é possível também configurar as notificações indicando os métodos
a receber notificação e o tipo de notificação a receber para cada método selecionado.

Caso uma empresa entre no perfil esta visualizará um botão para aceder à gestão de recursos humanos.

\begin{figure}[htb]
    \centering
    \includegraphics[width=0.5\textwidth]{images/mockups/user_profile.png}
    \caption{Página de perfil de utilizador}
    \label{fig:23}
\end{figure}

\subsection{Página de gestão de recursos humanos}

Sempre que é necessário registar novas contas de técnico a empresa deverá direcionar-se à página de gestão
de recursos humanos, nesta página esta poderá registar novos técnicos ou gerir os técnicos já registados,
conseguindo aceder aos seus perfis. Por fim a empresa consegue também pesquisar por técnico.

\begin{figure}[htb]
    \centering
    \includegraphics[width=0.5\textwidth]{images/mockups/human_resources.png}
    \caption{Página de gestão de recursos humanos}
    \label{fig:24}
\end{figure}

\newpage

\subsection{Página de perfil de técnico registado}

Para ver as estatisticas de um técnico ou para realizar alguma operação sobre este, a empresa deverá
clicar no técnico desejado na página de recursos humanos, onde será direcionada para a página de perfil do
técnico e visualizará as estatisticas deste, assim como também as suas informações e tem a possibilidade
de impedir acesso à conta ou então remover a conta da plataforma.

\begin{figure}[htb]
    \centering
    \includegraphics[width=0.5\textwidth]{images/mockups/professional_profile.png}
    \caption{Página de perfil de técnico registado}
    \label{fig:24}
\end{figure}

\subsection{Página de registo de novo técnico}

Sempre que uma empresa deseja registar um novo técnico esta deverá clicar em registar novo técnico onde será
direcionada para a página de registo de técnicos, nesta página é pedido o número de contribuinte e email do 
técnico.

\begin{figure}[htb]
    \centering
    \includegraphics[width=0.5\textwidth]{images/mockups/account_registering.png}
    \caption{Página de registo de novo técnico}
    \label{fig:24}
\end{figure}

% \section{Diagramas de atividades}
De forma a ser possível detalhar de forma simples as ações do ator nos diferentes ecrãs foram 
desenvolvidos diagramas de atividades, com isto 

\subsection{Diagrama de atividades página inicial}

Na página inicial da aplicação é possível se deslocar para o fórum, ver as notificações, realizar operações
de catálogo e deslocar-se para o perfil.

\begin{figure}[htb]
    \centering
    \includegraphics[width=0.6\textwidth]{images/diagramas/atividades/diagrama_atividades_home.png}
    \caption{Diagrama de atividades de página inicial da aplicação}
    \label{fig:34}
\end{figure}

\subsection{Diagrama de atividades página de perfil}

Um técnico poderá necessitar de alterar alguma configuração ou informação sua, pelo que deverá se dirigir
ao ecrã de perfil. Neste ecrã, este poderá altear a sua image de perfil, o seu nome e seu email. 
Além disso poderá também selecionar os métodos de notificação que deseja receber e os tipos de notificação
para estes métodos. Caso uma empresa veja o seu perfil este poderá além das operações acima mencionadas
gerir os seus recursos humanos sendo encaminhada para o ecrã de gestão de recursos humanos.

\begin{figure}[htb]
    \centering
    \includegraphics[width=\textwidth]{images/diagramas/atividades/diagrama_atividades_perfil.png}
    \caption{Diagrama de atividades de página de perfil}
    \label{fig:35}
\end{figure}

\newpage

\subsection{Diagrama de atividades página inicial do fórum}

Na página inicial do fórum o utilizador poderá então selecionar um dos tipos de pesquisa, escrita 
ou código QR, filtrar por tipo de tópico ou selecionar tópico. Este conseguirá também ver as 
listagens de tópicos em destaque, tópicos mais recentes e por fim tópicos por 
responder, estas listas poderão ser também filtradas por tipo, conseguindo o 
técnico depois realizar todas as ações novamente, sobre estas listas o utilizador também poderá 
selecionar um tópico o que o redirecionará para o ecrã de detalhes de tópico. Para além das 
destas operações, o técnico consegue também ver os seus tópicos e criar um novo tópico.

\begin{figure}[htb]
    \centering
    \includegraphics[width=\textwidth]{images/diagramas/atividades/diagrama_atividades_forum.png}
    \caption{Diagrama de atividades de página inicial do fórum}
    \label{fig:36}
\end{figure}



\newpage

\subsection{Diagrama de atividades página de criação de tópico}

Quando o técnico decide criar um tópico, este é encaminhado para o ecrã de criação de tópico 
onde este obrigatoriamente tem de indicar o título, descrição e tipo do tópico, por predefinição a 
visibilidade deste é pública, mas o técnico poderá alterar esta visibilidade também. 
Facultativamente o técnico poderá indicar o produto referente ao tópico, 
assim como anexar imagens, conseguindo também remover estas. A qualquer momento o técnico poderá 
confirmar a criação do tópico, quando esta ação inicia, é verificado se o título e descrição estão 
preenchidos, caso estes dados não estejam preenchidos é indicado que estes dados estão em falta, 
caso contrário este volta para o ecrã anterior.

\begin{figure}[htb]
    \centering
    \includegraphics[width=0.7\textwidth]{images/diagramas/atividades/diagrama_atividades_criar_tópico.png}
    \caption{Diagrama de atividades de página de criação de tópico}
    \label{fig:37}
\end{figure}

\newpage

\subsection{Diagrama de atividades página de detalhes do tópico}

Assim que o utilizador seleciona um tópico este é redirecionado para o ecrã de detalhes de tópico, 
no qual este poderá visualizar todas as respostas e ver as imagens anexadas em ponto grande.
Já o técnico poderá, além disso, apagar um comentário caso seja seu, gostar do tópico e de uma resposta, 
comentar, responder a um comentário. Caso o tópico seja do técnico, este poderá também alterar a 
visibilidade do tópico, marcar como concluído ou remover este voltando para o ecrã anterior. 
A qualquer momento o técnico poderá também retroceder para o ecrã anterior.

\begin{figure}[htb]
    \centering
    \includegraphics[width=0.9\textwidth]{images/diagramas/atividades/diagrama_atividades_detalhes_topico.png}
    \caption{Diagrama de atividades de página de detalhes do tópico}
    \label{fig:38}
\end{figure}

\newpage

\subsection{Diagrama de atividades páginas de autenticação}

Para realizar a ativação da conta de técnico, assim que este realiza o registo, confirmação de conta ou 
o login com uma conta que não se encontra ativada, este é encaminhado para o ecrã de 
ativação de conta, neste ecrã este poderá cancelar a ativação de conta, ou então indicar o código 
de ativação de conta, caso este código esteja errado, o técnico deverá inserir novamente o código, 
caso seja um código correto o técnico validará a sua conta e ficará autenticado. O técnico poderá 
também em caso de necessidade pedir o envio de um novo código de ativação de conta.

\begin{figure}[htb]
    \centering
    \includegraphics[width=0.8\textwidth]{images/diagramas/atividades/diagrama_atividades_autenticação.png}
    \caption{Diagrama de atividades de página de validação de conta}
    \label{fig:39}
\end{figure}

\newpage

% \subsection{Diagrama de atividades página de notificações}

% Sempre que o técnico recebe uma notificação, este poderá ver esta notificação no ecrã de notificações, 
% Este ecrã permite ao utilizador selecionar uma notificação sendo redirecionado para o tópico referente,
% caso esta esteja referente a um tópico, ou então poderá apagar a notificação.

% \begin{figure}[htb]
%     \centering
%     \includegraphics[width=0.5\textwidth]{images/diagramas/atividades/diagrama_atividades_noti.png}
%     \caption{Diagrama de atividades de página de notificações}
%     \label{fig:27}
% \end{figure}

% \subsection{Diagrama de atividades gestão de recursos humanos}

% Uma empresa poderá gerir as contas dos seus recursos humanos, para isso deverá se dirigir a este ecrã.
% Neste ecrã é possível registar um novo técnico sendo encaminhada para o ecrã de registo de técnico, 
% selecionar um técnico sendo encaminhada para o ecrã de perfil do técnico e poderá também pesquisar por 
% técnico.

% \begin{figure}[htb]
%     \centering
%     \includegraphics[width=\textwidth]{images/diagramas/atividades/diagrama_atividades_human_resources.png}
%     \caption{Diagrama de atividades de página de recursos humanos}
%     \label{fig:29}
% \end{figure}


% \subsection{Diagrama de atividades perfil de técnico}

% Sempre que uma empresa seleciona um técnico, esta é encaminhada para o ecrã de perfil de técnico. Neste 
% ecrã é possível impedir acesso à plataforma e remover a conta de técnico da plataforma.

% \begin{figure}[htb]
%     \centering
%     \includegraphics[width=0.5\textwidth]{images/diagramas/atividades/diagrama_atividades_prof_profile.png}
%     \caption{Diagrama de atividades de página de perfil de técnico}
%     \label{fig:30}
% \end{figure}

\newpage

\subsection{Diagrama de atividades registar técnico}

Assim que uma empresa inicia o registo de um técnico esta é redirecionada para a página de registo de técnico.
Nesta página esta terá de indicar o número de contribuinte e email do técnico, por fim poderá confirmar o registo
de conta sendo redirecionada para a página anterior.

\begin{figure}[htb]
    \centering
    \includegraphics[width=0.8\textwidth]{images/diagramas/atividades/diagrama_atividades_add_professional.png}
    \caption{Diagrama de atividades de página de registar técnico}
    \label{fig:31}
\end{figure}

\subsection{Diagrama de atividades confirmar conta}

Quando uma conta de técnico é registada um email de confirmação é enviado para o técnico, assim que este
recebe o email deverá clicar em confirmar a conta sendo redirecionado para a página de confirmação de conta.
Nesta página o técnico poderá alterar o seu email, indicar o seu nome, indicar a password e confirmar esta, 
finalizando quando decidir se registar.

\begin{figure}[htb]
    \centering
    \includegraphics[width=0.8\textwidth]{images/diagramas/atividades/diagrama_atividades_prof_register.png}
    \caption{Diagrama de atividades de página de confirmar conta de técnico}
    \label{fig:31}
\end{figure}

\section{Diagramas de estados}
Para especificar os principais processos do projeto foram desenvolvidos diagramas de estados, com o objetivo de demonstrar o processo de criação de um tópico do fórum por parte de um técnico, o processo de aceder e responder a um tópico e o login com ativação de conta, visto que, estas interações são as de maior significância e regradas no \textit{software}.

\subsection{Diagrama de estados criação de tópico}

Com o diagrama de estados de criação de tópico é pretendido demonstrar o processo por parte de um técnico. Assim sendo, primeiramente terá de estar autenticado, caso não esteja, será encaminhado para autenticação. De seguida criará um tópico. Após preencher os campos desejados este poderá confirmar. Caso confirme, é verificado se o tópico possui título. Caso não possua, é determinado como inválido o que levará o técnico a preencher os dados em falta. Contudo, se o título estiver preenchido, é verificado se possui descrição e tipo. Na condição de não possuir descrição ou tipo, é seguido o mesmo fluxo que o caso anterior. Caso contrário é criado um novo tópico. Por fim, na hipótese de o técnico não desejar confirmar o tópico, este poderá cancela-lo, o que o determina como cancelado.

\begin{figure}[htb]
  \centering
  \includegraphics[width=0.9\textwidth]{images/diagramas/estados/criar_topico.png}
  \caption{Diagrama de estados de criar tópico}
  \label{fig:40}
\end{figure}

\newpage

\subsection{Diagrama de estados responder a tópico}

Com o diagrama de estados de responder a tópico é pretendido demonstrar o processo de seleção e de responder a um tópico por parte de um técnico. Assim sendo, primeiramente terá de estar autenticado, caso não esteja, será encaminhado para autenticação. Após a autenticação, o técnico irá por predefinição ver os tópicos em destaque. Nesta listagem selecionará um tópico o que permite responder. Depois da criação do comentário, este poderá confirmar e caso confirme ficará criado, caso contrário ficará cancelado.

\begin{figure}[htb]
  \centering
  \includegraphics[width=0.7\textwidth]{images/diagramas/estados/responder_topico_tecnico.png}
  \caption{Diagrama de estados de criar tópico}
  \label{fig:41}
\end{figure}

\newpage

\subsection{Diagrama de estados autenticação e validação de conta}

Aquando a realização do login, o técnico indicará as suas credenciais. Todavia, se não estiverem corretas, a autenticação será determinada como incorreta. Caso as credenciais estejam corretas e a conta válida, o técnico ficará automáticamente autenticado. Contudo, na condição de não conter uma conta válida, esta deverá ser validada e para isso, este terá de inserir o código de validação. No caso de o código estar correto, a conta será validada e o técnico ficará autenticado. Caso contrário o código será inválido e este necessitará indicar o código de validação novamente.

\begin{figure}[htb]
  \centering
  \includegraphics[width=0.7\textwidth]{images/diagramas/estados/autenticacao.png}
  \caption{Diagrama de estados de autenticação e validação de conta}
  \label{fig:42}
\end{figure}

% \newpage

% \newpage

\section{Diagrama de sequência}

Visto que a realização da autenticação, ativação e confirmação de conta requer passos extras e regras a seguir, 
foi necessário criar diagramas de sequência para especificar a sequência de interações do com o sistema.

\subsection{Diagrama de sequência Login e ativação de conta}

Através deste diagrama (Figura~\ref{fig:43}) é entende-se que assim que o técnico deseja realizar o 
login, primeiramente tem de verificar as credenciais, caso estas se encontrem incorretas, este receberá 
uma mensagem de erro, caso as credenciais estejam válidas e a conta esteja ativada o técnico ficará 
autenticado. 

Caso o técnico coloque as credenciais corretas, mas a conta não esteja ativada, este irá realizar a 
ativação de conta, onde poderá enviar o código de ativação, caso esteja correto a sua conta será 
ativada, caso contrário este receberá uma mensagem de erro. Este poderá também cancelar a 
ativação de conta e pedir um novo \textit{email} de ativação, onde será pedido novo código ao servidor, 
este será gerado e enviado.

\begin{figure}[htb]
    \centering
    \includegraphics[width=0.67\textwidth]{images/diagramas/sequencia/diagrama_login.png}
    \caption{Diagrama de sequência de login e ativação de conta}
    \label{fig:43}
\end{figure}

\newpage

\subsection{Diagrama de sequência Registo e ativação de conta}

Através do diagrama abaixo representado (Figura~\ref{fig:44}) é possível perceber que quando uma
empresa realiza o registo este será enviado para o servidor, o qual registará a empresa com uma 
conta não ativada, esta conta será então validada pela Motorline sendo de seguida
gerado um código de ativação e enviado por \textit{email} para o 
\textit{email} de registo, após isto a empresa será encaminhado para a validação de conta, esta validação 
ocorre seguindo o mesmo processo mencionado no anteriormente.


\begin{figure}[htb]
    \centering
    \includegraphics[width=0.8\textwidth]{images/diagramas/sequencia/diagrama_registo.png}
    \caption{Diagrama de sequência de registo e validação de conta}
    \label{fig:44}
\end{figure}

\newpage

\subsection{Diagrama de sequência registo de técnicos}

Através do diagrama abaixo representado (Figura~\ref{fig:45}) é possível perceber que quando uma
empresa deseja registar um técnico, esta introduzirá os seus dados, sendo a sua conta criada. Após isto, 
um código de ativação é gerado e enviado para o técnico ativar a sua conta.

\begin{figure}[htb]
    \centering
    \includegraphics[width=0.8\textwidth]{images/diagramas/sequencia/registo_tecnico.png}
    \caption{Diagrama de sequência de registo de técnicos}
    \label{fig:45}
\end{figure}
% \chapter{Arquitetura de sistema}
Na Figura~\ref{fig:46} é possível visualizar a arquitetura do sistema que indica os principais 
componentes deste software. Entre estes componentes é possível visualizar a aplicação frontend, 
onde esta realiza pedidos a uma aplicação backend e espera respostas. A aplicação backend é composta 
de uma api rest que receberá os pedidos e responderá aos mesmos, este backend é composto também por 
uma base de dados a qual vai receber queries e devolver dados para a aplicação api rest.

\begin{figure}[htb]
    \centering
    
    \includegraphics[width=\textwidth]{images/Arquiteturas/arquitetura_de_solucao.png}
    \caption{Arquitetura do sistema}
    \label{fig:46}
\end{figure}

\newpage

\section{Arquitetura de funcional}

Para especificar a implementação da api rest foi então criada uma arquitetura de 
backend(Figura~\ref{fig:47}), nesta arquitetura é possível visualizar que sempre que a api 
recebe um request este é redirecionado primeiramente para o router, o router tem como função 
identificar a rota a ser pedida e redirecionar para os respetivos middlewares. 

Os middlewares tem como função realizar todo o código necessário antes de proceder à execução 
do código de rota, os middlewares existentes são o SessionTokenValidator, este middleware tem 
como função validar a sessão do utilizador a realizar o pedido, de forma similar o middleware 
RefreshTokenValidator, valida a sessão principal do utilizador, por fim o middleware RoleValidator, 
tem como função validar se o utilizador que realiza o pedido tem cargos suficientes . 
Caso o pedido não seja impedido por nenhum middleware este é então direcionado para o 
controller.

O controller tem como função principal extrair os dados do pedido, validar os dados, verificando 
se os dados obrigatórios existem e encaminhar o pedido para o serviço, procedendo depois à formação 
da resposta e devolução da mesma. No serviço serão primeiramente aplicadas as regras de negócio para 
validar o conteúdo do pedido, caso o pedido não seja impedido por nenhuma das validações de 
regras de negócio, este então, em caso de necessidade, irá proceder à interação com base de dados, 
podendo esta realizar diversas interações como, obter dados, atualizar dados, apagar dados e inserir 
dados. Por fim a resposta é formada e devolvida como resposta ao pedido recebido.

\begin{figure}[htb]
    \centering
    \includegraphics[width=\textwidth]{images/Arquiteturas/arquitetura_funcional.png}
    \caption{Arquitetura do funcional}
    \label{fig:47}
\end{figure}

\newpage

\section{Arquitetura de componentes}
Após a perceção de todas as necessidades da aplicação do frontend, foi então desenvolvida a 
arquitetura de componentes na qual estão contidos todos os serviços que deverão ser implementados na 
api frontend, identificando também qual ator poderá realizar estes pedidos.

\begin{figure}[htb]
    \centering
    \includegraphics[width=0.7\textwidth]{images/Arquiteturas/arquitetura_de_componentes_final.png}
    \caption{Arquitetura de componentes}
    \label{fig:48}
\end{figure}

\section{Tabela de endpoints}
De forma a evitar colisões de endpoints durante a implementação dos mesmos, foi então desenvolvida a 
tabela de endpoints que contém uma estrutura semelhante à arquitetura de componentes, mas que contém 
para cada serviço a rota e o método a utilizar.

% \usepackage{color}
% \usepackage{tabularray}
\definecolor{Concrete}{rgb}{0.952,0.952,0.952}
\begin{longtblr}
[
caption={Tabela de endpoints},
label={tab:19},
]{
  row{1} = {Concrete,c},
  hlines,
  vlines,
}
Serviço                                    & Ator       & Rota                                                             & Método \\
{Obter informações \\do utilizador}        & Cliente    & baseurl/client/:uid                                              & GET    \\
Realizar login                             & Utilizador & baseurl/login                                                    & POST   \\
Realizar registo                           & Utilizador & baseurl/register                                                 & POST   \\
{Esquecimento de \\password}               & Utilizador & baseurl/forgot-password                                          & GET    \\
Ativação de conta                          & Cliente & baseurl/client/:uid/activate                                     & POST   \\
{Reenvio de código \\de ativação de conta} & Cliente & baseurl/client/:uid/new-code                                     & GET    \\
{Obter tópicos em \\destaque}              & Cliente    & baseurl/client/topics/featured                                   & GET    \\
{Obter tópicos mais \\recentes}            & Cliente    & baseurl/client/topics/latest                                     & GET    \\
{Obter tópicos por \\responder}            & Cliente    & baseurl/client/topics/to-answer                                  & GET    \\
{Obter tópicos do \\utilizador}            & Cliente    & baseurl/client/topics                                            & GET    \\
{Obter tópicos \\privados}                 & Técnico    & baseurl/professional/topics/private                              & GET    \\
{Gostar de um \\tópico}                    & Cliente    & baseurl/client/topics/:topicId/like                              & PUT    \\
{Adicionar resposta \\a tópico}            & Cliente    & baseurl/client/topics/:topicId/answer                            & POST   \\
{Adicionar resposta \\a outra resposta}    & Cliente    & baseurl/client/answers/:answerId/                                & POST   \\
{Gostar de uma \\resposta}                  & Cliente    & baseurl/client/answers/:answerId/like                            & PUT    \\
{Marcar tópico \\como completo}            & Cliente    & baseurl/client/topics/:topicId/completed                         & PUT    \\
Remover Tópico                             & Cliente    & baseurl/client/topics/:topicId/                                  & DELETE \\
{Alterar visibilidade \\do tópico}         & Cliente    & baseurl/client/topics/:topicId/visibility                        & PUT    \\
{Adicionar melhor \\resposta do tópico}    & Cliente    & {baseurl/client/topics/:topicId/answers/\\:answerId/best-answer} & PUT    \\
Adicionar novo tópico                      & Cliente    & baseurl/client/topics/                                           & POST   
\end{longtblr}

\newpage

\section{Web scraper}

Após uma reunião com o cliente foi percebido que o catálogo de produtos Motorline não se encontra em um
servidor, esta informação encontra-se apenas diretamente no website da empresa, sendo assim viu-se a 
necessidade de criar um web scraper.

Web scraping é uma terminologia dada para o processo de obter uma página web, ler a página e obter
dados desta, geralmente utilizando bots. O grande problema com web scraping é que pode ser facilmente
detetado. Tendo em conta este problema surgiram duas grandes formas principais de realizar web scraping,
a mais comum sendo realizar um pedido para obter uma página web e ler então esta, sendo assim um processo
rápido e simples. A segunda forma de realizar web scraping é através da simulação da ação humana 
conseguindo abrir o navegador pesquisar pela página desejada, descarregar a página e daí ler esta, 
tornando-se então em um processo lento e complexo.A grande diferença entre estas duas formas é a 
velocidade, visto que a segunda forma tem de esperar que o navegador inicie, de seguida terá de 
esperar que a página carregue e apenas após este processo poderá ser lida a página web.

Na reunião mencionada anteriormente foi decidido que o web scraper iria apenas correr 1 vez por mês
de forma a evitar a sobrelotação do servidor, não existindo problema visto que o catálogo não é 
atualizado regularmente. Para agilizar a realização do web scraper foi disponibilidado pela 
empresa a estrutura do website a seguir para obter as informações da página web.

\subsection{Implemenção web scraper}
De forma a implementar e testar o web scraper sem sobrelotar o servidor, foi então descarregado todo
o wesite localmente, conseguindo assim simular o mesmo.

Para implementar o web scraper foi optado pela abordagem mais simples, realizar um pedido para obter a
pagina web, ler a página para obter os dados e guardar os dados.

Para isto foi optado pela linguagem python devido à facilidade desta lidar com grandes quantidades 
de dados. De forma a facilitar a localização dos dados na página foi utilizada a biblioteca bs4, também
conhecida como beautiful soup, esta biblioteca permite alimentar com uma página web e de seguida realizar
pesquisas sobre esta página baseado em tags e atributos dos elementos.

Tendo esta base em conta foi então primeiramente estudado que dados seriam necessários, sendo estes então:
\begin{enumerate}
    \item Categorias e subcategorias de produtos;
    \item Produtos de cada categoria e subcategoria;
    \item Documentação dos produtos;
    \item Imagens e videos dos produtos;
\end{enumerate}

\newpage

Para guardar estes dados foi utilizado um dicionário que contém primeiramente como chaves as categorias de produtos,
para cada categoria contém mais um dicionário com as subcategorias de produtos e para cada categoria existe uma lista
de produtos, contendo o nome de produto, imagem de amostra e url do mesmo.Por fim a chave produtos contém a lista de 
todos os produtos, sendo cada produto representado também por um dicionário, que contém como chaves os atributos do mesmo.
A utilização dicionarios e listas para guardar estes dados deve-se a que o objetivo será guardar estes dados 
em json e a transformação é simplificada utilizando estas estruturas devido à sua proximidade com a estrutura
json.

\begin{figure}[htb]
    \centering
    
    \includegraphics[width=0.7\textwidth]{images/implementacao/scraper/estrutura_scraper.png}
    \caption{Estrutura dos dados obtidos}
    \label{fig:49}
\end{figure}

\newpage

Após uma análise da estrutura do website foi percebido que a página geral de produtos possui todas as categorias
de produtos, assim como também as subcategorias de produtos com urls para as páginas que contém todos os produtos
das subcategorias. Sendo assim foi primeiramente percebido que cada conjunto é uma secção, pelo que é obtido
todas as secções de categorias e para cada uma destas secções é obtido o título da secção que equivale ao nome da
categoria e também todos os correspondentes a clicáveis. Os clicáveis corresponde às subcategorias de cada categoria
estes clicáveis contém também um url que redireciona para a página de produtos da subcategoria.

\begin{figure}[htb]
    \centering
    
    \includegraphics[width=0.55\textwidth]{images/implementacao/scraper/pagina_geral_produtos.png}
    \caption{Página geral de produtos}
    \label{fig:50}
\end{figure}

Sendo assim já é possível identificar cada categoria e subcategoria, assim como também o url da página de produtos
para cada subcategoria. Mas após alguma análise dos dados foi percebido que estas não contem acentuação devido à 
sua formatação no website. Para resolver este problema foi pesquisado por ferramentas capazes de corrigir estes
erros ortográficos. Pelo que foi descoberto que a biblioteca mais utilizada em python para resolver este problema 
é a biblioteca spellchecker, esta ferramenta é a mais utilizada devido à sua capacidade de corrigir erros ortográficos
em diversas linguagens. Sendo assim sempre que uma categoria e subcategoria é obtida, antes de ser guardada, esta é corrigida.

Após isto cada url é aberto e são obtidos os urls de produtos e imagens de amostra dos produtos, para isto foi obtido todos os
elementos clicáveis existentes na secção de produtos de cada página, sendo que cada um correponde a um produto, para obter o nome
do produto correspondente foi utilizado o nome contido no url da página de produto, sendo que todos os produtos seguem a mesma 
estrutura, sendo esta, /produtos/nome-produto. Sendo que em urls não é permitido utilizar acentuação e espaços, então todos os 
nomes foram corrigidos utilizando a mesma ferramenta mencionada anteriormente.

\begin{figure}[htb]
    \centering
    
    \includegraphics[width=0.55\textwidth]{images/implementacao/scraper/pagina_produtos_subcat.png}
    \caption{Página de produtos de uma subcategoria}
    \label{fig:51}
\end{figure}

\newpage
Neste momento após correr o código foi percebido que existiam algumas páginas de produtos em que este não conseguia obter produtos,
pelo que um erro era atirado, para perceber exatamente que páginas de produtos este erro acontecia, sempre que um erro era detetado
este url seria adicionado a uma nova chave do dicionario mencionado anteriormente, esta chave tem o nome misses e contém todos os urls
em que algum erro aconteceu. Foi então neste momento que foi percebido que nem todas as páginas de produtos são iguais e após uma
reunião com o cliente este expôs que existem páginas de produtos e de detalhes de produtos que são muito diferentes das restantes.

\begin{figure}[htb]
    \centering
    
    \includegraphics[width=0.7\textwidth]{images/implementacao/scraper/pagina_mconnect.png}
    \caption{Página de produtos de uma subcategoria distinta}
    \label{fig:52}
\end{figure}

De forma ao restante do projeto não ser atrasado foi então decidido primeiramente obter todos os produtos que contêm páginas semelhantes,
sendo assim para cada página de produto foi obtido o titulo que corresponde ao nome do produto, de seguida foi obtida a descrição do produto,
o elemento que contém esta tem como id produto-descrição. As imagens dos produtos são disponibilizadas através de urls na secção da galeria do produto, sendo assim são obtidas todas as imagens desta
galeria e de seguida todos os seus urls.

A documentação dos produtos pode ser disponibilizada através de urls para os manuais,
ou com uma lista dropdown com todos os manuais disponiveis para download, sendo assim são obtidos todos os urls da secção de documentação, 
assim como os seus nomes e todas as opções de documentação do dropdown se este existir.

\begin{figure}[htb]
    \centering
    
    \includegraphics[width=0.7\textwidth]{images/implementacao/scraper/pagina_detalhes_produto.png}
    \caption{Página de detalhes de produto, secção inicial}
    \label{fig:53}
\end{figure}

\newpage

As imagens de desenho técnico e informação geral estão disponibilizadas na secção correspondente
ao nome de cada uma, sendo assim obtidas estas secções e caso estas existam são obtidas as imagens e os seus urls.

\begin{figure}[htb]
    \centering
    
    \includegraphics[width=0.7\textwidth]{images/implementacao/scraper/pagina_detalhes_desenhos.png}
    \caption{Página de detalhes de produto, secção de informações}
    \label{fig:54}
\end{figure}

Os videos de produtos estão disponiveis na secção de videos, sendo que cada secção de videos contém o nome do video e por sua vez o video.
Estes videos são demonstrados utilizando um elemento iframe, este elemento contém um url para o video, mas após tentar visualizar este url,
foi percebido que não é possivel obter o vídeo a partir deste. Sendo assim foi investigada a plataforma vimeo, esta é a plataforma que contém
todos os videos de produtos, pelo que para cada um é gerado um id unico e este poderá ser acedido através do url geral da plataforma seguido 
do id do video. Este id está também colocado no elemento iframe, pelo que este é obtido e acrescentado ao url da plataforma conseguindo assim
guardar todos os videos de produtos.

\begin{figure}[htb]
    \centering
    
    \includegraphics[width=0.7\textwidth]{images/implementacao/scraper/pagina_detalhes_videos.png}
    \caption{Página de detalhes de produto, secção de videos}
    \label{fig:55}
\end{figure}

\newpage

\subsubsection{Implemenção no website}

Após se verificar que eram obtidos pelomenos 80\% dos produtos totais foi então decidido testar no website. Para isto foi utilizada a biblioteca
requests, com a qual é realizado um pedido get a cada url necessário para se obter a página web. Assim que o código foi corrido e a resposta analisada
foi percebido que o website bloqueia este tipo de solução recebendo a resposta demonstrada pela figura~\ref{fig:56}

\begin{figure}[htb]
    \centering
    
    \includegraphics[width=0.7\textwidth]{images/implementacao/scraper/forbiden_response.png}
    \caption{Resposta obtida aquando o pedido ao url da página web}
    \label{fig:56}
\end{figure}

Através da resposta obtida foi então percebido que seria necessário alterar a abordagem visto que a abordagem anterior não seria possível utilizar.
A abordagem opcional a seguir seria simular a ação humana abrindo um navegador e pesquisando pelo url desejado. 

Após uma investigação foi descoberto que existem ferramentas que permitem controlar o dispositivo onde correm impedindo a utilização deste enquanto se 
encontram a correr, assim como ferramentas que apenas recebem o navegador a utilizar e abrem uma nova janela deste navegador para realizar a pesquisa.
Visto que o processo de obter os produtos seria demorado, foi optado pela segunda opção visto que seria possivel continuar com trabalho em paralelo com a
obtenção de dados. Sendo assim a ferramenta mais recomendada para realizar esta operação é a biblioteca selenium, esta biblioteca permite realizar exatamente
o processo referido anteriormente com a possibilidade de escalar com multi threading, permitindo abrir diversas janelas do navegador simultaneamente,
diminuindo drasticamente o tempo de execução para obter os dados, esta funcionalidade não foi explorada devido a limitações de hardware, 
mas seria uma importante implementação futura.

Utilizando a biblioteca selenium foi primeiramente indicado qual o navegador a utilizar, neste caso foi escolhido o chrome devido a este já estar instalado
no dispositivo. Após se indicar qual o navegador a utilizar, é necessário para cada página indicar qual elemento esperar que carregue, pois assim que a pesquisa
é efetuada a página poderá demorar a carregar pelo quem se deverá indicar a espera pelo elemento que se deseja obter. Visto que a página carregada poderá não
conter o elemento a obter foi então implementado um tempo de espera máximo de 5 segundos, assim que este tempo expira a operação é abortada e o url é adicionado
à lista de urls com erros.

\newpage

\subsubsection{Armazenamento de dados}

Após se obter os dados dos produtos, é necessário guardar estes na base de dados para disponibilizar para a sua utilização no backend. Para realizar
esta operação existem duas opções, criar um serviço para inserir produtos e realizar um pedido a este serviço, ou então conectar diretamente
o web scraper à base de dados. Visto que não seria de grande interesse conectar diretamente à base de dados, foi decidido criar um serviço que recebe um produto e o 
insere na base de dados. O grande problema que surgiu com esta solução é que os pedidos ocorrem de forma sequencial, mas com pouco tempo de espera entre estes, o que
levava a que o limite máximo de conexões com a base de dados fosse extrapolado. Isto acontece porque para cada serviço chamado é criado uma nova conexão à base de 
dados, todas as operações são realizadas e por fim a conexão é terminada, mas enquanto estas operações estão a decorrer, o servidor poderá receber mais pedidos, o que 
leva a que mais conexões sejam criadas, atingindo assim rápidamente o limite de conexões da base de dados. Como solução para este problema surgiu a ideia de receber todos
os produtos a inserir em apenas um pedido e inserir estes na base de dados.

\subsubsection{Melhoria de implementação}

Após se completar o processo foi então decidido melhorar a implementação resolvendo os erros encontrados em urls especificos. De forma a perceber exatamente quais os urls
que possuem erros foi então direcionado os dados obtidos para um ficheiro json. Sendo assim os urls com erros eram os indicados na figura~\ref{fig:57}.

\begin{figure}[htb]
    \centering
    
    \includegraphics[width=0.7\textwidth]{images/implementacao/scraper/urls_erro_iteracao_1.png}
    \caption{Urls com erro primeira interação}
    \label{fig:57}
\end{figure}

Após uma primeira análise foi possível perceber que grande maioria dos erros provém de urls de acessórios de produtos, isto deve-se ao facto de os acessórios de produtos encontrarem-se
na página de produtos de subcategoria e serem tratados como um url de detalhes de produto, sendo assim sempre que se trata de um url de acessórios seria necessário correr código para 
obter dados de destes ao invés de detalhes de produtos. Para desenvolver este código foi primeiramente analisada a página de acessórios de produtos(Figura~\ref*{fig:58}), 
esta página contém para cada acessório um elemento do tipo artigo o qual contém uma imagem, titulo e descrição, esta descrição por vezes contém urls para os produtos aos quais este acessório se refere, pelo que sempre que estes
urls são detetados, os nomes dos produtos são guardados para futuramente realizar a ligação entre os acessórios e os produtos, visto que não existem produtos com nomes iguais. Foi percebido que nos urls a palavra accessórios está 
sempre contida pelo que sempre que esta é detetada em um url é corrido o código referente à obtenção de acessórios.

\begin{figure}[htb]
    \centering
    
    \includegraphics[width=0.7\textwidth]{images/implementacao/scraper/pagina_acessorios.png}
    \caption{Exemplo de página de acessórios}
    \label{fig:58}
\end{figure}

\newpage

Após correr o novo código criado foi percebido que a quantidade de urls com erros diminuiu, mas existiam produtos com páginas de detalhes de produtos comuns pelo que estas foram analisadas e foi percebido
que um erro ocorria devido a por vezes as páginas não conterem vídeos ou imagens de documentação, sendo que o código foi alterado para apenas obter estes dados se os elementos existirem na página. 
Após correr novamente o código foi percebido que a quantidade de falhas obtidas diminuiu drásticamente (Figura~\ref{fig:59}). Mas mesmo assim ainda existiam 3 falhas a ocorrer e após uma análise foi percebido que
estas falhas estavam a ocorrer devido a:

\begin{enumerate}
    \item Uma página de subcategoria de produtos conter um serviço;
    \item Um produto conter uma página de detalhes de produto com sub produtos;
    \item Existir uma página de adaptadores de produtos;
    \item Um produto conter uma página de detalhes diferente das demais;
\end{enumerate}

\begin{figure}[htb]
    \centering
    
    \includegraphics[width=0.7\textwidth]{images/implementacao/scraper/melhor_corrida.png}
    \caption{Exemplo de página de acessórios}
    \label{fig:59}
\end{figure}

Visto que a página de adaptadores de produtos segue uma estrutura similar à estrutura dos acessários este foi o primeiro a ser abordado e resolvido, correndo o código de obter detalhes de acessórios sempre a 
palavra acessórios ou adaptadores se encontra no url. De seguida foi percebido que para resolver o problema de existerem serviços e subprodutos o diagrama de base de entidade relação teria de ser alterado pelo que
primeiramente foi resolvido o problema do produto que contém uma página de detalhes diferente das demais. 

Este produto para além da dificuldade de ser uma página completamente diferente as informações encontram-se espalhadas pela página (Figura~\ref*{fig:60}), pelo que estas deveriam ser combinadas para construir os detalhes do produto.

\begin{figure}[htb]
    \centering
    
    \includegraphics[width=0.7\textwidth]{images/implementacao/scraper/flama.png}
    \caption{Exemplo de página de produto incomum}
    \label{fig:60}
\end{figure}
% \chapter{Implementação}



\section{Web scraper}

Após uma reunião com o cliente foi compreendido que o catálogo de produtos Motorline não se encontra num servidor. Estes dados apenas estão diretamente no \textit{website} da empresa, sendo assim, foi determinado a necessidade de criar um \textit{web scraper}.

Durante a reunião concluiu-se que o \textit{web scraper} iria apenas correr uma vez por mês para ser evitada a sobrelotação do servidor. Para agilizar a realização do \textit{web scraper} foi entregue pela empresa a estrutura do \textit{website} a seguir para serem obtidas as informações.

\subsection{Implemenção web scraper}
A implementação e testagem do \textit{web scraper}, sem sobrelotação do servidor, foi alcançada, uma vez que, préviamente foi descarregado todo o \textit{wesite} localmente, o que permitiu simular o catálogo.

Para a implementação do \textit{web scraper} optou-se por uma abordagem mais simples. Esta abordagem, consiste na realização de um pedido para ser obtida a página \textit{web}, ler e guardar os dados.

A linguagem utilizada foi \textit{Python} devido à facilidade de lidar com abundantes quantidades de dados. O tratamento dos dados das páginas \textit{web} foi realizado com o auxílio da biblioteca \textit{bs4}, também conhecida como \textit{beautiful soup}, esta permite alimentar com uma página \textit{web} e de seguida realizar pesquisas sobre esta com base em \textit{tags} e atributos dos elementos.

Após um estudo do catálogo foi compreendido que dados seriam necessários, sendo estes:
\begin{enumerate}
  \item Categorias e subcategorias de produtos;
  \item Produtos de cada categoria e subcategoria;
  \item Documentação dos produtos;
  \item Imagens e vídeos dos produtos;
\end{enumerate}

\newpage

Para guardar estes dados foi utilizado um dicionário, que contém como chaves as categorias de produtos. Para cada categoria contém mais um dicionário com as subcategorias de produtos. Cada categoria possui uma lista de produtos, sendo cada produto representado por um dicionário, que abarca como chaves os seus atributos.
A utilização de dicionários e listas para guardar estes dados deve-se a que o objetivo será guardar estes em \textit{\acrshort{json}} e a transformação é simplificada com a utilização destas estruturas devido à proximidade com a estrutura \textit{\acrshort{json}}.

\begin{figure}[htb]
  \centering
  
  \includegraphics[width=0.7\textwidth]{images/implementacao/scraper/estrutura_scraper.png}
  \caption{Estrutura dos dados obtidos}
  \label{fig:49}
\end{figure}

\newpage

Após uma análise da estrutura do \textit{website} foi constatado que a página geral dos produtos possui todas as categorias, assim como, as subcategorias com \textit{urls} para as páginas que contém todos os produtos das subcategorias. 

Sendo assim, em primeiro lugar percebeu-se que cada conjunto(categoria, subcategorias) é uma secção, pelo que, são obtidas todas as secções das categorias. Para cada uma destas secções, é obtido o título que equivale ao nome da categoria e também todos os elementos clicáveis. Estes, correspondem às subcategorias que contêm o nome e um \textit{url}, que redireciona para a página de produtos da subcategoria.

\begin{figure}[htb]
  \centering
  
  \includegraphics[width=0.55\textwidth]{images/implementacao/scraper/pagina_geral_produtos.png}
  \caption{Página geral de produtos}
  \label{fig:50}
\end{figure}

Posto isto, já é possível identificar cada categoria e subcategoria, assim como, o \textit{url} da página de produtos para cada subcategoria. Mas, após alguma análise dos dados foi percebido que estas não contêm acentuação visto que, existe uma formatação no \textit{website}. Para resolver este problema, foram pesquisadas ferramentas capazes de corrigir erros ortográficos. Pelo que, descobriu-se que a biblioteca mais utilizada em \textit{Python} para resolver este problema é a biblioteca \textit{spellchecker}. Esta ferramenta, por sua vez, é a mais utilizada dado à sua capacidade de correção dos erros ortográficos em diversas linguagens. Por fim, sempre que uma categoria e subcategoria é obtida, antes de ser guardada, é corrigida.


Aquando a finalização do processo anterior, cada \textit{url} é aberto e a partir disto, são obtidos os \textit{urls} de produtos e imagens de amostra dos produtos. Em cada página, foram obtidos todos os elementos pressionáveis existentes, sendo que cada um refere-se a um produto. Para obter o nome do produto correspondente, foi utilizado o nome contido no \textit{url} do elemento, pois todos os produtos seguem a mesma estrutura, sendo esta, /produtos/nome-produto. Como em \textit{urls} não é permitido utilizar acentuação e espaços, todos os nomes foram corrigidos com a mesma ferramenta de correção ortográfica mencionada anteriormente.

\begin{figure}[htb]
  \centering
  
  \includegraphics[width=0.55\textwidth]{images/implementacao/scraper/pagina_produtos_subcat.png}
  \caption{Página de produtos de uma subcategoria}
  \label{fig:51}
\end{figure}

\newpage
Neste momento, depois de correr o código foi deduzido que existem algumas páginas de produtos em que este não conseguia obter dados, pelo que, um erro era atirado. Para compreender exatamente em que páginas de produtos surgia este erro, sempre que um erro era detetado, este \textit{url} era adicionado a uma nova chave do dicionário mencionado anteriormente. Esta chave, tem o nome \textit{misses} e contém todos os \textit{urls} em que algum erro aconteceu. Então, nesta ocasião percebeu-se que nem todas as páginas de produtos são iguais, contudo, após uma reunião com o cliente este expôs que existem páginas de produtos e de detalhes de produtos que são muito diferentes das restantes.

\begin{figure}[htb]
  \centering
  
  \includegraphics[width=0.7\textwidth]{images/implementacao/scraper/mconnect.png}
  \caption{Página de produtos de uma subcategoria distinta}
  \label{fig:52}
\end{figure}

Com o objetivo de não atrasar o restante projeto foi determinado primeiramente, que todos os produtos que contêm páginas semelhantes deveriam ser obtidos. Para cada página de produto, foi obtido o título que corresponde ao nome do produto e o elemento que contém o \textit{id} produto-descrição, que corresponde à sua descrição. As imagens dos produtos são disponibilizadas através de \textit{urls} na secção da galeria do produto, pelo que, são obtidos todos os seus \textit{urls}.

A documentação dos produtos pode ser disponibilizada através de \textit{urls} para os manuais, ou, com uma lista \textit{dropdown} com todos os manuais em formato \textit{url}, o que permite, serem obtidos através de todos os \textit{urls} da secção de documentação, assim como, os nomes e todas as opções do \textit{dropdown} se este existir.

\begin{figure}[htb]
  \centering
  
  \includegraphics[width=0.7\textwidth]{images/implementacao/scraper/pagina_detalhes_produto.png}
  \caption{Página de detalhes de produto, secção inicial}
  \label{fig:53}
\end{figure}

\newpage

As imagens de desenho técnico e informação geral encontram-se disponíveis na secção correspondente ao nome de cada uma, caso existam, são obtidas as imagens e os seus \textit{urls}.

\begin{figure}[htb]
  \centering
  
  \includegraphics[width=0.7\textwidth]{images/implementacao/scraper/pagina_detalhes_desenhos.png}
  \caption{Página de detalhes de produto, secção de informações}
  \label{fig:54}
\end{figure}

Os vídeos de produtos estão disponíveis na secção de vídeos, sendo que, cada uma contém o nome e o vídeo. Estes são demonstrados através da utilização de um elemento \textit{iframe}, este contém um \textit{url} para o vídeo, mas após a tentativa de visualização deste \textit{url}, constatou-se que não é possível obter o vídeo a partir deste. Sendo assim, foi investigada a plataforma \textit{vimeo}. Esta, contém todos os vídeos de produtos, pelo que, para cada um é gerado um \textit{id} único. Este vídeo poderá ser acedido através do \textit{url} geral da plataforma seguido do \textit{id}. O \textit{id}, está colocado no elemento \textit{iframe}, pelo que, é obtido e acrescentado ao \textit{url} da plataforma, o que permite guardar todos os vídeos de produtos.

\begin{figure}[htb]
  \centering
  
  \includegraphics[width=0.7\textwidth]{images/implementacao/scraper/pagina_detalhes_videos.png}
  \caption{Página de detalhes de produto, secção de videos}
  \label{fig:55}
\end{figure}

\newpage

\subsubsection{Implemenção no website}

Logo que se verificou que pelo menos 80\% dos produtos totais eram obtidos, decidiu-se testar no \textit{website}. Para isto, foi utilizada a biblioteca \textit{requests}, com a qual é realizado um pedido \textit{GET} a cada \textit{url} necessário para obter a página \textit{web}. Assim que o código foi corrido e a resposta analisada compreendeu-se que o \textit{website} bloqueia este tipo de solução como é indicado na figura~\ref{fig:56}

\begin{figure}[htb]
  \centering
  \includegraphics[width=0.7\textwidth]{images/implementacao/scraper/forbiden_response.png}
  \caption{Resposta obtida aquando o pedido ao \textit{url} da página \textit{web}}
  \label{fig:56}
\end{figure}

Através da resposta obtida, foi compreendida a necessidade de alteração da abordagem, visto que, a anterior não poderia ser utilizada. Opcionalmente seria possível seguir a abordagem de simular a ação humana através da abertura de um navegador programáticamente e pesquisar pelo \textit{url} desejado. 

Após uma investigação descobriu-se que existem ferramentas que permitem controlar o dispositivo onde correm, mas estas impedem a utilização enquanto se encontram a correr. Também, foram descobertas ferramentas que apenas recebem o navegador a utilizar e abrem uma nova janela deste para realizar a pesquisa.
O processo de obter os produtos é demorado, pelo que, optou-se pela segunda opção, uma vez que, seria possível continuar com o trabalho em paralelo com o processo de obter dados. Sendo assim, a ferramenta mais recomendada para realizar esta operação é a biblioteca \textit{Selenium}.

Com a utilização da biblioteca \textit{Selenium} indicou-se qual o navegador a utilizar. Neste caso foi escolhido o \textit{Chrome}, dado que, encontrava-se instalado no dispositivo. Após ser indicado qual o navegador a utilizar, é necessário para cada página referir qual o elemento a esperar que carregue, pois, por vezes existem elementos que demoram mais tempo a carregar do que a página. A página carregada poderá não conter o elemento a obter, pelo que foi implementado um tempo de espera máximo de 5 segundos, assim que este tempo expira, a operação é cancelada e o \textit{url} é adicionado à lista de \textit{urls} com erros.


\subsubsection{Melhoria de implementação}

Depois de completo o processo, foi decidido melhorar a implementação com a resolução dos erros encontrados em \textit{urls} específicos. Para perceber exatamente quais os \textit{urls} que possuem erros, foram direcionados os dados obtidos para um ficheiro \textit{\acrshort{json}}. Sendo assim, os \textit{urls} com erros eram os indicados na Figura~\ref{fig:57}.

Posteriormente a uma primeira análise percebeu-se que grande maioria os erros provêm de \textit{urls} de acessórios de produtos, isto deve-se ao facto destes encontrarem-se na página de produtos de subcategoria e serem tratados como um produto, sendo assim, sempre que se trata de um url de acessórios seria necessário correr código para obter dados de acessórios ao invés de detalhes de produtos. Deste modo, compreendeu-se que nos \textit{urls} a palavra accessórios está sempre contida, o que permite que sempre que esta é detetada num \textit{url}, seja corrido o código referente a obter de acessórios. 

\begin{figure}[htb]
  \centering
  
  \includegraphics[width=0.7\textwidth]{images/implementacao/scraper/urls_erro_iteracao_1.png}
  \caption{Urls com erro primeira interação}
  \label{fig:57}
\end{figure}

Para desenvolver este código foi primeiramente analisada a página de acessórios de produtos (Figura~\ref*{fig:58}), esta contém para cada acessório um elemento do tipo artigo o qual detém uma imagem, título e descrição. Esta descrição por vezes possui \textit{urls} para os produtos aos quais este acessório se refere. Sempre que estes \textit{urls} são detetados, os nomes dos produtos são guardados para futuramente realizar a ligação entre os acessórios e os produtos, dado que, não existem produtos com nomes iguais. 

\begin{figure}[htb]
  \centering
  
  \includegraphics[width=0.55\textwidth]{images/implementacao/scraper/pagina_acessorios.png}
  \caption{Exemplo de página de acessórios}
  \label{fig:58}
\end{figure}

Na iteração seguinte determinou-se que a quantidade de \textit{urls} com erros diminuiu, mas mesmo assim existiam produtos com erro, pelo que, estes foram analisados e percebeu-se que o erro ocorria, visto que, por vezes as páginas não continham os vídeos ou imagens de documentação. Para resolver este problema, o código foi alterado para somente obter estes dados se os elementos existirem na página. Após correr novamente o código foi compreendido que a quantidade de falhas obtidas diminuiu drasticamente (Figura~\ref{fig:59}), mas mesmo assim, ainda existiam quatro falhas a ocorrer e após uma análise foi determinado que estas ocorriam devido a uma página de subcategoria de produtos conter um serviço, um produto conter uma página de detalhes de produto com sub produtos, existir uma página de adaptadores de produtos e um produto conter uma página de detalhes diferente das demais.

% \begin{enumerate}
%   \item Uma página de subcategoria de produtos conter um serviço;
%   \item Um produto conter uma página de detalhes de produto com sub produtos;
%   \item Existir uma página de adaptadores de produtos;
%   \item Um produto conter uma página de detalhes diferente das demais;
% \end{enumerate}

\begin{figure}[htb]
  \centering
  
  \includegraphics[width=0.9\textwidth]{images/implementacao/scraper/melhor_corrida.png}
  \caption{Indicação das páginas com falha}
  \label{fig:59}
\end{figure}



A página de adaptadores de produtos segue uma estrutura similar à dos acessórios, pelo que foi a primeira a ser abordada e resolvida através do código de obter detalhes de acessórios. Neste sentido este foi alterado para executar sempre que a palavra acessórios ou adaptadores encontra-se no \textit{url}.

A resolução do problema de existirem serviços e subprodutos implica uma alteração no diagrama de entidade relação, pelo que, o problema do produto que contém uma página de detalhes diferente das demais foi resolvido primeiro. Este produto para além da dificuldade de ser uma página completamente diferente, as informações encontram-se espalhadas (Figura~\ref*{fig:60}), o que leva a que estas tenham de ser combinadas para construir os detalhes do produto. Após obter-se estes dados foi determinado que as imagens do produto têm dados escritos que não se encontram na imagem original. Para a solução deste problema o cliente recomendou obter com \textit{screenshots} e guardar num servidor de imagens.

\begin{figure}[htb]
  \centering
  \includegraphics[width=0.9\textwidth]{images/implementacao/scraper/flama.png}
  \caption{Exemplo de página de produto incomum}
  \label{fig:60}
\end{figure}

O problema de existirem produtos com subprodutos foi resolvido através da alteração da base de dados, à qual foi adicionada uma nova ligação sobre si mesma na tabela produtos, o que permite que um subproduto possua uma ligação com produto principal. Após ser feita esta alteração, o código para obter os dados do catálogo foi alterado. Em primeiro lugar, foram obtidos todos os dados do produto principal e de seguida os dados específicos a cada subproduto, o que leva a que a organização do produto principal seja igual aos restantes, mas, com um dado extra com os subprodutos. 

\begin{figure}[htb]
  \centering
  \includegraphics[width=0.6\textwidth]{images/implementacao/scraper/zuma.png}
  \caption{Exemplo de página de produto com subprodutos}
  \label{fig:61}
\end{figure}

O último problema a solucionar é a existência de serviços, pelo que, iniciou-se pela análise deste tipo de produto. Este possui descrição e imagens como os produtos, mas também vídeos direcionados a plataformas diferentes, produtos do serviço, registo de atualizações do serviço, planos de pagamento e cada plano contém diversas ofertas. 
Através da estrutura de um serviço foi adicionado à base de dados as tabelas de serviço, planos do serviço, ofertas dos planos, vídeos do serviço, informações de atualizações dos serviços, imagens do serviço e a ligação à tabela produtos, o que permite a identificação dos produtos do serviço. Aqui, foi identificada a necessidade de manter guardado os \textit{links} para todas as imagens e vídeos na própria base de dados, visto que, existem imagens que não sabem ainda qual é o \textit{id} do produto referente, pelo que não seria possível no futuro obtê-las sem alteração manual, o que levou a que estas tabelas também existam para os produtos. De seguida compreendeu-se que existem dados que não são necessários na descrição, o cliente recomendou guardar estes dados como \textit{screenshots}.
\begin{figure}[htb]
  \centering
  \includegraphics[width=0.6\textwidth]{images/implementacao/scraper/mconnect.png}
  \caption{Exemplo de página de serviço}
  \label{fig:62}
\end{figure}

\newpage

\subsubsection{Diagrama de base de dados final}

Após as alterações mencionadas à base de dados, o seu diagrama final encontra-se na Figura~\ref{webscraper_bd}. Encontra-se no documento de anexos, no anexo 19, uma versão mais detalhada da Figura~\ref*{webscraper_bd}.

\begin{figure}[htb]
  \centering
  \includegraphics[width=0.8\textwidth]{images/diagramas/bd_final.png}
  \caption{atualização da base de dados}
  \label{webscraper_bd}
\end{figure}


\subsubsection{Armazenamento de dados}

Após obter-se os dados dos produtos, é necessário guardar na base de dados para serem disponibilizados ao \textit{backend}. Para realizar esta operação existiam duas opções, criar um serviço para inserir produtos e realizar um pedido a este serviço, ou então, conectar diretamente o \textit{web scraper} à base de dados. Para evitar falhas de segurança, foi decidido criar um serviço que recebe um produto e o insere. O grande problema que surgiu com esta solução é que os pedidos ocorrem de forma sequencial, mas com pouco tempo de espera, o que levou a que o limite máximo de conexões com a base de dados fosse extrapolada. Isto acontece porque para cada serviço que recebe uma chamada é desenvolvida uma nova conexão à base de dados, todas as operações são realizadas e por fim a conexão é terminada, mas enquanto estas operações estão a decorrer, o servidor poderá receber mais pedidos, o que leva a que mais conexões sejam criadas, o que atinge assim rápidamente o limite de conexões da base de dados. Como solução para este problema foi acrescentado uma espera de 0.5 segundos a cada pedido. A inserção de serviços decorreu com o mesmo processo.

A inserção de categorias decorre através do envio das categorias a inserir num \textit{array}, o que leva a que estas sejam inseridas todas num pedido. As subcategorias, visto que, não se sabe o \textit{id} da categoria referente, foi utilizado o nome da categoria, pois este é único, sendo assim, é enviado o nome da categoria e as subcategorias referentes. Todas são inseridas com a referência para a sua categoria.


\newpage

\section{Serviços Backend}

De forma a realizar a integração entre a aplicação \emph{frontend} e os dados, foi necessário desenvolver uma API para dar suporte a todos os serviços necessários para a aplicação.
API sigla para \emph{Application Programming Interface} disponibiliza um conjunto de funções e dados que facilita as interações entre aplicações e permite que troquem informação ~\cite{rest_cookbook}.
Esta ferramenta apesar de ser desenvolvida para trabalhar em conjunto com outros programas, ela são em sua grande maioria desenvolvidas para serem entendidas e utilizadas por outros programadores no
desenvolvimento dos seus programas ~\cite{api_design}.

\subsection{Serviços REST Full}
Explicar o que é

\subsection{Organização do projeto}
Antes de iniciar a implementação definiu-se a estrutura do projeto a seguir. \textit{MVC} foi a estrutura escolhida, uma vez que, é a mais comum e bem estabelecida. Sendo assim, a organização do projeto seguiu a seguinte estrutura:
\begin{figure}[htb]
  \centering
  \includegraphics[width=0.2\textwidth]{images/implementacao/api/project_organization.png}
  \caption{Exemplo de página de produto incomum}
  \label{fig:63}
\end{figure}

\begin{itemize}
  \item \textbf{docs} - Documentação gerada;
  \item \textbf{src} - Base de todo o projeto;
  \item \textbf{config} - Ficheiros de configuração do projeto;
  \item \textbf{controllers} - Controladores para cada pedido;
  \item \textbf{helpers} - Ficheiros com funções gerais utilizadas regularmente;
  \item \textbf{middlewares} - Ficheiros com os middlewares da api;
  \item \textbf{models} - Classes criadas para representação de base de dados e outras entidades;
  \item \textbf{routes} - Rotas existentes;
  \item \textbf{services} - Serviços para cada pedido;
  \item \textbf{templates} - Templates de \textit{email} a serem enviados;
  \item \textbf{tests} - Testes de código realizados;
  \item \textbf{validations} - Validações a realizar do modelo de negócio e dos dados;
  \item \textbf{app} - Ficheiro de início do projeto;
\end{itemize}

\subsection{Definição de rotas base}
Após a definição da estrutura do projeto foi então definido as rotas base a existir, estas são rotas
que se referem a cada tipo de utilizador. Para melhor organização destas rotas e aplicação de regras foram definidos 3 routers, user para utilizadores se sessão, professional para técnicos e company para empresas. De forma a definir para o projeto qual o router a utilizar em cada pedido foi então definido que:
\begin{itemize}
  \item \textbf{http://baseurl:port/professional} - Encaminhar para router de técnicos;
  \item \textbf{http://baseurl:port/company} - Encaminhar para router de empresas;
  \item \textbf{Restantes} - Encaminhar para router de user;
\end{itemize}

\subsection{Middlewares} 
Um middleware comporta-se como uma ligação entre porções de código, sendo possível este também executar código.

\subsubsection{Linguagem}
O bem essencial em uma boa comunicação entre duas partes é a utilização da mesma linguagem, sendo assim foi necessário perceber qual a linguagem a utilizar quando se responde a um pedido. Para este fim foi então desenvolvido um middleware, o objetivo deste é verificar se existe a chave language no cabeçalho do pedido, caso esta exista é então obtido a linguagem e guardada nas variáveis locais do pedido. Em caso de esta tag não existir, foi então decidido que a aplicação responderá em português por omissão, este valor poderá ser futuramente alterado de forma simples.

\newpage

\subsubsection{Autenticação}
De forma a assegurar a autenticação dos utilizadores que necessitam desta foi então decidido implementar JsonWebToken, este tipo de autenticação baseia-se em a utilização de tokens com tempo de expiração, sendo que enquando o token estiver válido, o utilizador poderá realizar pedidos e assim que este token expirar este terá de se autenticar novamente para obter um novo token.
A utilização de tokens permite também assegurar que os pedidos são realizados com tokens gerados pela api através de utilização de uma chave de assinatura de token, impedindo assim a utilização de tokens gerados por utilizadores.
\begin{figure}[htb]
  \centering
  \includegraphics[width=0.5\textwidth]{images/implementacao/api/jwt_session.png}
  \caption{Exemplo de página de produto incomum}
  \label{fig:64}
\end{figure}

A grande valia da utilização a técnica de autenticação mencionada anteriormente é a segurança desta, mas este nivel de segurança leva a que as aplicações que não necessitam de um nivel de segurança muito alto se tornem impráticas. Isto acontece porque estes  tokens têm geralmente uma duração muito curta como por exemplo 15 minutos, e sempre que um token de sessão expira o utilizador 
teria de realizar novamente o login.

A solução deste problema sem a perda de segurança significativa veio pelo meio da utilização de tokens de duração maior em conjunto com os tokens de duração curta, sendo que enquanto o token de grande duração estiver válido, novos tokens de curta duração são gerados para o utilizador nunca perdendo assim a sua sessão. Estes tokens de grande duração tem por nome tokens de refresh e os tokens de curta duração têm por nome tokens de sessão. Sempre que o utilizador termina a sua sessão o token de refresh deverá ser apagado.

Sempre que um utilizador realiza um pedido o seu token de sessão deverá ser validado, caso este seja válido, o seu token de refresh deverá também ser validado e apenas após isto o utilizador estará autenticado. Caso o token de sessão ou de refresh esteja expirado, este continuará a estar sem autorização para realizar o pedido, mas poderá pedir um novo token de sessão enquanto o seu token de refresh estiver válido, isto acontece sem realizar novamente o login e sem o utilizador perceber.

 Além das funcionalidades atrás mencionadas é possível também associar dados em formato json a um token jwt, esta funcionalidade foi utilizada para enviar o id do utilizador a qual pertence este token e também o cargo do mesmo.

 \newpage

\subsubsection{Validação de Papel}

Com finalidade de garantir que apenas empresas podem realizar os pedidos de empresas e apenas técnicos e empresas podem realizar os pedidos de técnicos foi então criado um middleware que valida se o utilizador que realizou o pedido tem permissões para o mesmo. Este middleware interliga-se com o middleware anterior pois como mencionado o cargo do utilizador em questão é enviado no token, sendo assim é obtido este cargo e realizada uma comparação, com o cargo desejado. Para isto foram criados 2 middlewares diferentes, um valida o cargo de empresas e o outro o cargo de técnicos. Visto que as empresas podem realizar operações de técnicos então no middleware de técnicos é verificado se o token corresponde a um utilizador empresa ou a um utilizador técnico, já no middleware de validação de empresa é verificado se o utilizador tem cargo de empresa.


% //TODO : Adicionar esquema a mostrar como estes middlewares funcionam


\subsection{Controllers}
Assim que um pedido consegue passar por todos os middlewares sem ser impedido, este é então redirecionado para um controller.
Um controller é %Todo: explicar o que é um controller

\subsubsection{Estruturação dos controllers}
De forma a evitar que o código destes controllers varie em termos de estrutura, foi então decidido desenhar uma estrutura de controller e 
aplicar esta perante o demais código. A estrutura deste segue as seguintes etapas:
\begin{enumerate}
 \item Obter dados do pedido
 \item Validar se os dados obrigatórios são obtidos
 \item Validar o pedido perante o modelo de negócio
 \item Executar a lógica do pedido
 \item Formular a resposta e enviar
 \item Em caso de erro este deverá ser capturado e processado de forma a enviar um erro para o utilizador
\end{enumerate}

Para garantir que esta estrutura sempre será aplicada foram utilizados snippets de código que permitem criar um modelo de estrutura de código sendo apenas necessário escrever a palavra chave e toda a estrutura é escrita, necessitando de seguida de efetuar as alterações necessárias perante 
o contexto.

\newpage

\subsubsection{Execução da lógica de negócio}
A execução da lógica de negócio passa por direcionar os dados para a ação correta, sendo que esta ação geralmente resulta em uma operação de base de dados. Inicialmente foi desenvolvida toda a validação de código e todas as operações de base de dados diretamente na execução da lógica de negócio, após uma revisão desta organização de código com o professor orientador, foi decidido separar esta funcionalidades, surgindo assim a componente de validação de dados, a componete de operações de base de dados e por fim a componente de lógica de negócio que implementa a componente de operações de base de dados. Sendo assim de forma a evitar que estas operações sobre a base de dados estejam em conjunto com o direcionamento dos dados, foram então criados modelos para cada tabela. Cada modelo contém um conjunto de operações sobre a sua tabela correspondente, estas operações estão contidas sobre métodos que podem receber dados para executar na operação e devolver a resposta da mesma.

\subsubsection{Validação dos dados}
A validação dos dados é necessária de forma a evitar erros a nivel de servidor com dados em falta e também para aplicar as regras de negócio, garantindo assim que estas são cumpridas. Para realizar estas validações é primeiramente verificado que todos os dados são recebidos, de seguida estes são enviados para um validador. O validador executa todas as verificações necessárias a nivel de regras de negócio e em caso de alguma regra não ser cumprida, é então atirado um erro.

\subsubsection{Formulação da resposta}
Assim como mencionado anteriormente o bem mais importante numa boa comunicação é a utilização da mesma linguagem, sendo assim a resposta do servidor deverá  utilizar a linguagem indicada pelo utilizador. De forma a realizar esta tradução foi utilizado o mesmo conceito que é utilizado para a tradução de aplicações android onde nestas é criado um ficheiro que contém um conjunto de chaves e a cada chave corresponde um texto, para cada tradução estas chaves têm de existir de forma a ser possível obter o texto correto para cada chave. Sendo assim foi utilizado um ficheiro json contento as chaves das linguagens suportadas, a cada linguagem corresponde um conjunto de outras chaves que contém todas as traduções necessárias, utilizando neste caso numeração, em vez de palavras. Esta abordagem permite que de forma fácil futuramente seja possível adicionar outras linguagens ao servidor.

% todo mostrar exemplo

Para dar suporte a este ficheiro foi criada uma operação que recebe a chave desejada e a linguagem desejada, devolvendo o texto correspondente, sendo assim na Formulação da resposta esta operação é executada indicando a chave da resposta a enviar e a linguagem desejada obtendo o texto traduzido, sendo então este devolvido para o utilizador.

\subsubsection{Processamento de erros}
Visto que não é de interesse enviar para o utilizador erros do próprio servidor, foi então decidido controlar estes, para isso foi criado um erro customizado, tendo este por base o erro da própria linguagem. Este erro recebe por parâmetro o código da tradução da mensagem de erro. Esta abordagem permite também evitar que sempre que um erro é lançado o sistema pare. Mesmo com esta abordagem acontece que sempre que um erro é lançado por base de dados, erro de código ou de biblioteca, o erro original é chamado, pelo que foi decidido que sempre que é detetado um erro que não é do tipo do erro customizado, então será devolvido um erro com mensagem de erro de servidor evitando que dados sensíveis e desnecessários para o utilizador sejam devolvidos.

\subsection{Logging}
Logging é um processo que permite guardar informação(logs) sobre um evento. Neste contexto logging poderá ser utilizado para realizar a monitorização de pedidos e/ou monitorização de utilização de recursos do software através da análise de pedidos. Estas informações poderão até auxiliar na toma de decisões sobre o software e em quais funcionalidades deste software colocar mais atenção.

Neste projeto logging foi aplicado sobre os pedidos recebidos, assim como também os erros registados, pois uma vez que os erros são tratados, uma dificuldade encontrada foi a identificação dos erros, para resolver este problema foi então decidido que sempre que um erro que não é customizado é detetado, é registado um log, este log contém informações sobre o pedido, data e hora do pedido, dados recebidos e assim como também a descrição original do erro. Esta implementação permite assim realizar monitorização de erros auxiliando assim na identificação dos serviços mais problemáticos e para quais serviços dirigir mais recursos.

\subsubsection{Morgan}
Morgan é uma ferramenta que permite extrair dados de um pedido, assim como também a criação de logs, 
este atua como um middleware do servidor, recebendo qual o tipo de log a ser escrito, sendo estes tipos definidos pela ferramenta, neste caso foi utilizado o tipo combinado que permite obter todas as informações referentes ao pedido, este tipo de log recebe também a ligação ao ficheiro onde ecrever estes logs. Os principais dados obtidos pela ferramenta são a data e hora do pedido, o tipo de pedido, o serviço pedido, os dados recebidos, a resposta devolvida e também a descrição do sistema utilizado para realizar o pedido, com estes dados é possível saber que plataforma é mais utilizada no software, quais as horas de maior utilização e quais os serviços mais executados, estes dados permitem direcionar mais recursos para uma indicada plataforma e/ou serviço, assim como também escolher os melhores horários de manutenção dos servidores.

\input{sections/chap5/2.servicos_backend/7.documentacao.tex}

\subsection{Testes de código}
Aquando o fim do desenvolvimento de cada serviço é necessário testar este de forma a verificar se a funcionalidade se encontra de acordo com o desejado e/ou se existem erros de código. Para realizar estes testes poderão ser utilizadas ferramentas de auxílio ou então poderão ser realizados manualmente. O grande problema de testes manuais é que 

\subsection{Encriptação de passwords}
De forma a garantir a segurança das passwords dos utilizadores é necessário encriptar estas, a encriptação poderá ser feita manualmente ou com o auxílio de ferramentas, a grande diferença é que manualmente poderá não se obter uma cifra tão segura como com o auxílio de uma ferramenta, sendo assim foi decidido utilizar uma ferramenta para encriptar passwords, a ferramenta escolhida foi bcrypt, esta foi escolhida devido a ser vastamente utilizada e até ao momento sem problemas em relação à sua cifra sendo que não existem registos de ataques bem sucedidos a esta ferramenta. Esta ferramenta oferece um conjunto de métodos sendo tendo sido utilizados os métodos de cifra e de comparação. O método de cifra permite através de um valor indicar a complexidade a aplicar sobre a cifra INDICAR O RANGE DE VALORES E O UTILIZADO, sendo de seguida devolvida a password cifrada sendo esta guardada na base de dados. O método comparação permite comparar uma password cifrada com uma password sem cifra, devolvendo verdadeiro ou falso conforme as passwords sejam iguais ou não.

\subsection{Envio de emails}
Para o envio de emails para os utilizadores, foi utilizada a ferramenta nodemailer, EXPLICAR O QUE é nodemailer..... Esta ferramenta foi escolhida devido a ser uma das mais utilizadas para esta tipo de necessidade, o que permite que exista mais informação sobre a mesma facilitando a resolução e identificação de erros.

Para além de uma ferramenta para enviar emails é preciso também de ter acesso a um servidor de emails, sendo assim para vias de teste foi utilizado um servidor gratuito de email sendo este hospedado por DIZER QUEM HOSPEDAVA, após a fase de testes foi então alterado para o servidor de email da empresa permitindo assim que este email seja identificado como empresa Motorline. A configuração deste servidor foi então colocada em um ficheiro de configuração que devolve o objeto do servidor, os dados de acesso ao servidor foram guardados em um ficheiro .env sendo este futuramente encriptado.

Para desenvolver o conteúdo dos emails foi utilizada a ferramenta Tabular Email, esta ferramenta permite realizar o design do conteúdo de um email, sendo possível de seguida exportar o mesmo para html, a dificuldade desta ferramenta é que não permite a utilização de acentuação e visto que o html é gerado por uma máquina este torna-se complicado de navegar e traduzir. Após a resolução dos erros mencionados, este email é colocado em uma função que recebe por parâmetro os dados necessários a enviar e escreve no email sendo assim este devolvido na criação. Este processo foi então repetido apra cada tipo de email que é necessário enviar.

Após obter o servidor a utilizar e o conteúdo a enviar, é então utilizado o objeto do servidor de email e no envio do email é definido o destinatário, o assunto e o conteúdo do email.


\subsection{Agendador de emails}

Um requisito para este projeto é o envio de emails com relatório diário de notificação todos os dias ao final do dia. Para realizar este primeiramente foi pesquisado que ferramentas existem para realizar este tipo de ações, pelo que foram encontradas o cronetab e o node-cron. A grande diferença este estas duas ferramentas é, o cronetab funciona a nivel de servidor sendo que sempre que se encontra na hora programada, este executa um comando indicado, este comando poderá por exemplo executar um código para enviar emails.Já o node-cron trata-se de um biblioteca de NodeJs que trabalha com base no crontab, este permite o fácil agendamento de tarefas de forma programática assim como também a indicação do código a ser executado 
sem necessidade de criar comandos de execução de código.

Visto que este a hora de execução do código de envio de emails poderá variar e necessitar de reprogramação foi então optado pela utilização do node-cron devido á sua facilidade de utilização, agilizando assim o processo de reprogramação de horas de envio de relatório.

Sendo assim foi inicialmente programado para este enviar o relatório de notificações todos os dias às 22 horas, para realizar esta configuração e necessário indicar primeiramente a programação de horário de envio, para isso é utilizada a estrutura segundo, minuto, hora, dia do mês, mês, dia da semana. Visto que o objetivo é as 22 horas os segundos e minutos foram indicados como 0 e as horas foram indicadas como 22, já o restante foi indicado com o símbolo * que indica que o processo deverá occorrer em todas as instâncias dos restantes valores, significando assim todos os dias.

\subsection{Cifragem de configurações do servidor}
De forma a garantir um nivel de segurança maior foram realizadas pesquisas sobre as principais falhas de segurança no NodeJs, pelo que foi descoberto que as principais formas de ataque a esta ferramenta é o desenvolvimento de bibliotecas de malware e o ataque às bibliotecas com o objetivo de obter dados de acesso a servidores que se encontram nos ficheiros de configuração.

Por norma todas as configurações de servidores são colocadas num ficheiro env, este ficheiro no momento de iniciar o servidor é utilizado para carregar todas as variáveis para o ambiente do mesmo, sendo assim qualquer um com acesso ao ficheiro ou às variáveis de ambiente poderá ver todas as configurações de todos os servidores.

Para resolver este problema foi então decidido cifrar o ficheiro com as variáveis de ambiente e as próprias variaveis. Para realizar a cifragem foi utilizada a biblioteca secur-env, esta permite realizar a cifragem de um ficheiro indicando uma password. A password deverá ser indicada no processo de inicialização do servidor de forma a ser possível ao mesmo decifrar o ficheiro, sendo que a gestão das variáveis cifradas passa então a estar encarregue desta biblioteca. Para obter a password é pedido ao utilizador que vai iniciar o servidor para colocar uma password sendo esta usada para decifrar o ficheiro, caso esta esteja errada é então devolvido um erro. De forma a evitar que seja possível visualizar o histórico do terminal para obter a password, este processo é realizado utilizando a biblioteca readline-sync que permite pedir ao utilizador algum dado e indicar a funcionalidade de esconder os dados escritos.


\bibliography{biblio}

\end{document}  
