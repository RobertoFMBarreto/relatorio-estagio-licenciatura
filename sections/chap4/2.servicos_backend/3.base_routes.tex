\subsection{Definição de rotas base}
Após ser definida a estrutura do projeto foram estruturadas as rotas base a existir, estas referem-se a cada ator. Para uma melhor organização das rotas e da aplicação de regras, foram definidos 3 \textit{routers}, \textit{user}, para utilizadores sem sessão, \textit{professional}, para técnicos, e \textit{company} para empresas. Para indicar qual o \textit{router} a utilizar em cada pedido foi definido que:
\begin{itemize}
 \item \textbf{http://baseurl:port/professional} - Encaminhar para \textit{router} de técnicos;
 \item \textbf{http://baseurl:port/company} - Encaminhar para \textit{router} de empresas;
 \item \textbf{Restantes} - Encaminhar para \textit{router} de utilizadores;
\end{itemize}