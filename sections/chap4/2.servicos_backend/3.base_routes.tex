\subsection{Definição de rotas base}
Após a definição da estrutura do projeto foram definidas as rotas base a existir, estas são rotas que se referem a cada ator. Para melhor organização destas rotas e aplicação de regras foram definidos 3 \textit{routers}, \textit{user} para utilizadores sem sessão, \textit{professional} para técnicos e \textit{company} para empresas. Para indicar qual o \textit{router} a utilizar em cada pedido foi então definido que:
\begin{itemize}
  \item \textbf{http://baseurl:port/professional} - Encaminhar para \textit{router} de técnicos;
  \item \textbf{http://baseurl:port/company} - Encaminhar para \textit{router} de empresas;
  \item \textbf{Restantes} - Encaminhar para \textit{router} de utilizadores;
\end{itemize}