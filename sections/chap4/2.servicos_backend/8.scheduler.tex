\subsection{Agendamento de tarefas}

Com a utilização da ferramenta \textit{Node-Cron}, foi inicialmente programado para enviar o relatório de notificações, todos os dias, às vinte e duas horas. Esta configuração foi relizada através da indicação da programação de horário de envio, para isso, é utilizada a estrutura segundo, minuto, hora, dia do mês, mês, dia da semana. Como o objetivo é enviar às vinte e duas horas, os segundos e minutos foram indicados como zero e as horas foram indicadas como vinte e dois, já o restante foi indicado com o símbolo "*", que indica que o processo deverá ocorrer em todas as instâncias dos restantes valores, o que significa, todos os dias. A configuração final obtida foi "0 0 22 * * *".

Quando o método do agendamento é invocado, são obtidos todos os utilizadores com a configuração de relatório de notificações ativa. Em primeiro lugar, para cada um destes, são obtidas todas as notificações do dia, de seguida é criado o objeto de servidor e invocado o método para obter o conteúdo de notificações através da indicação de todas as notificações do utilizador. Por fim, é enviado o \textit{email} para o utilizador com todas as suas notificações e um \textit{link} para aceder rápidamente ao ecrã de notificações.