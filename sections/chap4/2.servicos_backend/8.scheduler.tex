\subsubsection{Agendamento de tarefas}

Utilizando a ferramenta node-cron foi inicialmente programado para este enviar o relatório de notificações todos os dias às 22 horas, para realizar esta configuração e necessário indicar primeiramente a programação de horário de envio, para isso é utilizada a estrutura segundo, minuto, hora, dia do mês, mês, dia da semana. Visto que o objetivo é as 22 horas os segundos e minutos foram indicados como 0 e as horas foram indicadas como 22, já o restante foi indicado com o símbolo "*" que indica que o processo deverá occorrer em todas as instâncias dos restantes valores, significando assim todos os dias. A configuração final obtida foi então "0 0 22 * * *".

Quando o método do agendamento é invocado, é então obtido todos os utilizadores com a configuração de relatório de notificações ativa, sendo de seguida para cada um destes obtidas todas as notificações do dia. De seguida é criado o objeto de servidor e invocado o método para obter o \textit{email} de notificações enviando todas as notificações do utilizador para o mesmo. Por fim é enviado o \textit{email} para o utilizador com todas as suas notificações e um link para aceder rápidamente ao ecrã de notificações.