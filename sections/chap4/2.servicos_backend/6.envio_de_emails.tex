\subsubsection{Envio de \textit{emails}}
Como mencionado na apresentação da ferramenta tabular, esta não permite a utilização de acentuação na escrita de \textit{emails}, sendo assim estes erros necessitam de ser resolvidos. Para permitir que os dados dos \textit{emails} sejam personalizaveis, como por exemplo dados de utilizador e link para clicar, estes \textit{emails} são colocados dentro de métodos que recebem por parametro os dados personalizaveis, sendo estes então colocados dentro do html do \textit{email}, este método por fim devolve em string o html do \textit{email} a enviar.

Para o envio de \textit{emails} é ncessario um servidor, sendo assim para vias de teste foi utilizado um servidor gratuito de \textit{email} sendo este hospedado por DIZER QUEM HOSPEDAVA, após a fase de testes foi então alterado para o servidor de \textit{email} da empresa permitindo assim que este \textit{email} seja identificado como empresa Motorline.

A configuração do servidor de \textit{emails} do nod\textit{email}er é realizada através de uma chamada ao objeto 
de servidor indicando as configurações do mesmo que estão no ficheiro .env, esta chamada ao servidor por sua vez devolve um objeto que fornece métodos para enviar \textit{emails}, sendo este então criado e utilizado sempre que se deseja enviar \textit{emails} no servidor.


De forma a evitar que sempre que se deseja enviar um \textit{email} seja necessário indicar todos os dados, foi então criado um método que cria e devolve o objeto de servidor sempre que necessário. Sendo assim sempre que se deseja enviar um \textit{email} é então primeiramente obtido o objeto do servidor, de seguida é invocado o método para obter o conteúdo html do \textit{email} e por fim é utilizado o objeto do servidor para chamar o método de envio de \textit{emails} no qual se terá de indicar o \textit{email} do destinatário, o assunto e o conteúdo do \textit{email}.
