\subsection{Envio de \textit{emails}}
Como mencionado no capítulo \ref{sec:emails_send}, não é permitido a utilização de acentuação na escrita de \textit{emails} no \textit{Tabular}, pelo que, primeiramente todos os problemas foram resolvidos. Para permitir que os dados dos \textit{emails} sejam personalizaveis, como por exemplo, dados de utilizador e \textit{link} para clicar, estes \textit{emails} são colocados em métodos que recebem por parâmetro os dados para serem colocados dentro do \textit{html} do \textit{email}, este método, por fim, devolve em \textit{string} o conteúdo a enviar.

Para o envio de \textit{emails} é necessário um servidor e um serviço de envio. Para vias de teste foi utilizado um servidor gratuito de \textit{email} hospedado por \textit{Mailjet}. Após a fase de testes foi alterado para o servidor de \textit{email} da empresa, o que permitiu que seja identificado como empresa Motorline.

O serviço de envio de \textit{emails} foi utilizado o \textit{Nodemailer}. A configuração do servidor de \textit{emails} do \textit{Nodemailer} é realizada através de uma chamada ao objeto do servidor com a indicação das configurações que estão no ficheiro \textit{.env}. Esta chamada ao servidor, por sua vez, devolve um objeto que fornece métodos para enviar \textit{emails}, sendo este, criado e utilizado sempre que se deseja enviar \textit{emails} no servidor.

Para evitar que sempre que se deseja enviar um \textit{email} seja necessário indicar todos os dados de configuração do servidor, foi elaborado um método que cria e devolve o objeto do servidor sempre que necessário. Sendo assim, sempre que se deseja enviar um \textit{email} é primeiramente obtido o objeto do servidor, de seguida é invocado o método para obter o conteúdo \textit{html} do \textit{email} e por fim, é utilizado o objeto do servidor para chamar o método de envio de \textit{emails}, no qual, se terá de indicar o \textit{email} do destinatário, o assunto e o conteúdo.
