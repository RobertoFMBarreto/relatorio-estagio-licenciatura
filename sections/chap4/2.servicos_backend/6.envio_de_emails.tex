\subsection{Envio de \textit{emails}}
Como mencionado na apresentação da ferramenta tabular, esta não permite a utilização de acentuação na escrita de \textit{emails}, pelo que primeiramente todos os problemas foram resolvidos. Para permitir que os dados dos \textit{emails} sejam personalizaveis, como por exemplo, dados de utilizador e \textit{link} para clicar, estes \textit{emails} são colocados em métodos que recebem por parâmetro os dados para serem colocados dentro do \textit{html} do \textit{email}, este método por fim devolve em \textit{string} o conteúdo a enviar.

Para o envio de \textit{emails} é necessario um servidor, para vias de teste foi utilizado um servidor gratuito de \textit{email} sendo este hospedado por \textit{Mailjet}, após a fase de testes foi alterado para o servidor de \textit{email} da empresa o que permitiu que este seja identificado como empresa Motorline.

A configuração do servidor de \textit{emails} do \textit{nodemailer} é realizada através de uma chamada ao objeto de servidor com a indicação das configurações que estão no ficheiro \textit{.env}, esta chamada ao servidor, por sua vez, devolve um objeto que fornece métodos para enviar \textit{emails}, sendo este então criado e utilizado sempre que se deseja enviar \textit{emails} no servidor.


Para evitar que sempre que se deseja enviar um \textit{email} seja necessário indicar todos os dados, foi elaborado um método que cria e devolve o objeto de servidor sempre que necessário. Sendo assim sempre que se deseja enviar um \textit{email} é então primeiramente obtido o objeto do servidor, de seguida é invocado o método para conseguir o conteúdo \textit{html} do \textit{email} e por fim, é utilizado o objeto do servidor para chamar o método de envio de \textit{emails} no qual se terá de indicar o \textit{email} do destinatário, o assunto e o conteúdo.
