\subsection{Encriptação de \textit{passwords}}
Aquando o registo de clientes é indicada a \textit{password}, pelo que, esta deverá ser guardada sobre encriptação. Para isto, é utilizada a biblioteca \textit{Bcrypt}. No processo de registo, antes do envio dos dados para a base de dados, é invocado o método de encriptação da \textit{password} com a indicação do \textit{salt} com valor de oito. Após a execução é obtida a \textit{password} encriptada e enviada para a base de dados em conjunto com os restantes dados.

\subsubsection{Encriptação de configurações do servidor}
Para a encriptação das configurações do servidor, foi executado o comando de encriptar da biblioteca \textit{secure-env}, com a indicação da \textit{password}. Como resultado foi criado o ficheiro encriptado. O ficheiro original deverá ser eliminado ou, em caso de necessidade, guardado num local seguro para evitar que durante algum ataque seja obtido os dados do mesmo.

Após encriptar o ficheiro, o servidor foi configurado para quando inicia pedir ao utilizador para indicar a \textit{password} do ficheiro. Para esta configuração foi utilizada a biblioteca \textit{Readline-Sync} no modo de \textit{password}. Quando o utilizador indica a \textit{password}, na eventualidade de esta estar errada, o servidor irá falhar, pois a biblioteca tentará desencriptar o ficheiro de configurações e falhará não concluindo o processo de inicialização. Caso contrário, este continuará o processo, no qual, o primeiro passo é a utilização da \textit{password}, para desencriptar o ficheiro de configurações e carregar as variáveis de ambiente.