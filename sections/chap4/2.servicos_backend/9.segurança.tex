\subsection{Cifra de \textit{passwords}}
Aquando o registo de clientes é indicada a \textit{password}, pelo que, esta deverá ser guardada cifrada. Para isto é utilizada a biblioteca \textit{bcrypt}, no processo de registo, antes do envio dos dados para a base de dados, é invocado o método de cifra da \textit{password} com a indicação do \textit{salt} com valor de 8, após a execução é obtida a \textit{password} cifrada e enviada para a base de dados com conjunto com os restantes dados.

\subsubsection{Cifra de configurações do servidor}
Para a cifra das configurações do servidor, foi então executado o comando de cifra da biblioteca \textit{secure-env}, com a indicação da \textit{password} da mesma, como resultado foi criado o ficheiro cifrado. O ficheiro original deverá ser eliminado ou em caso de necessidade guardado num local seguro para evitar que durante algum ataque seja possível obter os dados do mesmo.

Após a cifra do ficheiro, o servidor foi configurado para aquando a sua inicialização, pedir para o utilizador indicar a \textit{password} do ficheiro, para esta configuração foi utilizada a biblioteca \textit{readline-sync} no modo de \textit{password}. Quando o utilizador indica a \textit{password}, na eventualidade de esta estar errada o servidor irá falhar, pois a biblioteca de cifra irá tentar decifrar o ficheiro de configurações e irá falhar não concluindo o processo de inicialização. Caso contrário, este continuará o processo de inicialização, no qual o primeiro passo é a utilização da \textit{password} para decifrar o ficheiro de configurações e carregar as variáveis de ambiente.