\subsection{Links}
Uma funcionalidade da aplicação necessária é também a utilização de links, para isto programação mobile oferece duas soluções, deep links e dynamic links. Como mencionado na secção de tecnologias foi decidido implementar a solução de dynamic links da firebase.

Para implementar esta solução primeiramente a nível de backend foi necessário gerar os links, para isto, existem 2 soluções, implementação do firebase no próprio backend ou então uma chamada ao firebase utilizando uma chave de pedido. Primeiramente foi testado a implementação do firebase no próprio backend, mas esta implementação surge com alguns problemas pois existem diversas configurações especificas necessárias, sendo então recomendado pelos colegas de trabalho a chamada ao firebase devido à sua simplicidade.

Sendo assim para a realização de chamadas ao firebase foi utilizado o axios, realizando assim um método post para o serviço de dynamic links do firebase, indicando o o prefixo do projeto. De forma a permitir a reutilização deste código, este foi então colocado em um método onde este é chamado indicando o link desejado e os dados a enviar. O link é utilizado para por exemplo como em uma página web, indicar qual página se deseja direcionar o utilizador, já os parametros, assim como em um url web, são enviados através do próprio link, sendo assim estes dados são colocados na string do link permitindo a configuração do mesmo perante diversas situações.

Para a implementação de frontend foi necessário primeiramente importar a biblioteca de dynamic links do firebase, sendo de seguinda no código de inicialização da aplicação colocado um método para em caso de a aplicação ser aberta a partir de um link, este o ler. Quando este método lê o link primeiramente este extrai a página indicada pelo link e de seguida extrai a lista de parametros recebidos, sendo então o utilizador redirecionado para a página do link indicando como parametros os dados recebidos no link.

Aquando a testagem da implementação diversas tentativas de abertura de links foram realizadas mas sem sucesso, a grande dificuldade desta implementação é que os links dinamicos não permitem realizar debug, sendo que ou funcionará na totalidade ou então não funcionará lenvando que seja complicado identificar bugs, pelo que a documentação do serviço foi de grande auxilio pois estava em falta a indicação do nome do pacote da aplicação para android e ios, após a colocação destes dados, tudo seguiu em completo funcionamento.