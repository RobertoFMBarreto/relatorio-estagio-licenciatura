\subsection{Links}
Uma das funcionalidades necessárias da aplicação é a utilização de \textit{links}. Para isto, a programação \textit{mobile} oferece duas soluções, \textit{app links}, \textit{deep links} e \textit{dynamic links}. Como mencionado na secção de tecnologias foi decidido implementar a solução de \textit{dynamic links} da \textit{Firebase}.

Para implementar esta solução, primeiramente a nível de \textit{backend} foi necessário gerar os \textit{links}, para isto, existem duas opções, implementação do \textit{Firebase} no próprio \textit{backend} ou então uma chamada ao \textit{Firebase} com a utilização de uma chave de pedido. Em primeiro lugar foi testada a implementação do \textit{Firebase} no próprio \textit{backend}, contudo, esta implementação surgiu com alguns problemas, uma vez que, existem diversas configurações específicas necessárias, sendo então recomendado pelos colegas de trabalho a chamada ao \textit{Firebase} dada à sua simplicidade.

Sendo assim, para a realização de chamadas ao \textit{Firebase} foi utilizado o \textit{axios}. Este permitiu realizar um pedido com o método \textit{POST} para o serviço de \textit{dynamic links} do \textit{Firebase}, com indicação do prefixo do projeto. Para permitir a reutilização deste código foi colocado num método onde é chamado com indicação do \textit{link} desejado e os dados a enviar. O \textit{link} é utilizado para, por exemplo, como numa página \textit{web}, indicar qual página se deseja direcionar o utilizador, já os parâmetros, assim como em um \textit{url web}, são enviados através do próprio \textit{link}, pelo que, estes dados são colocados na \textit{string} do \textit{link} o que permite a configuração perante diversas situações. Por fim, a \textit{Firebase} retorna o \textit{link} criado e este é colocado no \textit{email} desejado.

Para a implementação do \textit{frontend} foi necessário importar a biblioteca de \textit{dynamic links} do \textit{Firebase} e de seguida no código de inicialização da aplicação colocar um método para em caso de a aplicação ser aberta a partir de um \textit{link}, este o ler. Quando este método lê o \textit{link}, extrai a página indicada e a lista de parâmetros recebidos. Deste modo, o utilizador é redirecionado para a página do \textit{link} com os dados recebidos.

Aquando o teste da implementação, diversas tentativas de abertura de \textit{links} foram realizadas, mas, contudo, sem sucesso. A grande dificuldade desta implementação foi os \textit{links} dinâmicos não permitem realizar \textit{debug}, sendo que, ou funcionará na totalidade, ou não funcionará, o que leva a que seja complicado identificar \textit{bugs}. A documentação do serviço foi de grande auxílio, uma vez que, estava em falta a indicação do nome do pacote da aplicação para \textit{android} e \textit{ios}. Após a colocação destes dados, tudo seguiu em completo funcionamento.