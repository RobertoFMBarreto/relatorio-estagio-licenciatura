\subsection{Links}
Uma funcionalidade da aplicação necessária é também a utilização de \textit{links}, para isto programação mobile oferece duas soluções,\textit{app links}, \textit{deep links} e \textit{dynamic links}. Como mencionado na secção de tecnologias foi decidido implementar a solução de \textit{dynamic links} da \textit{Firebase}.

Para implementar esta solução primeiramente a nível de backend foi necessário gerar os \textit{links}, para isto, existem 2 soluções, implementação do \textit{Firebase} no próprio \textit{backend} ou então uma chamada ao \textit{Firebase} com a utilização de uma chave de pedido. Foi testado primeiro lugar a implementação do \textit{Firebase} no próprio \textit{backend}, mas esta implementação surge com alguns problemas, pois existem diversas configurações especifica necessárias, sendo então recomendado pelos colegas de trabalho a chamada ao \textit{Firebase} devido à sua simplicidade.

Sendo assim para a realização de chamadas ao \textit{Firebase} foi utilizado o \textit{axios}. Este permitiu realizar um pedido com o método \textit{post} para o serviço de \textit{dynamic links} do \textit{Firebase}, com indicação do prefixo do projeto. Para permitir a reutilização deste código, este foi colocado num método onde é chamado com indicação do \textit{link} desejado e os dados a enviar. O \textit{link} é utilizado para, por exemplo ,como num página \textit{web}, indicar qual página se deseja direcionar o utilizador, já os parâmetros, assim como em um \textit{url web}, são enviados através do próprio \textit{link}, pelo que estes dados são colocados na \textit{string} do \textit{link} o que permite a configuração perante diversas situações. Por fim a \textit{Firebase} retorna o \textit{link} criado sendo este colocado no \textit{email} desejado.

Para a implementação de \textit{frontend} foi necessário importar a biblioteca de \textit{dynamic links} do \textit{Firebase} e seguida no código de inicialização da aplicação, colocado um método para em caso de a aplicação ser aberta a partir de um \textit{link}, este o ler. Quando este método lê o \textit{link}, extrai a página indicada pelo e de seguida extrai a lista de parâmetros recebidos, sendo então o utilizador redirecionado para a página do \textit{link} com os dados recebidos.

Aquando a testagem da implementação, diversas tentativas de abertura de \textit{links} foram realizadas, mas sem sucesso, a grande dificuldade desta implementação é que os \textit{links} dinâmicos não permitem realizar \textit{debug}, sendo que ou funcionará na totalidade, ou então não funcionará o quie leva a que seja complicado identificar \textit{bugs}. A documentação do serviço foi de grande auxílio, pois, estava em falta a indicação do nome do pacote da aplicação para \textit{android} e \textit{ios}, após a colocação destes dados, tudo seguiu em completo funcionamento.