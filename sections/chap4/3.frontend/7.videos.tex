\subsection{Vídeos}

A maior dificuldade detetada nos vídeos foi a necessidade de um comportamento diferente nestes quando se encontram no ecrã de visualização de imagens e vídeos, visto que, ao contrário das imagens, os vídeos necessitam de um \textit{player}, sendo sempre necessário verificar qual o tipo de ficheiro antes de carregar o \textit{widget} do mesmo.

Para resolver o problema de utilizar um \textit{player} testou-se uma biblioteca que permite a utilização do \textit{player} nativo do dispositivo, ou seja, \textit{android} utilizaria o \textit{player} do \textit{android} e \textit{ios} utilizaria o \textit{player} de \textit{ios}. O grande problema com esta solução é que o \textit{player} de \textit{android} tem os botões completamente brancos, sem nenhum tipo de fundo para os destacar, o que significa que se um vídeo branco for visualizado, o utilizador não conseguirá visualizar os botões do \textit{player}.

Deste modo decidiu-se desenvolver um \textit{player} próprio. Após a implementação de diversas funções como, pausar, resumir, avançar 5 segundos e recuar 5 segundos, esconder e apresentar os botões, existiam dois grandes problemas, demonstrar o vídeo em ponto grande, voltar para o mesmo tempo do vídeo em ponto pequeno e também o comportamento do \textit{player} não ser completamente fluido.

Depois de uma vasta pesquisa compreendeu-se que o \textit{player} do \textit{ios} resolvia os problemas do \textit{player} do \textit{android} através da colocação de um fundo nos botões do \textit{player}. Através da biblioteca \textit{appinio} é possível utilizar e configurar o \textit{player} nativo de \textit{ios} em \textit{android}. Sendo assim, a utilização do \textit{player} de \textit{ios} em ambos os sistemas operativos resolveria o problema. Este \textit{player} permitiu a resolução de um problema menor, a visualização de vídeos em ponto grande, sendo que, dependendo da orientação do vídeo, o \textit{player} altera a orientação da aplicação automáticamente voltando à orientação vertical, uma vez que, termina a visualização do vídeo em ponto grande.