\subsection{Videos}

A grande dificuldade encontrada com os vídeos foi o facto de ser necessário um comportamento diferente nestes quando se encontram no ecrã de visualização de imagens e vídeos, pois ao contrário de imagens, os vídeos necessitam de um \textit{player}, sendo sempre necessário verificar qual o tipo de ficheiro antes de carregar o \textit{widget} do mesmo.

Para resolver o problema de utilizar um \textit{player} foi testada uma biblioteca que permite a utilização do \textit{player} nativo do dispositivo, ou seja, \textit{android} utilizaria o \textit{player} do \textit{android} e \textit{ios} utilizaria o \textit{player} de \textit{ios}. O grande problema com esta solução é que o \textit{player} de \textit{android} tem o botões completamente brancos, sem nenhum tipo de fundo para os destacar, o que significa que se um vídeo branco for visualizado, o utilizador não conseguirá visualizar os botões do \textit{player}.

Após o problema anterior, foi decidido desenvolver o \textit{player} próprio, após a implementação de funções como, pausar, resumir, avançar 5 segundos e voltar 5 segundos, esconder e mostrar os botões, existiam 2 grandes problemas, sendo estes mostrar o vídeo em ponto grande, retornar para o mesmo tempo de vídeo em ponto pequeno e também o comportamento do \textit{player} não era completamente fluido.

Após uma pesquisa foi percebido que o \textit{player} de \textit{ios} resolvia os problemas do \textit{player} de \textit{android} através da colocação de um fundo nos botões do \textit{player}. Através da biblioteca \textit{appinio}, é possível utilizar o \textit{player} nativo de \textit{ios} em \textit{android} e configurar. Sendo assim foi decidido utilizar o \textit{player} de \textit{ios} em ambos os sistemas operativos o que resolve o problema. Este \textit{player} também permitiu a resolução de um problema menor que era a visualização de vídeos em ponto grande, sendo que dependendo a orientação do vídeo, o \textit{player} altera a orientação da aplicação automáticamente, voltando à orientação vertical uma vez que se termina a visualização do vídeo em ponto grande.