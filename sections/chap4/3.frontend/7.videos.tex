\subsection{Vídeos}

A maior dificuldade detetada nos vídeos foi a necessidade de um comportamento diferente nestes quando se encontram no ecrã de visualização de imagens e vídeos, visto que, ao contrário das imagens, os vídeos necessitam de um reprodutor, sendo sempre necessário verificar qual o tipo de ficheiro antes de carregar o \textit{widget} do mesmo.

Para resolver o problema de utilizar um reprodutor testou-se uma biblioteca que permite a utilização do reprodutor nativo do dispositivo, ou seja, \textit{Android} utilizaria o reprodutor do \textit{Android} e \textit{iOS} utilizaria o reprodutor de \textit{iOS}. O grande problema com esta solução é que o reprodutor de \textit{Android} tem os botões completamente brancos, sem nenhum tipo de fundo para os destacar, o que significa que se um vídeo branco for visualizado, o utilizador não conseguirá visualizar os botões do reprodutor.

Deste modo decidiu-se desenvolver um reprodutor próprio. Após a implementação de diversas funções como, pausar, resumir, avançar 5 segundos e recuar 5 segundos, esconder e apresentar os botões, existiam dois grandes problemas, demonstrar o vídeo em ponto grande, voltar para o mesmo tempo do vídeo em ponto pequeno e também o comportamento do reprodutor não ser completamente fluido.

Depois de uma vasta pesquisa compreendeu-se que o reprodutor do \textit{iOS} resolvia os problemas do reprodutor do \textit{Android} através da colocação de um fundo nos botões do reprodutor. Através da biblioteca \textit{appinio} é possível utilizar e configurar o reprodutor nativo de \textit{iOS} em \textit{Android}. Sendo assim, a utilização do reprodutor de \textit{iOS} em ambos os sistemas operativos resolveria o problema. Este reprodutor permitiu a resolução de um problema menor, a visualização de vídeos em ponto grande, sendo que, dependendo da orientação do vídeo, o reprodutor altera a orientação da aplicação automáticamente voltando à orientação vertical, uma vez que, termina a visualização do vídeo em ponto grande.