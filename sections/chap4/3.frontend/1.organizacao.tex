\subsection{Organização do projeto}
Assim como o \textit{backend} o modelo a seguir no \textit{frontend} foi o \textit{MVC}. 

Como recomendado de boas práticas de código limpo da \textit{framework} as cores do tema da aplicação foram todas colocadas num ficheiro separado para garantir a facilidade de troca de tema da aplicação. Outras aplicações de boas práticas de código limpo foram, sempre que possível particionar o código das páginas em vários \textit{widgets} para ser de fácil navegação e também a criação de \textit{widgets} reutilizaveis para evitar a repetição de código e também para agilizar o desenvolvimento. Por fim, a estrutura do projeto foi organizada da seguinte forma:
\begin{figure}[htb]
  \centering
  \includegraphics[width=0.35\textwidth]{images/implementacao/frontend/organizacao_projeto.png}
  \caption{Organização do projeto}
  \label{fig:69}
\end{figure}

\begin{itemize}
  \item \textbf{l10n} - Traduções da aplicação;
  \item \textbf{models} - Modelos de classes como handlers, helpers, providers, entre outros;
  \item \textbf{pages} - Páginas da aplicação;
  \item \textbf{widgets} - Widgets referentes às páginas;
\end{itemize}
\vspace{60mm}

