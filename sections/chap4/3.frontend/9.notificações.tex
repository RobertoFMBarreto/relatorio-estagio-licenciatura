\subsection{Notificações}

A implementação das notificações revelou ser a de maior dificuldade, visto que, surgiram diversos imprevistos. Para isto foi utilizado o serviço de notificações do \textit{Firebase}. Este surge como os \textit{links}, em duas secções, primeiramente \textit{backend} e de seguida \textit{frontend}.

A nível do \textit{backend} foi necessário utilizar \textit{axios} para realizar um pedido ao serviço de notificações do \textit{Firebase}. Mas assim como na criação dos \textit{links} o serviço não indica quaisquer informações sobre erros, apenas o dispositivo poderá ou não receber a notificação. 

Em primeiro lugar foi utilizado o conteúdo a enviar indicado pela documentação do serviço, mas, apenas retornava uma mensagem de erro. Posteriormente foi pesquisado outras implementações de outros utilizadores e testadas, mas, novamente surgia um erro, pelo que, decidiu-se utilizar o conteúdo indicado pelo professor de programação de dispositivos móveis, tendo este funcionado sem qualquer indicação de erro. 

No conteúdo da notificação é enviado em formato \textit{\acrshort{json}} a mensagem a apresentar na notificação e como estas serão sempre referentes a tópicos ou comentários é indicado o \textit{id} do tópico, do comentário e em caso de necessidade o \textit{id} do comentário pai.

Este processo de notificação foi aproveitado para direcionar os mesmos dados para notificação de \textit{email} em caso do utilizador possuir ativo as notificações por \textit{email}, sendo gerado um \textit{link} dinâmico com os dados na notificação.

A nível do \textit{frontend} foi importada a biblioteca do serviço de notificações do \textit{Firebase}, sendo esta implementada conforme a documentação. Como é necessário detetar notificações no iniciar da aplicação, durante a utilização e quando esta encontra-se em segundo plano aproveitaram-se estas deteções para implementar o direcionamento do utilizador para as páginas referentes às notificações, com a utilização dos dados recebidos.

O grande problema detetado nas notificações é que o \textit{icon} não era apresentado corretamente, sendo que ou era demonstrado um quadrado escuro, ou nenhum \textit{icon}. A biblioteca de notificações do \textit{Firebase} não permite a customização do \textit{icon}, pelo que foi decidido utilizar a biblioteca \textit{flutter\_local\_notifications}. Esta permite a total customização das notificações, na qual é enviado o icon desenhado para a aplicação. Mesmo assim, as notificações continuavam com o mesmo erro, pelo que, decidiu-se realizar uma pesquisa e percebeu-se que \textit{Android} possui um novo sistema para os \textit{icons} das notificações, deste modo, é necessário transformar estes em preto ou branco com fundo transparente e de seguida tratados pelo próprio \textit{Android}.

Sendo assim foi realizada a transformação e novamente alterados os \textit{icons} das notificações. Após um novo teste compreendeu-se que mesmo assim apenas o \textit{icon} da notificação expandida teria sido alterado. Em seguida a uma nova pesquisa percebeu-se que \textit{Android} necessita de dois \textit{icons} de notificação para aplicar a ambas as situações, tendo sido resolvido o problema em questão.

Nesta implementação apenas um problema continuou sem resolução, a abertura de notificações quando a aplicação encontra-se terminada. Aqui foram testadas várias soluções, mas após a leitura da documentação e das soluções de outros utilizadores, compreendeu-se que o \textit{Flutter} não permite a reconfiguração do comportamento das notificações quando a aplicação está terminada. O \textit{Flutter} possui como futura implementação a permissão de reconfiguração do comportamento do \textit{click} neste tipo de notificação, mas, de momento não dispõe de solução.