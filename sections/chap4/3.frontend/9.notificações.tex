\subsection{Notificações}

A implementação de notificações revelou-se ser a implementação de maior dificuldade devido a diversos imprevistos da mesma. Para isto foi utilizado o serviço de notificações do \textit{Firebase}, este surge assim como os \textit{links}, em 2 secções, primeiramente \textit{backend} e de seguida \textit{frontend}.

A nível de \textit{backend} foi necessário utilizar \textit{axios} para realizar um pedido ao serviço de notificações do \textit{Firebase}, mas assim como na criação dos \textit{links} o serviço não indica quaisquer informações sobre erros, apenas o dispositivo poderá ou não receber a notificação. 

Em primeiro lugar, foi utilizado o conteúdo a enviar indicado pela documentação do serviço, mas apenas retornava uma mensagem de erro, de seguida foi pesquisado outras implementações de outros utilizadores e testado, mas novamente surgia um erro, pelo que foi decidido utilizar o conteúdo indicado pelo professor de programação de dispositivos móveis, tendo este funcionado sem qualquer indicação de erro. 

No conteúdo da notificação é enviado em formato \textit{json} a mensagem a mostrar na notificação e como estas serão sempre referentes a tópicos ou comentários, então é indicado o \textit{id} do tópico, do comentário e em caso de necessidade o \textit{id} do comentário pai.

Este processo de notificação foi também aproveitado para direcionar os mesmos dados para notificação de \textit{email} em caso do utilizador possuir ativo as notificações por \textit{email}, sendo gerado um \textit{link} dinâmico com os dados na notificação.

A nível de \textit{frontend} foi importada a biblioteca do serviço de notificações do \textit{Firebase}, sendo esta implementada conforme a documentação. Uma vez que é necessário detetar notificações no iniciar da aplicação, durante a utilização e quando esta se encontra em segundo plano, foi aproveitado estas deteções para implementar o direcionamento do utilizador para as páginas referentes às notificações, utilizando os dados recebidos.

O grande problema encontrado com as notificações é que o \textit{icon} não era mostrado corretamente sendo que ou era mostrado um quadrado escuro, ou nenhum \textit{icon}. A biblioteca de notificações do \textit{Firebase} não permite a customicação do \textit{icon} da mesma, pelo que foi decidido utilizar a biblioteca \textit{flutter\_local\_notifications}, esta permite a total customização das notificações sendo então enviado o icon desenhado para a aplicação. Mesmo assim as notificações continuavam com o mesmo erro, pelo que foi decidido realizar uma pesquisa e foi percebido que \textit{android} possui um novo sistema para os \textit{icons} das notificações pelo que é necessário transformar estes em preto ou branco com fundo transparente e de seguida tratados pelo próprio \textit{android}.

Sendo assim foi realizada a transformação e novamente alterados os \textit{icons} das notificações, após uma nova testagem foi percebido que mesmo assim apenas o \textit{icon} da notificação expandida teria sido alterado, após uma nova pesquisa foi percebido que \textit{android} necessita de 2 \textit{icons} de notificação para aplicar a ambas as situações, tendo sido assim resolvido o problema em questão.

Nesta implementação apenas um problema continuou sem resolução, a abertura de notificações quando a aplicação se encontra terminada, foram testadas várias soluções, mas após a leitura de documentação e de soluções de outros utilizadores, foi percebido que o \textit{Flutter} não permite a reconfiguração do comportamento de notificações quando a aplicação se encontra terminada. O \textit{Flutter} possui como futura implementação a permissão de reconfiguração do comportamento do \textit{click} neste tipo de notificação, mas de momento não dispõe de solução.