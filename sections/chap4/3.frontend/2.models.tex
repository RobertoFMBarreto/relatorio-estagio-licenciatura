\subsection{Extensions}
A linguagem de programação \textit{dart}, assim como outras linguagens orientadas a objetos, permite alterar e adicionar métodos aos objetos base da linguagem. Para realizar estas ações foram criadas extensões do objeto, neste caso, uma extensão para o objeto \textit{string} que permite capitalizar um texto.

\subsection{Handlers}
Os \textit{handlers} são porções de código que possibilitam a execução de ações perante um evento, como por exemplo, a realização de uma ação num clique no ecrã. Neste caso, os \textit{handlers} foram utilizados para detetar o estado da aplicação e realizar ações diante destes estados. Os estados da aplicação referem-se a se a aplicação se encontra em primeiro plano, segundo plano, a resumir ou então desligada. Com estes \textit{handlers} é possível alterar o funcionamento da aplicação de acordo com estes estados, como por exemplo, tratar de notificações da aplicação.

\subsection{Providers}
Os \textit{providers} são classes criadas para ajudar com comunicações externas, neste caso chamadas à \textit{API}. Estes foram desenvolvidos conforme os diversos modelos de dados que se recebem, como por exemplo, uma tópico do fórum. Um \textit{provider} oferece, na presença de um modelo, um conjunto de métodos para as diferentes chamadas necessárias à \textit{API}. Como o exemplo anterior, numa tópico existem métodos para eliminar, adicionar, editar e obter dados, sendo que, cada método terá os seus requisitos do serviço que invoca.

Estes \textit{providers} automaticamente detetam a linguagem da aplicação e realizam o pedido ao serviço com a utilização dessa linguagem.

\subsection{Helpers}
Os \textit{helpers}, como o próprio nome indica, são classes que ajudam com a realização de tarefas. Neste caso, os \textit{helpers} foram utilizados para o auxílio de mensagens de notificações. Estas mensagens deveriam conter o nome do utilizador e a ação, traduzida na linguagem da aplicação, pelo que, foi criada a classe de \textit{helper} de notificação, que contém um método estático para obter a mensagem de uma notificação segundo a sua ação.