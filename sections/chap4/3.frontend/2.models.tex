\subsection{Extensions}
A linguagem de programação dart, assim como outras linguagens de programação orientadas as objetos permite alterar e adicionar métodos aos objetos base da linguagem, para realizar isso são criadas extensões do objeto, neste caso foi criada uma extensão para o objeto string de forma a facilmente capitalizar um texto.

\subsection{Handlers}
Os handlers são porções de código que permitem a execução de código perante um evento como por exemplo realizar uma ação perante um clique no ecrã, neste caso os handlers foram utilizados para detetar o estado da aplicação e realizar ações perante estes estados. Os estados da aplicação referem-se a se a aplicação se encontra em primeiro plano, segundo plano, a resumir ou então desligada. Com estes handlers é possível alterar o funcionamento da aplicação perante estes estados, como por exemplo tratar de notificações da aplicação.

\subsection{Providers}
Os providers são classes criadas para ajudar com comunicações externas, neste caso chamdas à API, estes providers foram criados de acordo com os diversos modelos de dados que se recebem, como por exemplo, uma publicação do forum. Um provider oferece perante um modelo um conjunto de métodos para as diferentes chamadas necessárias à API, utilizando o exemplo anterior, para uma publicação existem métodos para eliminar, adicionar, editar e obter dados, sendo que acada método terá os seus requisitos do serviço que invocam.

Estes providers automaticamente detetam a linguagem da aplicação e realizam o pedido ao serviço utilizando essa linguagem.

\subsection{Helpers}
Os helpers como o próprio nome indica são classes que ajudam com a realização de uma tarefa, neste caso os helpers, foram utilizados no auxílio de mensagens de notificações, estas mensagens deveriam conter o nome do utilizador e também a ação do mesmo traduzida na linguagem da aplicação, sendo assim foi criada a classe de helper de notificação que contém um método estático para obter a mensagem de uma notificação de acordo com a ação da mesma.