\subsection{Ios}
Após o desenvolvimento completo da aplicação para dispositivos \textit{android}, foi proposto pela Motorline testar como o esta se comportava em ambiente \textit{ios}, para isso, esta disponibilizou acesso a um dispositivo móvel \textit{apple}, um computador, uma conta de desenvolvedor e um colega de trabalho que já tinha a experiência de desenvolvimento \textit{ios} com \textit{flutter}.

Para a compilação primeiramente foi necessário configurar a ferramenta \textit{XCode}, esta configuração foi realizada em conjunto com o colega de trabalho.

Após o processo de configuração, o projeto foi compilado para \textit{ios}, nesta primeira compilação toda a aplicação funcionava corretamente, sendo apontado pelo colega de trabalho algumas configurações de \textit{design} comuns em \textit{ios}, como por exemplo, os botões de cancelar e confirmar se encontrarem no topo do ecrã e também a explicação de como lidar com navegação, uma vez que, \textit{ios} não dispõe de função de voltar para trás.

Para resolver o problema de navegação foram acrescentados botões de navegação em todas as páginas necessárias. Já para implementar a sugestão dos botões de cancelar e confirmar foi então declarado que se o dispositivo em que a aplicação está a correr for um dispositivo \textit{apple}, então estes botões estarão no topo da página, caso contrário ficarão no fundo.

Após esta implementação foi decidido testar as notificações do sistema, pelo que foi percebido que estas não funcionavam. Para isto foi pesquisado qual seria a possível fonte do problema sendo detetado que as permissões poderiam não estar configuradas. Para realizar a configuração das mesmas foi necessário atribuir as permissões de \textit{android} para ios, mas sendo estas agora implementadas através de um ficheiro com o nome de \textit{info.plist}.

Após esta adição a aplicação foi compilada novamente, mas mesmo assim as notificações não funcionavam pelo que foi indicado pelo colega de trabalho que no envio do pedido de notificações para o \textit{Firebase} é necessário indicar as configurações de \textit{ios} também, e foi então procurado na documentação como se enviava as configurações de \textit{ios} e com a utilização destas, foi então alterado o pedido de notificações e de links dinâmicos, uma vez que, é um pedido do mesmo estilo, evitando assim problemas futuros.

Após esta resolução foi testado novamente, mas mesmo assim o problema persistia e após alguma pesquisa foi percebido que as notificações e \textit{links} têm também de ser configurados no \textit{XCode}, após esta configuração foi novamente testado e todas as notificações, \textit{links} de \textit{emails} e permissões funcionaram como planeado.