\subsection{Ios}
Após o desenvolvimento completo da aplicação para dispositivos android, foi proposto pela Motorline testar como o esta se comportava em ambiente ios, para isso esta disponibilizou acesso a um dispositivo móvel apple, um computador, uma conta de desenvolvedor e um colega de trabalho que já tinha a experiência de desenvolvimento ios com flutter.

Para a compilação primeiramente foi necessário configurar a ferramenta XCode, esta configuração foi realizada em conjunto com o colega de trabalho.

Após o processo de configuração o projeto foi compilado para ios, nesta primeira compilação todas a aplicação funcionava corretamente, sendo apontado pelo colega de trabalho algumas configurações de design comuns em ios como por exemplo os botões de cancelar e confirmar se encontrarem no topo do ecrã e também a explicação de como lidar com navegação uma vez que ios não dispõem de função de voltar para trás.

Para resolver o problema de navegação foram acrescentados botões de navegação para trás em todas as páginas que provêm de páginas principais. Já para implementar a sugestão dos botões de cancelar e confirmar foi então declarado que se o dispositivo em que a aplicação está a correr for um dispositivo apple então estes botões estarão no topo da página, caso contrário ficarão no fundo.

Após esta implementação foi decidio testar as notificações do sistema, pelo que foi percebido que estas não funcionavam. Para isto foi pesquisado qual seria a possível fonte do problema pelo que foi detetado que as permissões poderiam não estar configuradas, para realizar a configuração das mesmas foi necessário atribuir as mesmas permissões de android para ios, mas sendo estas agora implementadas através de um ficheiro com o nome de info.plist.

Após esta adição a aplicação foi compilada novamente, mas mesmo assim as notificações não funcionavam pelo que foi indicado pelo colega de trabalho que no envio do pedido de notificações para o firebase é necessário indicar as configurações de ios também, e foi então procurado na documentação como se enviava as configurações de ios e utilizando estas foi então alterado o pedido de notificações e também o pedido de links dinámicos, uma vez que é um pedido do mesmo estilo, evitando assim problemas futuros.

Após esta resolução foi testado novamente, mas mesmo assim o problema persistia e após alguma pesquisa foi percebido que as notificações e links têm também de ser configurados no XCode, após esta configuração foi novamente testado e todas as notificações, links de emails e permissões funcionaram como planeado.