\subsection{Ios}
Após o desenvolvimento completo da aplicação para dispositivos \textit{Android}, foi proposto pela Motorline testar como o esta se comportava num ambiente \textit{iOS}, para isso, esta disponibilizou acesso a um dispositivo móvel \textit{apple}, um computador, uma conta de programador e um colega de trabalho que já possuía experiência de desenvolvimento \textit{iOS} com \textit{flutter}.

Para a compilação, primeiramente foi necessário configurar a ferramenta \textit{XCode}. Esta configuração foi realizada em conjunto com o colega de trabalho.

Depois do processo de configuração, o projeto foi compilado para \textit{iOS}. Nesta primeira compilação toda a aplicação funcionava corretamente, mas, o colega de trabalho em questão indicou algumas configurações de \textit{design} comuns em \textit{iOS}, como por exemplo, os botões de cancelar e confirmar se encontrarem no topo do ecrã e a explicação de como lidar com navegação, uma vez que, \textit{iOS} não dispõe de função de voltar para trás. Outros problemas encontrados foram as notificações e o \textit{links} de aplicação.

Para a resolução do problema de navegação foram acrescentados botões de navegação em todas as páginas necessárias. Já para implementar a sugestão dos botões de cancelar e confirmar foi declarado que se o dispositivo em que a aplicação está a correr for um dispositivo \textit{apple}, estes botões estarão no topo da página, caso contrário, ficarão no fundo.

A resolução do problema de notificações proveio de uma pesquisa sobre qual seria a possível fonte do problema, sendo detetado que as permissões poderiam não estar configuradas. Para realizar a configuração foi necessário atribuir as permissões de \textit{Android} para \textit{iOS}, contudo, estas foram implementadas através de um ficheiro com o nome de \textit{info.plist}.

Posteriormente a esta adição, o colega de trabalho indicou que no envio do pedido de notificações para o \textit{Firebase} é necessário também indicar as configurações de \textit{iOS}. Deste modo foi procurado na documentação como se envia as configurações de \textit{iOS}. Com a utilização destas configurações foi alterado o pedido de notificações e de links dinâmicos, visto que, é um pedido do mesmo estilo.

Para além das alterações anteriores foi necessário também configurar  as notificações e \textit{links} no \textit{XCode}. Depois desta configuração foi testado e todas as notificações, \textit{links} de \textit{emails} e permissões funcionaram como planeado.