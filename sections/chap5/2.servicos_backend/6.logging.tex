\subsection{Logging}
Logging é um processo que permite guardar informação(logs) sobre um evento. Neste contexto logging poderá ser utilizado para realizar a monitorização de pedidos e/ou monitorização de utilização de recursos do software através da análise de pedidos. Estas informações poderão até auxiliar na toma de decisões sobre o software e em quais funcionalidades deste software colocar mais atenção.

Neste projeto logging foi aplicado sobre os pedidos recebidos, assim como também os erros registados, pois uma vez que os erros são tratados, uma dificuldade encontrada foi a identificação dos erros, para resolver este problema foi então decidido que sempre que um erro que não é customizado é detetado, é registado um log, este log contém informações sobre o pedido, data e hora do pedido, dados recebidos e assim como também a descrição original do erro. Esta implementação permite assim realizar monitorização de erros auxiliando assim na identificação dos serviços mais problemáticos e para quais serviços dirigir mais recursos.

\subsubsection{Morgan}
Morgan é uma ferramenta que permite extrair dados de um pedido, assim como também a criação de logs, 
este atua como um middleware do servidor, recebendo qual o tipo de log a ser escrito, sendo estes tipos definidos pela ferramenta, neste caso foi utilizado o tipo combinado que permite obter todas as informações referentes ao pedido, este tipo de log recebe também a ligação ao ficheiro onde ecrever estes logs. Os principais dados obtidos pela ferramenta são a data e hora do pedido, o tipo de pedido, o serviço pedido, os dados recebidos, a resposta devolvida e também a descrição do sistema utilizado para realizar o pedido, com estes dados é possível saber que plataforma é mais utilizada no software, quais as horas de maior utilização e quais os serviços mais executados, estes dados permitem direcionar mais recursos para uma indicada plataforma e/ou serviço, assim como também escolher os melhores horários de manutenção dos servidores.