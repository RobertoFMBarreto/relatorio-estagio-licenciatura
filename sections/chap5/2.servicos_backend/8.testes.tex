\subsection{Testes de código}
Aquando o fim do desenvolvimento de cada serviço é necessário testar este de forma a verificar se a funcionalidade se encontra de acordo com o desejado e/ou se existem erros de código. Para realizar estes testes poderão ser utilizadas ferramentas de auxílio ou então poderão ser realizados manualmente. O grande problema de testes manuais é que são exaustivos devido a serem longos e propensos a erros, pelo que estes testes são realizados em menor escala do projeto. Para os testes deste projeto foram utilizadas ferramentas de auxílio, sendo as ferramentas escolhidas mocha e chai.

APRESENTAR O QUE É MOCHA E PORQUÊ DE SE TER ESCOLHIDO

APRESENTAR O QUE É CHAI E PORQUÊ DE SE TER ESCOLHIDO

A realização de estes de código foi muito importante pois com esta ferramenta foi possível encontrar erros de lógica de negócio bem como também erros de código tanto a nível da formulação de respostas como a nivel de código.



