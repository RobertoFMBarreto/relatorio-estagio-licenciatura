\subsubsection{Envio de emails}
Como mencionado na apresentação da ferramenta tabular, esta não permite a utilização de acentuação na escrita de emails, sendo assim estes erros necessitam de ser resolvidos. Para permitir que os dados dos emails sejam personalizaveis, como por exemplo dados de utilizador e link para clicar, estes emails são colocados dentro de métodos que recebem por parametro os dados personalizaveis, sendo estes então colocados dentro do html do email, este método por fim devolve em string o html do email a enviar.

Para o envio de emails é ncessario um servidor, sendo assim para vias de teste foi utilizado um servidor gratuito de email sendo este hospedado por DIZER QUEM HOSPEDAVA, após a fase de testes foi então alterado para o servidor de email da empresa permitindo assim que este email seja identificado como empresa Motorline.

A configuração do servidor de emails do nodemailer é realizada através de uma chamada ao objeto 
de servidor indicando as configurações do mesmo que estão no ficheiro .env, esta chamada ao servidor por sua vez devolve um objeto que fornece métodos para enviar emails, sendo este então criado e utilizado sempre que se deseja enviar emails no servidor.


De forma a evitar que sempre que se deseja enviar um email seja necessário indicar todos os dados, foi então criado um método que cria e devolve o objeto de servidor sempre que necessário. Sendo assim sempre que se deseja enviar um email é então primeiramente obtido o objeto do servidor, de seguida é invocado o método para obter o conteúdo html do email e por fim é utilizado o objeto do servidor para chamar o método de envio de emails no qual se terá de indicar o email do destinatário, o assunto e o conteúdo do email.
