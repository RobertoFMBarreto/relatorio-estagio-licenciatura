\subsection{Cifragem de configurações do servidor}
De forma a garantir um nivel de segurança maior foram realizadas pesquisas sobre as principais falhas de segurança no NodeJs, pelo que foi descoberto que as principais formas de ataque a esta ferramenta é o desenvolvimento de bibliotecas de malware e o ataque às bibliotecas com o objetivo de obter dados de acesso a servidores que se encontram nos ficheiros de configuração.

Por norma todas as configurações de servidores são colocadas num ficheiro env, este ficheiro no momento de iniciar o servidor é utilizado para carregar todas as variáveis para o ambiente do mesmo, sendo assim qualquer um com acesso ao ficheiro ou às variáveis de ambiente poderá ver todas as configurações de todos os servidores.

Para resolver este problema foi então decidido cifrar o ficheiro com as variáveis de ambiente e as próprias variaveis. Para realizar a cifragem foi utilizada a biblioteca secur-env, esta permite realizar a cifragem de um ficheiro indicando uma password. A password deverá ser indicada no processo de inicialização do servidor de forma a ser possível ao mesmo decifrar o ficheiro, sendo que a gestão das variáveis cifradas passa então a estar encarregue desta biblioteca. Para obter a password é pedido ao utilizador que vai iniciar o servidor para colocar uma password sendo esta usada para decifrar o ficheiro, caso esta esteja errada é então devolvido um erro. De forma a evitar que seja possível visualizar o histórico do terminal para obter a password, este processo é realizado utilizando a biblioteca readline-sync que permite pedir ao utilizador algum dado e indicar a funcionalidade de esconder os dados escritos.