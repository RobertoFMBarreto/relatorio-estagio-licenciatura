\subsubsection{Cifragem de passwords\\}
Aquando o registo de clientes é indicada a password, pelo que esta deverá ser guardada cifrada. Para isto é utilizada a biblioteca bcrypt, sendo que no processo de registo, antes do envio dos dados do utilizador para o servidor, é invocado o método de cifragem da password, indicando como salt o valor 8, após a execução é então obtida a password cifrada, sendo esta enviada para a base de dados com conjunto com os restantes dados.

\subsubsection{Cifragem de configurações do servidor\\}
Para a cifragem das configurações do servidor, foi então executado o comando de cifragem da biblioteca secure-env, sendo pedido a password da mesma, após esta indicação, foi criado o ficheiro cifrado, pelo que o ficheiro original deverá ser eliminado ou em caso de necessidade guardado em um local seguro de forma a evitar que durante algum ataque seja possível obter os dados do mesmo.

Após a cifragem do ficheiro o servidor foi configurado para aquando a sua inicialização, este deverá pedir para o utilizador indicar a password do mesmo, sendo esta a indicada na cifragem do ficheiro de configurações do servidor, para esta configuração foi utilizada a biblioteca readline-sync no modo de password. Quando o utilizador indica a password em caso de esta estar errada o servidor irá falhar, pois a biblioteca de cifragem irá tentar decifrar o ficheiro de configurações e irá falhar não concluindo o processo de inicialização, caso contrário, este continuará o processo de inicialização, sendo o primeiro passo a utilização da password para decifrar o ficheiro de configurações e carregar as variáveis de ambiente.