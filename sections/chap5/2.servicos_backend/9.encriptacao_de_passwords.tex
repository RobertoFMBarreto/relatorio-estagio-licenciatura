\subsection{Encriptação de passwords}
De forma a garantir a segurança das passwords dos utilizadores é necessário encriptar estas, a encriptação poderá ser feita manualmente ou com o auxílio de ferramentas, a grande diferença é que manualmente poderá não se obter uma cifra tão segura como com o auxílio de uma ferramenta, sendo assim foi decidido utilizar uma ferramenta para encriptar passwords, a ferramenta escolhida foi bcrypt, esta foi escolhida devido a ser vastamente utilizada e até ao momento sem problemas em relação à sua cifra sendo que não existem registos de ataques bem sucedidos a esta ferramenta. Esta ferramenta oferece um conjunto de métodos sendo tendo sido utilizados os métodos de cifra e de comparação. O método de cifra permite através de um valor indicar a complexidade a aplicar sobre a cifra INDICAR O RANGE DE VALORES E O UTILIZADO, sendo de seguida devolvida a password cifrada sendo esta guardada na base de dados. O método comparação permite comparar uma password cifrada com uma password sem cifra, devolvendo verdadeiro ou falso conforme as passwords sejam iguais ou não.