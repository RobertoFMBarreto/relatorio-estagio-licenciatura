\section{Diagrama Entidade Relação}

O software Install\&Go é suportado por uma base de dados relacional, 
sendo esta esquematizada tendo por base as necessidades do projeto.

\begin{figure}[htb]
    \centering
    
    \includegraphics[width=\textwidth]{images/diagramas/diagrama_bd.png}
    \caption{Diagrama Entidade Relação base de dados Install\&Go}
    \label{fig:20}
\end{figure}

\newpage

\subsection{Escolha de diagrama de entidade relação}
Durante o desenvolvimento do diagrama de entidade relação surgiu 
a opção de se separar as empresas dos seus técnicos em duas tabelas como 
exemplificado na Figura~\ref{fig:21}.O diagrama anteriormente 
mostrado foi escolhido devido à sua simplicidade, 
visto que no diagrama representado na Figura~\ref{fig:21} sempre que se
deseja, por exemplo, obter o utilizador que criou um tópico teria de se
verificar se o uid contido é de uma empresa ou de um técnico e apenas 
de seguida se obter o utilizador que criou o tópico, sendo este um exemplo
entre os demais do mesmo tipo. Tendo em conta este problema foi decido optar
pelo diagrama anteriormente demonstrado.

\begin{figure}[htb]
    \centering
    
    \includegraphics[width=\textwidth]{images/diagramas/diagrama_bd_alt.png}
    \caption{Diagrama Entidade Relação alternativo}
    \label{fig:21}
\end{figure}

\newpage

\subsection{Dicionário de termos}

De forma a ser possível entender o propósito de cada tabela e atributo 
foi então criado um dicionário de termos para a base de dados(Tabela X).


\definecolor{Concrete}{rgb}{0.952,0.952,0.952}
\begin{longtblr}
[
caption={Dicionário de termos da base de dados},
label={tab:22},
]{
  width = \linewidth,
  colspec = {Q[170]Q[292]Q[240]Q[215]},
  row{1} = {Concrete},
  column{1} = {c},
  cell{2}{1} = {r=19}{},
  cell{2}{2} = {r=19}{},
  cell{22}{1} = {r=2}{},
  cell{22}{2} = {r=2}{},
  cell{24}{1} = {r=2}{},
  cell{24}{2} = {r=2}{},
  cell{26}{1} = {r=2}{},
  cell{26}{2} = {r=2}{},
  cell{28}{1} = {r=7}{},
  cell{28}{2} = {r=7}{},
  cell{35}{1} = {r=10}{},
  cell{35}{2} = {r=10}{},
  cell{45}{1} = {r=8}{},
  cell{45}{2} = {r=8}{},
  cell{53}{1} = {r=2}{},
  cell{53}{2} = {r=2}{},
  cell{55}{1} = {r=3}{},
  cell{55}{2} = {r=3}{},
  cell{58}{1} = {r=4}{},
  cell{58}{2} = {r=4}{},
  cell{62}{1} = {r=11}{},
  cell{62}{2} = {r=11}{},
  vlines,
  hline{1-2,21-22,24,26,28,35,45,53,55,58,62,73} = {-}{},
  hline{3-20,23,25,27,29-34,36-44,46-52,54,56-57,59-61,63-72} = {3-4}{},
}
Tabela         & Descrição                                                                            & Atributos            & Descrição                                           \\
users          & Tabela encarregue de guardar todos os dados referentes aos utilizadores da aplicação & uid                  & Id do utilizador                                    \\
               &                                                                                      & company\_id          & Id da empresa referente ao técnico                  \\
               &                                                                                      & n\_contribuinte      & Número de contribuinte do utilizador                \\
               &                                                                                      & name                 & Nome do utilizador                                  \\
               &                                                                                      & email                & Email do utilizador                                 \\
               &                                                                                      & password             & Password do utilizador                              \\
               &                                                                                      & profile\_pic         & Imagem de perfil do utilizador                      \\
               &                                                                                      & verif\_code          & Código de verificação do utilizador                 \\
               &                                                                                      & is\_confirmed        & Verificação de se o código está confirmado          \\
               &                                                                                      & refresh\_token       & Token de refresh                                    \\
               &                                                                                      & is\_professional     & Verificação se é profissional                       \\
               &                                                                                      & has\_access          & Verificação se tem acesso à conta                   \\
               &                                                                                      & has\_email\_noti     & Verificação se ativou notificações de email         \\
               &                                                                                      & has\_push\_noti      & Verificação se ativou notificações push             \\
               &                                                                                      & email\_noti\_type    & Tipo de notificação de email                        \\
               &                                                                                      & push\_noti\_type     & Tipo de notificação push                            \\
               &                                                                                      & is\_deleted          & Verificação se a conta se encontra apagada          \\
               &                                                                                      & is\_certified        & Verificação se é um técnico certificado             \\
               &                                                                                      & is\_official         & Verificação se é um técnico oficial                 \\
black\_list    & Tabela que guarda os tokens a bloquear                                               & token                & Token a bloquear                                    \\
liked\_topics  & Tabela encarregue de guardar todos os tópicos gostados pelo utilizador               & uid                  & Id do utilizador                                    \\
               &                                                                                      & topic\_id            & Id do tópico                                        \\
liked\_answers & Tabela encarregue de guardar todas as respostas que receberam gosto do utilizador    & uid                  & Id do utilizador                                    \\
               &                                                                                      & answer\_id           & Id da resposta                                      \\
topic\_types   & Tabela encarregue de guardar os tipos de tópico existentes                           & type\_id             & Id do tipo de tópico                                \\
               &                                                                                      & name                 & Nome do tipo de tópico                              \\
notifications  & Tabela encarregue de guardar todas as notificações do técnico                        & noti\_id             & Id da notificação                                   \\
               &                                                                                      & receiver\_id         & Recetor da notificação                              \\
               &                                                                                      & sender\_id           & Emissor da notificação                              \\
               &                                                                                      & topic\_id            & Id do tópico em caso de estar referente a um tópico \\
               &                                                                                      & is\_deleted          & Verificação se a notificação está apagada           \\
               &                                                                                      & message              & Mensagem da notificação                             \\
               &                                                                                      & date                 & Data de emissão da notificação                      \\
forum\_topics  & Tabela encarregue de guardar todos os tópicos existentes na aplicação                & topic\_id            & Id do tópico                                        \\*
               &                                                                                      & uid                  & Id do dono do tópico                                \\*
               &                                                                                      & product\_id          & Produto referente ao tópico                         \\*
               &                                                                                      & tiype\_id            & Id do tipo referente ao tópico                      \\*
               &                                                                                      & creation\_date       & Data de criação do tópico                           \\*
               &                                                                                      & description          & Descrição do tópico                                 \\*
               &                                                                                      & title                & Título do tópico                                    \\*
               &                                                                                      & is\_complete         & Verificação se o tópico está finalizado             \\*
               &                                                                                      & is\_private          & Verificação se o tópico é privado                   \\*
               &                                                                                      & is\_deleted          & Verificação se o tópico está apagado                \\*
topic\_answers & Tabela encarregue de guardar todas as respostas a um tópico                          & answer\_id           & Id da resposta                                      \\
               &                                                                                      & topic\_id            & Id do tópico                                        \\
               &                                                                                      & uid                  & Id do dono da resposta                              \\
               &                                                                                      & creation\_date       & Data de criação da resposta                         \\
               &                                                                                      & description          & Descrição da resposta                               \\
               &                                                                                      & is\_best             & Verificação se é a melhor resposta                  \\
               &                                                                                      & parent\_id           & Id da resposta pai                                  \\
               &                                                                                      & is\_deleted          & Verificação se o topico se encontra apagado         \\
categories     & Tabela encarregue de guardar todas as categorias de produtos existentes              & category\_id         & Id da categoria                                     \\*
               &                                                                                      & name                 & Nome da categoria                                   \\*
subcategories  & Tabela encarregue de guardar as subcategorias de produtos existentes                 & subcategory\_id      & Id da subcategoria                                  \\*
               &                                                                                      & category\_id         & Id da categoria                                     \\*
               &                                                                                      & name                 & Nome da subcategoria                                \\*
cb\_manuals    & Tabela encarregue de guardar os manuais de utilização das placas de controlo         & cb\_manual\_id       & Id do manual                                        \\
               &                                                                                      & product\_id          & Id do produto                                       \\
               &                                                                                      & name                 & Nome da placa de controlo                           \\
               &                                                                                      & manual               & Url do manual                                       \\
products       & Tabela encarregue de guardar as informações dos produtos do catálogo da empresa      & product\_id          & Id do produto                                       \\
               &                                                                                      & name                 & Nome do produto                                     \\
               &                                                                                      & description          & Descrição do produto                                \\
               &                                                                                      & user\_manual         & Url do manual de utilização do produto              \\
               &                                                                                      & general\_information & Url da informação geral do produto                  \\
               &                                                                                      & technical\_draw      & Url do desenho técnico do produto                   \\
               &                                                                                      & install\_video       & Url do video de instalação do produto                \\
               &                                                                                      & program\_video       & Url do video de programação do dispositivo          \\
               &                                                                                      & category\_id         & Id da categoria de produto                          \\
               &                                                                                      & subcategory\_id      & Id da subcategoria de produto                       \\
               &                                                                                      & is\_featured         & Verificação se o produto é um destaque              
\end{longtblr}