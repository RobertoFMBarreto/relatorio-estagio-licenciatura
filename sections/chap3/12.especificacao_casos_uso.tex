\section{Casos de uso}
Para transformar as \textit{user stories} em ações e especificar todas as reações do sistema com o qual o ator interage foram desenvolvidos casos de uso \textit{\acrfull{uc}}.

% \usepackage{color}
% \usepackage{tabularray}
\definecolor{Concrete}{rgb}{0.952,0.952,0.952}
\definecolor{Gallery}{rgb}{0.937,0.937,0.937}
\begin{longtblr}
[
caption={Tabela de casos de uso},
label={tab:4},
]
{
  width = \linewidth,
  colspec = {Q[130]Q[81]Q[120]Q[242]Q[375]},
  row{1} = {Concrete},
  row{2} = {Concrete,c},
  row{4} = {Concrete,c},
  row{8} = {Concrete,c},
  row{10} = {Concrete,c},
  row{18} = {Concrete,c},
  row{23} = {Concrete,c},
  row{26} = {Concrete,c},
  row{28} = {Concrete,c},
  row{31} = {Concrete,c},
  column{2} = {c},
  column{3} = {c},
  cell{1}{1} = {c},
  cell{2}{1} = {c=5}{0.936\linewidth},
  cell{4}{1} = {c=5}{0.936\linewidth},
  cell{8}{1} = {c=5}{0.936\linewidth},
  cell{10}{1} = {c=5}{0.936\linewidth},
  cell{18}{1} = {c=5}{0.936\linewidth},
  cell{23}{1} = {c=5}{0.936\linewidth},
  cell{26}{1} = {c=5}{0.936\linewidth},
  cell{28}{1} = {c=5}{0.936\linewidth},
  cell{31}{1} = {c=5}{0.936\linewidth},
  hlines,
  vlines,
}
\#                         & User Story         & Ator       & Nome                                & Descrição                                                   \\
Criar tópico               &                    &            &                                     &                                                             \\
UC 1.0                     & US15               & Técnico    & Criar novo tópico                   & Criação de um novo tópico no fórum                          \\
Pesquisa de tópicos        &                    &            &                                     &                                                             \\
UC 1.1                     & US11               & Técnico    & Pesquisar tópicos específicos       & Pesquisar por tópicos no fórum                              \\
UC 1.1.1                   & US12 e US13        & Técnico    & Pesquisa escrita                    & Pesquisar tópicos por assunto                               \\
UC 1.1.2                   & US14               & Técnico    & Pesquisa por código QR              & Pesquisar tópicos referentes a um produto                   \\
Listagens de tópicos       &                    &            &                                     &                                                             \\
UC 1.2                     & US05               & Técnico    & Ver tópicos                         & Ver listagens de tópicos do fórum                           \\
Detalhes de tópico         &                    &            &                                     &                                                             \\
UC 1.3                     & US08               & Técnico    & Selecionar tópico                   & Ver detalhes de um tópico selecionado                       \\
UC 1.3.1                   & US21               & Técnico    & Finalizar tópico                    & Finalizar um tópico para indicar que está respondido        \\
UC 1.3.2                   & US20               & Técnico    & Selecionar melhor resposta          & Selecionar a melhor resposta do tópico                      \\
UC 1.3.3                   & US22               & Técnico    & Eliminar tópico                     & Eliminar um tópico do fórum                                 \\
UC 1.3.4                   & US23               & Técnico    & Alterar visibilidade do tópico      & Alterar a visibilidade de um tópico entre publico e privado \\
UC 1.3.5                   & US28               & Técnico    & Comentar o tópico                   & Comentar um tópico                                          \\
UC 1.3.6                   & US25               & Técnico    & Gostar de tópico                    & Gostar de um tópico                                         \\
Comentários                &                    &            &                                     &                                                             \\
UC 1.3.7                   & US24               & Técnico    & Ver comentários                     & Ver comentários do tópico                                   \\
UC 1.3.7.1                 & US27               & Técnico    & Apagar comentário                   & Apagar comentário de um tópico                              \\
UC 1.3.7.2                 & US29               & Técnico    & Responder a comentário              & Responder a um comentário de um tópico                      \\
UC 1.3.7.3                 & US31               & Técnico    & Gostar de comentário                & Gostar de um comentário                                     \\
Ativação de conta          &                    &            &                                     &                                                             \\
UC 1.4                     & -                  & Técnico    & Ativação de conta                   & Ativar conta de cliente                                     \\
UC 1.4.1                   & US04               & Técnico    & Pedir reenvio de código de ativação & Pedir reenvio de email de código de ativação                \\
Perfil                     &                    &            &                                     &                                                             \\
UC 1.5                     & US31 - US32 - US33 & Técnico    & Ver Perfil                          & Ver perfil de utilizador                                    \\
Notificações               &                    &            &                                     &                                                             \\
UC1.6                      & US34 e US36        & Técnico    & Ver notificações                    & Ver todas as notificações                                   \\
UC1.7                      & US34 e US35        & Técnico    & Configuração de notificações        & Configurar o modo e tipo de notificações a receber          \\
Gestão de recursos humanos &                    &            &                                     &                                                             \\
UC1.8                      & US37               & Empresa    & Registar Técnico                    & Registar conta de técnico da empresa                        \\
UC1.9                      & US38               & Empresa    & Impedir acesso a técnico            & Registar conta de técnico da empresa                        \\
UC1.10                     & US39               & Empresa    & Remover conta de técnico            & Registar conta de técnico da empresa                        
\end{longtblr}
\newpage

\subsection{Especificação de casos de uso}

Para demonstrar todas as interações entre os atores e o sistema, assim como todas as ações e fluxos possíveis, foram realizadas especificações de casos de uso.

\subsubsection{Especificação de caso de uso de listagem de tópicos}

Através da listagem de tópicos é possível visualizar todas as listagens que o utilizador poderá visualizar, sendo que, o técnico oficial consegue para além destas listagens, ver os seus tópicos e os tópicos privados.

%\definecolor{Concrete}{rgb}{0.952,0.952,0.952}
\begin{longtblr}
[
caption={Tabela de especificação de caso de uso de listagem de tópicos do utilizador},
label={tab:8},
]
{
  width = \linewidth,
  colspec = {Q[225]Q[331]Q[383]},
  row{6} = {Concrete},
  cell{1}{1} = {Concrete},
  cell{1}{2} = {c=2}{0.714\linewidth},
  cell{2}{1} = {Concrete},
  cell{2}{2} = {c=2}{0.714\linewidth},
  cell{3}{1} = {Concrete},
  cell{3}{2} = {c=2}{0.714\linewidth},
  cell{4}{1} = {Concrete},
  cell{4}{2} = {c=2}{0.714\linewidth,c},
  cell{5}{1} = {Concrete},
  cell{5}{2} = {c=2}{0.714\linewidth,c},
  cell{6}{2} = {c},
  cell{6}{3} = {c},
  cell{7}{1} = {r=2}{Concrete},
  cell{9}{1} = {r=2}{Concrete},
  cell{11}{1} = {r=2}{Concrete},
  vlines,
  hline{1-7,9,11,13} = {-}{},
  hline{8,10,12} = {2-3}{},
}
Caso de Uso           & Ver listagem de tópicos                                               &                                  \\
Descrição             & Ver a listagem de tópicos existentes no fórum por diversas categorias &                                  \\
Ator                  & Utilizador                                                            &                                  \\
Pré-condição          & -                                                                     &                                  \\
Pós-condição          & -                                                                     &                                  \\
                      & Ator                                                                  & Sistema                          \\
Fluxo Principal       & 1-Ver tópicos populares                                               &                                  \\
                      &                                                                       & 2-Lista de tópicos populares     \\
Fluxo Alternativo(A1) & 1-Ver tópicos mais recentes                                           &                                  \\
                      &                                                                       & 2-Lista de tópicos mais recentes \\
Fluxo Alternativo(A2) & 1-Ver tópicos por responder                                           &                                  \\
                      &                                                                       & 2-Lista de tópicos por responder 
\end{longtblr}

\definecolor{Concrete}{rgb}{0.952,0.952,0.952}
\begin{longtblr}
[
caption={Tabela de especificação de caso de uso de listagem de tópicos do técnico},
label={tab:5},
]
{
  width = \linewidth,
  colspec = {Q[225]Q[331]Q[383]},
  row{6} = {Concrete},
  cell{1}{1} = {Concrete},
  cell{1}{2} = {c=2}{0.714\linewidth},
  cell{2}{1} = {Concrete},
  cell{2}{2} = {c=2}{0.714\linewidth},
  cell{3}{1} = {Concrete},
  cell{3}{2} = {c=2}{0.714\linewidth},
  cell{4}{1} = {Concrete},
  cell{4}{2} = {c=2}{0.714\linewidth,c},
  cell{5}{1} = {Concrete},
  cell{5}{2} = {c=2}{0.714\linewidth,c},
  cell{6}{2} = {c},
  cell{6}{3} = {c},
  cell{7}{1} = {r=2}{Concrete},
  cell{9}{1} = {r=2}{Concrete},
  cell{11}{1} = {r=2}{Concrete},
  cell{13}{1} = {r=2}{Concrete},
  vlines,
  hline{1-7,9,11,13,15} = {-}{},
  hline{8,10,12,14} = {2-3}{},
}
Caso de Uso           & Ver listagem de tópicos                                               &                                  \\
Descrição             & Ver a listagem de tópicos existentes no fórum por diversas categorias &                                  \\
Ator                  & Técnico                                                            &                                  \\
Pré-condição          & -                                                                     &                                  \\
Pós-condição          & -                                                                     &                                  \\
                      & Ator                                                                  & Sistema                          \\
Fluxo Principal       & 1-Ver tópicos populares                                               &                                  \\
                      &                                                                       & 2-Lista de tópicos populares     \\
Fluxo Alternativo(A1) & 1-Ver tópicos mais recentes                                           &                                  \\
                      &                                                                       & 2-Lista de tópicos mais recentes \\
Fluxo Alternativo(A2) & 1-Ver meus tópicos                                                    &                                  \\
                      &                                                                       & 2-Lista de tópicos do técnico    \\
Fluxo Alternativo(A3) & 1-Ver tópicos por responder                                           &                                  \\
                      &                                                                       & 2-Lista de tópicos por responder 
\end{longtblr}

\newpage

\subsubsection{Especificação de caso de uso de criar novo tópico}

Aquando a criação de um tópico, um técnico poderá realizar diversas ações sendo que obrigatoriamente terá de indicar o título, descrição e tipo de tópico, para além desta informação o técnico poderá anexar imagens, referenciar um produto e indicar a visibilidade.

\definecolor{Concrete}{rgb}{0.952,0.952,0.952}
\begin{longtblr}
[
caption={Tabela de especificação de caso de uso login},
label={tab:6},
]{
 width = \linewidth,
 colspec = {Q[212]Q[360]Q[369]},
 row{6} = {Concrete},
 cell{1}{1} = {Concrete},
 cell{1}{2} = {c=2}{0.725\linewidth},
 cell{2}{1} = {Concrete},
 cell{2}{2} = {c=2}{0.725\linewidth},
 cell{3}{1} = {Concrete},
 cell{3}{2} = {c=2}{0.725\linewidth},
 cell{4}{1} = {Concrete},
 cell{4}{2} = {c=2}{0.725\linewidth},
 cell{5}{1} = {Concrete},
 cell{5}{2} = {c=2}{0.725\linewidth,c},
 cell{6}{2} = {c},
 cell{6}{3} = {c},
 cell{7}{1} = {r=10}{Concrete,c},
 cell{17}{1} = {Concrete},
 cell{18}{1} = {r=6}{Concrete},
 vlines,
 hline{1-7,17-18,24} = {-}{},
 hline{8-16,19-23} = {2-3}{},
}
Caso de Uso      & Criar novo tópico           &                    \\
Descrição       & Criação de um novo tópico no fórum   &                    \\
Ator         & Técnico                &                    \\
Pré-condição     & Clicar em adicionar novo tópico    &                    \\
Pós-condição     & -                   &                    \\
           & Ator                  & Sistema                \\
Fluxo Principal    & 1-Indicar o título do tópico      &                    \\
           & 2-Indicar a descrição do tópico    &                    \\
           & 3-Indicar se o tópico é privado    &                    \\
           & 4-Indicar o tipo do tópico       &                    \\
           & 5-Indicar o produto referido no tópico &                    \\
           & 6-Adicionar imagens de anexo      &                    \\
           & 7-Confirmar a criação do tópico    &                    \\
           &                    & 8-Verificar se titulo está inserido  \\
           &                    & 9-Verificar se descrição está inserida \\
           &                    & 10-Inserir novo tópico no fórum    \\
Fluxo Alternativo(A1) & 1-Cancelar a criação do tópico     &                    \\
Fluxo Alternativo(A2) & 1-Indicar o título do tópico      &                    \\*
           & 2-Indicar se o tópico é privado    &                    \\*
           & 3-Confirmar a criação do tópico    &                    \\*
           &                    & 4-Verificar se titulo está inserido  \\*
           &                    & 5-Verificar se descrição está inserida \\*
           &                    & 6-Erro descrição em falta     
\end{longtblr}

\newpage

\subsubsection{Especificação de caso de uso de pesquisar tópicos por escrito}

Assim que um técnico deseje pesquisar por um assunto específico de tópico este poderá realizar uma pesquisa escrita onde conseguirá realizar filtragem por tipo e categoria de tópico.

% \usepackage{color}
% \usepackage{tabularray}
\definecolor{Concrete}{rgb}{0.952,0.952,0.952}
\begin{longtblr}
[
caption={Tabela de especificação de caso de uso de pesquisa por escrito},
label={tab:6},
]{
  width = \linewidth,
  colspec = {Q[190]Q[217]Q[533]},
  row{6} = {Concrete},
  cell{1}{1} = {Concrete},
  cell{1}{2} = {c=2}{0.702\linewidth},
  cell{2}{1} = {Concrete},
  cell{2}{2} = {c=2}{0.702\linewidth},
  cell{3}{1} = {Concrete},
  cell{3}{2} = {c=2}{0.702\linewidth},
  cell{4}{1} = {Concrete},
  cell{4}{2} = {c=2}{0.702\linewidth},
  cell{5}{1} = {Concrete},
  cell{5}{2} = {c=2}{0.702\linewidth,c},
  cell{6}{2} = {c},
  cell{6}{3} = {c},
  cell{7}{1} = {r=4}{Concrete},
  cell{11}{1} = {r=2}{Concrete},
  vlines,
  hline{1-7,11,13} = {-}{},
  hline{8-10,12} = {2-3}{},
}
Caso de Uso           & Pesquisa por escrita           &                                            \\
Descrição             & Pesquisar por tópicos no fórum &                                            \\
Ator                  & Utilizador                     &                                            \\
Pré-condição          & Selecionar pesquisa de fórum   &                                            \\
Pós-condição          & -                              &                                            \\
                      & Ator                           & Sistema                                    \\
Fluxo Principal       & 1-Pesquisar assunto            &                                            \\
                      &                                & 2-Lista de tópicos do assunto              \\
                      & 3-Filtrar por tipo             &                                            \\
                      &                                & 3-Filtragem de tópicos do assunto por tipo \\
Fluxo Alternativo(A1) & 1-Pesquisar assunto            &                                            \\
                      &                                & 2-Lista de tópicos do assunto              
\end{longtblr}


\subsubsection{Especificação de caso de uso de ver finalizar tópico}

Quando um técnico encontra-se satisfeito com a solução do problema este poderá indicar que o tópico está finalizado, o que é sinalizado para outros técnicos.

% \usepackage{color}
% \usepackage{tabularray}
\definecolor{Concrete}{rgb}{0.952,0.952,0.952}
\begin{table}[htb]
\centering
\label{tab:8}
\caption{Tabela de especificação de caso de uso de finalizar tópico}
\begin{tblr}{
  width = \linewidth,
  colspec = {Q[254]Q[319]Q[365]},
  row{6} = {Concrete},
  cell{1}{1} = {Concrete},
  cell{1}{2} = {c=2}{0.683\linewidth},
  cell{2}{1} = {Concrete},
  cell{2}{2} = {c=2}{0.683\linewidth},
  cell{3}{1} = {Concrete},
  cell{3}{2} = {c=2}{0.683\linewidth},
  cell{4}{1} = {Concrete},
  cell{4}{2} = {c=2}{0.683\linewidth},
  cell{5}{1} = {Concrete},
  cell{5}{2} = {c=2}{0.683\linewidth},
  cell{6}{2} = {c},
  cell{6}{3} = {c},
  cell{7}{1} = {r=2}{Concrete},
  cell{9}{1} = {Concrete},
  cell{9}{2} = {c},
  cell{9}{3} = {c},
  vlines,
  hline{1-7,9-10} = {-}{},
  hline{8} = {2-3}{},
}
Caso de Uso           & Finalizar tópico                                     &                                  \\
Descrição             & Finalizar um tópico para indicar que está respondido &                                  \\
Ator                  & Técnico                                              &                                  \\
Pré-condição          & Clicar no tópico desejado                            &                                  \\
Pós-condição          & Alterar tópico para finalizado                       &                                  \\
                      & Ator                                                 & Sistema                          \\
Fluxo Principal       & 1-Clicar em finalizar tópico                         &                                  \\
                      &                                                      & 2-Alterar tópico para finalizado \\
Fluxo Alternativo(A1) & -                                                    & -                                
\end{tblr}
\end{table}

\newpage

\subsubsection{Especificação de caso de uso de selecionar melhor resposta}

Sempre que o técnico encontrar uma resposta no seu tópico que se destaca na solução da sua questão, este poderá colocar esta resposta como a melhor resposta do tópico. Caso já exista uma melhor resposta, esta automáticamente é removida e a nova é colocada como melhor resposta.

% \usepackage{color}
% \usepackage{tabularray}
\definecolor{Concrete}{rgb}{0.952,0.952,0.952}
\begin{table}[htb]
\centering
\label{tab:9}
\caption{Tabela de especificação de caso de uso de selecionar melhor resposta}
\begin{tblr}{
 width = \linewidth,
 colspec = {Q[181]Q[235]Q[525]},
 row{6} = {Concrete},
 cell{1}{1} = {Concrete},
 cell{1}{2} = {c=2}{0.76\linewidth},
 cell{2}{1} = {Concrete},
 cell{2}{2} = {c=2}{0.76\linewidth},
 cell{3}{1} = {Concrete},
 cell{3}{2} = {c=2}{0.76\linewidth},
 cell{4}{1} = {Concrete},
 cell{4}{2} = {c=2}{0.76\linewidth},
 cell{5}{1} = {Concrete},
 cell{5}{2} = {c=2}{0.76\linewidth},
 cell{6}{2} = {c},
 cell{6}{3} = {c},
 cell{7}{1} = {r=3}{Concrete},
 cell{10}{1} = {r=4}{Concrete},
 cell{14}{1} = {r=4}{Concrete},
 vlines,
 hline{1-7,10,14,18} = {-}{},
 hline{8-9,11-13,15-17} = {2-3}{},
}
Caso de Uso      & Selecionar melhor resposta       &                                 \\
Descrição       & Selecionar a melhor resposta do tópico &                                 \\
Ator         & Técnico                 &                                 \\
Pré-condição     & Clicar no tópico desejado        &                                 \\
Pós-condição     & Alterar a resposta para melhor resposta &                                 \\
           & Ator                  & Sistema                             \\
Fluxo Principal    & 1-Clicar em melhor resposta       &                                 \\
           &                     & 2-Verificar se já existe uma melhor resposta - Não        \\
           &                     & 3- Colocar a resposta como melhor resposta do tópico       \\
Fluxo Alternativo(A1) & 1-Clicar em melhor resposta       &                                 \\
           &                     & 2-Verificar se já existe uma melhor resposta - Sim        \\
           &                     & 3- Verificar se a resposta existente é a mesma selecionada - Não \\
           &                     & 4-Alterar melhor resposta                    \\
Fluxo Alternativo(A2) & 1-Clicar em melhor resposta       &                                 \\
           &                     & 2-Verificar se já existe uma melhor resposta - Sim        \\
           &                     & 3- Verificar se a resposta existente é a mesma selecionada - Sim \\
           &                     & 4-Remover melhor resposta                    
\end{tblr}
\end{table}

\newpage

\subsubsection{Especificação de caso de uso de eliminar tópico}

O técnico sempre que desejar poderá eliminar o tópico, o que permite remover do fórum e não volta a ser apresentado.

% \usepackage{color}
% \usepackage{tabularray}
\definecolor{Concrete}{rgb}{0.952,0.952,0.952}
\begin{table}[htb]
\centering
\label{tab:12}
\caption{Tabela de especificação do caso de uso de eliminar tópico}
\begin{tblr}{
  width = \linewidth,
  colspec = {Q[290]Q[371]Q[273]},
  row{6} = {Concrete},
  cell{1}{1} = {Concrete},
  cell{1}{2} = {c=2}{0.644\linewidth},
  cell{2}{1} = {Concrete},
  cell{2}{2} = {c=2}{0.644\linewidth},
  cell{3}{1} = {Concrete},
  cell{3}{2} = {c=2}{0.644\linewidth},
  cell{4}{1} = {Concrete},
  cell{4}{2} = {c=2}{0.644\linewidth},
  cell{5}{1} = {Concrete},
  cell{5}{2} = {c=2}{0.644\linewidth},
  cell{6}{2} = {c},
  cell{6}{3} = {c},
  cell{7}{1} = {r=2}{Concrete},
  cell{9}{1} = {Concrete},
  cell{9}{2} = {c},
  cell{9}{3} = {c},
  vlines,
  hline{1-7,9-10} = {-}{},
  hline{8} = {2-3}{},
}
Caso de Uso           & Eliminar tópico             &                    \\
Descrição             & Eliminar um tópico do fórum &                    \\
Ator                  & Técnico                     &                    \\
Pré-condição          & Clicar no tópico desejado   &                    \\
Pós-condição          & Remoção do tópico           &                    \\
                      & Ator                        & Sistema            \\
Fluxo Principal       & 1-Clicar em remover tópico  &                    \\
                      &                             & 3-Remover o tópico \\
Fluxo Alternativo(A1) & -                           & -                  
\end{tblr}
\end{table}

\subsubsection{Especificação de caso de uso de alterar visibilidade de um tópico}

Quando um técnico pública um tópico este pode desejar alterar a sua visibilidade para apenas técnicos oficiais ou todos os técnicos.

% \usepackage{color}
% \usepackage{tabularray}
\definecolor{Concrete}{rgb}{0.952,0.952,0.952}
\begin{table}[htb]
\centering
\label{tab:11}
\caption{Tabela de especificação de caso de uso de alteração de visibilidade de um tópico}
\begin{tblr}{
 width = \linewidth,
 colspec = {Q[242]Q[344]Q[354]},
 row{6} = {Concrete},
 cell{1}{1} = {Concrete},
 cell{1}{2} = {c=2}{0.698\linewidth},
 cell{2}{1} = {Concrete},
 cell{2}{2} = {c=2}{0.698\linewidth},
 cell{3}{1} = {Concrete},
 cell{3}{2} = {c=2}{0.698\linewidth},
 cell{4}{1} = {Concrete},
 cell{4}{2} = {c=2}{0.698\linewidth},
 cell{5}{1} = {Concrete},
 cell{5}{2} = {c=2}{0.698\linewidth},
 cell{6}{2} = {c},
 cell{6}{3} = {c},
 cell{7}{1} = {r=2}{Concrete},
 cell{9}{1} = {Concrete},
 cell{9}{2} = {c},
 cell{9}{3} = {c},
 vlines,
 hline{1-7,9-10} = {-}{},
 hline{8} = {2-3}{},
}
Caso de Uso      & Alterar visibilidade do tópico               &                  \\
Descrição       & Alterar a visibilidade de um tópico entre público e privado &                  \\
Ator         & Técnico                           &                  \\
Pré-condição     & Clicar no tópico desejado                  &                  \\
Pós-condição     & Alterar visibilidade do tópico               &                  \\
           & Ator                            & Sistema              \\
Fluxo Principal    & 1-Clicar em alterar visibilidade              &                  \\
           &                               & 2-Inverter visibilidade do tópico \\
Fluxo Alternativo(A1) & -                              & -                 
\end{tblr}
\end{table}

\newpage

\subsubsection{Especificação de caso de uso gostar de um tópico}

O técnico sempre que encontra um tópico que identifica como útil, este poderá colocar \textit{like} o que gera destaque.

% \usepackage{color}
% \usepackage{tabularray}
\definecolor{Concrete}{rgb}{0.952,0.952,0.952}
\begin{table}[htb]
\centering
\label{tab:12}
\caption{Tabela de especificação de caso de uso de gostar de um tópico}
\begin{tblr}{
  width = \linewidth,
  colspec = {Q[260]Q[221]Q[458]},
  row{6} = {Concrete},
  cell{1}{1} = {Concrete},
  cell{1}{2} = {c=2}{0.679\linewidth},
  cell{2}{1} = {Concrete},
  cell{2}{2} = {c=2}{0.679\linewidth},
  cell{3}{1} = {Concrete},
  cell{3}{2} = {c=2}{0.679\linewidth},
  cell{4}{1} = {Concrete},
  cell{4}{2} = {c=2}{0.679\linewidth},
  cell{5}{1} = {Concrete},
  cell{5}{2} = {c=2}{0.679\linewidth},
  cell{6}{2} = {c},
  cell{6}{3} = {c},
  cell{7}{1} = {r=3}{Concrete},
  cell{10}{1} = {r=3}{Concrete},
  vlines,
  hline{1-7,10,13} = {-}{},
  hline{8-9,11-12} = {2-3}{},
}
Caso de Uso           & Gostar do tópico          &                                        \\
Descrição             & Gostar de um tópico       &                                        \\
Ator                  & Técnico                   &                                        \\
Pré-condição          & Clicar no tópico desejado &                                        \\
Pós-condição          & Alterar gostos do tópico  &                                        \\
                      & Ator                      & Sistema                                \\
Fluxo Principal       & 1-Clicar em gosto         &                                        \\
                      &                           & 2-Verificar se o gosto já existe - Não \\
                      &                           & 3-Acrescentar gosto ao tópico          \\
Fluxo Alternativo(A1) & 1-Clicar em gosto         &                                        \\
                      &                           & 2-Verificar se o gosto já existe - Sim \\
                      &                           & 3-Remover gosto do tópico              
\end{tblr}
\end{table}


\subsubsection{Especificação de caso de uso gostar de um comentário}

Sempre que um técnico identificar um comentário como útil este poderá colcoar \textit{like} o que gera destaque.

% \usepackage{color}
% \usepackage{tabularray}
\definecolor{Concrete}{rgb}{0.952,0.952,0.952}
\begin{table}[htb]
\centering
\label{tab:13}
\caption{Tabela de especificação de caso de uso de gostar de comentário}
\begin{tblr}{
  width = \linewidth,
  colspec = {Q[260]Q[221]Q[458]},
  row{6} = {Concrete},
  cell{1}{1} = {Concrete},
  cell{1}{2} = {c=2}{0.679\linewidth},
  cell{2}{1} = {Concrete},
  cell{2}{2} = {c=2}{0.679\linewidth},
  cell{3}{1} = {Concrete},
  cell{3}{2} = {c=2}{0.679\linewidth},
  cell{4}{1} = {Concrete},
  cell{4}{2} = {c=2}{0.679\linewidth},
  cell{5}{1} = {Concrete},
  cell{5}{2} = {c=2}{0.679\linewidth,c},
  cell{6}{2} = {c},
  cell{6}{3} = {c},
  cell{7}{1} = {r=3}{Concrete},
  cell{10}{1} = {r=3}{Concrete},
  vlines,
  hline{1-7,10,13} = {-}{},
  hline{8-9,11-12} = {2-3}{},
}
Caso de Uso           & Gostar de comentário      &                                        \\
Descrição             & Gostar de um comentário   &                                        \\
Ator                  & Técnico                   &                                        \\
Pré-condição          & Clicar no tópico desejado &                                        \\
Pós-condição          & -                         &                                        \\
                      & Ator                      & Sistema                                \\
Fluxo Principal       & 1-Clicar em gosto         &                                        \\
                      &                           & 2-Verificar se o gosto já existe - Não \\
                      &                           & 3-Acrescentar gosto ao comentário      \\
Fluxo Alternativo(A1) & 1-Clicar em gosto         &                                        \\
                      &                           & 2-Verificar se o gosto já existe - Sim \\
                      &                           & 3-Remover gosto do tópico              
\end{tblr}
\end{table}

\newpage

\subsubsection{Especificação de caso de uso de comentar tópico}

Sempre que um técnico encontra um tópico sobre uma questão que poderá ajudar, este consegue responder através de um comentário, onde este também poderá adicionar imagens.

% \usepackage{color}
% \usepackage{tabularray}
\definecolor{Concrete}{rgb}{0.952,0.952,0.952}
\begin{table}[htb]
\centering
\label{tab:14}
\caption{Tabela de especificação de caso de uso de comentar um tópico}
\begin{tblr}{
 width = \linewidth,
 colspec = {Q[225]Q[348]Q[367]},
 row{6} = {Concrete},
 cell{1}{1} = {Concrete},
 cell{1}{2} = {c=2}{0.715\linewidth},
 cell{2}{1} = {Concrete},
 cell{2}{2} = {c=2}{0.715\linewidth},
 cell{3}{1} = {Concrete},
 cell{3}{2} = {c=2}{0.715\linewidth},
 cell{4}{1} = {Concrete},
 cell{4}{2} = {c=2}{0.715\linewidth},
 cell{5}{1} = {Concrete},
 cell{5}{2} = {c=2}{0.715\linewidth},
 cell{6}{2} = {c},
 cell{6}{3} = {c},
 cell{7}{1} = {r=4}{Concrete},
 cell{11}{1} = {r=3}{Concrete},
 cell{14}{1} = {Concrete},
 vlines,
 hline{1-7,11,14-15} = {-}{},
 hline{8-10,12-13} = {2-3}{},
}
Caso de Uso      & Comentar o tópico         &                   \\
Descrição       & Comentar um tópico        &                   \\
Ator         & Técnico              &                   \\
Pré-condição     & Clicar no tópico desejado     &                   \\
Pós-condição     & Inserir a resposta no tópico   &                   \\
           & Ator               & Sistema               \\
Fluxo Principal    & 1-Indicar a descrição da resposta &                   \\
           & 2-Anexar Imagem          &                   \\
           & 3-Confirmar a resposta      &                   \\
           &                  & 4-Inserir novo comentário no tópico \\
Fluxo Alternativo(A1) & 1-Indicar a descrição da resposta &                   \\
           & 2-Confirmar a resposta      &                   \\
           &                  & 3-Inserir novo comentário no tópico \\
Fluxo Alternativo(A2) & 1-Cancelar a criação do tópico  &                   
\end{tblr}
\end{table}

\newpage

\subsubsection{Especificação de caso de uso ativar conta}

O técnico para ativar a sua conta deverá introduzir o código de ativação correto, caso contrário não será possível ativar.

% \usepackage{color}
% \usepackage{tabularray}
\definecolor{Concrete}{rgb}{0.952,0.952,0.952}
\begin{longtblr}
  [
  caption={Tabela de especificação de caso de uso ativação de conta},
  label={tab:15},
  ]{
  width = \linewidth,
  colspec = {Q[219]Q[300]Q[419]},
  row{6} = {Concrete},
  cell{1}{1} = {Concrete},
  cell{1}{2} = {c=2}{0.719\linewidth},
  cell{2}{1} = {Concrete},
  cell{2}{2} = {c=2}{0.719\linewidth},
  cell{3}{1} = {Concrete},
  cell{3}{2} = {c=2}{0.719\linewidth},
  cell{4}{1} = {Concrete},
  cell{4}{2} = {c=2}{0.719\linewidth,c},
  cell{5}{1} = {Concrete},
  cell{5}{2} = {c=2}{0.719\linewidth,c},
  cell{6}{2} = {c},
  cell{6}{3} = {c},
  cell{7}{1} = {r=4}{Concrete},
  cell{11}{1} = {Concrete},
  cell{12}{1} = {r=4}{Concrete},
  vlines,
  hline{1-7,11-12,16} = {-}{},
  hline{8-10,13-15} = {2-3}{},
}
Caso de Uso           & Ativar conta                  &                                           \\
Descrição             & Ativar conta de aplicação     &                                           \\
Ator                  & Técnico                       &                                           \\
Pré-condição          & -                             &                                           \\
Pós-condição          & -                             &                                           \\
                      & Ator                          & Sistema                                   \\
Fluxo Principal       & 1-Inserir código de validação &                                           \\
                      & 2-Validar conta               &                                           \\
                      &                               & 3-Verificar se o código está correto- Sim \\
                      &                               & 4-Validar conta                           \\
Fluxo Alternativo(A1) & 1-Cancelar Ativação de conta  &                                           \\
Fluxo Alternativo(A2) & 1-Inserir código de validação &                                           \\
                      & 2-Validar conta               &                                           \\
                      &                               & 3-Verificar se o código está correto- Não \\
                      &                               & 4-Código de valdiação incorreto           
\end{longtblr}

\newpage

\subsubsection{Especificação de caso de uso configurar notificações}

As notificações da aplicação poderão ser personalizadas, o que possibilita escolher entre \textit{email}, \textit{push} e ambos,
para além destas configurações, é também possível personalizar o tipo de notificação para cada método, seja relatório diário de todas as notificações ou notificações em tempo real.

% \usepackage{color}
% \usepackage{tabularray}
\definecolor{Concrete}{rgb}{0.952,0.952,0.952}
\begin{table}[htb]
\centering
\label{tab:16}
\caption{Tabela de especificação de caso de uso de configuração de notificações}
\begin{tblr}{
 width = \linewidth,
 colspec = {Q[258]Q[575]Q[108]},
 row{6} = {Concrete},
 cell{1}{1} = {Concrete},
 cell{1}{2} = {c=2}{0.682\linewidth},
 cell{2}{1} = {Concrete},
 cell{2}{2} = {c=2}{0.682\linewidth},
 cell{3}{1} = {Concrete},
 cell{3}{2} = {c=2}{0.682\linewidth},
 cell{4}{1} = {Concrete},
 cell{4}{2} = {c=2}{0.682\linewidth},
 cell{5}{1} = {Concrete},
 cell{5}{2} = {c=2}{0.682\linewidth},
 cell{6}{2} = {c},
 cell{6}{3} = {c},
 cell{7}{1} = {r=2}{Concrete},
 cell{9}{1} = {Concrete},
 vlines,
 hline{1-7,9-10} = {-}{},
 hline{8} = {2-3}{},
}
Caso de Uso      & Configuração de notificações           &     \\
Descrição       & Configuração de notificações do técnico     &     \\
Ator         & Técnico                     &     \\
Pré-condição     & -                        &     \\
Pós-condição     & -                        &     \\
           & Ator                       & Sistema \\
Fluxo Principal    & 1-Indicar preferência de receção de notificações &     \\
           & 2-Indicar tipo de receção de notificações    &     \\
Fluxo Alternativo(A1) & 1-Ver notificações                &     
\end{tblr}
\end{table}

\subsubsection{Especificação de caso de uso registar técnico}

Sempre que uma empresa deseja realizar o registo de técnicos em seu nome, esta poderá indicar o nº contribuinte e \textit{email}, com isto, este receberá um \textit{email} para confirmar o registo de conta.

% \usepackage{color}
% \usepackage{tabularray}
\definecolor{Concrete}{rgb}{0.952,0.952,0.952}
\begin{table}[htb]
\centering
\label{tab:17}
\caption{Tabela de especificação de caso de uso de registar técnico}
\begin{tblr}{
  width = \linewidth,
  colspec = {Q[331]Q[454]Q[142]},
  row{6} = {Concrete},
  cell{1}{1} = {Concrete},
  cell{1}{2} = {c=2}{0.627\linewidth},
  cell{2}{1} = {Concrete},
  cell{2}{2} = {c=2}{0.627\linewidth},
  cell{3}{1} = {Concrete},
  cell{3}{2} = {c=2}{0.627\linewidth},
  cell{4}{1} = {Concrete},
  cell{4}{2} = {c=2}{0.627\linewidth},
  cell{5}{1} = {Concrete},
  cell{5}{2} = {c=2}{0.627\linewidth},
  cell{6}{2} = {c},
  cell{6}{3} = {c},
  cell{7}{1} = {r=3}{Concrete},
  cell{10}{1} = {Concrete},
  cell{10}{2} = {c},
  cell{10}{3} = {c},
  vlines,
  hline{1-7,10-11} = {-}{},
  hline{8-9} = {2-3}{},
}
Caso de Uso           & Registar técnico                     &                    \\
Descrição             & Registar conta de técnico da empresa &                    \\
Ator                  & Empresa                              &                    \\
Pré-condição          & -                                    &                    \\
Pós-condição          & -                                    &                    \\
                      & Ator                                 & Sistema            \\
Fluxo Principal       & 1-Indicar o nºcontribuinte           &                    \\
                      & 2-Indicar \textit{email}                      &                    \\
                      &                                      & 3-Registar técnico \\
Fluxo Alternativo(A1) & -                                    & -                  
\end{tblr}
\end{table}


% \subsubsection{Especificação de caso de uso responder a comentário}

% O técnico poderá manter uma conversa com outros técnicos através da resposta a outros comentários, 
% a qual poderá também incluir imagens.

% % \usepackage{color}
% \usepackage{tabularray}
\definecolor{Concrete}{rgb}{0.952,0.952,0.952}
\begin{table}[htb]
\centering
\label{tab:18}
\caption{Tabela de especificação de caso de uso de responder a comentário}
\begin{tblr}{
  width = \linewidth,
  colspec = {Q[229]Q[381]Q[333]},
  row{6} = {Concrete},
  cell{1}{1} = {Concrete},
  cell{1}{2} = {c=2}{0.714\linewidth},
  cell{2}{1} = {Concrete},
  cell{2}{2} = {c=2}{0.714\linewidth},
  cell{3}{1} = {Concrete},
  cell{3}{2} = {c=2}{0.714\linewidth},
  cell{4}{1} = {Concrete},
  cell{4}{2} = {c=2}{0.714\linewidth},
  cell{5}{1} = {Concrete},
  cell{5}{2} = {c=2}{0.714\linewidth},
  cell{6}{2} = {c},
  cell{6}{3} = {c},
  cell{7}{1} = {r=2}{Concrete},
  cell{9}{1} = {Concrete},
  cell{9}{2} = {c},
  cell{9}{3} = {c},
  vlines,
  hline{1-7,9-10} = {-}{},
  hline{8} = {2-3}{},
}
Caso de Uso           & Responder a comentário                 &                                 \\
Descrição             & Responder a um comentário de um tópico &                                 \\
Ator                  & Técnico                                &                                 \\
Pré-condição          & Clicar no tópico desejado              &                                 \\
Pós-condição          & Novo comentário                        &                                 \\
                      & Ator                                   & Sistema                         \\
Fluxo Principal       & 1-Clicar em responder a comentário     &                                 \\
                      &                                        & 2-Inserir resposta a comentário \\
Fluxo Alternativo(A1) & -                                      & -                               
\end{tblr}
\end{table}

%---------------------------------------------------------------------------------

% \subsubsection{Especificação de caso de uso registo}

% A empresa deverá realizar o registo utilizando o número de contribuinte, \textit{email} e \textit{password}. Após este 
% registo, a Motorline validará o registo e de seguida um \textit{email} é enviado para confirmar o registo na app.

% % \usepackage{color}
% \usepackage{tabularray}
\definecolor{Concrete}{rgb}{0.952,0.952,0.952}
\begin{table}[htb]
\centering
\label{tab:21}
\caption{Tabela de especificação de caso de uso de registo}
\begin{tblr}{
 width = \linewidth,
 colspec = {Q[267]Q[348]Q[323]},
 row{6} = {Concrete},
 cell{1}{1} = {Concrete},
 cell{1}{2} = {c=2}{0.671\linewidth},
 cell{2}{1} = {Concrete},
 cell{2}{2} = {c=2}{0.671\linewidth},
 cell{3}{1} = {Concrete},
 cell{3}{2} = {c=2}{0.671\linewidth},
 cell{4}{1} = {Concrete},
 cell{4}{2} = {c=2}{0.671\linewidth},
 cell{5}{1} = {Concrete},
 cell{5}{2} = {c=2}{0.671\linewidth},
 cell{6}{2} = {c},
 cell{6}{3} = {c},
 cell{7}{1} = {r=7}{Concrete},
 cell{14}{1} = {Concrete},
 vlines,
 hline{1-7,14-15} = {-}{},
 hline{8-13} = {2-3}{},
}
Caso de Uso      & Registo de cliente ou técnico       &            \\
Descrição       & Registo de cliente ou técnico na aplicação &            \\
Ator         & Cliente                  &            \\
Pré-condição     & -                     &            \\
Pós-condição     & Email de verificação de código       &            \\
           & Ator                    & Sistema        \\
Fluxo Principal    & 1-Indicar Nº Contribuinte         &            \\
           & 2-Indicar nome de empresa         &            \\
           & 3-Indicar Email              &            \\
           & 4-Password                 &            \\
           & 5-Confirmação Password           &            \\
           &                      & 6-Verificar Registo  \\
           &                      & 7-Mensagem de Sucesso \\
Fluxo Alternativo(A1) & 1-Cancelar Registo             &            
\end{tblr}
\end{table}

%---------------------------------------------------------------------------------

% \subsubsection{Especificação de caso de uso pedir reenvio de código de verificação}

% O técnico poderá aquando a validação da sua conta pedir o reenvio de um novo código de validação em caso 
% de algum imprevisto.

% % \usepackage{color}
% \usepackage{tabularray}
\definecolor{Concrete}{rgb}{0.952,0.952,0.952}
\begin{table}[htb]
\centering
\begin{tblr}{
  width = \linewidth,
  colspec = {Q[223]Q[356]Q[362]},
  row{6} = {Concrete},
  cell{1}{1} = {Concrete},
  cell{1}{2} = {c=2}{0.706\linewidth},
  cell{2}{1} = {Concrete},
  cell{2}{2} = {c=2}{0.706\linewidth},
  cell{3}{1} = {Concrete},
  cell{3}{2} = {c=2}{0.706\linewidth},
  cell{4}{1} = {Concrete},
  cell{4}{2} = {c=2}{0.706\linewidth,c},
  cell{5}{1} = {Concrete},
  cell{5}{2} = {c=2}{0.706\linewidth},
  cell{6}{2} = {c},
  cell{6}{3} = {c},
  cell{7}{1} = {r=3}{Concrete},
  cell{10}{1} = {Concrete},
  cell{10}{2} = {c},
  cell{10}{3} = {c},
  vlines,
  hline{1-7,10-11} = {-}{},
  hline{8-9} = {2-3}{},
}
Caso de Uso           & Pedir reenvio de código de ativação          &                                 \\
Descrição             & Pedir reenvio de \textit{email} de código de ativação &                                 \\
Ator                  & Técnico                                      &                                 \\
Pré-condição          & -                                            &                                 \\
Pós-condição          & Email de verificação de código               &                                 \\
                      & Ator                                         & Sistema                         \\
Fluxo Principal       & 1-Pedir novo código de ativação              &                                 \\
                      &                                              & 2-Gerar novo código de ativação \\
                      &                                              & 3-Enviar novo \textit{email}             \\
Fluxo Alternativo(A1) & -                                            & -                               
\end{tblr}
\end{table}

%---------------------------------------------------------------------------------

% \subsubsection{Especificação de caso de uso confirmar conta}

% Sempre que uma conta de técnico é criada, este deverá proceder à confirmação da conta, nesta confirmação
% o técnico tem de inidicar o seu nome de utilizador, poderá alterar o \textit{email} de registo e terá de indicar a
% \textit{password} e confirmação de \textit{password}, procedendo depois à ativação da conta.

% % \usepackage{color}
% \usepackage{tabularray}
\definecolor{Concrete}{rgb}{0.952,0.952,0.952}
\begin{table}[htb]
\centering
\begin{tblr}{
 width = \linewidth,
 colspec = {Q[258]Q[429]Q[254]},
 row{6} = {Concrete},
 cell{1}{1} = {Concrete},
 cell{1}{2} = {c=2}{0.683\linewidth},
 cell{2}{1} = {Concrete},
 cell{2}{2} = {c=2}{0.683\linewidth},
 cell{3}{1} = {Concrete},
 cell{3}{2} = {c=2}{0.683\linewidth},
 cell{4}{1} = {Concrete},
 cell{4}{2} = {c=2}{0.683\linewidth,c},
 cell{5}{1} = {Concrete},
 cell{5}{2} = {c=2}{0.683\linewidth,c},
 cell{6}{2} = {c},
 cell{6}{3} = {c},
 cell{7}{1} = {r=6}{Concrete},
 cell{13}{1} = {Concrete},
 vlines,
 hline{1-7,13-14} = {-}{},
 hline{8-12} = {2-3}{},
}
Caso de Uso      & Confirmar conta          &            \\
Descrição       & Confirmar conta de técnico    &            \\
Ator         & Técnico              &            \\
Pré-condição     & -                 &            \\
Pós-condição     & -                 &            \\
           & Ator               & Sistema        \\
Fluxo Principal    & 1-Inserir nome de utilizador   &            \\
           & 2-Indicar Password        &            \\
           & 3-Indicar Confirmação de \textit{password} &            \\
           &                  & 4-Verificar se registo \\
           &                  & 5-Conta registada   \\
           &                  & 6-Validar conta    \\
Fluxo Alternativo(A1) & 1-Cancelar Ativação de conta   &            
\end{tblr}
\end{table}

%---------------------------------------------------------------------------------

% \subsubsection{Especificação de caso de uso ver perfil}

% Sempre que um técnico desejar alterar alguma informação sua, este poderá se dirigir ao seu perfil onde 
% consegue alterar o seu \textit{email} e imagem de perfil.

% % \usepackage{color}
% \usepackage{tabularray}
\definecolor{Concrete}{rgb}{0.952,0.952,0.952}
\begin{table}[htb]
\centering
\begin{tblr}{
 width = \linewidth,
 colspec = {Q[252]Q[310]Q[379]},
 row{6} = {Concrete},
 cell{1}{1} = {Concrete},
 cell{1}{2} = {c=2}{0.689\linewidth},
 cell{2}{1} = {Concrete},
 cell{2}{2} = {c=2}{0.689\linewidth},
 cell{3}{1} = {Concrete},
 cell{3}{2} = {c=2}{0.689\linewidth},
 cell{4}{1} = {Concrete},
 cell{4}{2} = {c=2}{0.689\linewidth},
 cell{5}{1} = {Concrete},
 cell{5}{2} = {c=2}{0.689\linewidth},
 cell{6}{2} = {c},
 cell{6}{3} = {c},
 cell{7}{1} = {r=2}{Concrete},
 cell{9}{1} = {r=2}{Concrete},
 vlines,
 hline{1-7,9,11} = {-}{},
 hline{8,10} = {2-3}{},
}
Caso de Uso      & Ver perfil         &                 \\
Descrição       & Ver perfil do técnico   &                 \\
Ator         & Técnico          &                 \\
Pré-condição     & -             &                 \\
Pós-condição     & -             &                 \\
           & Ator            & Sistema             \\
Fluxo Principal    & 1-Alterar \textit{email}      &                 \\
           &              & 2-Alteração de \textit{email}      \\
Fluxo Alternativo(A1) & 1-Alterar imagem de perfil &                 \\
           &              & 2-Alteração de imagem de perfil 
\end{tblr}
\end{table}

%---------------------------------------------------------------------------------

% \subsubsection{Especificação de caso de uso ver notificações}

% Sempre que um técnico desejar ver todas as suas notificações este poderá ver esta listagem, 
% conseguindo também apagar notificações que já não deseja ver.

% % \usepackage{color}
% \usepackage{tabularray}
\definecolor{Concrete}{rgb}{0.952,0.952,0.952}
\begin{table}[htb]
\centering
\begin{tblr}{
  width = \linewidth,
  colspec = {Q[287]Q[285]Q[363]},
  row{6} = {Concrete},
  cell{1}{1} = {Concrete},
  cell{1}{2} = {c=2}{0.647\linewidth},
  cell{2}{1} = {Concrete},
  cell{2}{2} = {c=2}{0.647\linewidth},
  cell{3}{1} = {Concrete},
  cell{3}{2} = {c=2}{0.647\linewidth},
  cell{4}{1} = {Concrete},
  cell{4}{2} = {c=2}{0.647\linewidth},
  cell{5}{1} = {Concrete},
  cell{5}{2} = {c=2}{0.647\linewidth},
  cell{6}{2} = {c},
  cell{6}{3} = {c},
  cell{7}{1} = {r=2}{Concrete},
  cell{9}{1} = {r=4}{Concrete},
  vlines,
  hline{1-7,9,13} = {-}{},
  hline{8,10-12} = {2-3}{},
}
Caso de Uso           & Ver Notificações            &                            \\
Descrição             & Ver notificações do técnico &                            \\
Ator                  & Técnico                     &                            \\
Pré-condição          & -                           &                            \\
Pós-condição          & -                           &                            \\
                      & Ator                        & Sistema                    \\
Fluxo Principal       & 1-Ver notificações          &                            \\
                      &                             & 2-Listagem de notificações \\
Fluxo Alternativo(A1) & 1-Ver notificações          &                            \\
                      &                             & 2-Listagem de notificações \\
                      & 3-Apagar notificação        &                            \\
                      &                             & 4-Eliminar notificação     
\end{tblr}
\end{table}

%---------------------------------------------------------------------------------

% \subsubsection{Especificação de caso de uso apagar comentário}

% Sempre que o técnico cria um comentário este tem a possibilidade de o remover a qualquer momento que 
% desejar.

% % \usepackage{color}
% \usepackage{tabularray}
\definecolor{Concrete}{rgb}{0.952,0.952,0.952}
\begin{table}[htb]
\centering
\label{tab:17}
\caption{Tabela de especificação de caso de uso de apagar comentário}
\begin{tblr}{
 width = \linewidth,
 colspec = {Q[271]Q[394]Q[273]},
 row{6} = {Concrete},
 cell{1}{1} = {Concrete},
 cell{1}{2} = {c=2}{0.667\linewidth},
 cell{2}{1} = {Concrete},
 cell{2}{2} = {c=2}{0.667\linewidth},
 cell{3}{1} = {Concrete},
 cell{3}{2} = {c=2}{0.667\linewidth},
 cell{4}{1} = {Concrete},
 cell{4}{2} = {c=2}{0.667\linewidth},
 cell{5}{1} = {Concrete},
 cell{5}{2} = {c=2}{0.667\linewidth},
 cell{6}{2} = {c},
 cell{6}{3} = {c},
 cell{7}{1} = {r=2}{Concrete},
 cell{9}{1} = {Concrete},
 cell{9}{2} = {c},
 cell{9}{3} = {c},
 vlines,
 hline{1-7,9-10} = {-}{},
 hline{8} = {2-3}{},
}
Caso de Uso      & Apagar comentário       &           \\
Descrição       & Apagar comentário de um tópico &           \\
Ator         & Técnico            &           \\
Pré-condição     & Clicar no tópico desejado   &           \\
Pós-condição     & Comentário apagado       &           \\
           & Ator              & Sistema       \\
Fluxo Principal    & 1-Clicar em apagar comentário &           \\
           &                & 2-Apagar comentário \\
Fluxo Alternativo(A1) & -               & -          
\end{tblr}
\end{table}

%---------------------------------------------------------------------------------

% \subsubsection{Especificação de caso de uso login}

% O técnico deverá realizar o login utilizando o número de contribuinte e \textit{password}.

% % \usepackage{color}
% \usepackage{tabularray}
\definecolor{Concrete}{rgb}{0.952,0.952,0.952}
\begin{longtblr}
  [
  caption={Tabela de especificação de caso de uso criação de novo tópico},
  label={tab:5},
  ]{
    width = \linewidth,
    colspec = {Q[262]Q[373]Q[306]},
    row{6} = {Concrete},
    cell{1}{1} = {Concrete},
    cell{1}{2} = {c=2}{0.679\linewidth},
    cell{2}{1} = {Concrete},
    cell{2}{2} = {c=2}{0.679\linewidth},
    cell{3}{1} = {Concrete},
    cell{3}{2} = {c=2}{0.679\linewidth},
    cell{4}{1} = {Concrete},
    cell{4}{2} = {c=2}{0.679\linewidth},
    cell{5}{1} = {Concrete},
    cell{5}{2} = {c=2}{0.679\linewidth},
    cell{6}{2} = {c},
    cell{6}{3} = {c},
    cell{7}{1} = {r=4}{Concrete},
    cell{11}{1} = {Concrete},
    cell{12}{1} = {Concrete},
    cell{13}{1} = {r=4}{Concrete},
    vlines,
    hline{1-7,11-13,17} = {-}{},
    hline{8-10,14-16} = {2-3}{},
  }
  Caso de Uso           & Login                            &                          \\
  Descrição             & Iniciar sessão na aplicação      &                          \\
  Ator                  & Técnico                          &                          \\
  Pré-condição          & Código de verificação confirmado &                          \\
  Pós-condição          & Home Screen                      &                          \\
                        & Ator                             & Sistema                  \\
  Fluxo Principal       & 1-Indicar Nº Contribuinte        &                          \\
                        & 2-Indicar Password               &                          \\
                        &                                  & 3-Verificar Login        \\
                        &                                  & 4-Devolver Sessão        \\
  Fluxo Alternativo(A1) & 1-Clicar em não iniciar sessão   &                          \\
  Fluxo Alternativo(A2) & 1-Cancelar Login                 &                          \\
  Fluxo Alternativo(A3) & 1-Indicar Nº Contribuinte        &                          \\
                        & 2-Indicar Password               &                          \\
                        &                                  & 3-Verificar Login        \\
                        &                                  & 4-Erro conta não ativada 
\end{longtblr}

%---------------------------------------------------------------------------------

%\subsubsection{Especificação de caso de uso de pesquisar tópicos por código QR}

%Assim que um utilizador deseje pesquisar por tópicos relativos a um produto, este poderá utilizar 
%o código QR do mesmo, conseguindo também realizar uma pesquisa por escrito.  

%% \usepackage{color}
% \usepackage{tabularray}
\definecolor{Concrete}{rgb}{0.952,0.952,0.952}
\begin{longtblr}
[
caption={Tabela de especificação de caso de uso de pesquisa por código QR},
label={tab:7},
]{
 width = \linewidth,
 colspec = {Q[175]Q[219]Q[546]},
 row{6} = {Concrete},
 cell{1}{1} = {Concrete},
 cell{1}{2} = {c=2}{0.74\linewidth},
 cell{2}{1} = {Concrete},
 cell{2}{2} = {c=2}{0.74\linewidth},
 cell{3}{1} = {Concrete},
 cell{3}{2} = {c=2}{0.74\linewidth},
 cell{4}{1} = {Concrete},
 cell{4}{2} = {c=2}{0.74\linewidth},
 cell{5}{1} = {Concrete},
 cell{5}{2} = {c=2}{0.74\linewidth,c},
 cell{6}{2} = {c},
 cell{6}{3} = {c},
 cell{7}{1} = {r=6}{Concrete},
 cell{13}{1} = {r=4}{Concrete},
 vlines,
 hline{1-7,13,17} = {-}{},
 hline{8-12,14-16} = {2-3}{},
}
Caso de Uso      & Pesquisa por código QR            &                            \\
Descrição       & Pesquisar por tópicos no fórum por código QR &                            \\
Ator         & Utilizador                  &                            \\
Pré-condição     & Selecionar pesquisa de fórum         &                            \\
Pós-condição     & -                      &                            \\
           & Ator                     & Sistema                        \\
Fluxo Principal    & 1-Pesquisar por código QR          &                            \\
           &                       & 2-Lista de tópicos do produto             \\
           & 2-Pesquisar assunto             &                            \\
           &                       & 3-Filtragem de tópicos do produto por assunto     \\
           & 4-Filtrar por tipo              &                            \\
           &                       & 5-Filtragem de tópicos do produto por assunto e tópico \\
Fluxo Alternativo(A1) & 1-Pesquisar por código QR          &                            \\
           &                       & 2-Lista de tópicos do produto             \\
           & 2-Pesquisar assunto             &                            \\
           &                       & 3-Filtragem de tópicos do produto por assunto     
\end{longtblr}

%\newpage