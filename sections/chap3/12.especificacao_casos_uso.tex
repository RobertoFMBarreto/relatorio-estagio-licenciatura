\section{Casos de uso}
De forma a transformar as histórias de utilizador em ações e especificar todas as ações dos atores do 
software e todas as reações do sistema com o qual o ator interage foram desenvolvidos casos de uso.

% \usepackage{color}
% \usepackage{tabularray}
\definecolor{Concrete}{rgb}{0.952,0.952,0.952}
\definecolor{Gallery}{rgb}{0.937,0.937,0.937}
\begin{longtblr}
[
caption={Tabela de casos de uso},
label={tab:4},
]
{
  width = \linewidth,
  colspec = {Q[130]Q[81]Q[120]Q[242]Q[375]},
  row{1} = {Concrete},
  row{2} = {Concrete,c},
  row{4} = {Concrete,c},
  row{8} = {Concrete,c},
  row{10} = {Concrete,c},
  row{18} = {Concrete,c},
  row{23} = {Concrete,c},
  row{26} = {Concrete,c},
  row{28} = {Concrete,c},
  row{31} = {Concrete,c},
  column{2} = {c},
  column{3} = {c},
  cell{1}{1} = {c},
  cell{2}{1} = {c=5}{0.936\linewidth},
  cell{4}{1} = {c=5}{0.936\linewidth},
  cell{8}{1} = {c=5}{0.936\linewidth},
  cell{10}{1} = {c=5}{0.936\linewidth},
  cell{18}{1} = {c=5}{0.936\linewidth},
  cell{23}{1} = {c=5}{0.936\linewidth},
  cell{26}{1} = {c=5}{0.936\linewidth},
  cell{28}{1} = {c=5}{0.936\linewidth},
  cell{31}{1} = {c=5}{0.936\linewidth},
  hlines,
  vlines,
}
\#                         & User Story         & Ator       & Nome                                & Descrição                                                   \\
Criar tópico               &                    &            &                                     &                                                             \\
UC 1.0                     & US15               & Técnico    & Criar novo tópico                   & Criação de um novo tópico no fórum                          \\
Pesquisa de tópicos        &                    &            &                                     &                                                             \\
UC 1.1                     & US11               & Utilizador & Pesquisar tópicos específicos       & Pesquisar por tópicos no fórum                              \\
UC 1.1.1                   & US12 e US13        & Utilizador & Pesquisa escrita                    & Pesquisar tópicos por assunto                               \\
UC 1.1.2                   & US14               & Utilizador & Pesquisa por código QR              & Pesquisar tópicos referentes a um produto                   \\
Listagens de tópicos       &                    &            &                                     &                                                             \\
UC 1.2                     & US05               & Utilizador & Ver tópicos                         & Ver listagens de tópicos do fórum                           \\
Detalhes de tópico         &                    &            &                                     &                                                             \\
UC 1.3                     & US08               & Utilizador & Selecionar tópico                   & Ver detalhes de um tópico selecionado                       \\
UC 1.3.1                   & US21               & Técnico    & Finalizar tópico                    & Finalizar um tópico para indicar que está respondido        \\
UC 1.3.2                   & US20               & Técnico    & Selecionar melhor resposta          & Selecionar a melhor resposta do tópico                      \\
UC 1.3.3                   & US22               & Técnico    & Eliminar tópico                     & Eliminar um tópico do fórum                                 \\
UC 1.3.4                   & US23               & Técnico    & Alterar visibilidade do tópico      & Alterar a visibilidade de um tópico entre publico e privado \\
UC 1.3.5                   & US28               & Técnico    & Comentar o tópico                   & Comentar um tópico                                          \\
UC 1.3.6                   & US25               & Técnico    & Gostar de tópico                    & Gostar de um tópico                                         \\
Comentários                &                    &            &                                     &                                                             \\
UC 1.3.7                   & US24               & Utilizador & Ver comentários                     & Ver comentários do tópico                                   \\
UC 1.3.7.1                 & US27               & Técnico    & Apagar comentário                   & Apagar comentário de um tópico                              \\
UC 1.3.7.2                 & US29               & Técnico    & Responder a comentário              & Responder a um comentário de um tópico                      \\
UC 1.3.7.3                 & US31               & Técnico    & Gostar de comentário                & Gostar de um comentário                                     \\
Ativação de conta          &                    &            &                                     &                                                             \\
UC 1.4                     & -                  & Técnico    & Ativação de conta                   & Ativar conta de cliente                                     \\
UC 1.4.1                   & US04               & Técnico    & Pedir reenvio de código de ativação & Pedir reenvio de email de código de ativação                \\
Perfil                     &                    &            &                                     &                                                             \\
UC 1.5                     & US31 - US32 - US33 & Técnico    & Ver Perfil                          & Ver perfil de utilizador                                    \\
Notificações               &                    &            &                                     &                                                             \\
UC1.6                      & US34 e US36        & Técnico    & Ver notificações                    & Ver todas as notificações                                   \\
UC1.7                      & US34 e US35        & Técnico    & Configuração de notificações        & Configurar o modo e tipo de notificações a receber          \\
Gestão de recursos humanos &                    &            &                                     &                                                             \\
UC1.8                      & US37               & Empresa    & Registar Técnico                    & Registar conta de técnico da empresa                        \\
UC1.9                      & US38               & Empresa    & Impedir acesso a técnico            & Registar conta de técnico da empresa                        \\
UC1.10                     & US39               & Empresa    & Remover conta de técnico            & Registar conta de técnico da empresa                        
\end{longtblr}
\newpage

\subsection{Especificação de casos de uso}

De forma a demonstrar todas as interações entre os atores e o sistema, assim como também todas as ações 
destes e fluxos possíveis, foram realizadas especificações de casos de uso.

\subsubsection{Especificação de caso de uso de criar novo tópico}

Aquando a criação de um tópico um técnico poderá realizar diversas ações sendo que obrigatoriamente 
terá sempre de indicar o título, descrição do tópico e tipo de tópico, para além desta informação o técnico poderá também
anexar imagens, referenciar um produto e indicar a visibilidade.

\definecolor{Concrete}{rgb}{0.952,0.952,0.952}
\begin{longtblr}
[
caption={Tabela de especificação de caso de uso login},
label={tab:6},
]{
  width = \linewidth,
  colspec = {Q[212]Q[360]Q[369]},
  row{6} = {Concrete},
  cell{1}{1} = {Concrete},
  cell{1}{2} = {c=2}{0.725\linewidth},
  cell{2}{1} = {Concrete},
  cell{2}{2} = {c=2}{0.725\linewidth},
  cell{3}{1} = {Concrete},
  cell{3}{2} = {c=2}{0.725\linewidth},
  cell{4}{1} = {Concrete},
  cell{4}{2} = {c=2}{0.725\linewidth},
  cell{5}{1} = {Concrete},
  cell{5}{2} = {c=2}{0.725\linewidth,c},
  cell{6}{2} = {c},
  cell{6}{3} = {c},
  cell{7}{1} = {r=10}{Concrete,c},
  cell{17}{1} = {Concrete},
  cell{18}{1} = {r=6}{Concrete},
  vlines,
  hline{1-7,17-18,24} = {-}{},
  hline{8-16,19-23} = {2-3}{},
}
Caso de Uso           & Criar novo tópico                      &                                        \\
Descrição             & Criação de um novo tópico no fórum     &                                        \\
Ator                  & Técnico                                &                                        \\
Pré-condição          & Clicar em adicionar novo tópico        &                                        \\
Pós-condição          & -                                      &                                        \\
                      & Ator                                   & Sistema                                \\
Fluxo Principal       & 1-Indicar o título do tópico           &                                        \\
                      & 2-Indicar a descrição do tópico        &                                        \\
                      & 3-Indicar se o tópico é privado        &                                        \\
                      & 4-Indicar o tipo do tópico             &                                        \\
                      & 5-Indicar o produto referido no tópico &                                        \\
                      & 6-Adicionar imagens de anexo           &                                        \\
                      & 7-Confirmar a criação do tópico        &                                        \\
                      &                                        & 8-Verificar se titulo está inserido    \\
                      &                                        & 9-Verificar se descrição está inserida \\
                      &                                        & 10-Inserir novo tópico no fórum        \\
Fluxo Alternativo(A1) & 1-Cancelar a criação do tópico         &                                        \\
Fluxo Alternativo(A2) & 1-Indicar o título do tópico           &                                        \\*
                      & 2-Indicar se o tópico é privado        &                                        \\*
                      & 3-Confirmar a criação do tópico        &                                        \\*
                      &                                        & 4-Verificar se titulo está inserido    \\*
                      &                                        & 5-Verificar se descrição está inserida \\*
                      &                                        & 6-Erro descrição em falta         
\end{longtblr}

\subsubsection{Especificação de caso de uso de pesquisar tópicos por código QR}

Assim que um utilizador deseje pesquisar por tópicos relativos a um produto, este poderá utilizar 
o código QR do mesmo, conseguindo também realizar uma pesquisa por escrito.    

% \usepackage{color}
% \usepackage{tabularray}
\definecolor{Concrete}{rgb}{0.952,0.952,0.952}
\begin{longtblr}
[
caption={Tabela de especificação de caso de uso de pesquisa por código QR},
label={tab:7},
]{
  width = \linewidth,
  colspec = {Q[175]Q[219]Q[546]},
  row{6} = {Concrete},
  cell{1}{1} = {Concrete},
  cell{1}{2} = {c=2}{0.74\linewidth},
  cell{2}{1} = {Concrete},
  cell{2}{2} = {c=2}{0.74\linewidth},
  cell{3}{1} = {Concrete},
  cell{3}{2} = {c=2}{0.74\linewidth},
  cell{4}{1} = {Concrete},
  cell{4}{2} = {c=2}{0.74\linewidth},
  cell{5}{1} = {Concrete},
  cell{5}{2} = {c=2}{0.74\linewidth,c},
  cell{6}{2} = {c},
  cell{6}{3} = {c},
  cell{7}{1} = {r=6}{Concrete},
  cell{13}{1} = {r=4}{Concrete},
  vlines,
  hline{1-7,13,17} = {-}{},
  hline{8-12,14-16} = {2-3}{},
}
Caso de Uso           & Pesquisa por código QR                       &                                                        \\
Descrição             & Pesquisar por tópicos no fórum por código QR &                                                        \\
Ator                  & Utilizador                                   &                                                        \\
Pré-condição          & Selecionar pesquisa de fórum                 &                                                        \\
Pós-condição          & -                                            &                                                        \\
                      & Ator                                         & Sistema                                                \\
Fluxo Principal       & 1-Pesquisar por código QR                    &                                                        \\
                      &                                              & 2-Lista de tópicos do produto                          \\
                      & 2-Pesquisar assunto                          &                                                        \\
                      &                                              & 3-Filtragem de tópicos do produto por assunto          \\
                      & 4-Filtrar por tipo                           &                                                        \\
                      &                                              & 5-Filtragem de tópicos do produto por assunto e tópico \\
Fluxo Alternativo(A1) & 1-Pesquisar por código QR                    &                                                        \\
                      &                                              & 2-Lista de tópicos do produto                          \\
                      & 2-Pesquisar assunto                          &                                                        \\
                      &                                              & 3-Filtragem de tópicos do produto por assunto          
\end{longtblr}

\subsubsection{Especificação de caso de uso registo}

A empresa deverá realizar o registo utilizando o número de contribuinte, email e password. Após este 
registo, a Motorline validará o registo e de seguida um email é enviado para confirmar o registo na app.

% \usepackage{color}
% \usepackage{tabularray}
\definecolor{Concrete}{rgb}{0.952,0.952,0.952}
\begin{table}[htb]
\centering
\label{tab:21}
\caption{Tabela de especificação de caso de uso de registo}
\begin{tblr}{
 width = \linewidth,
 colspec = {Q[267]Q[348]Q[323]},
 row{6} = {Concrete},
 cell{1}{1} = {Concrete},
 cell{1}{2} = {c=2}{0.671\linewidth},
 cell{2}{1} = {Concrete},
 cell{2}{2} = {c=2}{0.671\linewidth},
 cell{3}{1} = {Concrete},
 cell{3}{2} = {c=2}{0.671\linewidth},
 cell{4}{1} = {Concrete},
 cell{4}{2} = {c=2}{0.671\linewidth},
 cell{5}{1} = {Concrete},
 cell{5}{2} = {c=2}{0.671\linewidth},
 cell{6}{2} = {c},
 cell{6}{3} = {c},
 cell{7}{1} = {r=7}{Concrete},
 cell{14}{1} = {Concrete},
 vlines,
 hline{1-7,14-15} = {-}{},
 hline{8-13} = {2-3}{},
}
Caso de Uso      & Registo de cliente ou técnico       &            \\
Descrição       & Registo de cliente ou técnico na aplicação &            \\
Ator         & Cliente                  &            \\
Pré-condição     & -                     &            \\
Pós-condição     & Email de verificação de código       &            \\
           & Ator                    & Sistema        \\
Fluxo Principal    & 1-Indicar Nº Contribuinte         &            \\
           & 2-Indicar nome de empresa         &            \\
           & 3-Indicar Email              &            \\
           & 4-Password                 &            \\
           & 5-Confirmação Password           &            \\
           &                      & 6-Verificar Registo  \\
           &                      & 7-Mensagem de Sucesso \\
Fluxo Alternativo(A1) & 1-Cancelar Registo             &            
\end{tblr}
\end{table}

\newpage

\subsubsection{Especificação de caso de uso de selecionar melhor resposta}

Sempre que o técnico encontrar uma resposta no seu tópico que se destaca na solução da sua questão, 
este poderá colocar esta resposta como melhor resposta do tópico, caso já exista uma melhor resposta no 
tópico, esta é removida de melhor resposta e a nova resposta é colocada como melhor resposta.

% \usepackage{color}
% \usepackage{tabularray}
\definecolor{Concrete}{rgb}{0.952,0.952,0.952}
\begin{table}[htb]
\centering
\label{tab:9}
\caption{Tabela de especificação de caso de uso de selecionar melhor resposta}
\begin{tblr}{
 width = \linewidth,
 colspec = {Q[181]Q[235]Q[525]},
 row{6} = {Concrete},
 cell{1}{1} = {Concrete},
 cell{1}{2} = {c=2}{0.76\linewidth},
 cell{2}{1} = {Concrete},
 cell{2}{2} = {c=2}{0.76\linewidth},
 cell{3}{1} = {Concrete},
 cell{3}{2} = {c=2}{0.76\linewidth},
 cell{4}{1} = {Concrete},
 cell{4}{2} = {c=2}{0.76\linewidth},
 cell{5}{1} = {Concrete},
 cell{5}{2} = {c=2}{0.76\linewidth},
 cell{6}{2} = {c},
 cell{6}{3} = {c},
 cell{7}{1} = {r=3}{Concrete},
 cell{10}{1} = {r=4}{Concrete},
 cell{14}{1} = {r=4}{Concrete},
 vlines,
 hline{1-7,10,14,18} = {-}{},
 hline{8-9,11-13,15-17} = {2-3}{},
}
Caso de Uso      & Selecionar melhor resposta       &                                 \\
Descrição       & Selecionar a melhor resposta do tópico &                                 \\
Ator         & Técnico                 &                                 \\
Pré-condição     & Clicar no tópico desejado        &                                 \\
Pós-condição     & Alterar a resposta para melhor resposta &                                 \\
           & Ator                  & Sistema                             \\
Fluxo Principal    & 1-Clicar em melhor resposta       &                                 \\
           &                     & 2-Verificar se já existe uma melhor resposta - Não        \\
           &                     & 3- Colocar a resposta como melhor resposta do tópico       \\
Fluxo Alternativo(A1) & 1-Clicar em melhor resposta       &                                 \\
           &                     & 2-Verificar se já existe uma melhor resposta - Sim        \\
           &                     & 3- Verificar se a resposta existente é a mesma selecionada - Não \\
           &                     & 4-Alterar melhor resposta                    \\
Fluxo Alternativo(A2) & 1-Clicar em melhor resposta       &                                 \\
           &                     & 2-Verificar se já existe uma melhor resposta - Sim        \\
           &                     & 3- Verificar se a resposta existente é a mesma selecionada - Sim \\
           &                     & 4-Remover melhor resposta                    
\end{tblr}
\end{table}

\newpage

\subsubsection{Especificação de caso de uso gostar de um tópico}

O técnico sempre que encontra um tópico que identifica como útil, este poderá gostar o tópico dando assim 
destaque a este.

% \usepackage{color}
% \usepackage{tabularray}
\definecolor{Concrete}{rgb}{0.952,0.952,0.952}
\begin{table}[htb]
\centering
\label{tab:12}
\caption{Tabela de especificação de caso de uso de gostar de um tópico}
\begin{tblr}{
 width = \linewidth,
 colspec = {Q[260]Q[221]Q[458]},
 row{6} = {Concrete},
 cell{1}{1} = {Concrete},
 cell{1}{2} = {c=2}{0.679\linewidth},
 cell{2}{1} = {Concrete},
 cell{2}{2} = {c=2}{0.679\linewidth},
 cell{3}{1} = {Concrete},
 cell{3}{2} = {c=2}{0.679\linewidth},
 cell{4}{1} = {Concrete},
 cell{4}{2} = {c=2}{0.679\linewidth},
 cell{5}{1} = {Concrete},
 cell{5}{2} = {c=2}{0.679\linewidth},
 cell{6}{2} = {c},
 cell{6}{3} = {c},
 cell{7}{1} = {r=3}{Concrete},
 cell{10}{1} = {r=3}{Concrete},
 vlines,
 hline{1-7,10,13} = {-}{},
 hline{8-9,11-12} = {2-3}{},
}
Caso de Uso      & Gostar do tópico     &                    \\
Descrição       & Gostar de um tópico    &                    \\
Ator         & Técnico          &                    \\
Pré-condição     & Clicar no tópico desejado &                    \\
Pós-condição     & Alterar gostos do tópico &                    \\
           & Ator           & Sistema                \\
Fluxo Principal    & 1-Clicar em gosto     &                    \\
           &              & 2-Verificar se o gosto já existe - Não \\
           &              & 3-Acrescentar gosto ao tópico     \\
Fluxo Alternativo(A1) & 1-Clicar em gosto     &                    \\
           &              & 2-Verificar se o gosto já existe - Sim \\
           &              & 3-Remover gosto do tópico       
\end{tblr}
\end{table}



\subsubsection{Especificação de caso de uso ativar conta}

O técnico de forma a ativar a sua conta deverá introduzir o código de ativação de conta correto, 
caso contrário não será possível ativar a sua conta.

% \usepackage{color}
% \usepackage{tabularray}
\definecolor{Concrete}{rgb}{0.952,0.952,0.952}
\begin{longtblr}
  [
  caption={Tabela de especificação de caso de uso ativação de conta},
  label={tab:15},
  ]{
  width = \linewidth,
  colspec = {Q[219]Q[300]Q[419]},
  row{6} = {Concrete},
  cell{1}{1} = {Concrete},
  cell{1}{2} = {c=2}{0.719\linewidth},
  cell{2}{1} = {Concrete},
  cell{2}{2} = {c=2}{0.719\linewidth},
  cell{3}{1} = {Concrete},
  cell{3}{2} = {c=2}{0.719\linewidth},
  cell{4}{1} = {Concrete},
  cell{4}{2} = {c=2}{0.719\linewidth,c},
  cell{5}{1} = {Concrete},
  cell{5}{2} = {c=2}{0.719\linewidth,c},
  cell{6}{2} = {c},
  cell{6}{3} = {c},
  cell{7}{1} = {r=4}{Concrete},
  cell{11}{1} = {Concrete},
  cell{12}{1} = {r=4}{Concrete},
  vlines,
  hline{1-7,11-12,16} = {-}{},
  hline{8-10,13-15} = {2-3}{},
}
Caso de Uso           & Ativar conta                  &                                           \\
Descrição             & Ativar conta de aplicação     &                                           \\
Ator                  & Técnico                       &                                           \\
Pré-condição          & -                             &                                           \\
Pós-condição          & -                             &                                           \\
                      & Ator                          & Sistema                                   \\
Fluxo Principal       & 1-Inserir código de validação &                                           \\
                      & 2-Validar conta               &                                           \\
                      &                               & 3-Verificar se o código está correto- Sim \\
                      &                               & 4-Validar conta                           \\
Fluxo Alternativo(A1) & 1-Cancelar Ativação de conta  &                                           \\
Fluxo Alternativo(A2) & 1-Inserir código de validação &                                           \\
                      & 2-Validar conta               &                                           \\
                      &                               & 3-Verificar se o código está correto- Não \\
                      &                               & 4-Código de valdiação incorreto           
\end{longtblr}

\subsubsection{Especificação de caso de uso gostar de um comentário}

Sempre que um técnico identificar um comentário como útil este poderá gostar dos comentários, dando 
assim destaque a este.

\input{tables/casos_de_uso/gostar_comentario}

\newpage

\subsubsection{Especificação de caso de uso confirmar conta}

Sempre que uma conta de técnico é criada, este deverá proceder à confirmação da conta, nesta confirmação
o técnico tem de inidicar o seu nome de utilizador, poderá alterar o email de registo e terá de indicar a
password e confirmação de password, procedendo depois à ativação da conta.

% \usepackage{color}
% \usepackage{tabularray}
\definecolor{Concrete}{rgb}{0.952,0.952,0.952}
\begin{table}[htb]
\centering
\begin{tblr}{
  width = \linewidth,
  colspec = {Q[258]Q[429]Q[254]},
  row{6} = {Concrete},
  cell{1}{1} = {Concrete},
  cell{1}{2} = {c=2}{0.683\linewidth},
  cell{2}{1} = {Concrete},
  cell{2}{2} = {c=2}{0.683\linewidth},
  cell{3}{1} = {Concrete},
  cell{3}{2} = {c=2}{0.683\linewidth},
  cell{4}{1} = {Concrete},
  cell{4}{2} = {c=2}{0.683\linewidth,c},
  cell{5}{1} = {Concrete},
  cell{5}{2} = {c=2}{0.683\linewidth,c},
  cell{6}{2} = {c},
  cell{6}{3} = {c},
  cell{7}{1} = {r=6}{Concrete},
  cell{13}{1} = {Concrete},
  vlines,
  hline{1-7,13-14} = {-}{},
  hline{8-12} = {2-3}{},
}
Caso de Uso           & Confirmar conta                   &                        \\
Descrição             & Confirmar conta de técnico        &                        \\
Ator                  & Técnico                           &                        \\
Pré-condição          & -                                 &                        \\
Pós-condição          & -                                 &                        \\
                      & Ator                              & Sistema                \\
Fluxo Principal       & 1-Inserir nome de utilizador      &                        \\
                      & 2-Indicar Password                &                        \\
                      & 3-Indicar Confirmação de password &                        \\
                      &                                   & 4-Verificar se registo \\
                      &                                   & 5-Conta registada      \\
                      &                                   & 6-Validar conta        \\
Fluxo Alternativo(A1) & 1-Cancelar Ativação de conta      &                        
\end{tblr}
\end{table}
