\section{Descrição dos intervenientes}
O projeto envolve um conjunto de intervenientes, sendo estes, o utilizador, a empresa e o técnico. 
Estes desempenham um papel fundamental e podem realizar diferentes ações.

O utilizador sem sessão iniciada terá um fácil e rápido acesso ao catálogo de produtos, o que irá 
facilitar quando este deseja realizar a consulta do mesmo, terá também acesso apenas de visão de questões 
públicas do fórum, não conseguindo realizar nenhuma operação.

O técnico conseguirá realizar as mesmas ações que o utilizador, mas este ator conseguirá também ter 
acesso total ao fórum e a questões privadas. O fórum permite expor questões, com auxílio de imagens, 
permite a ligação da 
questão a uma categoria de questões e um produto em específico para facilitar a resolução da sua questão. 
As questões poderão ser públicas para assim qualquer um poder ver, ou então estas podem ser 
privadas para apenas técnicos conseguirem ver. Assim que o técnico estiver satisfeito com a sua 
questão este poderá indicar a melhor resposta que obteve para a destacar e então finalizar o tópico, 
para assim este ser indicado como finalizado.

O técnico pode também realizar pesquisas por questões em caso de ter algum problema. Com isto, este 
evita um telefonema ou o preenchimento de um formulário para contactar um técnico. Ao realizar pesquisas 
por questões este pode responder a outras questões, pode também pesquisar por questões em categoria 
e produto. As respostas podem conter imagens anexadas e podem também responder a outras 
respostas para manter uma comunicação continua. O técnico poderá também destacar tópicos e respostas de 
tópicos gostando destas.

A empresa pode realizar a gestão de contas de técnicos, conseguindo criar, impedir acesso e 
eliminar em caso de necessidade.
