\section{Descrição dos intervenientes}
O projeto envolve um conjunto de intervenientes, sendo estes, o utilizador, a empresa e o técnico. Estes desempenham um papel fundamental e podem realizar diferentes ações.

O utilizador, sem sessão iniciada terá apenas acesso ao catálogo de produtos.

O técnico conseguirá realizar as mesmas ações que o utilizador, mas este ator conseguirá também ter acesso total ao fórum e se for técnico oficial tem também acesso a questões privadas. O fórum permite expor questões, com auxílio de imagens, a ligação da questão a uma categoria e um produto para facilitar a sua resolução. As questões poderão ser públicas, ou então privadas, para apenas técnicos oficiais conseguirem ver. Assim que o técnico estiver satisfeito com a questão poderá indicar a melhor resposta obtida ao problema sendo esta destacada e o tópico finalizado.

O técnico pode realizar pesquisas por questões, onde evita um telefonema ou o preenchimento de um formulário para contactar um técnico oficial. Com estas pesquisas poderá responder a outras questões. As respostas podem conter imagens anexadas e responder a outras respostas para manter comunicação. O técnico poderá destacar tópicos e respostas de tópicos com um \textit{like}.

A empresa pode realizar a gestão de contas de técnicos, onde cria, impede acesso e elimina a conta em caso de necessidade.
