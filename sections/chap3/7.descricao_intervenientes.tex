\section{Descrição dos intervenientes}
O projeto envolve um conjunto de intervenientes, sendo estes, o utilizador, a empresa e o técnico. Estes desempenham um papel fundamental e podem realizar diferentes ações.

O utilizador, sem sessão iniciada terá um fácil e rápido acesso ao catálogo de produtos, o que irá facilitar quando este desejar realizar a consulta do mesmo.

O técnico conseguirá realizar as mesmas ações que o utilizador, mas este ator conseguirá também ter acesso total ao fórum e se for técnico oficial tem também acesso a questões privadas. O fórum permite expor questões, com auxílio de imagens, a ligação da questão a uma categoria de questões e um produto em específico para facilitar a resolução da questão. As questões poderão ser públicas para qualquer um, ou então estas podem ser privadas para apenas técnicos oficiais conseguirem ver. Assim que o técnico estiver satisfeito com a questão poderá indicar a melhor resposta obtida ao problema sendo esta destacada e a publicação finalizada.

O técnico pode realizar pesquisas por questões, onde evita um telefonema ou o preenchimento de um formulário para contactar um técnico oficial. Com estas pesquisas poderá responder a outras questões, pode pesquisar por questões em categoria. As respostas podem conter imagens anexadas e podem responder a outras respostas para manter uma comunicação continua. O técnico poderá destacar publicações e respostas de publicações com um \textit{like}.

A empresa pode realizar a gestão de contas de técnicos, onde cria, impede acesso e 
eliminar em caso de necessidade.
