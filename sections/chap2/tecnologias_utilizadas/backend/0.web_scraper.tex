\subsection{Web Scraper}
\textit{Web scraping} é terminologia dada à "\emph{construction of an agent to download, parse, and organize data from the web in an automated manner}"\citep{web_scraping}. O grande problema com \textit{web scraping} é que poderá ser considerado ilegal e é facilmente detetado. Tendo em conta este problema surgiram duas grandes formas de realizar \textit{web scraping}. A forma mais comum de \textit{web scraping} é realizar um pedido para obter uma página web e ler esta, sendo assim um processo rápido e simples.

A segunda forma de realizar \textit{web scraping} é através da simulação da ação humana com da abertura do navegador programáticamente, pesquisa pela página desejada, descarregar e daí ler os dados. Este torna-se um processo lento e complexo. 

A grande diferença entre estas duas formas é a velocidade, visto que a segunda forma tem de esperar que o navegador inicie, de seguida terá de esperar que a página carregue e apenas após este processo se poderá ler os dados.

\subsubsection{Selenium}

O \emph{Selenium} é uma ferramenta "\emph{for a range of tools and libraries that enable and support the automation of web browsers}"\citep{selenium}, esta provém "\emph{extensions to emulate user interaction with browsers}"\citep{selenium}. Na sua base esta "\emph{is WebDriver, an interface to write instruction sets that can be run interchangeably in many browsers}"\citep{selenium}. Esta ferramenta provém também a possibilidade de escalar com \textit{multi threading}, o que permite abrir diversas janelas do navegador simultaneamente e obter dados destas o que diminui drasticamente o tempo de execução, esta funcionalidade não foi explorada devido a limitações de \textit{hardware}, mas seria uma importante implementação futura.