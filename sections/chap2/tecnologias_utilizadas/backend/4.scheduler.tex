\subsubsection{Agendamento de tarefas}

Um requisito deste projeto foi o envio diário de \textit{emails} com o relatório de notificações ao final do dia. Primeiramente, para realizar o agendamento de tarefas, foi feita uma análise das ferramentas existentes para a realização deste tipo de ações. Deste modo, foram encontradas o \textit{cronetab} e o \textit{node-cron}. A grande diferença entre estas duas ferramentas é que o \textit{cronetab} funciona a nível de servidor, sendo que, sempre que se encontra na hora programada, este executa um comando indicado, este comando poderá por exemplo executar um código para enviar \textit{emails}. Já o \textit{node-cron} trata-se de uma biblioteca de \textit{NodeJs} que trabalha com base no \textit{crontab}, este permite o fácil agendamento de tarefas de forma programática, assim como a indicação do código a ser executado sem necessidade de criar comandos.

A hora de execução do código de envio de \textit{emails} poderá variar e necessitar de reprogramação, pelo que, foi optada a utilização do \textit{node-cron}, uma vez que, facilita a utilização e agilizando o processo de reprogramação de horas de envio do relatório de notificações.