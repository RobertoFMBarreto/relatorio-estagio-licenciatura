\subsubsection{Agendamento de tarefas}

Um requisito deste projeto foi o envio diário de \textit{emails} com o relatório de notificações ao final do dia. Primeiramente, para realizar o agendamento de tarefas, foi feita uma análise das ferramentas existentes para a realização deste tipo de ações. Deste modo, foram encontradas o \textit{Cronetab} e o \textit{Node-Cron}. A grande diferença entre estas duas ferramentas é que o \textit{Cronetab} funciona a nível de servidor, sendo que, o funcionamento tem por base "(...)\emph{run this command at this time
on this date}(...)"\citep{crontab}, este comando poderá por exemplo executar um código para enviar \textit{emails}. Já o \textit{Node-Cron} trata-se de uma biblioteca de \textit{NodeJs} "(...)\emph{in pure JavaScript for node.js based on GNU crontab}(...)"\citep{node_cron}, este permite o fácil agendamento de tarefas de forma programática, assim como a indicação do código a ser executado sem necessidade de criar comandos.

A hora de execução do código de envio de \textit{emails} poderá variar e necessitar de reprogramação, pelo que, foi optada a utilização do \textit{Node-Cron}, uma vez que, facilita a utilização e agiliza o processo de reprogramação de horas de envio do relatório de notificações.