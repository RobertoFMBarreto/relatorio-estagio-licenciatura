\subsubsection{Cifragem de \textit{passwords}}
Para garantir a segurança das \textit{passwords} dos utilizadores é necessário cifrar estas, a cifragem poderá ser realizada manualmente ou com o auxílio de ferramentas, a grande diferença é que manualmente poderá não se obter uma cifra tão segura como com o auxílio de uma ferramenta. Sendo assim, foi decidido utilizar uma ferramenta para cifrar \textit{passwords}, a ferramenta escolhida foi \textit{bcrypt}. Esta foi escolhida devido a ser vastamente utilizada e tem por base a \emph{hash} \textit{bcrypt}, esta "\emph{uses a variant of the Blowfish encryption algorithm’s keying schedule, and introduces a work factor, which allows you to determine how expensive the hash function will be}"\citep{bcrypt} isto permite que esta acompanhe a lei de \emph{Moore}, pois "\emph{As computers get faster you can increase the work factor and the hash will get slower}"\citep{bcrypt} . Esta ferramenta oferece um conjunto de métodos dos quais foram utilizados os de cifra e de comparação. O método de \emph{hash} permite através de um valor, chamado \textit{salt}, que não indica a complexidade a aplicar sobre a cifra, sendo de seguida devolvida a \textit{\textit{password}} cifrada. O método comparação permite comparar uma \textit{\textit{password}} cifrada com uma \textit{\textit{password}} sem cifra, devolvendo verdadeiro ou falso conforme as \textit{passwords} sejam iguais ou não.

\subsubsection{Cifragem de configurações do servidor}
Com o objetivo de garantir um nível de segurança maior foram realizadas pesquisas sobre as principais falhas de segurança no \textit{NodeJs}. Nestas, foi descoberto que as principais formas de ataque são as bibliotecas de \textit{malware} e o ataque direto com o objetivo de obter dados de acesso a servidores que se encontram nos ficheiros de configuração.

Por norma todas as configurações de servidores são colocadas num ficheiro \textit{env}, este ficheiro no momento de iniciar o servidor é utilizado para carregar todas as variáveis para o ambiente do mesmo. Sendo assim qualquer um com acesso ao ficheiro ou às variáveis de ambiente poderá ver todas as configurações do servidor.

A solução mais indicada para este problema é a cifragem do ficheiro \textit{env} e das variáveis de ambiente. A biblioteca mais utilizada para este objetivo é a \textit{secur-env}, visto que, esta permite realizar a cifragem de um ficheiro com a indicação de uma \textit{\textit{password}}. A \textit{\textit{password}} deverá ser indicada no processo de inicialização do servidor de forma a ser possível ao mesmo decifrar o ficheiro, sendo que a gestão das variáveis cifradas passa então a estar encarregue desta biblioteca.

Embora exista esta solução, continuam a haver possibilidades de ataque, uma vez que, é possível ver o histórico do terminal do servidor, o que permite obter a \textit{\textit{password}} escrita. Para resolver este problema é indicada a biblioteca \textit{readline}, pois esta possui o modo de \textit{\textit{password}} que apaga o histórico do terminal sempre que utilizado. Esta contém a vertente assíncrona, \textit{readline} e a vertente síncrona, \textit{readline-sync}. Neste projeto, foi utilizada a versão síncrona da biblioteca visto que o objetivo é o servidor apenas inicie após a indicação da \textit{\textit{password}} sem nenhum serviço a correr em simultâneo.