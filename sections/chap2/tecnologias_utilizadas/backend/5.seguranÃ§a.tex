\subsubsection{Encriptação de passwords}
De forma a garantir a segurança das passwords dos utilizadores é necessário encriptar estas, a encriptação poderá ser feita manualmente ou com o auxílio de ferramentas, a grande diferença é que manualmente poderá não se obter uma cifra tão segura como com o auxílio de uma ferramenta, sendo assim foi decidido utilizar uma ferramenta para encriptar passwords, a ferramenta escolhida foi bcrypt, esta foi escolhida devido a ser vastamente utilizada e até ao momento sem problemas em relação à sua cifra sendo que não existem registos de ataques bem sucedidos a esta ferramenta. Esta ferramenta oferece um conjunto de métodos sendo tendo sido utilizados os métodos de cifra e de comparação. O método de cifra permite através de um valor, chamado salt, indicar a complexidade a aplicar sobre a cifra, sendo de seguida devolvida a password cifrada. O método comparação permite comparar uma password cifrada com uma password sem cifra, devolvendo verdadeiro ou falso conforme as passwords sejam iguais ou não.

\newpage

\subsubsection{Cifragem de configurações do servidor}
De forma a garantir um nivel de segurança maior foram realizadas pesquisas sobre as principais falhas de segurança no NodeJs, pelo que foi descoberto que as principais formas de ataque a esta ferramenta é o desenvolvimento de bibliotecas de malware e o ataque às bibliotecas com o objetivo de obter dados de acesso a servidores que se encontram nos ficheiros de configuração.

Por norma todas as configurações de servidores são colocadas num ficheiro env, este ficheiro no momento de iniciar o servidor é utilizado para carregar todas as variáveis para o ambiente do mesmo, sendo assim qualquer um com acesso ao ficheiro ou às variáveis de ambiente poderá ver todas as configurações do servidor.

A solução mais indicada para este problema é a cifragem do ficheiro env e das variáveis de ambiente. A biblioteca mais utilizada para este objetivo é a secur-env, esta permite realizar a cifragem de um ficheiro indicando uma password. A password deverá ser indicada no processo de inicialização do servidor de forma a ser possível ao mesmo decifrar o ficheiro, sendo que a gestão das variáveis cifradas passa então a estar encarregue desta biblioteca.

Mesmo com esta solução existem possibilidades de ataque, pois é possível ver o histórico do terminal do servidor, pelo que e possível obter a password escrita no mesmo, para resolver este problema é indicada a biblioteca readline, pois esta possui o modo de password que apaga o histórico do terminal sempre que utilizado, esta contém a vertente assincrona, readline e a vertente síncrona, readline-sync. Para o projeto foi utilizada a versão síncrona da biblioteca visto que o objetivo é o servidor apenas inciar após a indicação da password, sem nenhum serviço a correr em simultâneo.