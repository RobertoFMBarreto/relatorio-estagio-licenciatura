\subsubsection{Logs e Logging}
Logs \emph{are a very useful source of information for computer system resource management (printers, disk systems, battery backup systems, operating systems, etc.), user and application management (login and logout, application access, etc.), and security}\citep{Logging}. Um Log é \emph{what a computer system, device, software, etc. generates in response to some sort of stimuli.What exactly the stimuli are greatly depends on the source of the log message}\citep{Logging}, ou seja perante um estimulo desejado, um log poderá ser gerado. Os dados dos logs são \emph{the intrinsic meaning that a log message has}\citep{Logging}, significando isto que estes contêm apenas dados relevantes ao objetivo do log. Logging é o nome que se dá ao processo de geração de logs.

Neste contexto logging poderá ser utilizado para realizar a monitorização de pedidos e erros. Estas informações poderão até auxiliar na toma de decisões sobre o software e em quais funcionalidades deste software colocar mais atenção.

\subsubsection{Morgan}

Morgan é uma biblioteca que permite extrair dados de um pedido, assim como também a criação de logs, este atua como um middleware do servidor, recebendo qual o tipo de log a ser escrito, sendo estes tipos definidos pela biblioteca. Os principais dados obtidos pela biblioteca são a data e hora do pedido, o tipo de pedido, o serviço pedido, os dados recebidos, a resposta devolvida e também a descrição do sistema utilizado para realizar o pedido. Com estes dados é possível saber que plataforma é mais utilizada no software, quais as horas de maior utilização e quais os serviços mais executados, estes dados permitem direcionar mais recursos para uma indicada plataforma e/ou serviço, assim como também escolher os melhores horários de manutenção dos servidores.