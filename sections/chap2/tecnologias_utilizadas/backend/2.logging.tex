\subsubsection{Logs e Logging}
\textit{Logs} "(...)\emph{are a very useful source of information for computer system resource management (printers, disk systems, battery backup systems, operating systems, etc.), user and application management (login and logout, application access, etc.), and security}(...)"\citep{Logging}. Um Log é "(...)\emph{what a computer system, device, software, etc. generates in response to some sort of stimuli.What exactly the stimuli are greatly depends on the source of the log message}(...)"\citep{Logging}, ou seja, perante um estímulo desejado, um \textit{log} poderá ser gerado. Os dados dos \textit{logs} são "(...)\emph{the intrinsic meaning that a log message has}(...)"\citep{Logging}, o que significa que estes contêm apenas dados relevantes ao objetivo do \textit{log}. \textit{Logging} é o nome que se dá ao processo de geração de \textit{logs}.

Neste contexto \textit{logging} poderá ser utilizado para realizar a monitorização de pedidos e erros. Estas informações poderão até auxiliar na toma de decisões sobre o \textit{software} e em quais funcionalidades colocar mais atenção.

\subsubsection{Morgan}

\textit{Morgan} é uma biblioteca que permite extrair dados de um pedido, assim como a criação de \textit{logs}. Este atua como um \textit{middleware} do servidor, que recebe qual o tipo de \textit{log} a ser escrito, estes estão definidos na biblioteca. Os principais dados obtidos por este são, a data e hora do pedido, o tipo, o serviço, os dados recebidos, a resposta devolvida e também a descrição do sistema utilizado. Com estes dados é possível saber que plataforma é mais utilizada no \textit{software}, quais as horas de maior utilização e quais os serviços mais executados. Estes dados permitem direcionar mais recursos para uma indicada plataforma e/ou serviço, assim como escolher os melhores horários de manutenção dos servidores.