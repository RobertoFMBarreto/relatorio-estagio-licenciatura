\subsubsection{Firebase}
\emph{Firebase} é uma solução que foi comprada pela \textit{Google} em 2014. O seu objetivo é "(...)\emph{to provide the tools and infrastructure that you need to build great apps}(...)"\citep{Moroney2017}, esta alcança este objetivo através da oferta de serviços pré configurados, sendo que "(...)\emph{many of the technologies are available at no cost}(...)"\citep{Moroney2017}.

\textit{Firestorage} também conhecida como \textit{cloud storage}, é um serviço que dispõe "(...)\emph{a simple API that is backed up by Google Cloud Storage}(...)"\citep{Moroney2017}, que permite guardar e transmitir até um gigabyte de ficheiros de forma gratuita.

\textit{Cloud messaging} é também um serviço do \textit{Firebase} que permite "(...)\emph{to reliably deliver messages at no cost}(...)"\citep{Moroney2017}. Este garante que" \emph{Over 98\% of connected devices receive these messages in less than 500ms}(...)"\citep{Moroney2017}. \textit{Cloud messaging} permite a utilização de diversas formas de envio de notificações como "(...)\emph{driven by analytics to pick audiences, or using topics or other methods}(...)"\citep{Moroney2017}.

\textit{Dynamic links} é um serviço do \textit{Firebase} que permite a criação de "(...)\emph{links to an app that contain context about what you want the end user to see in the app}(...)"\citep{Moroney2017}.

Esta ferramenta foi escolhida devido à sua capacidade de fornecer serviços pré configurados de forma gratuita, o que evita o gasto monetário e a alocação de tempo para a configuração de servidores durante o desenvolvimento.