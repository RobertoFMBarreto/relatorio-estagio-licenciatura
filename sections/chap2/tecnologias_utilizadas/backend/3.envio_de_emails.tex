\subsubsection{Gestão de \textit{emails}}
O envio de \textit{emails} para os utilizadores, foi desenvolvido através da biblioteca \textit{nodemailer}, que permite a utilização de um servidor de \textit{SMTP}. Esta ferramenta foi escolhida devido a ser uma das mais utilizadas para este tipo de necessidade, o que permite que exista mais informação sobre a mesma, o que auxilia a resolução e identificação de erros. 

Para desenvolver o conteúdo dos \textit{emails} foi utilizada a ferramenta \textit{Tabular Email}, esta ferramenta permite realizar o \textit{design} do conteúdo de um \textit{email}, sendo possível exportar para \textit{html}. A maior dificuldade desta ferramenta é que não permite a utilização de acentuação, e visto que o \textit{html} é gerado por uma máquina este torna-se complicado de navegar e traduzir.