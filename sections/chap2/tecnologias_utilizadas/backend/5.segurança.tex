\subsubsection{Encriptação de \textit{passwords}}
Para garantir a segurança das \textit{passwords} dos utilizadores é necessário encriptar estas, a encriptação poderá ser realizada manualmente ou com o auxílio de ferramentas, a grande diferença é que manualmente poderá não se obter uma encriptação tão segura como com o auxílio de uma ferramenta. Sendo assim, foi decidido utilizar uma ferramenta para encriptação de \textit{passwords}, a ferramenta escolhida foi \textit{Bcrypt}. Esta foi escolhida devido a ser vastamente utilizada e tem por base a \emph{Hash} \textit{Bcrypt}, esta "(...)\emph{uses a variant of the Blowfish encryption algorithm’s keying schedule, and introduces a work factor, which allows you to determine how expensive the Hash function will be}(...)"\citep{Bcrypt} isto permite que esta acompanhe a lei de \emph{Moore}, pois "(...)\emph{As computers get faster you can increase the work factor and the Hash will get slower}(...)"\citep{Bcrypt}. Esta ferramenta oferece um conjunto de métodos dos quais foram utilizados os de \emph{Hash} e de \emph{compare}. O método de \emph{Hash} permite através de uma password e um valor chamado \textit{salt}, que não indica a complexidade a aplicar sobre a encriptação, obter a \textit{password} encriptada. O método \emph{compare} permite comparar uma \textit{password} encriptada com uma \textit{password} não encriptada, e devolve verdadeiro ou falso conforme as \textit{passwords} sejam iguais ou não.

\subsubsection{Encriptação de configurações do servidor}
Com o objetivo de garantir um nível de segurança maior foram realizadas pesquisas sobre as principais falhas de segurança no \textit{NodeJs}. Nestas, foi descoberto que as principais formas de ataque são as bibliotecas de \textit{malware} e o ataque direto com o objetivo de obter dados de acesso a servidores que se encontram nos ficheiros de configuração.

Por norma, nas metodologias mais recentes de desenvolvimento de \emph{software}, é sugerido que se coloque todas as configurações de servidores num ficheiro à parte, devido à praticidade de gerir estes dados, mas esta leva a um nível de segurança mais baixo, uma vez que, se alguém conseguir acesso a este ficheiro, consegue obter todos os dados de configuração de servidores. Neste projeto foi utilizado o ficheiro \textit{env} para este fim, este no momento de iniciar o servidor é utilizado para carregar todas as variáveis para o ambiente do mesmo. Sendo assim qualquer um com acesso ao ficheiro ou às variáveis de ambiente poderá ver todas as configurações do servidor.

A solução mais indicada para este problema é a encriptação do ficheiro \textit{env} e das variáveis de ambiente. A biblioteca mais utilizada para este objetivo é a \textit{Secur-Env}, visto que, esta permite realizar a encriptação de um ficheiro com a indicação de uma \textit{password}. A \textit{password} deverá ser indicada no processo de inicialização do servidor de forma a ser possível ao mesmo desencriptar o ficheiro, sendo que a gestão das variáveis encriptadas passa então a estar encarregue desta biblioteca.

Embora exista esta solução, continuam a haver possibilidades de ataque, uma vez que, é possível ver o histórico do terminal do servidor, o que permite obter a \textit{password} escrita. Para resolver este problema é indicada a biblioteca \textit{Readline}, pois esta possui o modo de \textit{password} que apaga o histórico do terminal sempre que utilizado. Esta contém a vertente assíncrona, \textit{Readline} e a vertente síncrona, \textit{Readline-Sync}. Neste projeto, foi utilizada a versão síncrona da biblioteca visto que o objetivo é o servidor apenas inicie após a indicação da \textit{password} sem nenhum serviço a correr em simultâneo.