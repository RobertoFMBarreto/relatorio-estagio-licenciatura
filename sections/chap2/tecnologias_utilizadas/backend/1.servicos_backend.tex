\subsection{Serviços Backend}

Para a realização da integração entre a aplicação \emph{frontend} e os dados, foi necessário desenvolver uma API para dar suporte a todos os serviços necessários para a aplicação.
API sigla para "\emph{Application Programming Interface} \emph{exposes a set of data and functions to facilitate interactions between computer programs and allow them to exchange information.}" ~\citep{rest_cookbook}.
Estas ferramentas apesar de serem "\emph{designed  to  work  with  other  programs,  they’re  mostly intended to be understood and used by humans writing those other programs}"\citep{api_design}".

\newpage

\subsubsection{Serviços RestFull e SOAP}
Os seviços RestFull "\emph{expose data as resources  and  use  standard  HTTP  methods  to  represent  Create,  Read,Update,  and  Delete  (CRUD)  transactions  against  these  resources}
\citep{api_design}", sendo que a resposta é no formato JSON "\emph{due  to  its  simplicity  and  ease  of  use  with  JavaScript,  JSON  has become the standard for modern APIs}"\citep{api_design}.

Já SOAP é "\emph{nothing  more  than  a  simple  XML-based  envelope  for  the  information  being  transferred,  and  a  set  of  rules  for  translating  application and platform-specific data types into XML representations}"\citep{Snell2002} sendo que a resposta é realizada em XML leva a que "\emph{It relies heavily on XML  standards  like  XML  Schema  and  XML  Namespaces  for  its  definition  and  function}"\citep{Snell2002}. A utilização de XML permite que "\emph{two  applications,  regardless  of  operating  system,  programming  language,  or  any  other  technical  implementation  detail,  may  openly  share  information  using  nothing  more  than  a  simple  message  encoded  in  a  way  that  both  applications  understand}"\citep{Snell2002}.

Por fim foi decidido utilizar Rest devido a ser \emph{much lighter compared to SOAP. It does not require formats like headers to be included in the message, like it is required in SOAP architecture}". Outra maior valia é que "\emph{ it parses JSON, a human readable language designed to allow data exchange and making it easier to parse and use by the computer. It is estimated to be at around one hundred times faster than XML}"\citep{Halili2018}.

\subsubsection{NodeJS}

Para o desenvolvimento do projeto backend foi escolhido \textit{NodeJS}, este surgiu quando "\emph{the original developers took JavaScript, something you could usually only run inside the browser, and they let it run on your machine as a standalone process}"\citep{design_node} isto significa que "\emph{applications using JavaScript outside the context of the browser}"\citep{design_node} .

Para correr \textit{JavaScript} a nível de servidor web e a nível de computador pessoal é utilizado o mesmo motor, "\emph{it's called the V8 JavaScript runtime engine}"\citep{design_node} , este é "\emph{it's an opensource engine that takes JavaScript code and compiles it into much faster machine code.And that's a big part of what makes Node.js so fast}"\citep{design_node} .

\textit{NodeJs} permite a utilização de bibliotecas externas "\emph{by providing the Node Package Manager}"\citep{design_node} , este permite ao desenvolvedor "\emph{to easily install, manage, and even provide  your own modules for a rapidly grown and well-maintained open source repository}"\citep{design_node} .

\textit{NodeJS} foi escolhido para o desenvolvimento \textit{backend}, uma vez que, permite a utilização de \textit{Typescript} para desenvolvimento e por ser vastamente utilizado, o que permite acesso a diversas fontes de informação para resolução de problemas, assim como auxílio ao desenvolvimento.

\newpage

\subsubsection{Typescript}
O typescript "\emph{is  a  bit  unusual  as  a  language  in  that  it  neither  runs  in  an  interpreter  (asPython  and  Ruby  do)  nor  compiles  down  to  a  lower-level  language  (as  Java  and  Cdo).}"\citep{typescript} isto porque este "\emph{compiles  to  another  high-level  language,  JavaScript.}"\citep{typescript}, isto faz com que o Typescript seja visto como "\emph{a  superset  of  JavaScript  in  a  syntactic  sense}"\citep{typescript}.

Todos os programas JavaScript" \emph{are  TypeScript  programs,  but  the  converse  is  not  true... This  is  because  TypeScript adds additional syntax for specifying types}". O sistema de tipagens do TypeScript tem como objetivo "\emph{to  detect  code  that  will  throw  anexception  at  runtime,  without  having  to  run  your  code. ... The  type  checker  cannot always spot code that will throw exceptions, but it will try}"\citep{typescript}.

Esta linguagem foi escolhida para o \textit{backend} devido à capacidade de assegurar as tipagens, o que proporciona um maior nível de segurança quando se lida com dados recebidos, assim como a agilização do processo de programação devido à capacidade de prever a maioria dos erros de código.

\subsection{PostgreSQL}
\emph{PostgreSQL} "is an open source object relational database management system"\citep{Juba2015}. Esta "emphasizes extensibility"\citep{Juba2015} o que permite a utilização de extensões desenvolvidas pela comunidade, mas "Also, there are several extensions to access, manage, and monitor PostgreSQL clusters, such as pgAdmin III"\citep{Juba2015}. Esta ferramenta é de aprendizagem simples devido ao facto de "it complies with ANSI SQL standards"\citep{Juba2015}, o que leva a que, qualquer individuo com conhecimentos prévios em \emph{SQL} consiga facilmente aprender esta tecnologia.

\emph{PostgreSQL} foi escolhido visto que a empresa já utiliza vastamente esta tecnologia, mas também porque que é \emph{open source} e não existirem custos associados para este tipo de utilização. A funcionalidade de extensões foi utilizada para implementar \emph{id's} que utilizam a estrutura \emph{uuid} que dificultam o ataque aos dados, visto que, é difícil de prever os valores de \emph{id's}.

\subsubsection{Logs e Logging}
\textit{Logs} "(...)\emph{are a very useful source of information for computer system resource management (printers, disk systems, battery backup systems, operating systems, etc.), user and application management (login and logout, application access, etc.), and security}(...)"\citep{Logging}. Um Log é "(...)\emph{what a computer system, device, software, etc. generates in response to some sort of stimuli.What exactly the stimuli are greatly depends on the source of the log message}(...)"\citep{Logging}, ou seja, perante um estímulo desejado, um \textit{log} poderá ser gerado. Os dados dos \textit{logs} são "(...)\emph{the intrinsic meaning that a log message has}(...)"\citep{Logging}, o que significa que estes contêm apenas dados relevantes ao objetivo do \textit{log}. \textit{Logging} é o nome que se dá ao processo de geração de \textit{logs}.

\newpage

Neste contexto \textit{logging} poderá ser utilizado para realizar a monitorização de pedidos e erros. Estas informações poderão até auxiliar na toma de decisões sobre o \textit{software} e em quais funcionalidades colocar mais atenção.

\subsubsection{Morgan}

\textit{Morgan} é uma biblioteca que permite extrair dados de um pedido, assim como a criação de \textit{logs}. Este atua como um \textit{middleware} do servidor, que recebe qual o tipo de \textit{log} a ser escrito, estes estão definidos na biblioteca. Os principais dados obtidos por este são, a data e hora do pedido, o tipo, o serviço, os dados recebidos, a resposta devolvida e também a descrição do sistema utilizado. Com estes dados é possível saber que plataforma é mais utilizada no \textit{software}, quais as horas de maior utilização e quais os serviços mais executados. Estes dados permitem direcionar mais recursos para uma indicada plataforma e/ou serviço, assim como escolher os melhores horários de manutenção dos servidores.

\subsubsection{Envio de emails}
Para o envio de emails para os utilizadores, foi utilizada a ferramenta nodemailer, EXPLICAR O QUE é nodemailer..... Esta ferramenta foi escolhida devido a ser uma das mais utilizadas para esta tipo de necessidade, o que permite que exista mais informação sobre a mesma facilitando a resolução e identificação de erros. 

Para desenvolver o conteúdo dos emails foi utilizada a ferramenta Tabular Email, esta ferramenta permite realizar o design do conteúdo de um email, sendo possível de seguida exportar o mesmo para html, a dificuldade desta ferramenta é que não permite a utilização de acentuação e visto que o html é gerado por uma máquina este torna-se complicado de navegar e traduzir.

Após obter o servidor a utilizar e o conteúdo a enviar, é então utilizado o objeto do servidor de email e no envio é definido o destinatário, o assunto e o conteúdo do email.


\subsubsection{Agendamento de tarefas}

Um requisito para este projeto é o envio de emails com relatório diário de notificação todos os dias ao final do dia. Para realizar este primeiramente foi pesquisado que ferramentas existem para realizar este tipo de ações, pelo que foram encontradas o cronetab e o node-cron. A grande diferença este estas duas ferramentas é, o cronetab funciona a nivel de servidor sendo que sempre que se encontra na hora programada, este executa um comando indicado, este comando poderá por exemplo executar um código para enviar emails. Já o node-cron trata-se de uma biblioteca de NodeJs que trabalha com base no crontab, este permite o fácil agendamento de tarefas de forma programática assim como também a indicação do código a ser executado sem necessidade de criar comandos de execução de código.

Visto que a hora de execução do código de envio de emails poderá variar e necessitar de reprogramação foi então optado pela utilização do node-cron devido á sua facilidade de utilização, agilizando assim o processo de reprogramação de horas de envio de relatório.

\newpage

\subsubsection{Encriptação de passwords}
De forma a garantir a segurança das passwords dos utilizadores é necessário encriptar estas, a encriptação poderá ser feita manualmente ou com o auxílio de ferramentas, a grande diferença é que manualmente poderá não se obter uma cifra tão segura como com o auxílio de uma ferramenta, sendo assim foi decidido utilizar uma ferramenta para encriptar passwords, a ferramenta escolhida foi bcrypt, esta foi escolhida devido a ser vastamente utilizada e até ao momento sem problemas em relação à sua cifra sendo que não existem registos de ataques bem sucedidos a esta ferramenta. Esta ferramenta oferece um conjunto de métodos sendo tendo sido utilizados os métodos de cifra e de comparação. O método de cifra permite através de um valor, chamado salt, indicar a complexidade a aplicar sobre a cifra, sendo de seguida devolvida a password cifrada. O método comparação permite comparar uma password cifrada com uma password sem cifra, devolvendo verdadeiro ou falso conforme as passwords sejam iguais ou não.

\newpage

\subsubsection{Cifragem de configurações do servidor}
De forma a garantir um nivel de segurança maior foram realizadas pesquisas sobre as principais falhas de segurança no NodeJs, pelo que foi descoberto que as principais formas de ataque a esta ferramenta é o desenvolvimento de bibliotecas de malware e o ataque às bibliotecas com o objetivo de obter dados de acesso a servidores que se encontram nos ficheiros de configuração.

Por norma todas as configurações de servidores são colocadas num ficheiro env, este ficheiro no momento de iniciar o servidor é utilizado para carregar todas as variáveis para o ambiente do mesmo, sendo assim qualquer um com acesso ao ficheiro ou às variáveis de ambiente poderá ver todas as configurações do servidor.

A solução mais indicada para este problema é a cifragem do ficheiro env e das variáveis de ambiente. A biblioteca mais utilizada para este objetivo é a secur-env, esta permite realizar a cifragem de um ficheiro indicando uma password. A password deverá ser indicada no processo de inicialização do servidor de forma a ser possível ao mesmo decifrar o ficheiro, sendo que a gestão das variáveis cifradas passa então a estar encarregue desta biblioteca.

Mesmo com esta solução existem possibilidades de ataque, pois é possível ver o histórico do terminal do servidor, pelo que e possível obter a password escrita no mesmo, para resolver este problema é indicada a biblioteca readline, pois esta possui o modo de password que apaga o histórico do terminal sempre que utilizado, esta contém a vertente assincrona, readline e a vertente síncrona, readline-sync. Para o projeto foi utilizada a versão síncrona da biblioteca visto que o objetivo é o servidor apenas inciar após a indicação da password, sem nenhum serviço a correr em simultâneo.

\newpage

\subsubsection{Firebase}
Firebase é uma framework comprada pela google em 2014, o seu objetivo é \emph{to provide the tools and infrastructure that you need to build great apps}\citep{Moroney2017}, esta alcança este objetivo através da oferta de serviços pré configurados, sendo que \emph{many of the technologies are available at no cost}\citep{Moroney2017}.

Firestorage também conhecida como cloud storage, é um serviço que dispõe \emph{a simple API that is backed up by Google Cloud Storage}\citep{Moroney2017}, que permite guardar e transmitir até 1GB de ficheiros de forma gratuita.

Cloud messaging é também um serviço do firebase que permite \emph{to reliably deliver messages at no cost}\citep{Moroney2017}. Este garante que \emph{Over 98\% of connected devices receive these messages in less than 500ms}\citep{Moroney2017}. Cloud messaging tambem permite a utilização de diversas formas de envio de notificações como \emph{driven by analytics to pick audiences, or using topics or other methods}\citep{Moroney2017}.

Dynamic links é um serviço do firebase que permite a criação de \emph{links to an app that contain context about what you want the end user to see in the app}\citep{Moroney2017}..

Esta ferramenta foi escolhida devido à sua capacidade de fornecer serviços pré configurados de forma gratuita, evitando assim o gasto monetário e de tempo durante o desenvolvimento.

\subsection{Axios}

Para ser possível realizar pedidos a outros serviços externos como \textit{Firebase}, é necessário utilizar uma biblioteca capaz do mesmo. Para isso foi optado por utilizar a biblioteca \textit{Axios}. Esta biblioteca é "\emph{a promise-based HTTP Client for node.js}"\citep{axios} que "\emph{uses the native node.js http module}"\citep{axios}. Esta disponibiliza um conjunto de métodos para a realização de pedidos a serviços externos, assim como a configuração total dos mesmos.