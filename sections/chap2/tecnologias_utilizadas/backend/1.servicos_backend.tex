\subsection{Serviços Backend}

Para a realização da integração entre a aplicação \emph{frontend} e os dados, foi necessário desenvolver uma \textit{\acrfull{api}} para dar suporte a todos os serviços necessários para a aplicação.
Uma \textit{\acrshort{api}} "(...)\emph{exposes a set of data and functions to facilitate interactions between computer programs and allow them to exchange information.}(...)"~\citep{rest_cookbook}.
Estas ferramentas apesar de serem "(...)\emph{designed to work with other programs, they’re mostly intended to be understood and used by humans writing those other programs}(...)"\citep{api_design}.

\newpage

\subsubsection{Serviços RestFull e SOAP}
Os seviços \emph{RestFull} "(...)\emph{expose data as resources and use standard HTTP methods to represent Create, Read,Update, and Delete (CRUD) transactions against these resources}(...)"
\citep{api_design}, sendo que a resposta é no formato \textit{\acrfull{json}} "(...)\emph{due to its simplicity and ease of use with JavaScript, JSON has become the standard for modern APIs}(...)"\citep{api_design}.

Já \emph{SOAP} é "(...)\emph{XML-based envelope for the information being transferred, and a set of rules for translating application and platform-specific data types into XML representations}(...)"\citep{Snell2002} sendo que a resposta é realizada em \emph{XML} leva a que "(...)\emph{It relies heavily on XML standards like XML Schema and XML Namespaces for its definition and function}(...)"\citep{Snell2002}. A utilização de \emph{XML} permite que "(...)\emph{two applications, regardless of operating system, programming language, or any other technical implementation detail, may openly share information using nothing more than a simple message encoded in a way that both applications understand}(...)"\citep{Snell2002}.

Por fim foi decidido utilizar \emph{RestFull} devido a ser "(...)\emph{much lighter compared to SOAP. It does not require formats like headers to be included in the message, like it is required in SOAP architecture}(...)"\citep{Halili2018}. Outra maior valia é que "(...)\emph{it parses JSON, a human readable language designed to allow data exchange and making it easier to parse and use by the computer. It is estimated to be at around one hundred times faster than XML}(...)"\citep{Halili2018}.

\subsubsection{NodeJS}

Para o desenvolvimento do projeto \emph{backend} foi escolhido \textit{NodeJS}, este surgiu quando "(...)\emph{the original developers took JavaScript, something you could usually only run inside the browser, and they let it run on your machine as a standalone process}(...)"\citep{design_node} isto significa que é possível correr código "(...)\emph{JavaScript outside the context of the browser}(...)"\citep{design_node} .

Para correr \textit{JavaScript} a nível de servidor \emph{web} e a nível de computador pessoal é utilizado o mesmo motor, "(...)\emph{it's called the V8 JavaScript runtime engine}(...)"\citep{design_node}, este é "(...)\emph{an opensource engine that takes JavaScript code and compiles it into much faster machine code. And that's a big part of what makes Node.js so fast}(...)"\citep{design_node} .

\textit{NodeJs} permite a utilização de bibliotecas externas "(...)\emph{by providing the Node Package Manager}(...)"\citep{design_node}, este provém ao programador "(...)\emph{to easily install, manage, and even provide your own modules for a rapidly grown and well-maintained open source repository}(...)"\citep{design_node}.

\textit{NodeJS} foi escolhido para o \textit{backend}, uma vez que, permite a utilização de \textit{Typescript} para o desenvolvimento e por ser vastamente utilizado, o que dá acesso a diversas fontes de informação para a resolução de problemas, assim como, para o auxílio ao desenvolvimento.

\newpage

\subsubsection{Typescript}
O Typescript "(...)\emph{is a bit unusual as a language in that it neither runs in an interpreter (as Python and Ruby do) nor compiles down to a lower-level language (as Java and C do).}(...)"\citep{typescript} isto porque este "(...)\emph{compiles to another high-level language, JavaScript}(...)"\citep{typescript}, isto faz com que o \emph{Typescript} seja visto como "(...)\emph{a superset of JavaScript in a syntactic sense}(...)"\citep{typescript}.

Todos os programas \emph{JavaScript} "(...)\emph{are TypeScript programs, but the converse is not true}(...)"\citep{typescript}, "(...)\emph{This is because TypeScript adds additional syntax for specifying types}(...)"\citep{typescript}. O sistema de tipagens do \emph{TypeScript} tem como objetivo "(...)\emph{to detect code that will throw an exception at runtime, without having to run your code}(...)"\citep{typescript}. "(...)\emph{The type checker cannot always spot code that will throw exceptions, but it will try}(...)"\citep{typescript}.

Esta linguagem foi escolhida para o \textit{backend}, visto que, assegura as tipagens, o que proporciona um maior nível de segurança quando se trabalha com dados recebidos, assim como, a agilização do processo de programação devido à capacidade de prever a maioria dos erros de código.

\subsubsection{PostgreSQL}
\emph{PostgreSQL} "(...)\emph{is an open source object relational database management system}(...)"\citep{Juba2015}. Esta "(...)\emph{emphasizes extensibility}(...)"\citep{Juba2015} o que permite a utilização de extensões desenvolvidas pela comunidade, mas "(...)\emph{Also, there are several extensions to access, manage, and monitor PostgreSQL clusters, such as pgAdmin III}(...)"\citep{Juba2015}. Esta ferramenta é de aprendizagem simples devido ao facto de "(...)\emph{it complies with ANSI SQL standards}(...)"\citep{Juba2015}, o que leva a que, qualquer indivíduo com conhecimentos prévios em \emph{SQL} consiga facilmente aprender esta tecnologia.

\emph{PostgreSQL} foi escolhido para a base de dados, dado que, a empresa já utiliza vastamente esta tecnologia, mas também porque é \emph{open-source} e não existem custos associados para este tipo de utilização. A funcionalidade de extensões foi utilizada para implementar \emph{id's}, que utilizam a estrutura \emph{uuid}, o que dificulta o ataque aos dados, uma vez que, é difícil de prever os valores de \emph{id's}.

\subsubsection{Logging}
Logging é um processo que permite guardar informação(logs) sobre um evento. Neste contexto logging poderá ser utilizado para realizar a monitorização de pedidos e erros. Estas informações poderão até auxiliar na toma de decisões sobre o software e em quais funcionalidades deste software colocar mais atenção.

\subsubsection{Morgan}

Morgan é uma ferramenta que permite extrair dados de um pedido, assim como também a criação de logs, este atua como um middleware do servidor, recebendo qual o tipo de log a ser escrito, sendo estes tipos definidos pela ferramenta. Os principais dados obtidos pela ferramenta são a data e hora do pedido, o tipo de pedido, o serviço pedido, os dados recebidos, a resposta devolvida e também a descrição do sistema utilizado para realizar o pedido. Com estes dados é possível saber que plataforma é mais utilizada no software, quais as horas de maior utilização e quais os serviços mais executados, estes dados permitem direcionar mais recursos para uma indicada plataforma e/ou serviço, assim como também escolher os melhores horários de manutenção dos servidores.

\subsubsection{Gestão de \textit{emails}}\label{sec:emails_send}
O envio de \textit{emails} para os utilizadores, foi desenvolvido através da biblioteca \textit{Nodemailer}, que permite a utilização de um servidor de \textit{SMTP}. Esta ferramenta foi escolhida devido a ser uma das mais utilizadas para este tipo de necessidade, o que permite que exista mais informação sobre a mesma que auxilia a resolução e identificação de erros. 

Para desenvolver o conteúdo dos \textit{emails} foi utilizada a ferramenta \textit{Tabular Email}, esta permite realizar o \textit{design} do conteúdo de um \textit{email}, sendo possível exportar para \textit{html}. A maior dificuldade desta ferramenta é que não permite a utilização de acentuação, e visto que o \textit{html} é gerado por uma máquina este torna-se complicado de navegar e traduzir.

\subsubsection{Agendamento de tarefas}

Um requisito deste projeto foi o envio diário de \textit{emails} com o relatório de notificações ao final do dia. Primeiramente, para realizar o agendamento de tarefas, foi feita uma análise das ferramentas existentes para a realização deste tipo de ações. Deste modo, foram encontradas o \textit{cronetab} e o \textit{node-cron}. A grande diferença entre estas duas ferramentas é que o \textit{cronetab} funciona a nível de servidor, sendo que, o funcionamento tem por base "\emph{run this command at this time
on this date}"\citep{crontab}, este comando poderá por exemplo executar um código para enviar \textit{emails}. Já o \textit{node-cron} trata-se de uma biblioteca de \textit{NodeJs} "\emph{in pure JavaScript for node.js based on GNU crontab}"\citep{node_cron}, este permite o fácil agendamento de tarefas de forma programática, assim como a indicação do código a ser executado sem necessidade de criar comandos.

A hora de execução do código de envio de \textit{emails} poderá variar e necessitar de reprogramação, pelo que, foi optada a utilização do \textit{node-cron}, uma vez que, facilita a utilização e agilizando o processo de reprogramação de horas de envio do relatório de notificações.

\input{sections/chap2/tecnologias_utilizadas/backend/5.segurança.tex}

\subsubsection{Firebase}
Firebase é uma solução que foi comprada pela \textit{Google} em 2014. O seu objetivo é "\emph{to provide the tools and infrastructure that you need to build great apps}"\citep{Moroney2017}, esta alcança este objetivo através da oferta de serviços pré configurados, sendo que "\emph{many of the technologies are available at no cost}"\citep{Moroney2017}.

\textit{Firestorage} também conhecida como \textit{cloud storage}, é um serviço que dispõe "\emph{a simple API that is backed up by Google Cloud Storage}"\citep{Moroney2017}, que permite guardar e transmitir até 1\textit{GB} de ficheiros de forma gratuita.

\textit{Cloud messaging} é também um serviço do \textit{Firebase} que permite "\emph{to reliably deliver messages at no cost}"\citep{Moroney2017}. Este garante que" \emph{Over 98\% of connected devices receive these messages in less than 500ms}"\citep{Moroney2017}. \textit{Cloud messaging} permite a utilização de diversas formas de envio de notificações como "\emph{driven by analytics to pick audiences, or using topics or other methods}"\citep{Moroney2017}.

\textit{Dynamic links} é um serviço do \textit{Firebase} que permite a criação de "\emph{links to an app that contain context about what you want the end user to see in the app}"\citep{Moroney2017}.

Esta ferramenta foi escolhida devido à sua capacidade de fornecer serviços pré configurados de forma gratuita, o que evita o gasto monetário e a alocação de tempo para a configuração de servidores durante o desenvolvimento.

\subsection{Axios}

Para ser possível realizar pedidos a outros serviços externos como \textit{Firebase}, é necessário utilizar uma biblioteca capaz do mesmo. Para isso foi optado por utilizar \textit{Axios}. Esta é "(...)\emph{a promise-based HTTP Client for node.js}(...)"\citep{axios} que "(...)\emph{uses the native node.js http module}(...)"\citep{axios}. Esta disponibiliza um conjunto de métodos para a realização de pedidos a serviços externos, assim como a configuração total dos mesmos.