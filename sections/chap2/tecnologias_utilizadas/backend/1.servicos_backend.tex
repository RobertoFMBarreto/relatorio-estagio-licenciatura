\subsection{Serviços Backend}

De forma a realizar a integração entre a aplicação \emph{frontend} e os dados, foi necessário desenvolver uma API para dar suporte a todos os serviços necessários para a aplicação.
API sigla para \emph{Application Programming Interface} disponibiliza um conjunto de funções e dados que facilita as interações entre aplicações e permite que troquem informação ~\cite{rest_cookbook}.
Esta ferramenta apesar de ser desenvolvida para trabalhar em conjunto com outros programas, ela são em sua grande maioria desenvolvidas para serem entendidas e utilizadas por outros programadores no
desenvolvimento dos seus programas ~\cite{api_design}.

\subsubsection{Serviços REST Full}
% //TODO:
Explicar o que é


\subsubsection{Typescript}
% //TODO:
Explicar o que é

\subsubsection{Logging}
Logging é um processo que permite guardar informação(logs) sobre um evento. Neste contexto logging poderá ser utilizado para realizar a monitorização de pedidos e erros. Estas informações poderão até auxiliar na toma de decisões sobre o software e em quais funcionalidades deste software colocar mais atenção.

\subsubsection{Morgan}

Morgan é uma ferramenta que permite extrair dados de um pedido, assim como também a criação de logs, este atua como um middleware do servidor, recebendo qual o tipo de log a ser escrito, sendo estes tipos definidos pela ferramenta. Os principais dados obtidos pela ferramenta são a data e hora do pedido, o tipo de pedido, o serviço pedido, os dados recebidos, a resposta devolvida e também a descrição do sistema utilizado para realizar o pedido. Com estes dados é possível saber que plataforma é mais utilizada no software, quais as horas de maior utilização e quais os serviços mais executados, estes dados permitem direcionar mais recursos para uma indicada plataforma e/ou serviço, assim como também escolher os melhores horários de manutenção dos servidores.

\newpage


\subsubsection{Gestão de \textit{emails}}\label{sec:emails_send}
O envio de \textit{emails} para os utilizadores, foi desenvolvido através da biblioteca \textit{Nodemailer}, que permite a utilização de um servidor de \textit{SMTP}. Esta ferramenta foi escolhida devido a ser uma das mais utilizadas para este tipo de necessidade, o que permite que exista mais informação sobre a mesma que auxilia a resolução e identificação de erros. 

Para desenvolver o conteúdo dos \textit{emails} foi utilizada a ferramenta \textit{Tabular Email}, esta permite realizar o \textit{design} do conteúdo de um \textit{email}, sendo possível exportar para \textit{html}. A maior dificuldade desta ferramenta é que não permite a utilização de acentuação, e visto que o \textit{html} é gerado por uma máquina este torna-se complicado de navegar e traduzir.

\subsubsection{Agendamento de tarefas}

Um requisito deste projeto foi o envio diário de \textit{emails} com o relatório de notificações ao final do dia. Primeiramente, para realizar o agendamento de tarefas, foi feita uma análise das ferramentas existentes para a realização deste tipo de ações. Deste modo, foram encontradas o \textit{cronetab} e o \textit{node-cron}. A grande diferença entre estas duas ferramentas é que o \textit{cronetab} funciona a nível de servidor, sendo que, o funcionamento tem por base "\emph{run this command at this time
on this date}"\citep{crontab}, este comando poderá por exemplo executar um código para enviar \textit{emails}. Já o \textit{node-cron} trata-se de uma biblioteca de \textit{NodeJs} "\emph{in pure JavaScript for node.js based on GNU crontab}"\citep{node_cron}, este permite o fácil agendamento de tarefas de forma programática, assim como a indicação do código a ser executado sem necessidade de criar comandos.

A hora de execução do código de envio de \textit{emails} poderá variar e necessitar de reprogramação, pelo que, foi optada a utilização do \textit{node-cron}, uma vez que, facilita a utilização e agilizando o processo de reprogramação de horas de envio do relatório de notificações.

\input{sections/chap2/tecnologias_utilizadas/backend/5.segurança.tex}

\subsubsection{Firebase}
Firebase é uma solução que foi comprada pela \textit{Google} em 2014. O seu objetivo é "\emph{to provide the tools and infrastructure that you need to build great apps}"\citep{Moroney2017}, esta alcança este objetivo através da oferta de serviços pré configurados, sendo que "\emph{many of the technologies are available at no cost}"\citep{Moroney2017}.

\textit{Firestorage} também conhecida como \textit{cloud storage}, é um serviço que dispõe "\emph{a simple API that is backed up by Google Cloud Storage}"\citep{Moroney2017}, que permite guardar e transmitir até 1\textit{GB} de ficheiros de forma gratuita.

\textit{Cloud messaging} é também um serviço do \textit{Firebase} que permite "\emph{to reliably deliver messages at no cost}"\citep{Moroney2017}. Este garante que" \emph{Over 98\% of connected devices receive these messages in less than 500ms}"\citep{Moroney2017}. \textit{Cloud messaging} permite a utilização de diversas formas de envio de notificações como "\emph{driven by analytics to pick audiences, or using topics or other methods}"\citep{Moroney2017}.

\textit{Dynamic links} é um serviço do \textit{Firebase} que permite a criação de "\emph{links to an app that contain context about what you want the end user to see in the app}"\citep{Moroney2017}.

Esta ferramenta foi escolhida devido à sua capacidade de fornecer serviços pré configurados de forma gratuita, o que evita o gasto monetário e a alocação de tempo para a configuração de servidores durante o desenvolvimento.