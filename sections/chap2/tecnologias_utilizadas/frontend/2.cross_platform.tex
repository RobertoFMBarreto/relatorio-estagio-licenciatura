\subsubsection{Desenvolvimento cross-platform}
O desenvolvimento de aplicações \textit{cross-platform} ou multi-plataforma, consiste no desenvolvimento de uma aplicação para diversas plataformas e este pode ser realizado de diversas formas, mas as principais formas conhecidas são WebView, nativo e outras abordagens.

As \emph{frameworks} nativas são "(...)\emph{the most stable choice for mobile application development}(...)"\citep{flutter} e dispõem de um grande comunidade e leque de aplicações desenvolvidas. O que torna estas \textit{frameworks} estáveis é o facto de "(...)\emph{the app in this framework talks directly to the system}(...)"\citep{flutter}. Todo o desenho no ecrã é realizado através de o que é chamado de \emph{OEM components} que são disponibilizados pela \emph{framework} mas não permitem customização total. A grande desvantagem desta abordagem é o facto de se o objetivo do projeto é o desenvolvimento para \textit{iOS} e \textit{Android}, então "(...)\emph{you need to learn two different languages}(...)"\citep{flutter}, porque estas são utilizadas para "(...)\emph{write two different apps with the same functionalities}(...)"\citep{flutter} o que significa que "(...)\emph{every modification must be duplicated on both platforms}(...)"\citep{flutter}.

Uma outra abordagem para o desenvolvimento para diversas plataformas através de uma única base de código é o \textit{WebView}. "(...)\emph{Cordova-, Ionic-, PhoneGap-, and WebView-based frameworks in general are good examples of cross-platform frameworks}(...)"\citep{flutter}, mas o grande problema desta abordagem é a "(...)\emph{lack in performance}(...)"\citep{flutter} pois esta é composta por um processo intermédio chamado \textit{WebView} que renderiza código \textit{HTML}, isto significa que "(...)\emph{the app is basically a website}(...)"\citep{flutter}.
Esta abordagem acrescenta também o componente de ponte que realiza o "(...)\emph{switch between JavaScript to the native system}(...)"\citep{flutter} para obter acesso aos serviços nativos.

Um concorrente à tecnologia mencionada na secção seguinte(\ref{flutter_explaining}) é o \textit{React Native}, este assim como as \textit{frameworks} nativas "(...)\emph{heavily relies on OEM components}(...)"\citep{flutter} e "(...)\emph{expands the bridge concept in the WebView systems, and uses it not only for services, but also to build widgets}(...)"\citep{flutter}, isto leva a grandes problemas em termos de performance devido a que "(...)\emph{a component may be built hundreds of times during an animation, but due to the expanded concept of the bridge, this component may slow down to a great extent}(...)"\citep{flutter}.