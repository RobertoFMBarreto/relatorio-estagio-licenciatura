\subsection{Processo de aprendizagem}
De forma a realizar a aprendizagem da ferramenta foi então procedido para a desenvolvedora da mesma, a \textit{Google}, esta dispõe de uma lista de \textit{workshops} e projetos a seguir para aprender as bases da ferramenta, sendo a lista a seguinte:

\begin{enumerate}
 \item MDC-101 no Flutter: noções básicas dos componentes do Material Design | Google Codelabs
 \item MDC-102 no Flutter: estrutura e layout do Material Design | Google Codelabs
 \item MDC-103 Flutter: temas do Material Design com cores, formas, elevação e tipo | Google Codelabs
 \item MDC-104 Flutter: componentes avançados do Material Design | Google Codelabs
 \item Apps adaptáveis no Flutter | Google Codelabs
\end{enumerate}

Após a realização destes \textit{workshops}, foi possível interiorizar como funciona a base desta ferramenta e também encontrar fontes para pesquisa de \textit{widgets} da comunidade a nível gráfico e funcional, estes provaram ser de grande auxílio no desenvolvimento da aplicação \textit{frontend}.