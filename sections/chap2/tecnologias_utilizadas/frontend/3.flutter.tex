\subsection{Flutter}
\textit{Flutter} é uma \textit{framework} desenvolvida pela \textit{Google}, de inicio "\emph{was an experiment, as the developers at Google were trying to remove a few compatibility supports from Chrome, to try to make it run smoother}"\citep{flutter}, por fim, acabaram por descobrir que "\emph{they had something that rendered 20 times faster than Chrome did and saw that it had the potential to be something great}"\citep{flutter}. Em suma, \textit{Google} desenvolveu "\emph{a layered framework that communicated directly with the CPU and the GPU in order to allow the developer to customize the applications as much as possible}"\citep{flutter}.

Para o \textit{Flutter} tudo é um \textit{widget}, "\emph{Orientation, layout, opacity,animation... everything is just a widget}"\citep{flutter}, isto permite que os utilizadores "\emph{choose composition over inheritance, making the construction of an app as simple as building a Lego tower}"\citep{flutter}. Todos estes \textit{widgets} oficiais estão identificados no catálogo de widgets do \textit{Flutter}. Como tudo no \textit{Flutter} é composto por \textit{widgets} "\emph{the more you learn how to use, create, and compose them,the better and faster you become at using Flutter}"\citep{flutter}.

A abordagem ao \textit{cross-platform} realizada pelo \textit{Flutter} é baseada em "\emph{AOT (Ahead Of Time) instead of JIT (Just In Time) like the JavaScript solutions}"\citep{flutter} mostradas anteriormente. Esta também permite a conversação direta com o cpu sem necessidade de ponte e "\emph{does not rely on the OEM platform}"\citep{flutter}. Esta permite "\emph{custom components to use all the pixels in the screen}"\citep{flutter}, o que significa que "\emph{the app displays the same on every version of Android and iOS}"\citep{flutter}. Esta também utiliza "\emph{Platform Channels to use the services}"\citep{flutter}, o que permite que "\emph{if you need to use a specific Android or iOS feature, you can do it easily}"\citep{flutter}.