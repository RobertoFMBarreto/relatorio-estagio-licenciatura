\subsection{Flutter}
Flutter é uma framework desenvolvida pela google, de inicio "\emph{lutter was an experiment, as the developers at Google were trying toremove a few compatibility supports from Chrome, to try to make it run smoother.}", por acabaram por descobrir que "\emph{hey had something that rendered 20 times faster than Chrome did and saw thatit had the potential to be something great}". Em suma google desenvolveu "\emph{a layered framework that communicated directly with the CPU and theGPU in order to allow the developer to customize the applications as much as possible}".

Para o flutter tudo é um widget, "\emph{Orientation, layout, opacity,animation... everything is just a widget}", isto permite que os utilizadores "\emph{choose composition over inheritance, making the construction of an app as simple as building a Lego tower}". Todos estes widgets oficiais estão catalogados no catalogo de widgets do flutter. Como tudo no flutter é composto por widgets "\emph{the more you learn how to use, create, and compose them,the better and faster you become at using Flutter}".

A abordagem ao cross-platform realizada pelo flutter é baseada em  "\emph{AOT (Ahead Of Time) instead of JIT (Just In Time) like the JavaScript solutions}" mostradas anteriormente. Esta também permite a conversação direta com o cpu sem necessidade de ponte e "\emph{does not rely on the OEM platform}". Esta permite "\emph{custom components to use all the pixels in the screen}, o que significa que "\emph{the app displays the same on every version of Android and iOS}". Esta também utiliza "\emph{Platform Channels to use the services}", o que permite que "\emph{if you need to use a specific Android or iOS feature, you can do it easily}".