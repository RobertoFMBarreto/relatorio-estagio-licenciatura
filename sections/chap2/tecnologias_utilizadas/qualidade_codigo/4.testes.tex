\subsubsection{Testes de código}
Aquando o fim do desenvolvimento de cada serviço é necessário testar este para verificar se a funcionalidade encontra-se de acordo com o desejado e/ou existem erros de código. Para realizar estes testes poderão ser utilizadas ferramentas de auxílio ou então poderão ser realizados manualmente. O grande problema dos testes manuais é a exaustividade, uma vez que, são longos e propensos a erros, pelo que, são realizados em menor escala. Para os testes deste projeto foram utilizadas ferramentas de auxílio, sendo as ferramentas escolhidas \textit{Mocha} e \textit{Chai}.

\textit{Mocha} é uma \textit{framework} de testes \textit{JavaScript} que permite teste assíncrono."(...)\emph{Mocha tests run serially, allowing for flexible and accurate reporting, while mapping uncaught exceptions to the correct test cases}(...)"\citep{mocha}. Esta ferramenta foi escolhida devido à sua simplicidade e à capacidade de testar código \textit{Typescript} para além de \textit{javascript}.

\textit{Chai} é "(...)\emph{a BDD / TDD assertion library for node and the browser that can be delightfully paired with any javascript testing framework}(...)"\citep{chai}, neste caso esta foi utilizada em conjunto com \emph{Mocha}. A utilização de \emph{BDD} permite a utilização de uma "(...)\emph{expressive language \& readable style}(...)"\citep{chai}, já \emph{TDD} é "(...)\emph{more classical feel}(...)"\citep{chai}. Para tornar os testes de código mais interpretáveis foi explorado o estilo de testes \emph{BDD}.

A realização de testes de código foi muito importante, visto que, com esta ferramenta foi possível encontrar erros de lógica de negócio, bem como, erros de código tanto a nível da formulação de respostas como a nível de código.



