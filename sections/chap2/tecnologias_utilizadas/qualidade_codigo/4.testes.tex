\subsubsection{Testes de código}
Aquando o fim do desenvolvimento de cada serviço é necessário testar este para verificar se a funcionalidade encontra-se de acordo com o desejado e/ou existem erros de código. Para realizar estes testes poderão ser utilizadas ferramentas de auxílio ou então poderão ser realizados manualmente. O grande problema dos testes manuais é a exaustividade, uma vez que, são longos e propensos a erros, pelo que, são realizados em menor escala. Para os testes deste projeto foram utilizadas ferramentas de auxílio, sendo as ferramentas escolhidas \textit{mocha} e \textit{chai}.

%//TODO APRESENTAR O QUE É MOCHA E PORQUÊ DE SE TER ESCOLHIDO

%//TODO APRESENTAR O QUE É CHAI E PORQUÊ DE SE TER ESCOLHIDO

A realização de testes de código foi muito importante, visto que, com esta ferramenta foi possível encontrar erros de lógica de negócio, bem como, erros de código tanto a nível da formulação de respostas como a nível de código.



