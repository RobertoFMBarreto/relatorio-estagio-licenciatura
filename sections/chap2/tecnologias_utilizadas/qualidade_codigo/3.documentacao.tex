\subsubsection{Documentação}
De forma a ser possível manter todo o projeto desenvolvido foi criada documentação a diferentes níveis, sendo esta desenvolvida a nivel de serviços explicando o objetivo do serviço e que dados este recebe, como também a nivel de código explicando o código desenvolvido em cada script existente. Para estes níveis de documentação foram utilizadas diferentes ferramentas, para documentação de serviços, foi utilizada a ferramenta swagger e para a documentação foi utilizado typedoc.

\subsubsection{Typedoc}
Typedoc é uma ferramenta que faz utilização de comentários de código para gerar a sua documentação, esta documentação utiliza chaves específicas para detetar as informações de documentação, estas permitem também criar categorias para melhor organizar toda a documentação. Esta documentação permite também a interligação entre si mesma permitindo ao visualizador desta seguir todo o processo. Após a realização de geração de documentação, a ferramenta gera um website onde é possível navegar por toda a documentação gerada.

\newpage

\subsubsection{Swagger}
Swagger é uma ferramenta que permite gerar documentação a nivel de serviços, esta documentação é acessível a partir do mesmo servidor indicando uma rota para o mesmo, evitando assim outro servidor para hospedar a documentação, a base de toda a documentação encontra-se em um ficheiro no formato json. Esta documentação poderá ser gerada a partir de comentários a nivel de código ou a partir de um ficheiro em formato json como mencionado anteriormente. Este ficheiro em formato json poderá ser mantido manualmente ou automaticamente.