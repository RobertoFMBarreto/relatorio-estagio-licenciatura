\subsubsection{Documentação}
Para ser possível manter todo o projeto desenvolvido, foi criada documentação a diferentes níveis, sendo esta desenvolvida a nível de serviços com a explicação do objetivo do serviço e que dados este recebe, como também a nível de código sendo explicado o código desenvolvido em cada \textit{script} existente. Para estes níveis de documentação foram utilizadas diferentes ferramentas, para documentação de serviços, foi utilizada a ferramenta \textit{swagger} e para a documentação de código foi utilizado o \textit{typedoc}.

\subsubsection{Typedoc}

\textit{Typedoc} é uma ferramenta que realiza a utilização de comentários de código para gerar a sua documentação, esta utiliza chaves específicas para detetar as informações e criar categorias para melhor organizar toda a documentação. Esta documentação possibilita a interligação entre si mesma, o que permite ao visualizador seguir todo o processo. Após a realização de geração de documentação, a ferramenta gera um \textit{website} no qual é possível navegar por toda a documentação gerada.

\newpage

\subsubsection{Swagger}

\textit{Swagger} é uma ferramenta que permite gerar documentação a nível de serviços, esta é acessível a partir do mesmo servidor com indicação de uma rota, o que evita outro servidor para hospedar a documentação, toda a base encontra-se num ficheiro no formato \textit{json}. Esta documentação poderá ser gerada a partir de comentários a nível de código ou a partir de um ficheiro em formato \textit{json} como mencionado anteriormente. Este ficheiro em formato \textit{json} poderá ser mantido manualmente ou automaticamente. Neste projeto foi decido manter este manualmente.