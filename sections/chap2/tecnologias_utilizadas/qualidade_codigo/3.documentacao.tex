\subsubsection{Documentação}
Para ser possível manter todo o projeto desenvolvido, foi criada documentação a diferentes níveis, sendo esta desenvolvida a nível de serviços com a explicação do objetivo do serviço e que dados este recebe, como também a nível de código sendo explicado o código desenvolvido em cada \textit{script} existente. Para estes níveis de documentação foram utilizadas diferentes ferramentas, para documentação de serviços, foi utilizada a ferramenta \textit{swagger} e para a documentação de código foi utilizado o \textit{typedoc}.

\subsubsection{Typedoc}

\textit{Typedoc} é um "\emph{documentation generator for TypeScript}"\citep{typedoc}, uma ferramenta "\emph{which reads your TypeScript source files, parses comments contained within them, and generates a static site containing documentation}"\citep{typedoc}. Esta utiliza chaves específicas para detetar as informações e criar categorias para melhor organizar toda a documentação. Esta documentação possibilita a interligação entre si mesma, o que permite ao visualizador seguir todo o processo.

\newpage

\subsubsection{Swagger}

\textit{Swagger} é uma ferramenta "\emph{built around the OpenAPI Specification}"\citep{swagger} que ajudam com "\emph{design, build, document and consume REST APIs}"\citep{swagger}, o \emph{OpenAPI} "\emph{is an API description format for REST APIs}"\citep{swagger} que poderá ser "\emph{written in YAML or JSON}"\citep{swagger}. Esta permite gerar documentação a nível de serviços, esta é acessível a partir do mesmo servidor com indicação de uma rota, o que evita outro servidor para hospedar a documentação. Esta documentação poderá ser gerada a partir de comentários a nível de código ou a partir de um ficheiro em formato \textit{json} ou \textit{yaml}, como mencionado anteriormente. Este ficheiro poderá ser mantido manualmente ou automaticamente. Neste projeto foi decido manter este manualmente em \emph{json}.