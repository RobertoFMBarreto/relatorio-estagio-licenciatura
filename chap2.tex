
\chapter{Estado da arte}
Para a organização de todo o trabalho a desenvolver, visto que este é dividido com mais uma colega, foi utilizada a técnica de desenvolvimento ágil, através da qual foi possível organizar todas as tarefas entre os elementos de desenvolvimento do projeto.

\section{Ferramentas de trabalho utilizadas}

A organização de todas as tarefas, foi realizada na ferramenta \textit{Github Projetcs}. Esta dispõe de funções que permitem ligar um projeto a um repositório de \textit{Github}, onde se consegue personalizar completamente todo o projeto e parâmetros das tarefas que resulta numa organização minuciosa.

O Microsoft Excel foi utilizado para a engenharia de \textit{software} onde foram descritos os requisitos do projeto, \textit{user stories} e também a especificação de casos de uso. Esta ferramenta também foi utilizada para a organização de reuniões com o cliente e redação de tópicos a abordar.

Para o desenvolvimento do \textit{design do software} foi utilizado a ferramenta \textit{figma}, que permite o \textit{design} de todas as componentes tendo em conta as reais dimensões de um dispositivo. Esta ferramenta dispõe de funções para criar apresentações interativas que conseguem demonstrar o comportamento da aplicação como resultado final, dando também suporte à implementação.

O \textit{draw.io} foi utilizado para os desenhos das arquiteturas do projeto tendo revelado grande auxílio, uma vez que, permite uma grande liberdade ilustrativa. Esta ferramenta permite conectar com o \textit{github} o que proporciona a facilidade de guardar projetos e ter acesso a partir de qualquer dispositivo.

A engenharia de \textit{software} foi realizada através da utilização \textit{Visual Studio paradigm}, esta é uma ferramenta muito completa que contém modelos e regras para a engenharia de \textit{software}. Esta tornou-se um grande recurso no desenvolvimento da base de dados, dado que é possível desenhar o modelo e exportar para um ficheiro de criação de base de dados.

\newpage

\section{Tecnologias utilizadas}

\subsection{Web Scraper}
\textit{Web scraping} é terminologia dada à "(...)\emph{construction of an agent to download, parse, and organize data from the web in an automated manner}(...)"\citep{web_scraping}. O grande problema com \textit{web scraping} é que poderá ser considerado ilegal e é facilmente detetado. Tendo em conta este problema surgiram duas grandes formas de realizar \textit{web scraping}. A forma mais comum de \textit{web scraping} é realizar um pedido para obter uma página web e ler esta, sendo assim um processo rápido e simples.

A segunda forma de realizar \textit{web scraping} é através da simulação da ação humana com da abertura do navegador programáticamente, pesquisa pela página desejada, descarregar e daí ler os dados. Este torna-se um processo lento e complexo. 

A grande diferença entre estas duas formas é a velocidade, visto que a segunda forma tem de esperar que o navegador inicie, de seguida terá de esperar que a página carregue e apenas após este processo se poderá ler os dados.

\subsubsection{Selenium}

O \emph{Selenium} é uma ferramenta "(...)\emph{for a range of tools and libraries that enable and support the automation of web browsers}(...)"\citep{selenium}, esta provém "(...)\emph{extensions to emulate user interaction with browsers}(...)"\citep{selenium}. Na sua base esta "(...)\emph{is WebDriver, an interface to write instruction sets that can be run interchangeably in many browsers}(...)"\citep{selenium}. Esta ferramenta provém também a possibilidade de escalar com \textit{multi threading}, o que permite abrir diversas janelas do navegador simultaneamente e obter dados destas o que diminui drasticamente o tempo de execução, esta funcionalidade não foi explorada devido a limitações de \textit{hardware}, mas seria uma importante implementação futura.
\subsection{Serviços Backend}

De forma a realizar a integração entre a aplicação \emph{frontend} e os dados, foi necessário desenvolver uma API para dar suporte a todos os serviços necessários para a aplicação.
API sigla para \emph{Application Programming Interface} disponibiliza um conjunto de funções e dados que facilita as interações entre aplicações e permite que troquem informação ~\cite{rest_cookbook}.
Esta ferramenta apesar de ser desenvolvida para trabalhar em conjunto com outros programas, ela são em sua grande maioria desenvolvidas para serem entendidas e utilizadas por outros programadores no
desenvolvimento dos seus programas ~\cite{api_design}.

\subsubsection{Serviços REST Full}
% //TODO:
Explicar o que é


\subsubsection{Typescript}
% //TODO:
Explicar o que é

\subsubsection{Logs e Logging}
\textit{Logs} "(...)\emph{are a very useful source of information for computer system resource management (printers, disk systems, battery backup systems, operating systems, etc.), user and application management (login and logout, application access, etc.), and security}(...)"\citep{Logging}. Um Log é "(...)\emph{what a computer system, device, software, etc. generates in response to some sort of stimuli.What exactly the stimuli are greatly depends on the source of the log message}(...)"\citep{Logging}, ou seja, perante um estímulo desejado, um \textit{log} poderá ser gerado. Os dados dos \textit{logs} são "(...)\emph{the intrinsic meaning that a log message has}(...)"\citep{Logging}, o que significa que estes contêm apenas dados relevantes ao objetivo do \textit{log}. \textit{Logging} é o nome que se dá ao processo de geração de \textit{logs}.

\newpage

Neste contexto \textit{logging} poderá ser utilizado para realizar a monitorização de pedidos e erros. Estas informações poderão até auxiliar na toma de decisões sobre o \textit{software} e em quais funcionalidades colocar mais atenção.

\subsubsection{Morgan}

\textit{Morgan} é uma biblioteca que permite extrair dados de um pedido, assim como a criação de \textit{logs}. Este atua como um \textit{middleware} do servidor, que recebe qual o tipo de \textit{log} a ser escrito, estes estão definidos na biblioteca. Os principais dados obtidos por este são, a data e hora do pedido, o tipo, o serviço, os dados recebidos, a resposta devolvida e também a descrição do sistema utilizado. Com estes dados é possível saber que plataforma é mais utilizada no \textit{software}, quais as horas de maior utilização e quais os serviços mais executados. Estes dados permitem direcionar mais recursos para uma indicada plataforma e/ou serviço, assim como escolher os melhores horários de manutenção dos servidores.

\newpage


\subsubsection{Envio de emails}
Para o envio de emails para os utilizadores, foi utilizada a ferramenta nodemailer, EXPLICAR O QUE é nodemailer..... Esta ferramenta foi escolhida devido a ser uma das mais utilizadas para esta tipo de necessidade, o que permite que exista mais informação sobre a mesma facilitando a resolução e identificação de erros. 

Para desenvolver o conteúdo dos emails foi utilizada a ferramenta Tabular Email, esta ferramenta permite realizar o design do conteúdo de um email, sendo possível de seguida exportar o mesmo para html, a dificuldade desta ferramenta é que não permite a utilização de acentuação e visto que o html é gerado por uma máquina este torna-se complicado de navegar e traduzir.

Após obter o servidor a utilizar e o conteúdo a enviar, é então utilizado o objeto do servidor de email e no envio é definido o destinatário, o assunto e o conteúdo do email.


\subsubsection{Agendamento de tarefas}

Um requisito para este projeto é o envio de emails com relatório diário de notificação todos os dias ao final do dia. Para realizar este primeiramente foi pesquisado que ferramentas existem para realizar este tipo de ações, pelo que foram encontradas o cronetab e o node-cron. A grande diferença este estas duas ferramentas é, o cronetab funciona a nivel de servidor sendo que sempre que se encontra na hora programada, este executa um comando indicado, este comando poderá por exemplo executar um código para enviar emails. Já o node-cron trata-se de uma biblioteca de NodeJs que trabalha com base no crontab, este permite o fácil agendamento de tarefas de forma programática assim como também a indicação do código a ser executado sem necessidade de criar comandos de execução de código.

Visto que a hora de execução do código de envio de emails poderá variar e necessitar de reprogramação foi então optado pela utilização do node-cron devido á sua facilidade de utilização, agilizando assim o processo de reprogramação de horas de envio de relatório.

\subsubsection{Encriptação de passwords}
De forma a garantir a segurança das passwords dos utilizadores é necessário encriptar estas, a encriptação poderá ser feita manualmente ou com o auxílio de ferramentas, a grande diferença é que manualmente poderá não se obter uma cifra tão segura como com o auxílio de uma ferramenta, sendo assim foi decidido utilizar uma ferramenta para encriptar passwords, a ferramenta escolhida foi bcrypt, esta foi escolhida devido a ser vastamente utilizada e até ao momento sem problemas em relação à sua cifra sendo que não existem registos de ataques bem sucedidos a esta ferramenta. Esta ferramenta oferece um conjunto de métodos sendo tendo sido utilizados os métodos de cifra e de comparação. O método de cifra permite através de um valor, chamado salt, indicar a complexidade a aplicar sobre a cifra, sendo de seguida devolvida a password cifrada. O método comparação permite comparar uma password cifrada com uma password sem cifra, devolvendo verdadeiro ou falso conforme as passwords sejam iguais ou não.

\newpage

\subsubsection{Cifragem de configurações do servidor}
De forma a garantir um nivel de segurança maior foram realizadas pesquisas sobre as principais falhas de segurança no NodeJs, pelo que foi descoberto que as principais formas de ataque a esta ferramenta é o desenvolvimento de bibliotecas de malware e o ataque às bibliotecas com o objetivo de obter dados de acesso a servidores que se encontram nos ficheiros de configuração.

Por norma todas as configurações de servidores são colocadas num ficheiro env, este ficheiro no momento de iniciar o servidor é utilizado para carregar todas as variáveis para o ambiente do mesmo, sendo assim qualquer um com acesso ao ficheiro ou às variáveis de ambiente poderá ver todas as configurações do servidor.

A solução mais indicada para este problema é a cifragem do ficheiro env e das variáveis de ambiente. A biblioteca mais utilizada para este objetivo é a secur-env, esta permite realizar a cifragem de um ficheiro indicando uma password. A password deverá ser indicada no processo de inicialização do servidor de forma a ser possível ao mesmo decifrar o ficheiro, sendo que a gestão das variáveis cifradas passa então a estar encarregue desta biblioteca.

Mesmo com esta solução existem possibilidades de ataque, pois é possível ver o histórico do terminal do servidor, pelo que e possível obter a password escrita no mesmo, para resolver este problema é indicada a biblioteca readline, pois esta possui o modo de password que apaga o histórico do terminal sempre que utilizado, esta contém a vertente assincrona, readline e a vertente síncrona, readline-sync. Para o projeto foi utilizada a versão síncrona da biblioteca visto que o objetivo é o servidor apenas inciar após a indicação da password, sem nenhum serviço a correr em simultâneo.

\newpage

\section{Frontend}
Um dos requisitos do projeto é o desenvolvimento do frontend utilizando a ferramenta Flutter, visto que a empresa já utiliza esta ferramenta, sendo assim necessário o aprendizado desta ferramenta e da sua linguagem de programação o dart.

\subsection{Flutter}
% //TODO:EXPLICAR O QUE E FLUTTER
EXPLICAR O QUE E FLUTTER
\subsection{Dart}
% //TODO:EXPLICAR O QUE É DART
EXPLICAR O QUE É DART
\input{sections/chap2/tecnologias_utilizadas/frontend/4.processo_aprendizagem.tex}


\newpage

\subsection{Qualidade de código}

\newpage

\section{Planificação do trabalho}\label{sec:planificacao trabalho}

Para obter uma visão geral do projeto e uma previsão de finalização foi realizada uma planificação expectável de tarefas.

\begin{figure}[htb]
    \centering
    
    \includegraphics[width=0.75\textwidth]{images/etapa1_sprint_planning.png}
    \caption{Planificação de sprints}
    \label{fig:1}
\end{figure}



